
\documentclass[a4paper,11pt]{article}

% Math symbols
\usepackage{amsmath}
\usepackage{amsfonts}
\usepackage{esvect}

% Hyperlink contents page
\usepackage{hyperref}
\hypersetup{
	colorlinks,
	citecolor=black,
	filecolor=black,
	linkcolor=black,
	urlcolor=black
}

% No indent on new paragraphs
\setlength{\parindent}{0mm}
\setlength{\parskip}{0.2cm}

% Alias \boldsymbol to \bb for vectors
\newcommand{\bb}{\boldsymbol}


\begin{document}

\title{Logarithmic Functions}
\author{Ben Anderson}
\date{\today}
\maketitle
\pagebreak

\tableofcontents
\pagebreak


\section{Logarithm}

A logarithmic function is the inverse of an exponential function.

Given:

$$
a^x = b
$$

Then:

$$
\log_a{b} = x
$$

The logarithm of a negative number does not produce a real valued result.


\subsection{Base}

The base of the logarithm is written in subscript below the function:

$$
\log_a{x}
$$

$\log{x}$ without any specified base is assumed to be base 10.

$\ln{x}$ is assumed to be base $e$.


\subsection{Identities}

The logarithm of 1 (in any base) is always 0:

$$
\log_a{1} = 0
$$

This is because any number raised to the 0th power is 1:

$$
a^0 = 1
$$

The logarithm of the base is 1:

$$
\log_a{a} = 1
$$

This is because any number raised to the power of 1 is itself:

$$
a^1 = a
$$



\section{Laws}

\subsection{Exponents}

Any exponent of a value in a logarithm can be brought out the front as a
coefficient:

$$
\log_a{b^n} = n \log_a{b}
$$


\subsection{Negation}

To simplify a negative logarithm:

$$
-\log_a{b} = \log_a(\frac{1}{b})
$$


\subsection{Addition}

The sum of two logarithms with the same base can be combined:

$$
\log_a{b} + \log_a{c} = \log_a{bc}
$$

This can be generalised to:

$$
\sum_{i = 1} \log_a{x_i} = \log_a(\prod_{i = 1} x_i)
$$


\subsection{Subtraction}

Applying the rule for negation and addition of logarithms, we can devise a rule
for the subtraction of logarithmic functions:

$$
\log_a{b} - \log_a{c} = \log_a{\frac{b}{c}}
$$


\subsection{Change of Base}

To change the base of a logarithm:

$$
\log_a{b} = \frac{\log_c{b}}{\log_c{a}}
$$

More usefully, we can convert logarithms to base 10:

$$
\log_a{b} = \frac{\log{b}}{\log{a}}
$$




\section{Graphs}

The graph of $\log{x}$ has the following features:

\begin{itemize}
\item A domain of $x \in \mathbb{R}, x > 0$.
\item An x intercept at $x = 1$.
\item A point at $(10, 1)$.
\item A vertical asymptote with equation $x = 0$
\item No y intercept.
\item No stationary points.
\end{itemize}

For a graph of $\log_a{x}$:

\begin{itemize}
\item Still has an x intercept at $x = 1$ and vertical asymptote $x = 0$.
\item A point at $(a, 1)$.
\end{itemize}

For a graph of $\log_a{x + b} + c$:

\begin{itemize}
\item Shifted $c$ units up.
\item Shifted $b$ units left.
\end{itemize}

For a graph of $b \log_a{cx}$:

\begin{itemize}
\item Vertical dilation of scale factor $b$.
\item Horizontal dilation of scale factor $\frac{1}{c}$.
\end{itemize}


\subsection{Semi-Logarithmic Graph Paper}

Where the y axis uses a logarithmic scale, and the x axis a linear one.

Exponential functions such as:

$$
y = ab^{cx}
$$

Appear as straight lines.

Any vertical or horizontal translations to the original exponential function
will no longer make it appear linear on a logarithmic scale.

\end{document}
