
\documentclass[a4paper,11pt]{article}

% Math symbols
\usepackage{amsmath}
\usepackage{amsfonts}
\usepackage{esvect}

% Hyperlink contents page
\usepackage{hyperref}
\hypersetup{
	colorlinks,
	citecolor=black,
	filecolor=black,
	linkcolor=black,
	urlcolor=black
}

% No indent on new paragraphs
\setlength{\parindent}{0mm}
\setlength{\parskip}{0.2cm}

% Alias \boldsymbol to \bb for vectors
\newcommand{\bb}{\boldsymbol}


\begin{document}

\title{Exponential Functions}
\author{Ben Anderson}
\date{\today}
\maketitle
\pagebreak

\tableofcontents
\pagebreak


\section{e}

$e$ is defined as:

$$
\lim_{x \to \infty} (1 + \frac{1}{x})^x = e
$$




\section{Exponential Functions}

Functions of base $e$, with $x$ in the exponent:

$$
f(x) = ae^{kx}
$$


\subsection{Derivative}

$e^x$ is the only function where:

$$
f'(x) = f(x)
$$

And:

$$
\int f(x) dx = f(x)
$$


\subsection{Multiple of Derivative}

When given the derivative of a function is a constant multiplied by the
original function:

$$
\frac{dy}{dx} = ky
$$

This implies the function $y$ is:

$$
y = ae^{kx}
$$


\subsection{Decay}

An exponential decay function has a negative coefficient in front of $x$ in the
exponent:

$$
y = ae^{-kx}
$$




\section{Half Life}

A radioactive substance which decays exponentially has a half life of 60 days.

After 43 days, 200 g of a lump of the substance remains.

The amount of substance remaining is given by the equation, where $t$ is the
time in days:

$$
y = a e^{kt}
$$

Solving for $k$:

$$
\begin{aligned}
\frac{a}{2} & = a e^{kt} \\
\frac{1}{2} & = e^{60k} \\
k & = -0.01155 \\
\end{aligned}
$$

Solving for $a$:

$$
\begin{aligned}
200 & = a e^{-0.01155 \times 43} \\
a & = 328.6\text{ g} \\
\end{aligned}
$$

Thus 328.6 g of the substance was present initially.


\subsection{Without Using e}

The amount of substance can also be expressed as:

$$
y = a 0.5^{\frac{t}{60}}
$$

Where the exponent represents the fractional progress towards the next half
life.

Solving for $a$:

$$
\begin{aligned}
200 & = a 0.5^{\frac{43}{60}} \\
a & = 328.6\text{ g} \\
\end{aligned}
$$

\end{document}
