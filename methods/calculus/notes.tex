
\documentclass[a4paper,11pt]{article}

% Math symbols
\usepackage{amsmath}
\usepackage{amsfonts}
\usepackage{esvect}

% Hyperlink contents page
\usepackage{hyperref}
\hypersetup{
	colorlinks,
	citecolor=black,
	filecolor=black,
	linkcolor=black,
	urlcolor=black
}

% No indent on new paragraphs
\setlength{\parindent}{0mm}
\setlength{\parskip}{0.2cm}

% Alias \boldsymbol to \bb for vectors
\newcommand{\bb}{\boldsymbol}


\begin{document}

\title{Calculus}
\author{Ben Anderson}
\date{\today}
\maketitle
\pagebreak

\tableofcontents
\pagebreak


\section{Differentiation}

\subsection{Notation}

The first derivative of the function $f(x)$ with respect to $x$ is $f'(x)$.

The second derivative is $f''(x)$.

The derivative of $y$ with respect to $x$ is $\frac{dy}{dx}$.

The second derivative of $y$ is $\frac{d^2y}{dx^2}$.


\subsection{First Principles}

The derivative of $f(x)$ is defined as:

$$
f'(x) = \lim_{h \to \infty} \frac{f(x + h) - f(x)}{h}
$$

This is referred to as differentiating using first principles.


\subsection{Addition and Subtraction}

Given the function $f(x) = g(x) + h(x)$, its derivative is:

$$
f'(x) = g'(x) + h'(x)
$$

Differentiation distributes across addition.


\subsection{Polynomials}

Given the polynomial function $f(x) = ax^n$, its derivative is:

$$
f'(x) = anx^{n - 1}
$$

Multiply by the current power and decrease the power by 1.

$n$ may be a negative or fractional exponent.


\subsection{Trigonometric Functions}

The derivative of $\sin{x}$:

$$
\frac{d}{dx} \sin{x} = \cos{x}
$$

The derivative of $\cos{x}$:

$$
\frac{d}{dx} \cos{x} = -\sin{x}
$$

The derivative of $\tan{x}$:

$$
\frac{d}{dx} \tan{x} = \frac{1}{\cos^2{x}}
$$



\subsection{Exponential Functions}

The derivative of $e^x$:

$$
\frac{d}{dx} e^x = e^x
$$

The derivative of $e^{f(x)}$:

$$
\frac{d}{dx} e^{f(x)} = f(x)e^{f(x)}
$$


\subsection{Common Derivatives}

Some common derivatives:

$$
\begin{aligned}
\frac{d}{dx} x^2 & = 2x \\
\frac{d}{dx} \frac{a}{x} & = -\frac{a}{x^2} \\
\frac{d}{dx} \sqrt{x} & = \frac{1}{2\sqrt{x}} \\
\end{aligned}
$$


\subsection{Product Rule}

Given the function $f(x) = g(x) h(x)$, its derivative is:

$$
f'(x) = g'(x) h(x) + h'(x) g(x)
$$


\subsection{Quotient Rule}

Given the function $f(x) = \frac{g(x)}{h(x)}$, its derivative is:

$$
f'(x) = \frac{h(x) g'(x) - g(x) h'(x)}{h(x)^2}
$$


\subsection{Chain Rule}

Given the function $f(x) = g(h(x))$, its derivative is:

$$
f'(x) = g'(h(x)) h'(x)
$$

Or for a function $y$:

$$
\frac{dy}{dx} = \frac{dy}{du} \times \frac{du}{dx}
$$




\section{Indefinite Integrals}

\subsection{Notation}

The indefinite integral of the function $f(x)$ is:

$$
\int f(x) dx
$$

Must add the constant of integration $c$ to all indefinite integrals.


\subsection{Polynomials}

Given the polynomial function $f(x) = ax^n$, the indefinite integral is:

$$
\int f(x) dx = \frac{ax^{n + 1}}{n + 1}
$$

Raise the power by 1, and divide by the new power.

$n$ can be a negative or fractional exponent.


\subsection{Trigonometric Functions}

The integral of $\sin{x}$ is:

$$
\int \sin{x} dx = -\cos{x}
$$

The integral of $\cos{x}$ is:

$$
\int \cos{x} dx = \sin{x}
$$


\subsection{Exponential Functions}

The integral of $e^x$ is:

$$
\int e^x dx = e^x
$$

The integral of $e^{f(x)}$ is:

$$
\int e^{f(x)} dx = \frac{e^{f(x)}}{f(x)}
$$


\subsection{Composite Functions}

For a composite function (eg. $12x(x^2 + 3)$), the coefficient must be a scalar
multiple of the derivative of the inner function.

Otherwise, expand the brackets and integrate the polynomial normally.


\subsection{Examples}

Polynomial example:

$$
\int 2x + 3x^2 dx = x^2 + x^3 + c
$$

Hyperbola example:

$$
\int \frac{3}{x^2} dx = -\frac{3}{x} + c
$$

With the chain rule:

$$
\int 16(2x + 3)^3 dx = 2(2x + 3)^4
$$




\section{Definite Integrals}

\subsection{Notation}

The definite integral of $f(x)$ from $a$ to $b$ is:

$$
\int_a^b f(x) dx
$$

Will result in a constant value.


\subsection{Calculating}

To calculate the definite integral of $f'(x)$ from $a$ to $b$:

$$
\int_a^b f'(x) dx = f(b) - f(a)
$$


\subsection{Inverting an Integral}

The definite integral of $f(x)$ can be inverted:

$$
\int_a^b f(x) dx = -\int_b^a f(x) dx
$$


\subsection{Inscribed and Circumscribed Area}

The value of a definite integral can be approximated by calculating the area of
rectangles under the curve.

Inscribed area is the area of rectangles that touch the function, not extending
above the function.

Circumscribed area is the area of rectangles that extend above the function.

Best approximation for the integral is the average of the inscribed and
circumscribed areas.


\subsection{Area Under a Curve}

The definite integral of a function for a section below the x axis is negative.

Thus the area between a curve and the x axis between two x values is not the
definite integral between the two values.


\subsubsection{Splitting the Integral}

Solve for the points at which the function crosses the x axis ($f(x) = 0$).

Determine which parts are above and below the x axis.

Split the integral into sections, negating the sections below the x axis.


\subsubsection{Absolute Value}

Solve the definite integral using the absolute value of the function:

$$
\int_a^b \lvert f(x) \rvert dx
$$


\subsection{Area Between Curves}

Find the area enclosed by two curves $f(x)$ and $g(x)$.

Solve for the x coordinates at which they intersect ($f(x) = g(x)$). In this
example, $x_1$ and $x_2$.

Determine which function is above the other between these two points. In this
example, $f(x)$ is above $g(x)$.

Solve the definite integral between the two points:

$$
\int_{x_1}^{x_2} f(x) - g(x) dx
$$




\section{Graphs}

\subsection{Stationary Points}

Where the first derivative is 0 ($f'(x) = 0$).

Includes turning and horizontal inflection points.


\subsection{Turning Points}

A local maximum or minimum point.

Solve for $f'(x) = 0$ and perform either the sign or second derivative test.


\subsubsection{Nature}

The nature of a turning point refers to whether it is a maximum or minimum.


\subsection{Concavity}

Whether a function is concave up or concave down at a point.

If $f''(x) > 0$, the function is concave up at $x$.

If $f''(x) < 0$, the function is concave down at $x$.

If $f''(x) = 0$, there is likely to be an inflection point. Verify further using
concavity or sign test.


\subsection{Sign Test}

Given $f(x)$ has a stationary point at $x_1$.

Find the sign of the first derivative either side of $x_1$.

\begin{center}
\begin{tabular}{c|c|c|c}
& $x < x_1$ & $x = x_1$ & $x > x_1$ \\
\hline
Minimum Turning Point         & - & 0 & + \\
Maximum Turning Point         & + & 0 & - \\
Rising Horizontal Inflection  & + & 0 & + \\
Falling Horizontal Inflection & - & 0 & - \\
\end{tabular}
\end{center}


\subsection{Second Derivative Test}

Given $f(x)$ has a stationary point at $x_1$.

Find the sign of the second derivative at $x_1$ ($f''(x_1)$).

\begin{center}
\begin{tabular}{c|c}
Sign & Meaning \\
\hline
+ & Minimum turning point \\
- & Maximum turning point \\
0 & Inconclusive, use sign test \\
\end{tabular}
\end{center}


\subsection{Concavity Test}

Given $f(x)$ has a stationary or inflection point at $x_1$.

Find the sign of the second derivative either side of $x_1$.

\begin{center}
\begin{tabular}{c|c|c}
& $x < x_1$ & $x > x_1$ \\
\hline
Minimum Turning Point    & + & + \\
Maximum Turning Point    & - & - \\
Rising Inflection Point  & + & - \\
Falling Inflection Point & - & + \\
\end{tabular}
\end{center}


\subsection{Inflection Points}

Occur when there is a change in concavity.

Solve for $f''(x) = 0$ and verify change in concavity either side of the point
using the concavity test.

A horizontal inflection occurs when the first derivative is also 0.


\subsection{Summary}

\begin{center}
\begin{tabular}{cl}
$f'(x) = 0$ & Stationary point at $x$ \\
$f'(x) = 0$ and $f''(x) > 0$ & Minimum turning point at $x$ \\
$f'(x) = 0$ and $f''(x) < 0$ & Maximum turning point at $x$ \\
$f'(x) = 0$ and $f''(x) = 0$ & Inconclusive, use sign test \\
$f'(x) \neq 0$ and $f''(x) = 0$ & Likely point of inflection, verify with sign test \\
$f'(x) > 0$ & Rising curve \\
$f'(x) < 0$ & Falling curve \\
$f''(x) > 0$ & Concave up \\
$f''(x) < 0$ & Concave down \\
\end{tabular}
\end{center}




\section{Graphing Derivatives}

\subsection{First Derivative}

For stationary points on $f(x)$, $f'(x)$ will pass through the x axis.

For portions of $f(x)$ with a positive gradient, $f'(x)$ will be above the x
axis.

For portions of $f(x)$ with a negative gradient, $f'(x)$ will be below the x
axis.

For rising inflection points on $f(x)$, $f'(x)$ will have a turning point above
the x axis.

For falling inflection points on $f(x)$, $f'(x)$ will have a turning point
below the x axis.


\subsection{Second Derivative}

Take derivative of first derivative graph.

For inflection points on $f(x)$, $f''(x)$ will pass through the x axis.

For concave up portions of $f(x)$, $f''(x)$ will be above the x axis.

For concave down portions of $f(x)$, $f''(x)$ will be below the x axis.




\section{Rates of Change}

Given the rate of change of a quantity with respect to another.

For example, the rate at which water is leaking from a bucket:

$$
\frac{dV}{dt}
$$


\subsection{Water Remaining}

Use an indefinite integral to find a function for the amount of water leaked
from the bucket:

$$
V = \int \frac{dV}{dt} dt
$$

And solve for the integrating constant ($c$).


\subsection{Water Leaked}

Use a definite integral to find the amount of water leaked within a certain
time period of $t_1$ to $t_2$:

$$
\int_{t_1}^{t_2} \frac{dV}{dt} dt
$$


\subsection{Time for Water to Leak}

To find the amount of time it takes for $k$ litres of water to leak from the
bucket, solve the equation:

$$
\int_0^a \frac{dV}{dt} dt = k
$$

For $a$.




\section{Marginal Rates of Change}

There are 3 economic functions, where $x$ is the number of items sold:

\begin{enumerate}
\item The cost function, $C(x)$
\item The revenue function, $R(x)$
\item The profit function, $P(x) = R(x) - C(x)$
\end{enumerate}

The marginal cost $C'(x)$ gives the cost associated with producing one more
item when we are currently producing $x$ items.

The marginal cost of the $x$th item is $C'(x - 1)$.




\section{Rectilinear Motion}

Relates position/displacement ($x$), velocity ($v$), and acceleration ($a$) as
functions of time.

Differentiating:

$$
\begin{aligned}
v & = \frac{dx}{dt} \\
a & = \frac{dv}{dt} \\
\end{aligned}
$$

Integrating:

$$
\begin{aligned}
x & = \int v dt \\
v & = \int a dt \\
\end{aligned}
$$

Position, velocity, and acceleration should be left as positive or negative
values, as they are vectors and have a direction.


\subsection{Terminology}

At the origin implies $x = 0$.

At rest or stationary implies $v = 0$.

"Initially" implies $t = 0$.


\subsection{Speed}

Speed is the absolute value of velocity. It is a scalar and should not have an
associated direction.


\subsection{Displacement}

If finding the displacement of an object between two times, integrate velocity
using a definite integral.

If finding the velocity of an object between two times, integrate acceleration
using a definite integral.


\subsection{Distance}

To find the distance an object travels between two times, integrate the absolute
value of the velocity function using a definite integral.

Or solve the displacement function for its turning points, and calculate the
total distance travelled between these points.




\section{Optimisation}

Problems that involve finding the maximum or minimum value of a function.


\subsection{Within a Domain}

Given the function $f(x)$, to find the maximum value within the domain
$a \leq f(x) \leq b$.

Solve for the turning points of $f(x)$ ($f'(x) = 0$).

Find the value of $f(x)$ at each turning point.

Find the value of $f(x)$ at $a$ and $b$ (the boundaries of the domain).

Chose whichever is the largest.


\subsection{Worded Problem}

Find an equation to optimise.

Differentiate and solve for turning points ($f'(x) = 0$).


\subsection{Two Variables}

Find two equations relating the two variables.

Rearrange one equation for a variable, and substitute it into the other
equation.

Optimise this equation as above.

Find the value of the other variable by substituting back into one of the
original equations.




\section{Small Change}

The derivative of a function will approximate a small change in the function:

$$
\frac{dy}{dx} \approx \frac{\delta y}{\delta x}
$$

This gives us the small change formula:

$$
\delta y \approx \frac{dy}{dx} \delta x
$$

Given a value of $x$, and a small change in $x$ at this value ($\delta x$), the
formula tells us the resulting small change in $y$ ($\delta y$).

The new value of $y$ can be found by adding the small change in $y$
($\delta y$) to the original value of $y$ at $x$.


\subsection{Percentage Change}

Find the approximate percentage change in $y$ when $x$ changes by $k$ percent.

A $k$ percent change in $x$ implies:

$$
\frac{\delta x}{x} = k
$$

The percentage change in $y$ as a result is:

$$
\frac{\delta y}{y}
$$

Given the small change formula:

$$
\delta y \approx \frac{dy}{dx} \delta x
$$

If we divide through by $y$:

$$
\frac{\delta y}{y} \approx \frac{dy}{dx} \times \frac{\delta x}{y}
$$

Substitute in the derivative and the function $y$.

The equation will simplify to some constant times $\frac{\delta x}{x}$
(ie. $k$).



\section{Fundamental Theorem of Calculus}

\subsection{Indefinite Integrals}

Integrating a derivative:

$$
\int f'(x) dx = f(x) + c
$$

Where $f(x)$ does not include the original constant term, replacing it with $c$.

Deriving an integral:

$$
\frac{d}{dx} \int f(x) dx = f(x)
$$


\subsection{Definite Integrals}

Integrating a derivative:

$$
\int_a^b \frac{d}{dx} f(x) dx = f(b) - f(a)
$$

Deriving an integral:

$$
\frac{d}{dx} \int_a^b f(t) dt = 0
$$

As the definite integral will be a constant, resulting in a derivative of 0.


\subsection{Definite Integrals with Chain Rule}

Deriving an integral:

$$
\frac{d}{dx} \int_a^{g(x)} f(t) dt = f(g(x)) g'(x)
$$

The function of $x$ (ie. $g(x)$) must always be on the top of the integral.
Invert the integral if it isn't.

Will not be asked to integrate a derivative (as the integrating constant
complicates the equation).


\subsection{Definite Integrals with Multiple Functions}

Deriving an integral with multiple functions of $x$:

$$
\frac{d}{dx} \int_{g(x)}^{h(x)} f(t) dt = \frac{d}{dx} \int_{0}^{h(x)} f(t) dt - \frac{d}{dx} \int_{0}^{g(x)} f(t) dt
$$

Split the integral into two integrals and invert the second.

Use the rule above for each of the two integrals.




\section{Limits}

Required to calculate the value of some limits.


\subsection{First Principles Limits}

If given a limit which fits the format of the definition of a derivative:

$$
\lim_{h \to \infty} \frac{f(x + h) - f(x)}{h}
$$

Convert to the derivative of $f(x)$ and derive normally to solve the limit.


\subsection{Required Limits}

The required limits are:

$$
\begin{aligned}
\lim_{x \to \infty} (1 + \frac{1}{x})^x & = e \\
\lim_{x \to \infty} (1 + \frac{1}{kx})^x & = e^{\frac{1}{k}} \\
\lim_{x \to 0} \frac{e^x - 1}{x} & = 1 \\
\lim_{x \to 0} \frac{\sin{x}}{x} & = 1 \\
\lim_{x \to 0} \frac{1 - \cos{x}}{x} & = 0 \\
\end{aligned}
$$


\subsection{L'Hopital's Rule}

To solve the last 3 required limits, apply L'Hopital's Rule:

$$
\lim_{x \to c} \frac{f(x)}{g(x)} = \lim_{x \to c} \frac{f'(x)}{g'(x)}
$$

Derive the numerator and denominator, and substitute in $c$ for $x$.

\end{document}
