
\documentclass[a4paper,11pt]{article}

% Math symbols
\usepackage{amsmath}
\usepackage{amsfonts}
\usepackage{esvect}

% Hyperlink contents page
\usepackage{hyperref}
\hypersetup{
	colorlinks,
	citecolor=black,
	filecolor=black,
	linkcolor=black,
	urlcolor=black
}

% Other functions
\DeclareMathOperator\Bin{Bin}
\DeclareMathOperator\Var{Var}
\DeclareMathOperator\SD{SD}

% No indent on new paragraphs
\setlength{\parindent}{0mm}
\setlength{\parskip}{0.2cm}

% Alias \boldsymbol to \bb for vectors
\newcommand{\bb}{\boldsymbol}


\begin{document}

\title{Probability}
\author{Ben Anderson}
\date{\today}
\maketitle
\pagebreak

\tableofcontents
\pagebreak


\section{Variables}

\subsection{Random}

The value a variable takes on is determined by a random process, sampling from
a probability distribution.

The value of the variable is weighted according to the weightings given in by
the probability distribution.


\subsection{Discrete}

The variable can take on only a finite number of values.




\section{Probability Distributions}

\subsection{Probability Distribution}

A function that assigns a weighting to all possibile values of a random
variable.

Also called probability distribution function.

An example distribution:

\begin{center}
\begin{tabular}{c|c|c|c|c|c|c}
$x$ & 1 & 2 & 3 & 4 & 5 & 6 \\
\hline
$P(X = x)$ & $\frac{1}{6}$ & $\frac{1}{12}$ & $\frac{1}{3}$ & $\frac{1}{12}$ & $\frac{1}{6}$ & $\frac{1}{6}$ \\
\end{tabular}
\end{center}


\subsection{Restrictions}

All probabilities in a distribution sum to 1.

All probabilities must be positive.


\subsection{Probability Function}

For a random variable $X$, the probability that it takes on the value of $x$ is:

$$
P(X = x)
$$


\subsection{Cumulative Distribution}

Where each column in the table represents the value of $P(X \leq x)$.

Each value in a cumulative probability distribution is the sum of all values to
the left of it in the probability distribution.

The value in the right-most column must be 1.


\subsection{Graph}

For a graph of a probability distribution function.

X axis has values that the random variable can take.

Y axis has probabilities.

Bar graph is drawn, with each bar centred on the corresponding x axis value.



\section{Expected Value}

Symbol: $\mu$.

The average/mean of all values in a probability distribution.

Calculated as the sum of the probability multiplied by the value for all
values in the distribution:

$$
E(X) = \sum^{n}_{i = 0} x_i p_i
$$

Where $x_i$ is the $i$th value, and $p_i$ the $i$th probability.


\subsection{Payout on a Game}

Given a random game with a probability distribution giving the respective
probabilities of different payouts.

The long term profit of playing the game is the expected value of this
distribution.

A fair game is one where the long term payout of the game equals the cost of
playing the game.


\subsection{Transformations}

Given the expected value $E(X)$ for a random variable $X$.

If all values in $X$ are transformed in the same manner, the expected value
will also undergo the same transformations:

$$
E(aX + b) = aE(X) + b
$$




\section{Variance}

A measure of how spread out values are in a probability distribution.

Effectively the average distance of each value from the mean.

Calculated as the sum of the probability multiplied by the value subtract the
mean squared for each value in the distribution:

$$
\Var(X) = \sum^{n}_{i = 0} p_i (x_i - \mu)^2
$$

Where $x_i$ is the $i$th value, $p_i$ the $i$th probability, and $\mu$ the mean
of $X$.


\subsection{Additional Formula}

Another equivalent way of calculating the variance is:

$$
\Var(X) = E(X^2) - E(X)^2
$$


\subsection{Transformations}

Given the variance $\Var(X)$ for a random variable $X$.

If all values in $X$ undergo the same transformation, the variance will be
transformed by the square of any linear multiple applied to $X$:

$$
\Var(aX + b) = a^2 \Var(X)
$$

As translating the values does not affect their average distance from the mean.




\section{Standard Deviation}

Symbol $\sigma$.

The square root of the variance:

$$
\SD(X) = \sqrt{\Var(X)}
$$


\subsection{Transformations}

Given the standard deviation $\SD(X)$ for a random variable $X$.

The standard deviation is transformed only by linear multiples applied to $X$:

$$
\SD(aX + b) = a \SD(X)
$$

As translating the values does not affect their average distance from the mean.




\section{Uniform Distribution}

The probability of each value of a random variable is equal.


\subsection{Distribution}

For a fair six sided dice, the probability distribution is uniform:

\begin{center}
\begin{tabular}{c|c|c|c|c|c|c}
$x$ & 1 & 2 & 3 & 4 & 5 & 6 \\
\hline
$P(X = x)$ & $\frac{1}{6}$ & $\frac{1}{6}$ & $\frac{1}{6}$ & $\frac{1}{6}$ & $\frac{1}{6}$ & $\frac{1}{6}$ \\
\end{tabular}
\end{center}


\subsection{Probability Function}

For a random variable $X$ which can take on values between 0 and $n$, the
probability function is:

$$
P(X = x) = \frac{1}{n}
$$




\section{Hypergeometric Distribution}

A finite population is divided into two categories. A sample is taken from this
population.

A hypergeometric distribution gives the probability that a certain number of
one of the categories will be found in the sample.


\subsection{Example}

A manufacturer orders 500 items. 20 are faulty and unusable. Consider a sample
of 50 items from the 500.

The probability there are 5 faulty items in the sample of 10:

$$
P(X = 5) = \frac{\binom{20}{5} \binom{480}{5}}{\binom{500}{10}}
$$

The probability that there are less than 3 faulty items in the sample of 10:

$$
P(X \leq 3) = \frac{\binom{20}{2} \binom{480}{8}}{\binom{500}{10}} +
\frac{\binom{20}{1} \binom{480}{9}}{\binom{500}{10}} +
\frac{\binom{20}{0} \binom{480}{10}}{\binom{500}{10}}
$$




\section{Bernoulli Trial}

A random event with a certain probability of success.

The probability distribution is:

\begin{center}
\begin{tabular}{c|c|c}
& Success & Failure \\
$x$ & 1 & 0 \\
\hline
$P(X = x)$ & $p$ & $1 - p$ \\
\end{tabular}
\end{center}

The distribution is parameterised by $p$.

For example, the probability of correctly guessing a multi-choice question from
5 possible choices is:

\begin{center}
\begin{tabular}{c|c|c}
$x$ & 1 & 0 \\
\hline
$P(X = x)$ & $\frac{1}{5}$ & $\frac{4}{5}$ \\
\end{tabular}
\end{center}


\subsection{Expected Value}

The expected value of a Bernoulli trial is:

$$
E(X) = p
$$


\subsection{Variance}

The variance of a Bernoulli trial is:

$$
\Var(X) = p(1 - p)
$$




\section{Binomial Distribution}

The probability of achieving $x$ successes after performing a number of
Bernoulli trials.

The probability function of $x$ successes from $n$ Bernoulli trials with a
probability of succes $p$ is:

$$
P(X = x) = \binom{n}{x} p^x (1 - p)^{n - x}
$$

To achieve $x$ successes in $n$ trials, we must have $x$ successes and $n - x$
failures, which are multiplied together. There are $\binom{n}{x}$ ways of
arranging these multiplications.


\subsection{Notation}

If $X$ is a random variable sampled from a binomial distribution with $n$
trials and a probability of success $p$, this can be written:

$$
X \sim \Bin(n, p)
$$


\subsection{Expected Value}

The expected value for a binomial distribution is:

$$
E(X) = np
$$


\subsection{Variance}

The variance for a binomial distribution is:

$$
\Var(X) = np(p - 1)
$$


\subsection{Example}

The probability a car has to stop at all intersections is 0.2. A road has
10 intersections along it.

The probability the car will have to stop at exactly 4 intersections:

$$
P(X = 4) = \binom{10}{4} 0.2^4 0.8^6
$$

The probability the car will have to stop at at least 8 intersections:

$$
P(X \geq 8) = \binom{10}{8} 0.2^8 0.8^2 + \binom{10}{9} 0.2^9 0.8 + \binom{10}{10} 0.2^{10}
$$


\subsection{ClassPad}

Calculate $P(X = x)$ for a binomial distribution using:

$$
\text{binomialPDf}(x, n, p)
$$

PDf stands for probability distribution function.

Calculate $P(X \leq x)$ for a binomial distribution using:

$$
\text{binomialCDf}(0, x, n, p)
$$

CDf stands for cumulative distribution function.

\end{document}
