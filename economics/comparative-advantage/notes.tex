
\documentclass[a4paper,11pt]{article}

% Math symbols
\usepackage{amsmath}
\usepackage{amsfonts}
\usepackage{esvect}

% AI files
\usepackage{graphicx}
\DeclareGraphicsRule{.ai}{pdf}{.ai}{}

% No indent on new paragraphs
\setlength{\parindent}{0mm}
\setlength{\parskip}{0.2cm}

% Alias \boldsymbol to \bb for vectors
\newcommand{\bb}{\boldsymbol}


\begin{document}

\title{Comparative Advantage}
\author{Ben Anderson}
\date{\today}
\maketitle
\pagebreak

\tableofcontents
\pagebreak



\section{Trade}

Countries trade to exploit differences in natural resources.

Trade will always benefit countries, unless neither country has an absolute
advantage over the other in any particular good.


\subsection{Causes}

\textbf{Natural Resources} \quad Australia has an abundance of raw minerals,
giving us a competitive advantage in the mining industry.

\textbf{Technological Progress} \quad Countries with developed technology in
specific sectors will likely have a comparative advantage in them due to
superior efficiency from better technology.

\textbf{Government Attitude} \quad Countries with unrestrictive legislation for
specific sectors will likely have a comparative advantage in them due to easier
and faster technological and infrastructure development.




\section{Absolute Advantage}

A country has an absolute advantage over another in the production of a good if
it can produce that good more efficiently.

Measured by the use of less resources to produce a given quantity of output.


\subsection{Assumptions}

\begin{enumerate}
\item Only 2 countries.
\item Each country produces and consumes 2 goods.
\item Resources are perfectly mobile. There are no costs associated with
displacement.
\item No transportation costs.
\item Trade is free (eg. no tariffs).
\item No underutilisation of resources.
\item Perfect competition in the market.
\end{enumerate}


\subsection{Production Possibility Frontier}

Only two countries: Australia and USA.

Only two goods: wheat and cars.

USA can produce 10 wheat or 20 cars. Australia can produce 20 wheat or 10 cars.

Australia has an absolute advantage over USA in the production of wheat.

USA has an absolute advantage over Australia in the production of cars.


\subsubsection{Wording}

If a country's production capabilities is worded as "10 wheat or 20 cars``,
this implies:

\begin{itemize}
\item If 100\% of the country's resources were allocated to wheat, it could
produce 10 wheat and 0 cars.
\item if 100\% of the country's resources were allocated to cars, it could
produce 20 cars and 0 wheat.
\end{itemize}

If the phrase is worded "10 wheat and 20 cars``, this implies if the country
allocated 50\% of its resources to the production of each good, it could produce
10 wheat and 20 cars. Thus:

\begin{itemize}
\item If 100\% of the country's resources were allocated to wheat, it could
produce 20 wheat and 0 cars.
\item If 100\% of the country's resources were allocated to cars, it could
produce 40 cars and 0 wheat.
\end{itemize}


\subsection{Before Specialisation}

Both countries are self sufficient, consuming on their production possibility
frontier (since it is assumed there is no underutilisation of resources).

For example, each country consumes at 50\% of the total production of each
good:

\begin{center}
\begin{tabular}{c|c|c}
& Wheat & Cars \\
\hline
Australia & 10 & 5  \\
USA       & 5  & 10 \\
Total     & 15 & 15 \\
\end{tabular}
\end{center}


\subsection{After Specialisation}

Each country should 100\% specialise in the production of the good they have an
absolute advantage in:

\begin{center}
\begin{tabular}{c|c|c}
& Wheat & Cars \\
\hline
Australia & 20 & 0  \\
USA       & 0  & 20 \\
Total     & 20 & 20 \\
\end{tabular}
\end{center}

Total world production of each good has increased.


\subsection{After Trade}

Each country should trade so that they are wealthier than they were before
specialisation.

Split the increase in total production after specialisation between each
country 50/50, adding it to the before specialisation totals.

\begin{center}
\begin{tabular}{c|c|c}
& Wheat & Cars \\
\hline
Australia & 13 & 8  \\
USA       & 7  & 12 \\
Total     & 20 & 20 \\
\end{tabular}
\end{center}

Both countries are now consuming outside their production possibility frontiers.

The point at which the countries are consuming is a consumption possibility
point.

The new frontier representing all possible consumption points after trade is
the consumption possibility frontier.




\section{Comparative Advantage}

A country has a comparative advantage over another in the production of a good
if, in a comparison of two goods between the nations, we find the nation can
produce the good at a lower opportunity cost in terms of the other good than
the other nation.

A country should specialise in the production of the good it has a comparative
advantage in.

Used when one country has an absolute advantage over another in the production
of both goods.


\subsection{Assumptions}

\begin{enumerate}
\item Only 2 countries.
\item Each country produces and consumes 2 goods.
\item Resources are perfectly mobile. There are no costs associated with
displacement.
\item No transportation costs.
\item Trade is free (eg. no tariffs).
\item No underutilisation of resources.
\item Perfect competition in the market.
\end{enumerate}


\subsection{Production Possibility Frontier}

Only two countries: Australia and USA.

Only two goods: wheat and coal.

Australia can produce 6 wheat or 12 coal. USA can produce 20 wheat or 20 coal.

Australia has an absolute advantage in the production of both wheat and coal.


\subsection{Before Specialisation}

Both countries are self sufficient, consuming on their production possibility
frontier.

Australia dedicates 67\% of its resources to wheat and 33\% to coal.

USA dedicates 70\% of its resources to wheat and 30\% to coal.

\begin{center}
\begin{tabular}{c|c|c}
& Wheat & Coal \\
\hline
Australia & 4  & 4  \\
USA       & 14 & 6  \\
Total     & 18 & 10 \\
\end{tabular}
\end{center}


\subsection{Opportunity Cost}

To calculate the opportunity cost of a good made by one country in terms of the
other good:

$$
\frac{\text{other good}}{\text{required good}}
$$

Isolate the country, and form a fraction where the good for which we wish to
know the opportunity cost of is on the denominator, and the other good in the
numerator.

The opportunity cost for each good is:

\begin{center}
\begin{tabular}{c|c|c}
& Wheat & Coal \\
\hline
Australia & 2 & 0.5 \\
USA       & 1 & 1   \\
\end{tabular}
\end{center}

The meaning of this table is:

\begin{itemize}
\item If Australia is to produce 1 more wheat, then they must forego 1 coal.
\item If Australia is to produce 1 more coal, they would forego 1 wheat.
\item If the USA is to produce 1 more wheat, they would forego 2 coal.
\item If the USA is to produce 1 more coal, they would forego 0.5 wheat.
\end{itemize}

Australia has a comparative advantage in producing coal, as its opportunity
cost in terms of wheat is lower than Australia's.

USA has a comparative advantage in producing wheat, as its opportunity cost in
terms of coal is lower than the USA's.


\subsection{After Specialisation}

Each country should specialise (not necessarily 100\%) in the production of the
good in which it has a comparative advatange.

Australia should specialise in coal, USA in wheat.

If each country 100\% specialises:

\begin{center}
\begin{tabular}{c|c|c}
& Wheat & Coal \\
\hline
Australia & 0  & 12 \\
USA       & 20 & 0  \\
Total     & 20 & 12 \\
\end{tabular}
\end{center}

Total world production has increased in both goods.


\subsection{After Trade}

Split the increase in total world production between the two countries evenly,
adding it to the pre-specialisation numbers.

\begin{center}
\begin{tabular}{c|c|c}
& Wheat & Coal \\
\hline
Australia & 5  & 5  \\
USA       & 15 & 7  \\
Total     & 20 & 12 \\
\end{tabular}
\end{center}

Both countries are now consuming outside their production possibility frontiers.


\subsection{Terms of Trade}

The terms of trade is calculated as a fraction, where the amount exported by a
country of the required good is in the denominator, and the amount exported by
the other country of the other good is in the numerator:

$$
\frac{\text{other good}}{\text{required good}}
$$

Australia exported 7 coal to USA.

USA exported 5 wheat to Australia.

1 coal is worth $\frac{5}{7} = 0.714$ wheat.

1 wheat is worth $\frac{7}{5} = 1.4$ coal.

The terms of trade for each good lies in between the opportunity costs for each
country.

For wheat:

\begin{itemize}
\item Australia can import wheat for 1.4 cars. They would otherwise have to
produce it for 2 cars domestically.
\item USA can export wheat for 1.4 cars. They would otherwise sell it
domestically for only 1 car.
\end{itemize}

For coal:

\begin{itemize}
\item Australia can export coal for 0.714 cars. They would otherwise sell it on
the domestic market for only 0.5 cars.
\item USA can import coal for 0.714 cars. They would otherwise have to produce
it domestically for 1 car.
\end{itemize}




\section{Partial Specialisation}

\subsection{Production Possibility Frontier}

Only two countries: Australia and UK.

Only two goods: food and clothing.

Australia can produce 100 food or 30 clothing. UK can produce 150 food or 120
clothing.

UK has the absolute advantage in the production of both goods.


\subsection{Before Specialisation}

If each country allocated 50\% of its resources to the production of each good:

\begin{center}
\begin{tabular}{c|c|c}
& Food & Clothing \\
\hline
Australia & 50  & 15 \\
UK        & 75  & 60 \\
Total     & 125 & 75 \\
\end{tabular}
\end{center}


\subsection{Opportunity Costs}

The opportunity costs for each good are listed below:

\begin{center}
\begin{tabular}{c|c|c}
& Food & Clothing \\
\hline
Australia & 0.3 & 3.33 \\
UK        & 0.8 & 1.25 \\
\end{tabular}
\end{center}

Australia has a comparative advantage in food, UK in clothing.


\subsection{100\% Specialisation}

If each country 100\% specialised in the production of the good in which they
have a comparative advantage:

\begin{center}
\begin{tabular}{c|c|c}
& Food & Clothing \\
\hline
Australia & 100 & 0   \\
UK        & 0   & 120 \\
Total     & 100 & 120 \\
\end{tabular}
\end{center}

Total world production has increased for clothing, and decreased for food.

Total world production must increase or stay the same for both goods, thus we
must partially specialise.


\subsection{Partial Specialisation}

The country with no absolute advantage must 100\% specialise in the production
of the good in which it has a comparative advantage (Australia).

Usually chose an 80/20 split for the other country (UK).

Use a 33/67 split here:

\begin{center}
\begin{tabular}{c|c|c}
& Food & Clothing \\
\hline
Australia & 100 & 0  \\
UK        & 40  & 80 \\
Total     & 140 & 80 \\
\end{tabular}
\end{center}

Total world production has now increased in both goods.


\subsection{After Trade}

Split the increase in total world production for both goods 50/50 between each
country:

\begin{center}
\begin{tabular}{c|c|c}
& Food & Clothing \\
\hline
Australia & 58  & 17 \\
UK        & 82  & 63 \\
Total     & 140 & 80 \\
\end{tabular}
\end{center}

Both countries are now consuming outside their production possibility frontier.


\subsection{Terms of Trade}

Australia has exported 42 food to UK.

UK has exported 17 clothing to Australia.

1 food is worth $\frac{17}{42} = 0.405$ clothing.

1 clothing is worth $\frac{42}{17} = 2.47$ food.

The terms of trade lies between the opportunity costs of each good for each
country.




\section{Supply and Demand Model}

\subsection{Exports}

A country has a comparative advantage over another if it is relatively more
efficient at producing a good.

This efficiency is reflected in a lower cost of production.

The price of the good in a closed domestic market will be lower than the world
price.

\begin{figure}
\begin{center}
\includegraphics{exports.ai}
\end{center}
\end{figure}

When the price is raised:

\begin{itemize}
\item Price rises from domestic price (PD) to world price (PW)
\item Quantity demanded falls to Q2
\item Quantity supplied rises to Q3
\item A market surplus is created
\item Surplus goods are exported, sold at the world price
\item Consumer surplus decreases
\item Producer surplus increases by more than the decrease in consumer surplus
\item Results in an increase in total surplus and a net welfare gain for society
\end{itemize}


\subsection{Imports}

A country has a competitive disadvantage over another if it is relatively
inefficient at producing a good.

This inefficiency is reflected in a higher cost of production.

The price of the good in a closed domestic market will be higher than the world
price.

\begin{figure}
\begin{center}
\includegraphics{imports.ai}
\end{center}
\end{figure}

When the price is lowered:

\begin{itemize}
\item Price falls from domestic price (PD) to world price
\item Quantity demanded increases to Q3
\item Quantity supplied decreases to Q2
\item A market shortage is created
\item Additional goods are imported at the world price to satisfy demand
\item Producer surplus decreases
\item Consumer surplus increases by more than the decrease in producer surplus
\item Results in an increase in total surplus and a net welfare gain for society
\end{itemize}

\end{document}
