
\documentclass[a4paper,11pt]{article}

% Math symbols
\usepackage{amsmath}
\usepackage{amsfonts}
\usepackage{esvect}

% Hyperlink contents page
\usepackage{hyperref}
\hypersetup{
	colorlinks,
	citecolor=black,
	filecolor=black,
	linkcolor=black,
	urlcolor=black
}

% AI files
\usepackage{graphicx}
\DeclareGraphicsRule{.ai}{pdf}{.ai}{}

% No indent on new paragraphs
\setlength{\parindent}{0mm}
\setlength{\parskip}{0.3cm}

% Alias \boldsymbol to \bb for vectors
\newcommand{\bb}{\boldsymbol}


\begin{document}

\title{Trade}
\author{Ben Anderson}
\date{\today}
\maketitle
\pagebreak

\tableofcontents
\pagebreak


\section{Trade}

The import and exports of goods and services between countries.


\subsection{Inflows}

\begin{itemize}
\item Imports
\item Foreign tourists travelling to Australia
\item Immigrants
\item Foreign investment in Australia
\end{itemize}


\subsection{Outflows}

\begin{itemize}
\item Exports
\item Australian tourists travelling overseas
\item Emigrants
\item Foreign investment in other countries
\end{itemize}


\subsection{Specialisation}

When a country dedicates most of its resources to the production of goods in
which it has a comparative advantage.


\subsection{Strategic Trade Theory}

Where a government recognises the challenges facing smaller firms in competing
on global markets (eg. protectionist policies in other countries).

Offers financial or regulatory assistance to increase competitiveness.




\section{Trade Intensity}

The total value of a country's exports and imports as a percentage of its GDP.

Indication of the openness and willingness of a country to trade.


\subsection{Formula}

$$
\frac{\text{X} + \text{M}}{GDP} \times 100
$$


\subsection{Statistics}

13\% (1970) to 21\% (2013)




\section{Factors Affecting International Transactions}

Factors that affect the quantity and size of international transactions
include:

\begin{description}
\item [World Growth] Increasing world growth increases demand for Australia's
	exports.
\item [Domestic Growth] Australia has a high marginal propensity to imports.
	Increased domestic growth with increase value of imports.
\item [Exchange Rate] A depreciation increases our international
	competitiveness, increasing export income. Increases the price of imports,
	decreasing import spending.
\item [Inflation Rate] A higher inflation rate relative to our trading partners
	will decrease our international competitiveness, decreasing export income.
\item [Interest Rate] A higher interest rate relative to the rest of the world
	will increase foreign investment into Australia.
\item [Productivity] Higher productivity decreases costs associated with
	producing an export, increasing competitiveness and export income.
\end{description}




\section{Effects of Trade}

\subsection{To Consumers}

\begin{description}
\item [Goods] Increased variety, quantity, and quality of goods and services
	available to consumers.
\item [Price] Decreased prices due increased competition and decreased costs
	from established economies of scale.
\item [Employment] Greater quantity of higher paying employment opportunities
	from domestic producers.
\end{description}


\subsection{To Producers}

\begin{description}
\item [Market Size] Larger markets resulting in more demand, increasing sales
	and revenue.
\item [Competition] Promotes growth and advancement in technology.
\item [Diversified Markets] Sustained demand for exports as less demand from
	some regions may be offset by increased demand in others.
\end{description}




\section{Trade Composition}

What a country imports and exports.


\subsection{Australia's Exports}

\subsubsection{Categories}

\begin{itemize}
\item Primary (64\%, minerals 29\%, fuels 21\%, rural 11\%)
\item Services (17\%)
\item Manufacturing (13\%, STMs 9\%, ETMs 4\%)
\item Other (9\%)
\end{itemize}

Primary exports include all natural resources (eg. agricultural and mining
goods, livestock, fish).

STMs are simply transformed manufactures. Commodities that have undergone
minimal processing or refining (eg. steel).

ETMs are elaborately transformed manufactures. Commodities that have undergone
extensive processing and refining (eg. cars).


\subsubsection{Top Exports}

\begin{enumerate}
\item Iron Ore (22.5\%)
\item Coal (12.1\%)
\item Natural gas (4.9\%)
\item Education services (4.7\%)
\item Foreign travel to Australia (4.2\%)
\end{enumerate}


\subsection{Australia's Imports}

\subsubsection{Categories}

\begin{itemize}
\item Intermediate goods (35\%, fuels 13\%)
\item Consumption goods (24\%, cars 6\%, food 4\%, clothing 4\%)
\item Services (21\%)
\item Capital goods (19\%, machinery 6\%)
\end{itemize}

Consumption goods are the final products bought by consumers.

Intermediate goods are those used in the production of a final, consumption
good (eg. steel in cars).

Capital goods are used to aid production of consumption goods (eg. machinery).


\subsubsection{Top Imports}

\begin{enumerate}
\item Tourism, Australians travelling overseas (7.5\%)
\item Crude petroleum (6.4\%)
\item Refined petroleum (5.7\%)
\item Cars (5.3\%)
\item Freight services (2.8\%)
\end{enumerate}


\subsection{Causes}

Reasons for why our trade composition is what it is.

\subsubsection{Exports}

Mining exports:

\begin{itemize}
\item Australia has a strong comparative advantage in the mining sector.
\item We have an abundance of natural resources and minerals.
\item Have had heavy foreign investment into the mining sector sustained over
	a decade long mining boom.
\end{itemize}

Lack of manufactures:

\begin{itemize}
\item Australia has a competitive disadvantage in manufacturing (we're
	inefficient).
\item Small population, leaving fewer workers willing to work for low wages.
\item High cost of labour.
\item Stringent health and safety regulations.
\end{itemize}

Education:

\begin{itemize}
\item Education services is a niche market.
\item Heavy government investment into the public education sector has created
 	a large private sector complement.
\end{itemize}


\subsubsection{Imports}

Capital goods used predominantly by the mining sector in excavating and
refining commodities.

Consumption goods imported due to Australia's competitive disadvantage in
manufacturing, meaning we lack domestic substitutes.


\subsection{Trends}

Trends in Australia's composition of trade:

\begin{itemize}
\item Decrease in rural exports due to strong mining growth.
\item Increase in commodity exports.
\item Decrease in services and manufactures due to strong appreciation in \$AUD
	during the mining boom.
\end{itemize}




\section{Trade Direction}

Who Australia trades with.


\subsection{Countries}

Our top trading partners are:

\begin{enumerate}
\item China
\item Japan
\item United States
\end{enumerate}

80\% of our exports are to the Asia pacific region.

65\% of our imports are from the Asia pacific region.


\subsection{Trends}

A shift in trade direction from the European region to the Asia pacific region.

Reasons include:

\begin{description}
\item [APEC Free Trade Agreement] The Asia Pacific Economic Cooperation free
	trade agreement lowers barriers to trade with the Asia pacific region.
\item [Proximity] Australia is geographically close to the Asia region,
	reducing transportation costs.
\item [Population] Asia has a much larger population than Europe, resulting in
	greater demand for our exports.
\item [European Union] The European Union has established trade barriers
	increasing the cost of trading with the region.
\end{description}

\end{document}
