
\documentclass[a4paper,11pt]{article}

% Math symbols
\usepackage{amsmath}
\usepackage{amsfonts}
\usepackage{esvect}

% Hyperlink contents page
\usepackage{hyperref}
\hypersetup{
	colorlinks,
	citecolor=black,
	filecolor=black,
	linkcolor=black,
	urlcolor=black
}

% AI files
\usepackage{graphicx}
\DeclareGraphicsRule{.ai}{pdf}{.ai}{}

% No indent on new paragraphs
\setlength{\parindent}{0mm}
\setlength{\parskip}{0.3cm}

% Alias \boldsymbol to \bb for vectors
\newcommand{\bb}{\boldsymbol}


\begin{document}

\title{Globalisation}
\author{Ben Anderson}
\date{\today}
\maketitle
\pagebreak

\tableofcontents
\pagebreak


\section{Globalisation}

The closer integration of countries and people bought about by the reduction
in cost of transport and communication, and the breaking down of artificial
barriers to the flow of goods, services, capital, knowledge, and people
across borders.

Currently in 3rd wave of globalisation.


\subsection{Statistics}

Trade intensity ratio: 13\% (1970) to 24\% (2008).

World exports: 19\% of world GDP (1990) to 30\% of world GDP (2013).



\section{Organisations}

\subsection{Organisation for Economic Cooperation and Development}

34 member countries.

Undertake research and reviews to advance own economic wellbeing.

Member countries must be considered advanced economies, with minimum \$30000
average annual income per capita.


\subsection{World Trade Organisation}

188 member countries.

Attempt to achieve greater economic integration through providing a forum to:

\begin{itemize}
\item Facilitate the opening of boarders to international trade flows.
\item Negotiate trade agreements.
\item Settle trade disputes.
\end{itemize}


\subsection{International Monetary Fund}

Attempt to ensure the stability of the global financial system by ensuring the
stability of all countries.

Particularly, stability in exchange, inflation, and interest rates.


\subsection{World Bank}

Provides funding for less developed countries in the form of low or no interest
loans for infrastructure development projects.

These countries have limited access to international credit markets.




\section{Corporations}

\subsection{Multinational Corporation}

A corporation with headquarters in one country, and operations in many.


\subsection{Transnational Corporation}

A corporation with operations in many countries, where the location of its
headquarters is not readily identifiable.




\section{Causes}

\subsection{Technology}

\begin{description}
\item [Cost] Decreased cost of transporation through development of technology
	such as the standardisation of shipping containers.
\item [Speed] Increased speed of transportation through use of aircraft and
	larger ships.
\item [Communication] Increased ease and speed of communication through the
	Internet, allowing for eternal markets.
\item [Sharing] Sharing of technology and special expertise between countries
	through multinational corporations establishing operations in multiple
	countries.
\end{description}



\subsection{Government Policy}

\begin{description}
\item [Opening of Markets] Opening of domestic markets to international
	producers through unilateral trade reform.
\item [Assistance] Reduced governmental assistance to domestic firms through
	reduction in protectionist policies.
\end{description}


\subsection{Organisations}

Organisations such as the WTO and IMF encouraging unilateral trade reform
and facilitating trade agreements.


\subsection{Multinational Corporations}

Corporations have driven globalisation through establishing international
supply chains in an effort to reduce costs.

Evident through use of:

\begin{description}
\item [Outsourcing] Purchasing an intermediate good or service from another
	firm that specialises in its production to reduce costs.
\item [Offshoring] Moving operations to other countries where labour or
	material costs are less.
\end{description}


\subsection{Media}

Advertising has influenced many to want what the rest of the world has,
creating demand for products in international markets.

Reinforced by tourism to other countries.




\section{Benefits}

The benefits of globalisation include:

\begin{description}
\item [Consumers] Provides access to a greater quantity of goods and services
	at lower prices to consumers, due to increased competition in the market.
	Increases real purchasing power, increasing standard of living.
\item [Employment] Provides higher paying and more numerous employment
	opportunities. Increases real income, increasing standard of living.
\item [Economic Growth] Promotes economic growth and employment through
	reducing unemployment, increasing real income, reducing inflation, and
	developing infrastructure through foreign investment. Reduces poverty and
	increasing standard of living.
\item [Productivity] Increases competition by opening up markets to
	international producers, increasing efficiency and productivity.
\item [Technology and Skills] Increased competition and formation of
	multinational corporations improves technology and increases the skill of a
	country's labour force.
\item [World Peace] The interdependence of economies promotes world peace, as
	war causes damaging economic shocks that propagate faster and are more
	harmful.
\item [Multiculturalism] An international labour force promotes tolerance of
	other cultures and races.
\end{description}




\section{Costs}

The costs of globalisation include:

\begin{description}
\item [Environmental Damage] Increases damage to the environment due to
	increased industrialisation and resource depletion.
\item [Developing Countries] Unfair to developing countries who lack the
	infrastructure, technology, and skilled labour to compete with multinational
	corporations in competitive world markets. Limits their economic growth.
\item [Economic Shocks] Interdependent economies cause economic shocks to
	propagate throughout the world faster, and do more damage.
\item [Political Sovereignty] Multinational corporations gain large political
	influence from their economic importance. Erodes political sovereignty,
	where politicians make decisions in favour of large producers over
	consumers.
\item [Child Labour] Entrenches the use of child labour as a mechanism for
	reducing costs in multinational corporations.
\item [Culture] Small local businesses cannot compete with the economies of
	scale established by multinational corporations, decreasing diversity
	and destroying a country's culture.
\end{description}



\section{Competitiveness}

The degree to which a country can produce goods and services that meet the
test of international markets, while simultaneously maintaining and expanding
the real income of its people in the long term.

Test of international markets involves competition based on price and
perception of quality.


\subsection{Factors Affecting}

\subsubsection{Productivity}

The amount of output (measured by GDP) produced per unit of input (cost).

Higher productivity reduces cost for producing a given output, decreasing
prices, increasing competitiveness.


\subsubsection{Inflation}

Lower inflation relative to other countries that compete with our exports
reduces our export prices, increasing competitiveness.


\subsubsection{Wage Rate}

Amount paid to workers per unit of time or per unit of output.

Lower wage rate decreases the cost of production, decreasing prices, increasing
competitiveness.


\subsubsection{Exchange Rate}

A depreciation in the exchange rate lowers the price of our exports relative to
any competitors, improving competitiveness.


\subsection{Measurement}

\subsubsection{Real Unit Labour Cost}

Wage rate divided by productivity.

Lower real unit labour costs decreases cost of production, reducing prices and
increasing competitiveness.


\subsubsection{Trade Weighted Index}

A measure of the movements in the Australian Dollar against a weighted average
of the currencies of 20 other currencies.

The other currencies are weighted in accordance with their importance to
Australia's international trade flows.

A lower trade weighted index decreases the cost of Australian exports to
overseas consumers, increasing our competitiveness.




\section{Purchasing Power Parity}

A measure to determine if a currency is overvalued or undervalued.

Theoretically, a good or service should cost the same in each country when
converted to a common currency.


\subsection{Formula}

Cost of good in one country (in that country's currency) divided by the cost of
the good in another country (in that other country's currency).

Compare the ratio to the exchange rate.


\subsection{Disadvantages}

A country may have some artificial condition which drives up the price of the
good (eg. a tariff).

\end{document}
