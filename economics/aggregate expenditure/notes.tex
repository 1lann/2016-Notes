
\documentclass[a4paper,11pt]{article}

% Math symbols
\usepackage{amsmath}
\usepackage{amsfonts}
\usepackage{esvect}

% Hyperlink contents page
\usepackage{hyperref}
\hypersetup{
	colorlinks,
	citecolor=black,
	filecolor=black,
	linkcolor=black,
	urlcolor=black
}

% AI files
\usepackage{graphicx}
\DeclareGraphicsRule{.ai}{pdf}{.ai}{}

% No indent on new paragraphs
\setlength{\parindent}{0mm}
\setlength{\parskip}{0.3cm}


\begin{document}

\title{Aggregate Expenditure}
\author{Ben Anderson}
\date{\today}
\maketitle
\pagebreak

\tableofcontents
\pagebreak


\section{Variables}

\subsection{Autonomous Variable}

A variable independent of the level of income.

\subsection{Induced Variable}

A variable influenced by the level of income.




\section{Aggregate Expenditure}

A measure of national income equal to the total value of all expenditures
by all entities in an economy.

Just under \$1.6 trillion in 2013 - 2014.




\section{Consumption}

Spending on all durable and non-durable goods and services.

Autonomous consumption is the value of spending on all durable and non-durable
goods and services when the level of national income is 0.


\subsection{Size}

56\% of aggregate expenditure.


\subsection{Nature}

Largest and most stable component of aggregate expenditure.

This is because of autonomous consumption, which remains relatively stable over
time as the goods and services that society deems necessities do not change
rapidly.


\subsection{Components}

\subsubsection{Non Durable Goods}

Goods that are consumed quickly after purchase.

35\% of total consumption.


\subsubsection{Durable Goods}

Goods that are expected to last or provide satisfaction for three or more
years after purchase.

Spending tends to be discretionary, where purchases can be delayed or advanced
depending on economic circumstances.

15\% of total consumption, because:

\begin{itemize}
\item Last for a long time.
\item Usually are more expensive than non durables.
\item These factors mean durables have a low frequency of purchase.
\end{itemize}


\subsubsection{Services}

Spending on non-commodities such as education, healthcare, and recreation.

50\% of total consumption.


\subsection{Breakeven}

$$
C = Y
$$

Consumption is at equilibrium (called the breakeven point) when consumption
spending is equal to national income.


\subsection{Consumption Function}

$$
C = a + bY
$$

Consumption can be split into its autonomous ($a$) and induced ($b$) components.

Autonomous consumption ($a$) is spending to finance basic needs for survivial,
which exists even when income is 0.


\subsubsection{Marginal Propensity to Consume}

$$
\mbox{MPC} = \frac{\Delta C}{\Delta Y}
$$

$b$ is the marginal propensity to consume (MPC). It is the proportion of one
additional dollar of income at the current level of national income that is
spent on consumption.

The MPC is a value between 0 and 1.

As income rises, the MPC remains constant.


\subsubsection{Average Propensity to Consume}

$$
\mbox{APC} = \frac{C}{Y}
$$

Average propensity to consume (APC) is the proportion of total income spent on
consumption.

As income rises, the APC falls.


\subsubsection{Graph}

\begin{description}
\item [Y Axis] Consumption spending (\$)
\item [X Axis] Income, output (\$)
\item [Scale] The scale on each axis must be equal
\item [Equilibrium Line] A line drawn at $45^\circ$ represents all possible
	points of equilibrium (where $C = Y$).
\item [Y Intercept] Autonomous spending
\item [Gradient] The marginal propensity to consume
\item [Breakeven] Where the consumption function intersects the $45^\circ$ line
\end{description}

A point on the $45^\circ$ line represents the level of output at that level of
national income.


\subsubsection{Inventories}

The distance between the $45^\circ$ line and the consumption function is the
change in inventories experienced by firms.

A change in inventories will prompt firms to either increase or decrease
output:

\begin{itemize}
\item When $C > Y$ and $S < 0$, inventories will fall, prompting firms to
	increase output.
\item When $C < Y$ and $S > 0$, inventories will increase, prompting firms to
	decrease output.
\end{itemize}


\subsubsection{Changes}

\begin{description}
\item [Autonomous Consumption] The consumption function will shift parallel
	(gradient remains constant, Y intercept changes).
\item [MPC] The graph pivots about its Y intercept (gradient changes, Y
	intercept remains constant as autonomous spending doesn't change).
\end{description}


\subsubsection{Solving for Breakeven}

Find the consumption function and equate it to income ($Y$).

For example, given the consumption function $C = 50 + 0.4Y$:

$$
\begin{aligned}
Y & = 50 + 0.6Y \\
0.4Y & = 50 \\
Y & = 125 \\
\end{aligned}
$$

Thus breakeven is at an income level of \$125.


\subsection{Savings Function}

$$
S = -a + (1 - b)Y
$$

The savings function can be split into its autonomous ($-a$) and induced
($1 - b$) components.

\begin{description}
\item [Dissavings] Where $S < 0$ (ie. $Y < C$), there is dissavings (where we
	spend more than our income).
\item [Autonomous] When $Y = 0$, no income is available to save, yet there is
	still spending on autonomous consumption. Thus savings is negative autonomous
	spending ($-a$).
\item [Breakeven] When $C = Y$ (ie. breakeven point), $S = 0$, since we are
	spending all our income on consumption, there is nothing
\end{description}


\subsubsection{Marginal Propensity to Save}

$$
\begin{aligned}
\mbox{MPS} & = \frac{\Delta S}{\Delta Y} \\
& = 1 - \mbox{MPC} \\
\end{aligned}
$$

$1 - b$ is the marginal propensity to save (MPS). It is the proportion of one
additional dollar of income at the current national income level that is saved.

The MPS is a value between 0 and 1.

As income rises, the MPS remains constant.


\subsubsection{Average Propensity to Save}

$$
\mbox{APS} = \frac{S}{Y}
$$

The average propensity to save (APS) is the proportion of total income that is
saved.

As income rises, the APS rises.



\subsubsection{Graph}

\begin{description}
\item [Y Intercept] Negative autonomous consumption
\item [Gradient] The marginal propensity to save
\item [X Intercept] Breakeven point (where $S = 0$ and $C = Y$)
\item [Dissavings] Where savings is negative ($S < 0$ and $C > Y$)
\item [Savings] Where savings is positive ($S > 0$ and $Y > C$)
\end{description}


\subsection{Factors Affecting}

\subsubsection{Disposable Income}

This is the most important factor that affects consumption.

For an increase in disposable income:

\begin{itemize}
\item Increases a consumer's ability to purchase desired goods which are not
	critical to their survival.
\item Increases their MPC, increasing aggregate consumption.
\end{itemize}

For a redistribution of income to lower income households:

\begin{itemize}
\item Consumers individually all have different levels of consumption (ie.
	different MPCs).
\item Lower income households have a higher MPC due to a greater proportion of
	their income being spent on necessities.
\item Thus they are more likely to spend any additional disposable income,
	rather than save it.
\item An increase in their disposable income will increase aggregate
	consumption.
\end{itemize}


\subsubsection{Interest Rates}

For an increase in the interest rate:

\begin{description}
\item [New Debt] Credit becomes more difficult and expensive to obtain, causing
	consumers to delay purchases they intended to finance through credit,
	reducing consumption.
\item [Existing Debt] Increases interest repayments on existing debt, reducing
	disposable income and thus consumption.
\end{description}


\subsubsection{Taxation}

Increased taxation rates by the government:

\begin{itemize}
\item Increases the personal income tax paid by consumers.
\item Decreases their disposable income.
\item Decreases their MPC, reducing aggregate consumption.
\end{itemize}

May increase government expenditure in the economy, slightly offseting the
decrease in consumption in terms of total aggregate expenditure.


\subsubsection{Assets}

For a sudden fall in the value of a household's assets:

\begin{itemize}
\item Households will have a target level of assets and wealth at every stage
	in their life.
\item A sudden fall will prompt them to consolidate their financial position by
	increasing their level of savings.
\item This will increase their marginal propensity to save, decreasing their
	marginal propensity to consume, and thus reducing consumption.
\end{itemize}

For a sudden increase in the value of a household's assets (eg. increase in
stock market, property prices, commodity prices):

\begin{itemize}
\item Reduced emphasis on savings, decreasing the marginal propensity to save.
\item Increases the marginal propensity to consume, increasing consumption.
\end{itemize}


\subsubsection{Expectations}

Individual consumers' attitudes to the economy will affect their level of
consumption, and thus aggregate consumption overall.

Expectations are influenced by numerous factors, which can include:

\begin{itemize}
\item Levels of debt
\item Employment stability
\item Volatility of markets
\end{itemize}

Poor expectations will cause consumers to delay purchases until expectations
improve, decreasing their MPC, reducing aggregate consumption.

High expectations will make consumers more likely to make a higher frequency of
purchases, likely of more expensive goods and services. This will increase their
MPC, increasing aggregate consumption.




\section{Investment}

Spending on producer or capital goods that are used by firms to produce final
goods and services.

Funds for investment can be obtained from undistributed profits, share issues,
or borrowing funds from financial markets.

Assumed to be autonomous in the Keynesian model (doesn't change depending on
the level of income).

Expenditure on new housing by consumers is not consumption, but investment,
as it increases the stock of physical capital.


\subsection{Size}

14 to 26\% of aggregate expenditure.


\subsection{Nature}

Investment is the most volatile component of aggregate expenditure.

This is because the success of investment projects is based on unpredictable
economic activity in the future. Causes volatility in the levels of investment
in the economy.

Over 50\% of small business ventures fail within the first 3 years.


\subsection{Components}

\begin{description}
\item [Depreciation Investment] Investment to replace aging capital.
\item [Additional Investment] New investment to increase the stock of physical
	capital.
\end{description}


\subsubsection{Net Investment}

Gross investment in the economy subtract depreciation investment.


\subsection{Model}

Investment can be modelled using an investment demand curve:

\begin{description}
\item [Y Axis] Real interest rates
\item [X Axis] Quantity of investment
\item [Gradient] The graph has a negative gradient, representing an inverse
	relationship between interest rates and the level of investment.
\end{description}

Movements along the curve:

\begin{description}
\item [Expansion] A movement down the curve, increasing the quantity of
	investment due to a decrease in interest rates.
\item [Contraction] A movement up the curve, decreasing the quantity of
	investment due to an increase in interest rates.
\end{description}

The curve itself is moved when a non interest rate factor changes, which has an
effect on the level of investment.


\subsubsection{Marginal Efficiency of Capital}

Also called the net productivity of capital.

Calculated by subtracting the cost of investment in new capital from the
expected return of the investment (revenue from sales).

The cost of investment is determined by:

\begin{description}
\item [Interest Rates] When the funds for the investment are borrowed, the cost
	of investing is the interest repayments on the loan.
\item [Opportunity Cost] When reinvesting profits (the value of other activity
	funded by these profits which would be foregone by undertaking the
	investment).
\end{description}

As interest rates fall, ceritus paribus:

\begin{itemize}
\item The cost of investing decreases.
\item The expected return remains constant.
\item The marginal efficiency of capital increases.
\end{itemize}


\subsection{Factors Affecting}

\subsubsection{Expectations}

Expectations is the main determinant of investment.

Expectations are influenced by:

\begin{description}
\item [Sales and Profits] Strong sales increase the profit firms have available
	to invest, and indicate that the investment is more likely to be successful.
\item [Uncertainty] Investing involves the assessment of risk and uncertainty
	in the economy. A higher degree of uncertainty will discourage investment.
	Influenced by factors like political decisions, overseas events, and
	changes in consumer preferences.
\item [Competition] Increased competition will prompt a producer to consolidate
	their position within a market, increasing investment to provide more
	attractive goods and services.
\item [Technology] A high level of advancement of technology in a producer's
	industry may prompt them to invest in improving their own technologies, so
	that they remain competitive in their industry and sustain profit levels.
\item [World Events] Domestic natural diasters or international economic shocks
	will reduce the level of investment due to increased market uncertainty
	and potential falls in sales and profits.
\item [Taxation] Increased government taxation in an industry will reduce the
	expected return on investment, discouraging investment.
\item [Government Policies] Increased rules and regulations in a certain
	industry imposed by changing government policies discourages investment in
	that industry.
\item [Inflation] Increased levels of inflation will increase the cost of
	investing by raising the price of capital equipment and potentially
	increasing the interest rate, discouraging firms from investing. High
	inflation also makes determining accurate costings for investment projects
	more difficult, introducing uncertainty, discouraging investment.
\end{description}


\subsubsection{Real Interest Rate}

Higher real interest rates:

\begin{description}
\item [Interest Repayments] Increase the cost of borrowing by increasing the
	size of interest repayments.
\item [Opportunity Cost] Increase the opportunity cost of the money used to
	fund the investment, since the size of interest repayments becomes larger.
\end{description}

Interest rates are particularly influential in boom conditions, because:

\begin{itemize}
\item Strong sales, profits, and economic expectations encourage firms to
	invest.
\item This places increased emphasis on the cost of investing for firms, which
	is determined by the interest rate.
\item Thus the interest rate is more influential in changing the level of
	investment in the economy.
\item In a recession, poor economic expectations, sales, profits, and growth
	discourage firms from even considering the affordability of investing.
\end{itemize}


\subsubsection{Output Levels}

Decreasing sales and demand for a firm's output reduces the need to replace
aging capital equipment (depreciation investment).

This is the reverse condition of the accelerator principle.




\section{Government Expenditure}

The total amount of spending by the public sector.


\subsection{Components}

\begin{description}
\item [G1] Expenditure that funds the day to day operating costs of the
	government and its core functions.
\item [G2] Expenditure on infrastructure or social overhead capital which
	provides for the needs of society in the future.
\end{description}


\subsection{Expenditure Types}

Structural expenditure involves ongoing spending commitments which do not
change dramatically over time. Includes:

\begin{description}
\item [Allocative Role] Public goods and services (roads, education, healthcare,
	etc).
\item [Regulatory Role] Influence in business activity.
\item [Redistributive Role] Welfare payments (Centrelink, aged pensions).)
\end{description}

Cyclical expenditure involves spending to stabilise the busines cycle.


\subsection{Size}

23\% of aggregate expenditure.


\subsection{Nature}

Very stable, second to consumption. This is because:

\begin{description}
\item [Compulsory] Much of the government's expenditure is compulsory, and must
	be spent despite whatever economic circumstances prevail at the time.
\item [Concerns] Government spending is also influenced by social and political
	issues, rather than just economic ones.
\end{description}


\section{Factors Affecting}

\begin{description}
\item [Policies] The level of government expenditure is determined predominantly
	by the prevailing government's policies.
\item [Size] The size of government will affect the cost of wages for government
	employees.
\end{description}




\section{Net Exports}

Net exports is the total value of exports subtract the total value of imports.

Exports are assumed to be autonomous in the Keynesian model.

Imports are assumed to be induced, and are subtracted from aggregate expenditure
as they represent activity for another economy, since money leaves Australia.


\subsection{Size}

-2\% of aggregate expenditure (imports exceed exports).

Usually a deficit, sitting around -4\% of aggregate expenditure during the
last boom phase.

Reached a surplus of \$20 billion in 2011, following a balance of goods and
services surplus in the current account.


\subsection{Nature}

Volatile component of aggregate expenditure, second to investment.

This is because of exports:

\begin{itemize}
\item The majority of our exports are primary goods (eg.  iron ore, coal, and
	natural gas account for 37\% of total exports).
\item Demand for our exports is inelastic as these goods have few substitutes.
\item Supply of our exports is inelastic as it requires expensive and time
	consuming investment in infrastructure to increase productive capacity.
\item Thus any small changes in demand or supply for our exports causes large
	changes in price, but small changes in quantity.
\item This causes large volatility in the value of our exports.
\end{itemize}

This volatility is made more severe by the volatility in the exchange rate.


\subsection{Factors Affecting}

\subsubsection{Structural Factors}

\begin{description}
\item [Exchange Rate] A depreciation in the exchange rate will increase the
	competitiveness of our export industries, increasing export sales. It will
	raise the price of imports for consumers, decreasing import sales. This
	will increase net exports.
\item [Terms of Trade] An unfavourable movement in our terms of trade will
	decrease the total value of exports (due to inelastic demand), and increase
	the value of imports. This will decrease net exports.
\item [Trade Policy] An increase in the number of bilateral and multilateral
	free trade agreements will increase export sales and import purchases,
	affecting net exports.
\end{description}


\subsubsection{Cyclical Factors}

\begin{description}
\item [World Growth] Falling world growth (especially in China) has decreased
	demand for exports, decreasing export sales and our net exports.
\item [Domestic Growth] Strong domestic growth will increase imports due to
	Australia's high marginal propensity to import, decreasing net exports.
\item [Shocks] Natural disasters (drought, floods) and economic shocks (GFC)
	causes volatility in the demand for exports and import, affecting net
	exports.
\end{description}




\section{Multiplier Effect}

A coefficient by which any injections or leakages (represented through an
autonomous change in aggregate expenditure) in the economy are multiplied,
resulting in a larger change in national income.


\subsection{Concept}

When an individual receives some form of income:

\begin{itemize}
\item Part of this income is spent on consumption (determined by the person's
	marginal propensity to consume).
\item The other part is saved (a leakage).
\item The income spent on consumption becomes another individual's income.
\item This action is repeated through a cyclical process.
\item The resulting change in national income is much larger than the initial
	increase in autonomous aggregate expenditure.
\end{itemize}


\subsection{Formula}

$$
\Delta \mbox{autonomous AE} \times k = \Delta Y
$$

Where $k$ is the multiplier coefficient.


\subsubsection{Coefficient}

The multiplier coefficient is determined by the size of the leakages in the
economy.

The simple multipler considers leakages only in the financial sector (ie. to
savings):

$$
\begin{aligned}
k & = \frac{1}{\mbox{MPS}} \\
& = \frac{1}{1 - \mbox{MPC}} \\
& = \frac{\Delta Y}{\Delta \mbox{autonomous AE}} \\
\end{aligned}
$$

The complex multiplier considers other forms of leakages in addition to savings,
such as tax and imports:

$$
k = \frac{1}{\mbox{MPS} + \mbox{MPT} + \mbox{MPM}}
$$

Where MPT is the marginal propensity of tax, and MPM is the marginal propensity
to import.

The complex multiplier will be smaller than the simple one, as more leakages
are included.


\subsection{Example}

Consider an investment of \$100 into an economy where the MPC is 0.8:

\begin{center}
\begin{tabular}{c|c c c}
Stage & Income & Consumption & Savings \\
\hline
Initial & \$100   & \$80    & \$20    \\
2       & \$80    & \$64    & \$16    \\
3       & \$64    & \$51.20 & \$12.80 \\
4       & \$51.20 & \$40.96 & \$10.24 \\
\end{tabular}
\end{center}

The stages in this table will continue indefinitely.

The multiplier is:

$$
\begin{aligned}
k & = \frac{1}{1 - \mbox{MPC}} \\
& = \frac{1}{1 - 0.8} \\
& = 5 \\
\end{aligned}
$$

The final change in national income is:

$$
\begin{aligned}
\Delta Y & = 100 \times 5 \\
& = \$500 \\
\end{aligned}
$$


\subsection{Model}

The multiplier effect can be demonstrated on an aggregate expenditure diagram
as a movement between the two points of equilibrium:

\begin{itemize}
\item From the intial equilibrium position, draw an arrow upwards the size of
	the autonomous change in aggregate expenditure. This represents the initial
	autonomous change in the economy (eg. an investment).
\item Draw an arrow of equal size to the right. This represents the increase in
	income for an individual due to this initial autonomous change.
\item Draw an arrow up, smaller by a factor determined by the MPC. This
	represents the second stage of consumption.
\item Draw an arrow right of equal size. This represents the second stage
	consumption forming another individual's income.
\item Continue this with increasingly smaller arrows until we reach the new
	equilibrium position.
\end{itemize}


\subsection{Reverse}

The multiplier principle also operates in reverse, where an autonomous decrease
in aggregate expenditure will have a reverse multiplier effect on national
income.

\end{document}
