
\documentclass[a4paper,11pt]{article}

% Math symbols
\usepackage{amsmath}
\usepackage{amsfonts}
\usepackage{esvect}

% Hyperlink contents page
\usepackage{hyperref}
\hypersetup{
	colorlinks,
	citecolor=black,
	filecolor=black,
	linkcolor=black,
	urlcolor=black
}

% AI files
\usepackage{graphicx}
\DeclareGraphicsRule{.ai}{pdf}{.ai}{}

% No indent on new paragraphs
\setlength{\parindent}{0mm}
\setlength{\parskip}{0.2cm}

% Alias \boldsymbol to \bb for vectors
\newcommand{\bb}{\boldsymbol}


\begin{document}

\title{Protection}
\author{Ben Anderson}
\date{\today}
\maketitle
\pagebreak

\tableofcontents
\pagebreak


\section{Exchange Rates}

\subsection{Definition}

The price of one country's currency in terms of another's.

\subsection{Foreign Exchange Market}

The international market where the currencies of different countries are bought
and sold.




\section{Measurement}

\subsection{Bilateral Exchange Rate}

The price of one country's currency directly in terms of another's.

Also called the cross exchange rate.


\subsubsection{Quoting}

Can be quoted in terms of a country's purchasing power:

$$
\$1.00\text{ AUD} = \$0.77\text{ USD}
$$

Or in terms of a common currency:

$$
\$1.00\text{ USD} = \$1.30\text{ AUD}
$$


\subsubsection{Calculation}

Given a bilateral exchange rate:

$$
\$1.00\text{ AUD} = \$x \text{ USD}
$$

To invert the rate, divide by $x$:

$$
\$\frac{1.00}{x}\text{ AUD} = \$1.00\text{ USD}
$$

To convert \$AUD to \$USD, modify the rate such that it is
$\$1\text{ AUD} = \$x\text{ USD}$, and multiply each side by the required
amount of \$AUD.

To convert \$USD to \$AUD, modify the rate such that it is
$\$1\text{ USD} = \$x\text{ AUD}$, and multiply each side by the required
amount of \$USD.


\subsection{Trade Weighted Index}

A measure of the movements in the Australian Dollar against a weighted average
of the currencies of 20 other currencies.

The other currencies are weighted in accordance with their importance to
Australia's international trade flows.


\subsubsection{Advantages over Bilateral Rate}

\begin{description}
\item [Accuracy] The TWI is more accurate and significant holistically.
\item [Balance of Payments] Due to the weighting in accordance with trade flow
	importance, the TWI reflects the performance in Australia's Balance of
	Payments over time.
\end{description}




\section{Types}

\subsection{Floating}

Where the market forces of supply and demand in the foreign exchange market
alone determine the price of a currency.

Australia floated our currency in 1983.


\subsection{Fixed}

Where the price of a country's currency is set at a particular value and
adjusted periodically by the country's central bank.


\subsection{Managed}

Where the price of a country's currency is allowed to fluctuate between an
accceptable upper and lower limit in accordance with the market forces of
supply and demand in the foreign exchange market.

The country's central bank will intervene if the exchange rate strays out of
these limits.

Referred to as a "dirty" float.


\subsubsection{Controlling the Exchange Rate}

To lower the exchange rate because it is approaching the upper limit, the
central bank could:

\begin{description}
\item [Increase Supply] Sell its reserves of Australian dollars into the market
	to increase supply, decreasing the price.
\item [Decrease Cash Rate] Make Australia less attractive for foreign
	investors by reducing the cash rate, decreasing demand, decreasing the
	price.
\end{description}




\section{Movements}

\begin{center}
\begin{tabular}{c|c}
Name & Definition \\
\hline
Appreciation & Increase in a floating exchange rate
Depreciation & Decrease in a floating exchange rate
Revaluation & Increase in a fixed exchange rate
Devaluation & Decrease in a fixed exchange rate
\end{tabular}
\end{center}




\section{Model}

The price of a country's currency is determined by the market forces of supply
and demand in the foreign exchange market.


\subsection{Demand}

For an international consumer purchasing an Australian export:

\begin{itemize}
\item Australian producer will expect to be paid in \$AUD.
\item International consumer will convert their domestic currency to \$AUD on
	the foreign exchange market.
\item The consumer demands \$AUD from the market.
\end{itemize}

Thus all credits in Australia's Balance of Payments represents demand for the
\$AUD.


\subsection{Supply}

For an Australian consumer purchasing an international export:

\begin{itemize}
\item International producer will expect to be paid in their domestic currency.
\item Australian consumer will convert \$AUD to the producer's domestic
	currency.
\item The consumer supplies \$AUD to the market.
\end{itemize}

Thus all debits in Australia's Balance of Payments represents supply of the
\$AUD.


\subsection{Factors Affecting Demand}

\subsubsection{Exports}

Since Australian exports are sold in \$AUD.

Increased demand for Australian exports will increase demand for the \$AUD.

For example: purchase of iron ore, international students studying in Australia,
tourists travelling to Australia.


\subsubsection{Foreign Investment Into Australia}

Foreign investment into Australia involves purchases made in \$AUD.

Increased foreign investment into Australia will increase demand for the \$AUD.

For example: construction of new mining infrastructure by an overseas firm.


\subsubsection{Income Receipts}

Australians working overseas are paid in \$AUD.

Increased income receipts for Australians working overseas increases demand for
the \$AUD.


\subsubsection{Dividend Payments and Profit Flows to Australian Firms}

Australian companies with operations overseas must convert profits into \$AUD.

Australian investors in overseas companies expect their dividends to be paid
in \$AUD.

Increase in dividends to Australian investors or profit flows to Australian
firms will increase demand for the \$AUD.


\subsection{Factors Affecting Supply}

\subsubsection{Imports}

Importing requires converting \$AUD into foreign currency.

Increased spending on imports increases supply of \$AUD.


\subsubsection{Foreign Investment Overseas}

Foreign investment into other countries requires converting \$AUD into foreign
currency to purchase investments.

Increased foreign investment abroad increases supply of \$AUD.


\subsubsection{Income Payments}

Overseas residents working in Australia expect to be paid in their domestic
currency.

Increased employment of international workers increases supply of \$AUD.


\subsubsection{Dividend Payments and Profit Flows Overseas}

Overseas firms and investors expect their profits and dividend payments to be
in their domestic currency.

Increased dividend payments or profit flows overseas increases supply of \$AUD.




\section{Factors Affecting}

\subsection{Terms of Trade}

\subsection{World Growth}

\subsection{Interest Rate Differential}

\subsection{International Foreign Investment}

\subsection{Relative Inflation Rate}

\subsection{Domestic Growth}

\subsection{Speculation}


\section{Effects}

\subsection{Exports}

\subsection{Imports}

\subsection{Balance of Goods and Services}

\subsection{Current Account Deficit}

\subsection{Foreign Debt}

\subsection{Inflation}

\subsection{Economic Growth}

\subsection{Employment}

\end{document}
