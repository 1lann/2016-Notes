
\documentclass[a4paper,11pt]{article}

% Math symbols
\usepackage{amsmath}
\usepackage{amsfonts}
\usepackage{esvect}

% Hyperlink contents page
\usepackage{hyperref}
\hypersetup{
	colorlinks,
	citecolor=black,
	filecolor=black,
	linkcolor=black,
	urlcolor=black
}

% AI files
\usepackage{graphicx}
\DeclareGraphicsRule{.ai}{pdf}{.ai}{}

% No indent on new paragraphs
\setlength{\parindent}{0mm}
\setlength{\parskip}{0.3cm}

% Alias \boldsymbol to \bb for vectors
\newcommand{\bb}{\boldsymbol}


\begin{document}

\title{Exchange Rates}
\author{Ben Anderson}
\date{\today}
\maketitle
\pagebreak

\tableofcontents
\pagebreak


\section{Exchange Rates}

\subsection{Definition}

The price of one country's currency in terms of another's.

\subsection{Foreign Exchange Market}

The international market where the currencies of different countries are bought
and sold.




\section{Measurement}

\subsection{Bilateral Exchange Rate}

The price of one country's currency directly in terms of another's.

Also called the cross exchange rate.


\subsubsection{Quoting}

Can be quoted in terms of a country's purchasing power:

$$
\$1.00\text{ AUD} = \$0.77\text{ USD}
$$

Or in terms of a common currency:

$$
\$1.00\text{ USD} = \$1.30\text{ AUD}
$$


\subsubsection{Calculation}

Given a bilateral exchange rate:

$$
\$1.00\text{ AUD} = \$x \text{ USD}
$$

To invert the rate, divide by $x$:

$$
\$\frac{1.00}{x}\text{ AUD} = \$1.00\text{ USD}
$$

To convert \$AUD to \$USD, modify the rate such that it is
$\$1\text{ AUD} = \$x\text{ USD}$, and multiply each side by the required
amount of \$AUD.

To convert \$USD to \$AUD, modify the rate such that it is
$\$1\text{ USD} = \$x\text{ AUD}$, and multiply each side by the required
amount of \$USD.


\subsection{Trade Weighted Index}

A measure of the movements in the Australian dollar against a basket of 20
other currencies weighted according to their importance to Australia's
international trade flows.


\subsubsection{Advantages over Bilateral Rate}

\begin{description}
\item [Accuracy] The TWI is more accurate and significant holistically.
\item [Balance of Payments] Due to the weighting in accordance with trade flow
	importance, the TWI reflects the performance in Australia's Balance of
	Payments over time.
\end{description}




\section{Types}

\subsection{Floating}

Where the market forces of supply and demand in the foreign exchange market
alone determine the price of a currency.

Australia floated our currency in 1983.


\subsection{Fixed}

Where the price of a country's currency is set at a particular value and
adjusted periodically by the country's central bank.


\subsection{Managed}

Where the price of a country's currency is allowed to fluctuate between an
accceptable upper and lower limit in accordance with the market forces of
supply and demand in the foreign exchange market.

The country's central bank will intervene if the exchange rate strays out of
these limits.

Referred to as a "dirty" float.


\subsubsection{Controlling the Exchange Rate}

To lower the exchange rate because it is approaching the upper limit, the
central bank could:

\begin{description}
\item [Increase Supply] Sell its reserves of Australian dollars into the market
	to increase supply, decreasing the price.
\item [Decrease Cash Rate] Make Australia less attractive for foreign
	investors by reducing the cash rate, decreasing demand, decreasing the
	price.
\end{description}




\section{Movements}

\begin{center}
\begin{tabular}{c|c}
Name & Definition \\
\hline
Appreciation & Increase in a floating exchange rate \\
Depreciation & Decrease in a floating exchange rate \\
Revaluation & Increase in a fixed exchange rate \\
Devaluation & Decrease in a fixed exchange rate \\
\end{tabular}
\end{center}




\section{Model}

The price of a country's currency is determined by the market forces of supply
and demand in the foreign exchange market.


\subsection{Demand}

For an international consumer purchasing an Australian export:

\begin{itemize}
\item Australian producer will expect to be paid in \$AUD.
\item International consumer will convert their domestic currency to \$AUD on
	the foreign exchange market.
\item The consumer demands \$AUD from the market.
\end{itemize}

Thus all credits in Australia's Balance of Payments represents demand for the
\$AUD.


\subsection{Supply}

For an Australian consumer purchasing an international export:

\begin{itemize}
\item International producer will expect to be paid in their domestic currency.
\item Australian consumer will convert \$AUD to the producer's domestic
	currency.
\item The consumer supplies \$AUD to the market.
\end{itemize}

Thus all debits in Australia's Balance of Payments represents supply of the
\$AUD.


\subsection{Factors Affecting Demand}

\subsubsection{Exports}

Since Australian exports are sold in \$AUD.

Increased demand for Australian exports will increase demand for the \$AUD.

For example: purchase of iron ore, international students studying in Australia,
tourists travelling to Australia.


\subsubsection{Foreign Investment Into Australia}

Foreign investment into Australia involves purchases made in \$AUD.

Increased foreign investment into Australia will increase demand for the \$AUD.

For example: construction of new mining infrastructure by an overseas firm.


\subsubsection{Income Receipts}

Australians working overseas are paid in \$AUD.

Increased income receipts for Australians working overseas increases demand for
the \$AUD.


\subsubsection{Dividend Payments and Profit Flows to Australian Firms}

Australian companies with operations overseas must convert profits into \$AUD.

Australian investors in overseas companies expect their dividends to be paid
in \$AUD.

Increase in dividends to Australian investors or profit flows to Australian
firms will increase demand for the \$AUD.


\subsection{Factors Affecting Supply}

\subsubsection{Imports}

Importing requires converting \$AUD into foreign currency.

Increased spending on imports increases supply of \$AUD.


\subsubsection{Foreign Investment Overseas}

Foreign investment into other countries requires converting \$AUD into foreign
currency to purchase investments.

Increased foreign investment abroad increases supply of \$AUD.


\subsubsection{Income Payments}

Overseas residents working in Australia expect to be paid in their domestic
currency.

Increased employment of international workers increases supply of \$AUD.


\subsubsection{Dividend Payments and Profit Flows Overseas}

Overseas firms and investors expect their profits and dividend payments to be
in their domestic currency.

Increased dividend payments or profit flows overseas increases supply of \$AUD.




\section{Causes}

Australia has seen (on average) a depreciation in its dollar since 2012.


\subsection{Terms of Trade}

The world has seen falling commodity prices since 2012, indicating the end of
the commodities boom:

\begin{itemize}
\item Australia's largest exports are all commodities (iron ore, coal, natural
	gas account for 39.5\% of our total exports).
\item Has decreased our export price index.
\item Price of iron ore fell from \$160 AUD a tonne at the height of the boom
	to \$40 recently in 2016.
\item Significant unfavourable movement in our terms of trade.
\item Since our exports are inelastic, this fall in price was not met by any
	increase in quantity sold.
\item Decreased our export income.
\item Reduced demand for the \$AUD, causing depreciation.
\end{itemize}


\subsection{World Growth}

\subsubsection{China}

Slowing growth from China over the past 4 years:

\begin{itemize}
\item China's average growth for the decade is around 10\%, whereas predictions
	for 2016 are around 6 to 7\%.
\item China is the largest consumer of Australian exports, and the country
	Australia imports the most from.
\item Resulted in decreased demand for Australian exports, depreciating the
	\$AUD.
\end{itemize}


\subsubsection{World Supply}

From 2014 onwards:

\begin{itemize}
\item Large infrastructure investment in the mining sector of countries like
	Brazil and Chile on the back of the commodities price boom.
\item Increased productive capacity of other countries.
\item Resulted in large surplus of base metals in the world market.
\item Reduced commodity prices and demand for Australia's exports, depreciating
	the \$AUD.
\end{itemize}


\subsection{Interest Rate Differential}

Australia is in an expansionary monetary policy phase:

\begin{itemize}
\item Reserve Bank is reducing the cash rate. 4.5\% in 2011 to 2.0\% in
	December 2015.
\item Reducing return on investment for foreign investors, making Australia a
	less favourable destination.
\item Investors are more likely to shift their holdings overseas as other
	countries raise their cash rates after the GFC,
\item Increases supply of \$AUD in the market, depreciating dollar.
\end{itemize}


\subsection{International Foreign Investment}

\begin{itemize}
\item Stabilisation of key markets (such as the US) since 2013.
\item Slowing growth in Australian mining sectors from falling commodity prices
	leading to low overall economic growth.
\item Decreases Australia's favourability as an investment destination.
\item Causes a decrease in direct and portfolio investment in Australia,
	decreasing demand for the \$AUD, causing a depreciation.
\item May cause Australians to shift their own investments overseas, supplying
	\$AUD to the market, compounding the depreciation.
\end{itemize}


\subsection{Relative Inflation Rate}

\begin{itemize}
\item Australia has had recently a low level of inflation (1.7\% December 2015,
	RBA target is between 2 and 3\%).
\item Would usually bestow a competitive advantage on Australia, increasing our
	international competitiveness, increasing export sales.
\item But other countries have had a lower level of inflation (European Union,
	1.2\% December 2015).
\item Hasn't influenced the depreciation significantly.
\end{itemize}


\subsection{Domestic Growth}

Australia has a high marginal propensity to import:

\begin{itemize}
\item Australia has had strong economic growth in recent years (3\% December
	2015).
\item High spending on imports, supplying \$AUD to the market, depreciating the
	dollar.
\item Despite this, economic growth is volatile and not a large factor
	influencing the exchange rate.
\end{itemize}

Do not include in an essay.


\subsection{Speculation}

\begin{itemize}
\item Speculative investment in Australia has a large effect on the exchange
	rate.
\item \$180 billion daily turnover on the foreign exchange market, 8th most
	traded currency.
\item Favourability of Australia to investors is declining on the back of slow
	growth and stabilisation of alternate markets.
\item Decreases demand for \$AUD, depreciating the dollar.
\end{itemize}




\section{Effects}

The effects of a depreciation on the Australian economy.


\subsection{Exports}

As the \$AUD depreciates:

\begin{itemize}
\item Australian exports become cheaper for international consumers.
\item Improves our international competitiveness after a time lag in accordance
	with the J-curve theory.
\item Increases export income and growth in export industries.
\end{itemize}

For exporters that must import intermediate goods used in the manufacture of
the final consumption good:

\begin{itemize}
\item Increases the price of these intermediate goods.
\item May eliminate any profits incurred from increased demand for exports.
\end{itemize}


\subsection{Imports}

As the \$AUD depreciates:

\begin{itemize}
\item Imports become more expensive for Australian consumers.
\item Decreases import spending.
\item Increases competitiveness of import-competing industries.
\item Encourages substitution of imported products with locally made
	substitutes.
\end{itemize}


\subsection{Balance of Goods and Services}

In the long term:

\begin{itemize}
\item Increase in export income from improved international competitiveness.
\item Decrease in import spending from increased prices.
\item Increase the balance of goods and services.
\end{itemize}


\subsection{Foreign Debt}

The valuation effect:

\begin{itemize}
\item 40 to 60\% of Australia's foreign debt is denominated in foreign currency.
\item A depreciation immediately increases the size of this debt in Australian
	dollar terms.
\item Increases the size of interest repayments in the primary income category
	of the current account.
\item Increases the deficit on the primary income category.
\end{itemize}


\subsection{Current Account Deficit}

\begin{itemize}
\item Expected improvement in the balance of goods and services.
\item Expected decrease in the primary income category deficit.
\item Since goods and services is roughly 4 times as large as primary income,
	we would expect the current account deficit to decrease in the long term.
\end{itemize}


\subsection{Inflation}

As the \$AUD depreciates:

\begin{itemize}
\item The price of imported goods rises in Australian dollar terms.
\item Increases tradeables inflation.
\item Tradeables represet a significant portion of the goods used to measure
	the consumer price index.
\item Could increase the inflation rate.
\end{itemize}


\subsubsection{Cost Push Inflation}

A depreciation increases the cost of imported intermediate goods for domestic
producers.

Leads to cost push inflation (inflation driven by rising costs for producers).


\subsubsection{Demand Pull Inflation}

An increase in aggregate demand from increased export sales may also increase
the price of non-tradeables through demand pull inflation.


\subsection{Economic Growth}

As the \$AUD depreciates:

\begin{itemize}
\item Growth in export industries.
\item Growth in import-competing industries.
\item Decline in import spending increasing aggregate demand.
\item Increases economic growth and improvement in standard of living in the
	long term.
\item Increases the derived demand for labour, decreasing unemployment.
\end{itemize}


\subsubsection{Dutch Disease}

One of the effects of the strong appreciation of the \$AUD during the mining
boom as demand for our exports increased dramatically:

\begin{itemize}
\item Strong growth in one industry (the mining sector) causing strong
	appreciation of the \$AUD.
\item Dramatically decreases competitiveness of other sectors due to high
	prices for international consumers.
\item Causes two speed economy, where one sector is thriving and others are
	failing to compete in the world market.
\item Can cause structural unemployment in failing industries. For example,
	Holden and Toyota ceasing Australian manufacturing operations in 2017.
\end{itemize}

\end{document}
