
\documentclass[a4paper,11pt]{article}

% Math symbols
\usepackage{amsmath}
\usepackage{amsfonts}
\usepackage{esvect}

% Hyperlink contents page
\usepackage{hyperref}
\hypersetup{
	colorlinks,
	citecolor=black,
	filecolor=black,
	linkcolor=black,
	urlcolor=black
}

% AI files
\usepackage{graphicx}
\DeclareGraphicsRule{.ai}{pdf}{.ai}{}

% No indent on new paragraphs
\setlength{\parindent}{0mm}
\setlength{\parskip}{0.3cm}

% Alias \boldsymbol to \bb for vectors
\newcommand{\bb}{\boldsymbol}


\begin{document}

\title{Trade Liberalisation}
\author{Ben Anderson}
\date{\today}
\maketitle
\pagebreak

\tableofcontents
\pagebreak


\section{Trade Liberalisation}

The ability of people to undertake economic transactions with people in other
countries free from any restraints imposed by governments or other regulators.


\subsection{Most Favoured Nation Treatment}

The principle that all of a country's trading partners should be treated
equally, and that none should be discriminated against or given an unfair
advantage.


\subsection{National Treatment}

The principle that all goods, imported or domestically produced, should be
treated equally once they have entered the country.


\subsection{Trade Intensity Ratio}

$$
\frac{\text{X} + \text{M}}{\text{GDP}} \times 100
$$


\subsection{Trade Penetration Ratio}

$$
\frac{\text{M}}{\text{AD}}
$$




\section{Trade Blocs}

\subsection{Free Trade Area}

A group of member countries abolish trade restrictions between themselves, but
retain restrictions against non-member countries.


\subsection{Customs Union}

A group of member countries abolish trade restrictions between themselves, and
adopt a common set of restrictions against non-member countries.


\subsection{Common Market}

Same as a customs union, but also allows the free movement of labour and
capital within the member countries.


\subsection{Monetary Union}

Same as a common market, but adopts a common currency and coordination of
monetary policy throughout member countries through a central bank.




\section{Trade Diversion}

When trade is diverted from a low cost producer outside a free trade bloc, to
a higher cost producer inside the trade bloc, who is cheaper as a consequence
of the restrictions imposed by the trade bloc on non-member countries.




\section{Trade Agreements}

\subsection{Unilateral}

A country abolishes certain trade restrictions against all other countries.


\subsection{Bilateral}

A trade agreement between two countries to abolish certain trade restrictions.


\subsection{Multilateral}

A trade agreement between more than two countries to abolish certain trade
restrictions.


\subsubsection{Benefits}

Greater potential for mutual gain than bilateral agreements.


\subsubsection{Costs}

More difficult to negotiate than bilateral agreements.

\end{document}
