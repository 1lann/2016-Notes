
\documentclass[a4paper,11pt]{article}

% Math symbols
\usepackage{amsmath}
\usepackage{amsfonts}
\usepackage{esvect}

% Hyperlink contents page
\usepackage{hyperref}
\hypersetup{
	colorlinks,
	citecolor=black,
	filecolor=black,
	linkcolor=black,
	urlcolor=black
}

% AI files
\usepackage{graphicx}
\DeclareGraphicsRule{.ai}{pdf}{.ai}{}

% No indent on new paragraphs
\setlength{\parindent}{0mm}
\setlength{\parskip}{0.3cm}

% Alias \boldsymbol to \bb for vectors
\newcommand{\bb}{\boldsymbol}


\begin{document}

\title{Fiscal Policy}
\author{Ben Anderson}
\date{\today}
\maketitle
\pagebreak

\tableofcontents
\pagebreak


\section{Fiscal Policy}

The manipulation of government expenditure and revenue in order to influence the
level of economic activity.



\section{Budget}

A statement of planned expenditure and revenue streams for the forthcoming year.
The main instrument of fiscal policy.

Delievered by the federal government in May each year.

Likely to be different from the actual outcome of the economy during the year.
Becuase of this the government releases a mid-year economic outlook in December
or January, with revisions responding to changes in economic conditions.

\subsection{Roles}

\subsubsection{Allocative}

Influence resource allocation through changes to the taxation system and
government spending decisions.

Promotes the production and consumption of goods and servies associated with
positive externalities, while limiting those associated with negative ones.

\subsubsection{Redistributive}

Influence income distribution through taxation and welfare systems.

A progressive taxation system means both marginal and average rates of taxation
rise as income rises, redistributing wealth from high income to low income
households.

The provision of public goods and services aids this function.

\subsubsection{Stabilising}

Used as a counter-cyclical measure to minimise the volatility in the business
cycle and to achieve certain macroeconomic objectives.

\subsection{Outcomes}

\subsubsection{Deficit}

Where government expenditure exceeds revenue.

On its own, a deficit has an expansionary effect on the economy.

\subsubsection{Surplus}

Where government expenditure is less than its revenue.

On its own, a surplus has a contractionary effect on the economy.

\subsubsection{Balanced}

Where government expenditure is equal to its revenue.

On its own, a surplus has neither an expansionary or contractionary effect on
the economy (neutral).



\end{document}
