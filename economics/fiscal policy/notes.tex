
\documentclass[a4paper,11pt]{article}

% Math symbols
\usepackage{amsmath}
\usepackage{amsfonts}
\usepackage{esvect}

% Hyperlink contents page
\usepackage{hyperref}
\hypersetup{
	colorlinks,
	citecolor=black,
	filecolor=black,
	linkcolor=black,
	urlcolor=black
}

% AI files
\usepackage{graphicx}
\DeclareGraphicsRule{.ai}{pdf}{.ai}{}

% No indent on new paragraphs
\setlength{\parindent}{0mm}
\setlength{\parskip}{0.3cm}

% Alias \boldsymbol to \bb for vectors
\newcommand{\bb}{\boldsymbol}


\begin{document}

\title{Fiscal Policy}
\author{Ben Anderson}
\date{\today}
\maketitle
\pagebreak

\tableofcontents
\pagebreak


\section{Fiscal Policy}

\subsection{Definitions}

\textbf{Fiscal policy} is the manipulation of government expenditure and
revenue in order to influence the level of economic activity.

The \textbf{budget} is a statement of planned expenditure and revenue streams
by the government for the forthcoming financial year, delivered each May.

The \textbf{net financing requirement} for the public sector is equivalent to
the value of a government's budget deficit or surplus, but is positive for a
deficit and negative for a surplus.

\textbf{Fiscal consolidation} is aimed at reducing budget deficits, curtailing
the accrual of debt.

The government's \textbf{fiscal stance} (contractionary, expansionary, or
neutral) is best measured by the structural component of the budget, as
exogenous events such as natural disasters will impact the cyclical component,
potentially obscuring the government's position.


\subsection{Strengths}

\begin{description}
\item [Discriminatory] Fiscal policy can favour certain industries or groups
	over others. Different sectors of the economy can be targeted with different
	policy approaches, effective in a multi-speed economy.
\item [Outside Time Lag] The effect time lag of any fiscal policy changes
	is relatively short.
\item [Spending Tap] In a trough or recession, a government can open up a
	``spending tap", increasing the level of economic activity to aid economic
	growth.
\item [Complementary to Business Cycle] Automatic stabilisers will aid in
	reducing the volatility of the business cycle by dampening spending in a
	boom, and promoting it in a trough. Welfare payments increase and tax
	receipts decrease in a trough to promote economic growth.
\end{description}


\subsection{Weaknesses}

\begin{description}
\item [Inflexible] The budget does not tend to make large changes to the
	patterns of allocation and distribution established in previous years, due
	to a long term commitment to spending programs and consistency in operating
	costs. The government must also adhere to social and political concerns,
	such as protests and strikes from unionised workers.
\item [Inside Time Lag] The budget is delivered only once a year and requires
	many months of consultation and formulation, leading to a long inside time
	lag. This prevents the government from responding quickly to sudden changes
	in the economic climate.
\item [Political Economic Cycle] Where government policy measures align far more
	with political interests within the 3 year election cycle, than with what
	would be most beneficial for the economy. For example, increasing spending
	in marginal seats in an election year.
\item [Crowding Out] Where a budget deficit increases demand for credit, causing
	a rise in interest rates, and a fall in investment and consumption (see
	below). This has the opposite effect than desired.
\end{description}



\section{Budget}

The main instrument of fiscal policy. Delievered by the federal government in
May each year.


\subsection{Components}

\begin{description}
\item [Cyclical] Changes to government spending or revenue caused by variations
	in economic activity through the various phases of the business cycle.
\item [Structural] Deliberate decisions made by the government in planning
	specific expenditures and revenue streams.
\end{description}


\subsection{Planned and Actual Budget Outcomes}

The actual budget outcome will be different to the forecasted one presented in
the budget at the beginning of the year.

Becuase of this the government releases a mid-year economic outlook in December
or January, with revisions responding to changes in economic conditions.

Reasons for this are predominantly cyclical, including the terms of trade,
exchange rate, interest rates, fluctuations in world markets (eg. oil prices),
natural disasters, terrorist attacks, etc.


\subsection{Roles}

\subsubsection{Allocative}

Influence resource allocation through changes to the taxation system and
government spending decisions.

Promotes the production and consumption of goods and servies associated with
positive externalities, while limiting those associated with negative ones.

\subsubsection{Redistributive}

Influence income distribution through taxation and welfare systems.

A progressive taxation system means both marginal and average rates of taxation
rise as income rises, redistributing wealth from high income to low income
households.

The provision of public goods and services aids this function.

\subsubsection{Stabilising}

Used as a counter-cyclical measure to minimise the volatility in the business
cycle and to achieve certain macroeconomic objectives.


\subsection{Outcomes}

\begin{description}
\item [Deficit] Where government expenditure exceeds revenue (net spender).
\item [Surplus] Where government expenditure is less than its revenue (net
	saver).
\item [Balanced] Where government expenditure is equal to its revenue.
\end{description}

\subsection{Effect}

In isolation:

\begin{description}
\item [Deficit] Expansionary, as the government is injecting funds into the
	economy overall.
\item [Surplus] Contractionary, as the government is withdrawing funds in the
	form of taxes overall.
\end{description}

But to better gauge the effect of the budget outcome, it should be compared to
the previous year:

\begin{itemize}
\item An increase in the size of a deficit is expansionary.
\item A decrease in the size of a deficit is contractionary.
\item An increase in the size of a surplus is contractionary.
\item A decrease in the size of a surplus is expansionary.
\end{itemize}


\subsection{Measures}

The budget outcome can be measured by:

\begin{description}
\item [Headline Balance] Total revenue subtract expenditure.
\item [Underlying Cash Balance] Equivalent to headline balance, but excluding
	all one-time transactions such as the sale of assets.
\item [Fiscal Balance] The accrual accounting equivalent of the underlying cash
	balance, including all future transcations that are yet to be paid.
\end{description}


\subsection{Balanced Budget Multiplier}

Any changes to government taxation has a multiplier effect where $k$ (the
multiplier coefficient) is:

$$
\begin{aligned}
k & = \frac{\text{MPC}}{1 - \text{MPC}} \\
& = \frac{1}{1 - \text{MPC}} - 1 \\
\end{aligned}
$$

The tax multiplier is always 1 less than the normal multiplier.

Tricks:

\begin{itemize}
\item The final change in income from a change in government spending and a
	change in taxation will always be in the direction of government spending.
\item If both government spending and taxation change by the same amount, then
	the final change in income is equal to the change in government spending.
\end{itemize}



\section{Automatic Stabilisers}

\subsection{Definitions}

\textbf{Automatic stabilisers} are economic policies and programs designed to
respond to and offset fluctuations in economic activity throughout the various
phases of the business cycle, without intervention from policymakers.

\textbf{Transfer payments} are one-way payments by the government to individuals
for which no good or service is exchanged.

Automatic stabilisers impact only the cyclical component of the budget.


\subsection{Mechanism}

In a boom:

\begin{itemize}
\item Rising levels of income increases tax revenue.
\item Falling unemployment decreases unemployment benefits, decreasing transfer
	payments.
\item This reduces government spending, moving the budget towards a surplus,
	having a contractionary effect on the economy.
\end{itemize}

In a trough:

\begin{itemize}
\item Falling levels of income decreases tax revenue.
\item Rising unemployment increases unemployment benefits, increasing transfer
	payments.
\item This increases government spending, moving the budget towards a deficit,
	having an expansionary effect on the economy.
\end{itemize}


\subsection{Model}

\begin{itemize}
\item Real GDP along X axis.
\item Government spending and taxation along Y axis.
\item Tax revenue curve with a positive gradient.
\item Welfare payments curve with a negative gradient.
\item The point where the two lines intersect is when the budget is balanced.
\item To the right is budget surplus.
\item To the left is budget deficit.
\end{itemize}



\section{Budget Deficit}

\subsection{Definitions}

A \textbf{bond} is a financial instrument which raises funds for its issuer, in
return for a rate of interest payable to the holder over a period of time.

\textbf{Crowding out} is where the sale of government bonds to finance a budget
deficit increases interest rates through an increased demand for credit,
reducing consumption and investment, opposing the desired effect.

The \textbf{twin deficits hypthesis} indicates a strong link between the
government's budget deficit, and the current account deficit.


\subsection{Financing a Budget Deficit}

The government can finance a budget deficit in 3 ways:

\begin{description}
\item [Bonds] Where the government sells bonds to investors to finance its
	deficit.
\item [Borrowing from RBA] Where the government borrows money from the RBA to
	finance its deficit.
\item [Borrowing from Overseas] Where the government borrows money from overseas
	investors (in the form of selling bonds) to finance its deficit.
\end{description}

The government cannot borrow from the private sector to finance its deficit
because it would have no net effect on the economy. The government would remove
money from the economy, only to inject it back through its spending.


\subsection{Crowding Out}

The sale of government bonds can cause the issue of crowding out:

\begin{itemize}
\item Financing a deficit through borrowing from consumers (through the sale of
	bonds) increases demand for funds in credit markets.
\item This increases interest rates, increasing the cost of borrowing, the
	opportunity cost of holding cash for consumers, and reduces their disposable
	income by increasing interest repayments on existing debt.
\item This reduces consumption and investment, having the opposite effect than
	desired on the economy.
\end{itemize}

This can be shown on an aggregate expenditure model:

\begin{itemize}
\item Draw an autonomous increase in government spending from $AE_1$ to $AE_2$.
\item Draw a further decrease in autonomous investment and consumption from
	$AE_2$ to $AE_3$ due to crowding out.
\end{itemize}


\subsection{Exchange Rate Effect}

Borrowing from overseas creates the issue of foreign debt, and has implications
for our exchange rate:

\begin{itemize}
\item Borrowing from overseas investors increases demand for the \$AUD, causing
	it to appreciate.
\item This reduces the international competitiveness of our exports, reducing
	export sales and income after a time lag.
\item This decreases net exports, having the opposite effect than desired on
	the economy.
\end{itemize}


\subsection{Model}

The effect of a budget deficit on an economy can be shown using a deflationary
gap model with a further government spending and taxation subpanel:

\begin{itemize}
\item Show an autonomous increase in government spending on an aggregate
	expenditure model.
\item Draw construction lines from each of the two equilibrium points to the
	X axis.
\item Draw a further subpanel underneath this.
\item Label the Y axis government spending and taxation.
\item Label the X axis income and output.
\item Draw a rise in the level of government spending through two horizontal
	lines, where the distance they are apart is equal to the original autonomous
	increase in the aggregate expenditure model.
\item Draw a tax curve with a positive gradient, which intersects the two
	government spending lines at each of the equilibrium points.
\end{itemize}


\subsection{Twin Deficits Hypothesis}

In a circular flow model in equilibrium:

$$
\begin{aligned}
\text{injections} & = \text{leakages} \\
I + T + M & = S + G + X \\
X - M & = (I - S) + (G - T) \\
\end{aligned}
$$

Where $G - T$ is the budget deficit, and $X - M$ is net exports, which
represents the largest component of the current account.

Assuming $S - I$ is constant, a larger government deficit will decrease net
exports, resulting in a larger current account deficit.



\section{Budget Surplus}

\subsection{Spending Targets}

The government can allocate the proceeds of a budget surplus to 3 main areas:

\begin{description}
\item [Retire Debt] Where the government retires public sector debt by
	repurchasing bonds, reducing its interest repayment burden.
\item [Deposit Funds with RBA] Where the government deposits money with the RBA
	to potentially finance future tax cuts or infrastructure projects.
\item [Special Funds] Where the proceeds of a budget surplus are depositied in
	specialised funds, which are allocated for a specific purpose only, and
	cannot be raided by future governments. For example, the Future Fund (paying
	for the government's superannuation liability to the public sector
	workforce), and the Education Investment Fund (funding for univeristies).
\end{description}


\subsection{Crowding In}

The buy-back of government bonds can cause the issue of crowding in:

\begin{itemize}
\item Buying back bonds reduces demand for credit, lowering interest rates.
\item This increases consumption and investment, increasing economic activity,
	having the opposite effect desired from a budget surplus.
\item This is not really an issue, because if the government is worried about
	an overactive economy, they can construct a larger surplus.
\end{itemize}



\section{Balanced Budget}

Prior to the adoption of Keynesian economic theory, governments atttempted to
balance their budgets.
This increased the volatility of the business cycle because:

\begin{itemize}
\item In troughs, income decreases, decreasing tax receipts, reducing government
	spending.
\item This had a further contractionary effect on the economy, worsening the
	trough.
\item In booms, income increases, increasing tax receipts, increasing government
	spending.
\item This had a further inflationary effect on the economy, increasing
	inflation.
\end{itemize}



\section{Fiscal Strategy}

\subsection{Deficits}

The budget deficit (underlying cash balance) for 2015/16 was \$39.9 billion
(2.4\% of GDP).

The forecasted budget deficit in 2016/17 (underlying cash balance) is \$37.1
billion (2.2\% of GDP).

The budget deficit (underlying cash balance) is projected to decrease to \$6
billion in 2018/19.

The government has recorded budget deficits since the GFC in 2008/09.


\subsection{Historical Deficits}

Over the past 5 years Australia has seen consistent deficits after the GFC
because:

\begin{description}
\item [Income Tax] Low growth in personal taxation revenue due to slower
	employment and wages growth.
\item [Company Tax] Reduced company taxation from falling terms of trade and
	lower demand (especially in China).
\item [Capital Gains Tax] Reduced capital gains tax revenue as share markets
	fail to return to pre-GFC highs.
\item [Expenditure] Growing expenditure to support economic growth and employment.
\end{description}


\subsection{Causes of the Forecasted Deficit}

Although fundamentally caused by insufficient tax revenue to finance spending,
other factors that contribute to the forecasted budget deficit include:

\begin{description}
\item [Nominal GDP] Weaker nominal GDP caused by low inflation, reducing tax
	revenue from lower nominal income levels.
\item [Terms of Trade] Falling terms of trade, reducing net exports and
	government company tax revenue.
\item [Wage Growth] Lower wages growth reducing income tax revenue.
\item [Unemployment Benefits] Higher than natural rate of unemployment (5.8\%
	in June 2016, where the natural rate is 4 - 5\%) increasing government
	spending on welfare payments, particularly unemployment benefits.
\end{description}


\subsection{Effects of the Forecasted Deficit}

\begin{itemize}
\item The slight decrease in the size of the budget deficit is expected to have
	a mildly contractionary effect on the economy.
\item Despite this, such a large deficit will still be expansionary.
\end{itemize}


\subsection{Strategy}

The government's medium term fiscal strategy:

\begin{itemize}
\item Achieve budget surplus (contractionary fiscal stance).
\item Boost productivity and workforce participation.
\item Control expenditure to reduce the government's share of the economy (ie.
	proportion of aggregate demand)
\item Improve the government's net financial worth over time.
\end{itemize}

This will be achieved primarily by:

\begin{itemize}
\item Lower company tax for small businesses (from 28.5\% to 27.5\%)
\item Target tax avoidance primarily by multinational corporations
\item Increase of the 32.5\% marginal tax bracket from \$80000 to \$87000
\item Superannuation reform to promote fairness
\item Increase in the tobacco excise by 12.5\% annually for 4 years
\end{itemize}


\subsection{2016-17 Forecast}

Revenue forecasts:

\begin{itemize}
\item Growth in individual income tax receipts by 4.5\%.
\item Growth in company tax by 6.6\% after a general rise in company
	profitability.
\end{itemize}

Expenditure forecasts:

\begin{itemize}
\item Increase in overall expenditure by 4.7\% in 2016-17 over the previous
	year.
\item Increased public hospital, education, youth employment, and national
	disability funding.
\end{itemize}

\end{document}
