
\documentclass[a4paper,11pt]{article}

% Math symbols
\usepackage{amsmath}
\usepackage{amsfonts}
\usepackage{esvect}

% Hyperlink contents page
\usepackage{hyperref}
\hypersetup{
	colorlinks,
	citecolor=black,
	filecolor=black,
	linkcolor=black,
	urlcolor=black
}

% AI files
\usepackage{graphicx}
\DeclareGraphicsRule{.ai}{pdf}{.ai}{}

% No indent on new paragraphs
\setlength{\parindent}{0mm}
\setlength{\parskip}{0.3cm}

% Alias \boldsymbol to \bb for vectors
\newcommand{\bb}{\boldsymbol}


\begin{document}

\title{Macroeconomic Objectives}
\author{Ben Anderson}
\date{\today}
\maketitle
\pagebreak

\tableofcontents
\pagebreak


\section{Macroeconomic Objectives}

To provide the highest standard of living for Australians today and into the
future, in both quantitative and qualitative forms.


\subsection{Internal Stability}

Internal stability is achieved when the country achieves a rate of economic
growth sufficient to achieve full employment, while maintaining price stability.

Most affected by the demand side of the economy.

Includes sustainable economic growth, price stability, and full employment.


\subsection{Sustainable Economic Growth}

Economic growth is the increasing capacity of the economy to satisfy the wants
of its members.

Measured by percentage change in real GDP.

Real GDP is the total value of all new, final goods and services produced in a
country during a year.

Target rate is 3 to 4\%, as established by the Mortimer report.

3.1\% in March 2016. The average growth rate for the last 30 years is 3.2\%.


\subsubsection{Effects}

Advantages:

\begin{description}
\item [Welfare] Increases standard of living and economic welfare across
	society.
\item [Income] Increased real income for consumers.
\item [Consumption] Greater consumption of goods and services of higher quality.
\end{description}

Disadvantages:

\begin{description}
\item [Structural Unemployment] Economic growth may cause structural
	unemployment in specific industries as technology or consumer preferences
	progress over time.
\item [Inflation] Economic growth often drives up prices (increasing
	aggregate demand).
\item [Environmental Degradation] Economic growth may deplete our natural
	resources and increase pollution levels.
\item [Social Dislocation] Disruption of social norms.
\end{description}


\subsection{Price Stability}

An economy achieves price stability when there is little increase in the general
price level.

Stable prices maintain consumer spending power and the international
competitiveness of the economy.

Measured by percentage change in CPI (inflation).

Disinflation is a falling inflation rate (but still positive). Deflation is a
negative inflation rate.

Target rate is 2 to 3\%, a specific target set by the RBA.

1\% in June 2016.


\subsubsection{Effects}

The disadvantages that come with high inflation include:

\begin{description}
\item [Interest Rates] Rising prices increase the demand for credit, placing
	upwards pressure on interest rates.
\item [Investment] Rising interest rates and prices (increasing projected costs
	of investment projects) will discourage investment.
\item [Purchasing Power] Rising prices erodes consumer purchasing power,
	reducing their real income.
\item [Competitiveness] Rising prices reduce our competitiveness in the
	international market, and increases the competitiveness of imports in the
	domestic market.
\item [Savings] Inflation causes people to lose faith in cash as a means of
	storing wealth, causing them to speculatively purchase physical assets that
	are likely to appreciate in value.
\end{description}


\subsection{Full Employment}

Measured by unemployment (the percentage of the population without a job that
are currently looking for work).

Target rate for unemployment is 4 to 5\%. 5.8\% in June 2016.

Can also be measured by the workforce participation rate (the percentage of the
population active in the workforce).

Target rate for workforce participation is greater than 65\%. 54.8\% in June
2016.


\subsubsection{Types of Unemployment}

Cyclical unemployment is the component of unemployment that fluctuates with
changes in the business cycle.

Structural unemployment is the component that does not fluctuate with changes
in the business cycle.

Frictional unemployment is the component that represents all people in the
process of moving between jobs, where there is an uncertain matching of demand
and supply across the labour market.


\subsubsection{Effects}

The disadvantages that come with high unemployment include:

\begin{description}
\item [Consumption] A lack of income for a larger proportion of the population
	decreases aggregate consumption and aggregate demand.
\item [Business Confidence] Lower unemployment decreases business confidence,
	which especially impacts the level of investment.
\item [Welfare Payments] Increases the value of welfare payments by the
	government.
\item [Taxation] Decreases the government's taxation income as they receive
	taxation receipts from fewer people.
\item [Opportunity Cost] There is an opportunity cost of the lost government and
	consumer income, which could be spent stimulating the economy.
\end{description}


\subsection{External Stability}

The capacity of a country to fulfil its financial obligations with the rest of
the world.

Measured through the CAD, the exchange rate, and public sector debt.

Target rate for the CAD is less than 5\% of GDP. 4.6\% in 2015.

Want the exchange rate to be seen as stable. \$0.76 USD in May 2016.

Target rate for public sector debt is less than 50\% of GDP, as recommended by
the IMF. 36.8\% in 2015.


\subsection{Efficient Resource Allocation}

To ensure the country achieves efficient resource allocation.

A supply side objective, tending to be long term.

Measured by 3 productivity rates:

\begin{description}
\item [Labour] Output produced per unit time of labour.
\item [Capital] Output produced per unit of capital machinery.
\item [Multifactor] Output produced per unit of all inputs.
\end{description}

No set target rate, but greater than 3\% multifactor productivity is excellent.


\subsection{Equitable Income Distribution}

To achieve an equitable (fair) income distribution.

Measured by the Gini coefficient of the Lorenz curve for the economy.

Target rate between 0.3 and 0.4\%.




\section{RBA Policy Objectives}

The RBA policy objectives are very similar to those included in internal
stability, with slightly different language:

\begin{itemize}
\item The stability of the currency of Australia (price stability).
\item Maintenance of full employment (low unemployment).
\item The economic prosperity and welfare of the people of Australia (standard
	of living and sustainable economic growth).
\end{itemize}




\section{Government Policy}

The government sets policy to achieve its macroeconomic objectives.


\subsection{Fiscal Policy}

Fiscal policy is the use of government spending and revenue raising to influence
economic activity.

Fiscal policy is set through the use of the government's budget, released in May
every year.

Managed by the government's Treasury department, whose minister is Scott
Morrison (the Treasurer).

Focused on the demand side of the economy, and is medium term.


\subsection{Monetary Policy}

Monetary policy is the use of the cash rate to influence other interest rates.

Managed by the RBA, whose Governor is Glenn Stevens.

Focused on the demand side of the economy, and is relatively short term.


\subsection{Microeconomic Reform}

Microeconomic reform is concerned with improving the efficiency and productivity
of the economy.

Managed by the Council of Australian Governments, which is a combination of
representatives from the state and federal governments.

The Productivity Commission provides the Council with advisory information to
inform policy decisions.

Focused on the supply side of the economy, and is long term.




\section{Compatible Objectives}

Compatible objectives are where achieving one objective aids in achieving the
other.


\subsection{Economic Growth and Full Employment}

Increased economic growth promotes employment:

\begin{itemize}
\item Economic growth will result in an increased demand for goods and services,
	increasing aggregate demand.
\item Derived demand for resources used to produce these goods also increases.
\item Increases derived demand for labour, reducing unemployment.
\end{itemize}


\subsection{Efficient Resource Allocation and Economic Growth}

Improving resource allocation promotes economic growth and price stability:

\begin{itemize}
\item Improving microeconomic reform increases the economy's efficiency and
	productivity.
\item This reduces costs of production, increasing aggregate supply.
\item This increases economic growth and price stability (as seen on an AD/AS
	model).
\end{itemize}


\subsection{Full Employment and Equitable Income Distribution}

Lower unemployment improves an economy's income distribution:

\begin{itemize}
\item Policies design to reduce unemployment reduce the proportion of people
	supported by government welfare.
\item This increases the income of people in a low socioeconomic class,
	resulting in a more equitable income distribution.
\end{itemize}




\section{Conflicting Objectives}

Conflicting objectives are ones where achiving one hinders our ability to
achieve the other.

They cannot be pursued simultaneously.


\subsection{Economic Growth and Price Stability}

Promoting economic growth reduces price stability:

\begin{itemize}
\item Policies may promote economic growth through stimulating aggregate demand.
\item This places pressure on the prices of resources and labour as output
	increases, leading to cost push inflation.
\item This increases inflation if the economy has little extra capacity
	(macroeconomic equilibrium lies in the intermediate range).
\item This reduces price stability.
\end{itemize}


\subsection{Full Employment and Price Stability}

Reducing unemployment reduces price stability:

\begin{itemize}
\item Policies may attempt to reduce unemployment through lifting the level of
	economic activity.
\item For the same reasons as above, this reduces price stability.
\end{itemize}


\subsection{Price Stability and Growth}

Increasing price stability reduces growth:

\begin{itemize}
\item Policies designed to increase price stability do so by reducing aggregate
	demand.
\item This reduces economic growth and employment.
\end{itemize}


\subsection{Economic Growth and Income Distribution}

Increasing economic growth may reduce the equitability of income distribution:

\begin{itemize}
\item Owners of businesses in growing industries and of appreicating assets may
	gain more than others in more disadvantaged situations.
\item This may distort income distribution towards the wealthy.
\end{itemize}


\subsection{Economic Growth and Structural Unemployment}

Increasing economic growth may result in structural unemployment:

\begin{itemize}
\item Large growth in emerging sectors may result in rapid structural change.
\item This will increase demand for more modern skills, reducing demand for
	outdated ones.
\item This will result in an increase in structural unemployment.
\end{itemize}

Try not to use this in a response, since structural unemployment isn't an actual
macroeconomic objective.




\section{Policy Time Lags}

\subsection{Recognition}

The time lag between when an economic shock or event occurs, and when it is
recognised by the government or reserve bank.

After recognition, the institution may chose to react to the event.


\subsection{Decision}

The time lag between when an economic event is recognised, and when a decision
is made on new policy to implement.


\subsection{Impact}

The time lag between implementing new policy in response to an economic event,
and when this new policy results in an observable change in the economy.

This change will usually be observable through economic indicators.

\end{document}
