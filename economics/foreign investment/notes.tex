
\documentclass[a4paper,11pt]{article}

% Math symbols
\usepackage{amsmath}
\usepackage{amsfonts}
\usepackage{esvect}

% Hyperlink contents page
\usepackage{hyperref}
\hypersetup{
	colorlinks,
	citecolor=black,
	filecolor=black,
	linkcolor=black,
	urlcolor=black
}

% AI files
\usepackage{graphicx}
\DeclareGraphicsRule{.ai}{pdf}{.ai}{}

% No indent on new paragraphs
\setlength{\parindent}{0mm}
\setlength{\parskip}{0.3cm}

% Alias \boldsymbol to \bb for vectors
\newcommand{\bb}{\boldsymbol}


\begin{document}

\title{Foreign Debt and Investment}
\author{Ben Anderson}
\date{\today}
\maketitle
\pagebreak

\tableofcontents
\pagebreak


\section{External Stability}

The ability for a country to meet its financial obligations with the rest of
the world.


\subsection{Components}

Three components are used to measure how externally stable a country is.


\subsubsection{Export Income}

Whether a country's export income is sufficient to finance its import spending.

A current account deficit of greater than 5\% of GDP is considered
unsustainable by the IMF.

A current account deficit implies a country relies on foreign liabilities to
finance spending, which is only sustainable if income grows at a similar rate
to liabilities.


\subsubsection{Foreign Debt}

Whether a country has a managable level of foreign debt.

Public sector debt greater than 50\% of GDP is considered unsustainable by the
IMF.

Public sector debt is seen as unproductive as it is predominantly used to
finance essential government services that do not contribute to economic
growth (eg. police and fire services).


\subsubsection{Exchange Rate}

Whether a country's exchange rate is believed to be stable.




\section{Foreign Investment}

The stock of all financial assets in Australia owned by foreign residents, and
any financial transactions recorded in Australia's balance of payments that
increase or decrease this stock.


\subsection{Gross Foreign Investment}

Value of all foreign investment into Australia.


\subsection{Net Foreign Investment}

All foreign investment into Australia subtract all Australian investment
abroad.


\subsection{Foreign Assets}

The stock of all assets located abroad that Australians own.


\subsection{Foreign Liabilities}

The stock of all Australian assets that are owned by foreign entities.


\subsection{Statistics}

Stock of foreign investment in Australia (June 2014): \$2609.7 billion

Stock of Australian investment abroad (June 2014): \$1745.5 billion

Net foreign investment (June 2014): \$864 billion

Growth in foreign investment: 70\% of GDP (1990) to 160\% of GDP (2014)

Growth in foreign investment: Around \$800 billion (2000) to over \$2600
billion (2014)



\section{Type Classification of Foreign Investment}

\subsection{Equity}

The sale of ownership of an Australian asset to a foreign investor.


\subsubsection{Effects}

\begin{description}
\item [Dividends and Profit Flows] The payment of dividends and profit flows
	back to the foreign investor as a return on investment.
\item [Foreign Ownership] Where decisions involving the asset may be made by
	foreign owners without regard for Australia's best interests (eg. tax
	avoidance).
\item [Loss of Future Profits] Selling an asset means the original owner loses
	out on their share of any future profits generated by the asset.
\end{description}


\subsection{Borrowing}

The loaning of financial capital from a foreign investor.


\subsubsection{Effects}

\begin{description}
\item [Interest Repayments] Interest repayments that must be made to the
	foreign investor.
\item [Foreign Debt] Increase in the stock of foreign debt.
\end{description}



\section{Size Classification of Foreign Investment}

\subsection{Direct}

Ownership of a significant controlling interest in the asset.

A significant controlling interest is defined as 10\% or greater ownership of
the asset.

Only includes equity transactions. All borrowing is classified as portfolio, as
no controlling interest is obtained by borrowing.


\subsubsection{Nature}

\begin{description}
\item [Productive based] Usually the transactions aim to improve economic
	growth and generate output.
\item [Long Term] Since the transactions are usually quite large, they tend to
	be long term.
\item [Stable] Resulting dividend and profit flow payments are usually quite
	stable.
\end{description}


\subsubsection{Effects}

As above:

\begin{itemize}
\item Dividend payments and profit flows.
\item The issue of foreign owernship.
\item Loss of share in future profits.
\end{itemize}


\subsubsection{Examples}

\begin{itemize}
\item Land aquisition
\item Sale of 10\% or more of shares in a company
\item Establishment of a subsidiary
\item Reinvestment of profits
\end{itemize}


\subsection{Portfolio}

Transfer of ownership of less than 10\% of an asset (an insignificant
controlling interest).

Includes all borrowing, as any amount of borrowing cannot give a significant
controlling interest in an asset.


\subsubsection{Nature}

\begin{description}
\item [Short Term] Most borrowing has short term interests in mind.
\item [Speculative] Most business ventures are funded through venture capital
	(sale of small amounts of equity). Since most business ventures fail,
	portfolio investment is speculative.
\end{description}


\subsubsection{Examples}

\begin{itemize}
\item All forms of borrowing (debt, government bonds, etc).
\item Sale of less than 10\% of the shares in a company.
\end{itemize}




\section{Investment Statistics}

75\% of foreign investment is invested in the private sector (25\% in the
public sector).


\subsection{Largest Investors}

The countries that invest the most in Australia:

\begin{enumerate}
\item United States (27\%)
\item United Kingdom (23\%)
\item Japan (5.3\%)
\end{enumerate}

China is the fastest growing, with a growth rate of 967\% between 2003 and
2013.


\subsection{Largest Industries}

The Australian industries that were invested the most in:

\begin{enumerate}
\item Mining (37\%)
\item Manufacturing (14\%)
\item Finance and Insurance (11\%)
\end{enumerate}




\section{Appearance to Investors}

Reasons for Australia's favourable appearance to foreign investors:

\begin{description}
\item [Interest Rate Differential] Australia has a higher interest rate than
	most of the rest of the world. Australia: 2\% (December 2015), USA: 0.5\%
	(December 2015)
\item [Natural Resources] Australia has an abundance of natural resources
	to export.
\item [Economic Growth] Australia has had consistent and strong economic growth
	over the past decade. Decade average of 2.9\% annual economic growth.
\item [Sense of Security] Australia has developed and well regulated financial
	markets, offering a sense of security to investors.
\item [Legal System] Australia has an excellent legal system for the
	enforcement of contracts.
\item [Labour Force] Australia has a skilled and productive labour force.
\end{description}



\section{Effects of Foreign Investment on the Balance of Payments}

When foreign investment is made into Australia:

\begin{itemize}
\item \$AUD is demanded, appreciating the exchange rate.
\item Credit in the financial account, increasing capital and financial account
	surplus.
\item Future interest repayments, dividends, etc are recorded as debits in the
	primary incomes category of the current account, increasing the current
	account deficit.
\end{itemize}


\subsection{Aggregate Demand}

In the short to medium term:

\begin{itemize}
\item Increases aggregate demand, since investment is a component of aggregate
	demand.
\item Increases consumption.
\item Increases firm spending on imported intermediate goods.
\item Increases import spending.
\item Decreases balance of goods and services.
\item Increases current account deficit.
\end{itemize}


\subsection{Productive Capacity}

In the mid to long term:

\begin{itemize}
\item Increases productive capacity through construction of new infrastructure,
	expansion of labour force, technology development, etc.
\item Increases export sales and export income.
\item Increases balance of goods and services.
\item Decreases current account deficit.
\end{itemize}




\section{Benefits of Foreign Investment}

\subsection{Access to Financial Capital}

Due to Australia's investment and spending imbalance (see balance of payments):

\begin{itemize}
\item Access to financial capital through foreign investment has allowed us to
	expand infrastructure, increase export sales, etc.
\item Enabled the Australian economy to grow at a much faster rate than it
	would have otherwise.
\item Improved our standard of living.
\end{itemize}


\subsubsection{Multiplier Effects}

Has had a number of multiplier effects on the economy:

\begin{itemize}
\item Where development in one sector increases demand for products from other
	industries in the economy.
\item Creates derived demand for labour and capital goods.
\end{itemize}


\subsection{Aggregate Demand}

Foreign investment is a component of aggregate demand:

\begin{itemize}
\item Increases aggregate demand, shifting aggregate demand curve outwards.
\item Increases the stock of physical capital (eg. infrastructure), increasing
	the capital to labour ratio, shifting the aggregate supply curve outwards.
\item Increases GDP, improving standard of living.
\end{itemize}


\subsection{Technology and Experience}

For direct investment only:

\begin{itemize}
\item Transfers improved technologies and managerial experience into Australia.
\item Increases efficiency of industries, improving productivity and output.
\item Increases economic growth.
\end{itemize}


\subsection{Tax Revenue}

Due to an increase in aggregate demand:

\begin{itemize}
\item Increased consumption, spending, imports, infrastructure development,
	economic growth, etc.
\item Increases government tax revenue.
\item Allows government to increase funding for essential yet unprofitable
	services like policemen or public education.
\item Also improves ability for the government to service its debt.
\end{itemize}




\section{Costs of Foreign Investment}

\subsection{Foreign Ownership}

For direct investment only:

\begin{itemize}
\item Direct investment involves the loss of ownership of an asset to a foreign
	entity.
\item Decisions can be made involving the asset without regard for Australia's
	best interests.
\item For example, sale of land overlooking Great Barrier Reef, which is
	developed into a tourist resort without regard for the environment and
	sustainability of the reef.
\end{itemize}


\subsection{Speculative}

For portfolio investment only:

\begin{itemize}
\item Relatively small size of transactions, not productive for the economy,
	short term and speculative.
\item Causes instability and volatility in the exchange rate due to fluctuating
	demand.
\end{itemize}


\subsection{Transfer Pricing and Tax Avoidance}

For direct investment only:

\begin{itemize}
\item Legal method of minimising amount of corporate tax paid in a country.
\item Where a company exports a good to itself in another country with a lower
	corporate tax rate, where it then sells the good to the world market for
	profit.
\item Cheats the Australian government out of a large portion of its tax
	revenue, reducing its income and hindering its ability to fund services.
\item For example, Apple paying roughly 0.7\% tax on its Australian
	subsidiaries in 2014.
\end{itemize}


\subsection{Overseas Operations}

For direct investment only:

\begin{itemize}
\item Transfer of ownership of Australian assets to overseas investors.
\item Can lead to the movement of operations originally in Australia to other
	countries with cheaper labour costs, better technology, etc.
\item Increases unemployment in specific industries, reduces economic growth
	and government tax revenue.
\item For example, Holden and Toyota moving its Australian manufacturing
	plants overseas in 2017.
\end{itemize}


\subsection{Structural Change}

\begin{itemize}
\item Heavy foreign investment into a particular sector over a long period of
	time reduces diversity in a country's industry base.
\item Creates a reliance on that sector's exports.
\item Any downswing in global demand for these exports can cause large
	structural unemployment in the economy.
\item For example, the mining boom.
\end{itemize}




\section{Foreign Debt}

\subsection{Gross Foreign Debt}

The total value of Australia's borrowings from overseas entities.


\subsection{Net Foreign Debt}

Australia's overseas borrowings subtract Australia's lendings to foreign
debtors.


\subsection{Private Sector Debt}

Debt owed by the private sector (non-official sector).

Typically productive based, used to finance investment that grows the
economy.

76\% of gross foreign debt (in 2014) is owed by the private sector.


\subsection{Public Sector Debt}

Debt owed by the public sector (official sector, government sector).

Typically unproductive for the economy, used to finance essential government
services which do not generate profit and contribute to economic growth.

26\% of gross foreign debt (in 2014) is owed by the public sector.


\subsection{Trend}

Foreign debt has been increasing over the past 2 decades.

32\% of GDP in 1990 to 55\% of GDP in 2014.

Highlights Australia's inclination to borrow rather than sell equity, in order
to retain a share of future profits.




\section{Foreign Liabilities}

A legal obligation for an entity to pay back something of financial value to
another entity located overseas.


\subsection{Total Foreign Liabilities}

The total value of all Australia's equity sales to foreign investors, and
borrowings from overseas entities.

Increased dramatically over the past 2 decades.

\$0.366 billion in 1992 to \$1006 billion in 2014.


\subsection{Net Foreign Liabilities}

Australia's total liabilities to the rest of the world, subtract the world's
total liabilities to Australia (Australia's investment abroad).

\end{document}
