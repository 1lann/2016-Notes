
\documentclass[a4paper,11pt]{article}

% Math symbols
\usepackage{amsmath}
\usepackage{amsfonts}
\usepackage{esvect}

% Hyperlink contents page
\usepackage{hyperref}
\hypersetup{
	colorlinks,
	citecolor=black,
	filecolor=black,
	linkcolor=black,
	urlcolor=black
}

% AI files
\usepackage{graphicx}
\DeclareGraphicsRule{.ai}{pdf}{.ai}{}

% No indent on new paragraphs
\setlength{\parindent}{0mm}
\setlength{\parskip}{0.3cm}

% Alias \boldsymbol to \bb for vectors
\newcommand{\bb}{\boldsymbol}


\begin{document}

\title{Business Cycle}
\author{Ben Anderson}
\date{\today}
\maketitle
\pagebreak

\tableofcontents
\pagebreak


\section{Indicators}

Indicators present us with information about the performance of the
macroeconomic environment.

No indicator is ever definite, but will either support or contradict different
suggetions.


\subsection{Leading}

Changes in a leading indicator can foreshadow the occurrence of future economic
events.

For example, business and consumer confidence.


\subsection{Coincidental}

A coincidental indicator changes as an event occurs.

For example, real GDP growth.


\subsection{Lagging}

A lagging indicator changes after an event has occurred, and can confirm the
existance of the event.

For example, unemployment.


\subsection{Pro-cyclical}

An indicator that moves in the same direction as the level of economic activity.

For example, real GDP growth.


\subsection{Counter-cyclical}

An indicator that moves in the opposite direction to the level of economic
activity.

For example, unemployment.




\section{Business Cycle}

The business cycle is the fluctuations in economic activity around a long term
trend over a period of time.

The business cycle has 4 phases:

\begin{itemize}
\item Boom
\item Downswing
\item Trough
\item Upswing
\end{itemize}

It is measured through changes in GDP.

Developed economies across the world tend to have roughly synchronised business
cycles, evidencing the interdependence of the world's economies.


\subsection{Graph}

The business cycle can be graphed:

\begin{description}
\item [Y Axis] Level of economic activity
\item [X Axis] Time
\item [Long Term Trend Line] An shallow, upwards sloping line that approximates
	the economic activity line.
\item [Economic Activity Line] A line that fluctuates about the long term trend
	line in regular cycles.
\end{description}

Different parts of the economic activity line are labelled:

\begin{description}
\item [Boom] A upper turning point
\item [Downswing] The downwards line after a boom
\item [Trough] A lower turning point
\item [Upswing] The upwards line after a boom
\end{description}


\subsection{Boom}

A boom is a period of higher than average economic growth.

Also called a peak.

It features:

\begin{itemize}
\item Strong economic growth
\item Economy is at or near full utilisation of resources (high levels of
	output, low cyclical unemployment, high utilisation of physical capital)
\item High level of investment and borrowing
\item Rising prices (inflation)
\item High business and consumer confidence
\item High consumption and spending on durable and luxury goods
\item Potential resource bottlenecks in certain markets
\item High import spending increasing the CAD
\end{itemize}

The last boom in Australia was during 2003 - 2004. GDP growth averaged above
4\%, and unemployment was below 5\%.


\subsection{Downswing}

A downswing is a period of slowing economic growth after a boom.

Also called a contraction.

Downswings tend to last for a shorter period of time than upswings.


\subsubsection{Recession}

A recession is defined as two consecutive quarters of negative economic growth.

It is not the same as a downswing, but effectively is a very severe downswing.


\subsection{Trough}

A trough is a period of lower than average economic growth.

It features (opposite of a boom):

\begin{itemize}
\item Weaker economic growth
\item Economy is below the level of full utilisation of resources (low levels
	of output, high cyclical unemployment, low utilisation of physical capital)
\item Low level of investment and borrowing
\item Reduced upwards pressure on prices (disinflation, or potentially
	deflation)
\item Low business and consumer confidence
\item Low consumption and spending on durable and luxury goods
\item Low import spending reducing the CAD
\end{itemize}


\subsection{Upswing}

An upswing is a period of rising economic growth after a trough.

Also called an expansion.

Upswings tend to last longer than downswings.




\section{Turning Points}

\subsection{Boom}

A boom does not continue forever, but effectively cannibalises itself through
inflation:

\begin{itemize}
\item The level of aggregate demand is at or above the full employment level of
	aggregate demand.
\item Prices rise due to demand pull inflation (as shown on an AD/AS model, with
	an aggregate demand line in the Classical range).
\item Consumers reduce their discretionary spending, causing a fall in
	consumption demand.
\item Sales and income falls for firms, prompting them to reduce investment (by
	the accelerator principle).
\item This fall in investment likely manifests itself in the form of a reduction
	in depreciation investment, where firms no longer replace aging capital
	because there is insufficient consumer demand to justify the purchase.
\item This decrease in investment further reduces income for firms (by the
	reverse multiplier principle).
\item This starts a cycle of falling investment and income, and the economy
	enters a downswing.
\end{itemize}

Inflation is the primary reason for the beginning of a downswing, but there are
other factors.

Other, demand side factors:

\begin{description}
\item [Profit Taking] Investors sell assets towards the end of a boom, reducing
	investment.
\item [Contractionary Government Policy] Little economic stimulus from the
	government since it was unnecessary during the boom.
\item [Contractionary RGA Policy] Rising interest rates discourage domestic
	investment.
\item [Speculative Investment] A shift towards speculative rather than
	productive investment.
\item [Exchange Rate] Appreciation in the exchange rate diminishes international
	competitiveness, reducing net exports after a time lag (according to the
	J-curve theory).
\end{description}

Other, supply side factors:

\begin{description}
\item [Capacity Constraints] Where demand in a market exceeds the available
	supply, causing a bottleneck. Will likely increase cost push inflation.
\end{description}


\subsection{Trough}

A trough does not continue forever:

\begin{itemize}
\item The economy will eventually reach a minimum level of consumption required
	to sustain a satisfactory lifestyle.
\item A minimum level of capital stock is required by firms to sustain this
	minimum level of consumption.
\item Once the level of capital stock falls below this minimum required level,
	firms will undertake depreciation investment, replacing aging capital to
	accomodate the minimum level of consumption.
\item This will increase income for firms (by the multiplier principle), causing
	further investment (by the accelerator principle).
\item This starts a cycle of rising income and investment, and the economy
	enters an upswing.
\end{itemize}

Other, demand side factors:

\begin{description}
\item [Fiscal Policy] Government stimulus packages (increase in government
	spending) or government funded infrastructure projects (increase in
	investment) will, by the multiplier principle, increase aggregate demand.
\item [Automatic Stabilisers] Automatic stabilisers also aid in increasing
	aggregate demand. Government tax receipts decrease due to decreased income,
	reducing the burden of tax on firms and households. Job search allowance
	payments increase.
\item [Expansionary RBA Policy] Low interest rates encourage borrowing, and
	spending financed through credit.
\item [Exchange Rate] Low exchange rate increases international competitiveness,
	improving net exports after a time lag (according to the J-curve theory).
\end{description}

\end{document}
