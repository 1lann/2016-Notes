
\documentclass[a4paper,11pt]{article}

% Math symbols
\usepackage{amsmath}
\usepackage{amsfonts}
\usepackage{esvect}

% Hyperlink contents page
\usepackage{hyperref}
\hypersetup{
	colorlinks,
	citecolor=black,
	filecolor=black,
	linkcolor=black,
	urlcolor=black
}

% AI files
\usepackage{graphicx}
\DeclareGraphicsRule{.ai}{pdf}{.ai}{}

% No indent on new paragraphs
\setlength{\parindent}{0mm}
\setlength{\parskip}{0.3cm}

% Alias \boldsymbol to \bb for vectors
\newcommand{\bb}{\boldsymbol}


\begin{document}

\title{Aggregate Demand and Supply}
\author{Ben Anderson}
\date{\today}
\maketitle
\pagebreak

\tableofcontents
\pagebreak


\section{Aggregate Demand}

The total amount of spending in the economy.

Equivalent to aggregate expenditure.


\subsection{Graph}

The aggregate demand curve can be graphed:

\begin{description}
\item [X Axis] Real GDP
\item [Y Axis] Price level
\item [Gradient] Downwards sloping, straight line
\end{description}


\subsection{Price Level}

Aggregate demand has a negative relationship with price level. As price level
increases, aggregate demand decreases.

There are 3 ways to explain this effect.


\subsubsection{Income Effect}

As price level increases:

\begin{itemize}
\item A consumer's real purchasing power decreases since goods cost more.
\item This reduces consumption.
\end{itemize}


\subsubsection{Interest Rate Effect}

As price level increases:

\begin{itemize}
\item Households and firms demand more credit to finance spending.
\item This places upwards pressure on interest rates to reduce this demand.
\item This increases the cost of borrowing, disincentivising spending.
\item This reduces consumption.
\end{itemize}


\subsubsection{Open Economy Effect}

As price level increases:

\begin{itemize}
\item A country's domestic industries become less competitive when compared
	against other producers in the international market.
\item Demand for our exports from international consumers will fall.
\item Demand for imports from domestic consumers will increase.
\item Net exports will decrease (falling exports and rising imports).
\item This reduces aggregate demand.
\end{itemize}


\subsecion{Factors Affecting}

A change in any factor other than price level will cause a shift in the
aggregate demand curve:

\begin{description}
\item [Increase] Curve shifts to the right.
\item [Decrease] Curve shifts to the left.
\end{description}

Factors that can cause this shift are any of those which affect aggregate
expenditure:

\begin{itemize}
\item Consumption
\item Investment
\item Government spending
\item Net exports
\end{itemize}

For example, the exchange rate and terms of trade will affect net exports,
changing aggregate demand. Interest rates will affect investment and consumption
(through affecting the cost of borrowing).


\subsection{Effects}

An increase in aggregate demand will:

\begin{itemize}
\item Increase real GDP
\item Decrease unemployment
\end{itemize}




\section{Aggregate Supply}

The relationship between the total production of goods and services (GDP) and
the general price level.

The aggregate supply curve can be split into two types.


\subsection{Long Run Aggregate Supply Curve}

Represents the maximum level of output (real GDP) for the economy when fully
utilising all resources.

In reality, this maximum level will not occur when all resources are fully
utilised, but at the natrual rate of utilisation for the economy.  For example,
the natural rate of unemployment in Australia is around 5\%.

Since the maximum level of output does not change with price level, the long run
aggregate supply curve is a straight line:

% TODO: Diagram


\subsubsection{Factors Affecting}

The position of the LRAS depends on the quantity of resources a country has.

Factors which affect this quantity of resources include all those that affect
the quantity of the factors of production available in the economy:

\begin{itemize}
\item Stock of physical capital
\item Labour force productivity
\item Size of the labour force
\item Technological advances
\item Natural disasters
\end{itemize}

An improvement in any of these factors will shift the LRAS to the right.

Over time these factors improve naturally, causing the LRAS to shift to the
right at a rate equal to the growth rate of the labour force and productivity
(roughly 3.5\% annually).


\subsection{Short Run Aggregate Supply Curve}

Represents the relationship between price level and economic growth, where
price level rises as economic growth increases.

A positively sloping line on a graph of price level vs. real GDP:

% TODO: Diagram


\subsubsection{Positive Gradient}

The SRAS is positively sloped because, as output increases (an increase in real
GDP along the X axis), demand pull inflation causes the price of resources
(such as the price of capital equipment or labour wages) to rise.

This increases the cost of production for firms, increasing cost push inflation.

This means that as output rises, the price level also rises.


\subsubsection{Factors Affecting}

Any change in a factor other than price level will cause the SRAS to shift.

Factors which can cause this include all those that affect the quantity of
factors of production available, as well as those that affect the price of
important inputs:

\begin{itemize}
\item Stock of physical capital
\item Labour force productivity
\item Size of the labour force
\item Technological advances
\item Natural disasters
\item Wage rates (the cost of labour)
\item Oil prices (through increasing transporation and fuel costs)
\end{itemize}

Shifts in the aggregate supply curve are referred to as supply shocks.


\subsection{Three Stage Aggregate Supply Curve}

Combines the SRAS and LRAS into a single curve in a graph of price level vs.
real GDP.

The curve is drawn as a flat line, curving upwards into a vertical line:

% TODO: Diagram


\subsubsection{Ranges}

\begin{description}
\item [Keynesian Range] The horizontal part of the line.
\item [Intermediate Range] The part of the line that curves upwards.
\item [Classical Range] The vertical part of the line.
\end{description}

In the Keynesian range:

\begin{itemize}
\item The economy has underutilised resources (capital and labour).
\item An increase in output increasing the utilisation of resources does not
	severely tighten labour and resource markets.
\item Firms will only increase their usage of existing capital, meaning they
	only incur additional variable costs, which are covered by increased sales.
\item This does not drive up the price goods and services.
\item Thus price level increases very little with increased economic growth.
\end{itemize}

In the intermediate range:

\begin{itemize}
\item The economy is approaching its full employment level of output.
\item An increase in output increases demand for labour and other resources.
\item This increases their price (through demand pull inflation).
\item This drives up the price of produced goods (cost push inflation).
\item Causes a rise in the general price level.
\end{itemize}

In the classical range:

\begin{itemize}
\item The economy is at full employment level of output.
\item Increases in economy activity cannot produce any more output (as there is
	insufficient resources to do so).
\item But, an increase in economic activity drives up demand for goods and
	services in the economy, raising their price.
\item This increases the price level.
\end{itemize}




\section{Macroeconomic Equilibrium}

We can combine the aggregate supply and aggregate demand curves on one graph to
show macroeconomic equilibrium.


\subsection{Short Run Equilibrium}

% TODO: Diagram

An economy is in short run equilibrium at the point where the aggregate demand
curve intersects the short run aggregate supply curve.

Short run equilibrium is the economy's actual level of production.


\subsection{Long Run Equilibrium}

% TODO: Diagram

An economy is in long run equilibrium when the aggregate demand, short
run, and long run aggregate supply curves all intersect at the same point.

This is where the level of output is equal to full employment GDP, and the
economy is at its natural rate of unemployment with a relatively low level of
inflation.

This is the most optimal position for the economy.


\subsection{Changes in Equilibrium}

An increase in aggregate demand (in the intermediate range):

\begin{itemize}
\item Increase in economic growth (advantage)
\item Increase in price level (disadvantage)
\end{itemize}

A decrease in aggregate demand (demand shock):

\begin{itemize}
\item Decrease in economic growth (disadvantage)
\item Decrease in price level (advantage)
\end{itemize}

The increase in economic growth that occurs when aggregate demand rises is far
more beneficial than the fall in the price level that occurs during a decrease
in aggregate demand.

An increase in aggregate supply:

\begin{itemize}
\item Increase in economic growth (advantage)
\item Decrease in price level (advantage)
\end{itemize}

This is the best outcome for the economy, but the most difficult to achieve, as
firms act in their best interest and do not create new productive capacity when
it is unneeded. More readily achieved through microeconomic reform or
improvements to technology.

A decrease in aggregate supply (supply shock):

\begin{itemize}
\item Decrease in economic growth (disadvantage)
\item Increase in price level (disadvantage)
\end{itemize}




\section{Expansionary and Contractionary Gaps}

\subsection{Expansionary Gap}

Equilibrium level of real GDP is above that of potential GDP.

Also called an inflationary gap.

Features of an economy in an expansionary gap include:

\begin{itemize}
\item Low cyclical unemployment (below the natural rate)
\item Infrastructure bottlenecks and skill shortages
\item High inflation
\item High consumer and business confidence
\item High interest rates (due to increased demand for funds)
\item High import spending increasing the CAD
\item Decreased government spending on welfare and other forms of assistance
\end{itemize}


\subsection{Contractionary Gap}

Equilibrium level of real GDP is less than the potential level of GDP. Occurs
when the economy has excess capacity.

Also called a deflationary gap, or a recessionary gap if relating to a
recession.

Features of an economy in a contractionary gap include:

\begin{itemize}
\item High cyclical unemployment (above the natural rate)
\item Low inflation
\item Low consumer and business confidence
\item Low interest rates (due to increased demand for funds)
\item Low import spending reducing the CAD
\item Increased government spending on welfare and other forms of assistance
\end{itemize}


\subsection{Aggregate Expenditure Model}

% TODO: Diagram

We can show expansionary and contractionary gaps using an aggregate expenditure
model:

\begin{itemize}
\item Two aggregate expenditure lines are drawn.
\item The equilibrium of one is labelled as full employment output along the X
	axis.
\item The equilibrium of the other is labelled as actual output.
\item If actual output is above full employment output, this is an expansionary
	gap. The gap between the two levels of output is the expansionary gap.
\item If actual output is below full employment output, this is a contractionary
	gap. The gap between the two levels of output is the contractionary gap.
\end{itemize}


\subsection{AD/AS Model}

% TODO: Diagram

We can show expansionary and contractionary gaps using an AD/AS model with
separate long and short run aggregate supply curves:

\begin{itemize}
\item If short run equilibrium is to the left of the LRAS, this is a
	contractionary gap.
\item If short run equilibrium is to the right of the LRAS, this is an
	expansionary gap.
\item The actual level of output in this model will be the point of intersection
	between the LRAS and the AD curve (ie. full employment GDP), not the level
	of output at shot run equilibrium.
\end{itemize}


\subsection{Three Stage Aggregate Supply Curve}

% TODO: Diagram

An expansionary gap cannot be shown on an AD/AS model using a three stage
aggregate supply curve.

A contractionary gap is shown whenever the aggregate demand curve doesn't lie on
the classical range. The horizontal distance between actual output and full
employment output is the contractionary gap.


\subsection{Correction Mechanism}

When the economy is in an expansionary gap:

\begin{itemize}
\item Tight resource and labour markets drive up the cost of resources.
\item Both demand pull and cost push inflation drive up the prices of goods and
	services, reducing aggregate demand.
\item An increase in wage rates also shifts the short run aggregate supply curve
	to the left.
\item This moves the economy back towards its long run equilibrium.
\end{itemize}

When the economy is in a contractionary gap:

\begin{itemize}
\item Wages and the cost of resources fall, reducing the price level due to
	reduced cost push inflation.
\item This shifts the short run aggregate supply curve to the right.
\item Aggregate demand may also increase from improving business and consumer
	confidence.
\end{itemize}




\section{Stagflation}

A simultaneous increase in the price level (inflation) and unemployment.

The result of a supply shock, which reduces aggregate supply.


\subsection{Causes}

Potential causes for stagflation include all those that could reduce aggregate
supply, such as:

\begin{itemize}
\item Natural disasters
\item Rising world oil prices
\end{itemize}




\section{Business Cycle}

The AD/AS model can demonstrate the phases of the business cycle.

During different stages of the business cycle, the aggregate demand curve will
intersect different ranges of a three stage aggregate supply curve.


\subsection{Boom}




\subsection{Aggregate Demand}

In a boom, the level of aggregate demand in the economy is at or above the full
employment level of aggregate demand.

\end{document}
