
\documentclass[a4paper,11pt]{article}

% Math symbols
\usepackage{amsmath}
\usepackage{amsfonts}
\usepackage{esvect}

% Hyperlink contents page
\usepackage{hyperref}
\hypersetup{
	colorlinks,
	citecolor=black,
	filecolor=black,
	linkcolor=black,
	urlcolor=black
}

% AI files
\usepackage{graphicx}
\DeclareGraphicsRule{.ai}{pdf}{.ai}{}

% No indent on new paragraphs
\setlength{\parindent}{0mm}
\setlength{\parskip}{0.3cm}

% Alias \boldsymbol to \bb for vectors
\newcommand{\bb}{\boldsymbol}


\begin{document}

\title{Aggregate Demand and Supply}
\author{Ben Anderson}
\date{\today}
\maketitle
\pagebreak

\tableofcontents
\pagebreak


\section{Aggregate Demand}

The total amount of spending in the economy.

Equivalent to aggregate expenditure.


\subsection{Graph}

The aggregate demand curve can be graphed:

\begin{description}
\item [X Axis] Real GDP
\item [Y Axis] Price level
\item [Gradient] Downwards sloping, straight line
\end{description}


\subsection{Price Level}

Aggregate demand has a negative relationship with price level. As price level
increases, aggregate demand decreases.

There are 3 ways to explain this effect.


\subsubsection{Income Effect}

As price level increases:

\begin{itemize}
\item A consumer's real purchasing power decreases since goods cost more.
\item This reduces consumption.
\end{itemize}


\subsubsection{Interest Rate Effect}

As price level increases:

\begin{itemize}
\item Households and firms demand more funds to finance spending, increasing
	demand for credit.
\item Places upwards pressure on interest rates to reduce this demand.
\item Increases the cost of borrowing, disincentivising spending and investment.
\item Reduces aggregate demand.
\end{itemize}


\subsubsection{Open Economy Effect}

As price level increases:

\begin{itemize}
\item A country's domestic industries become less competitive when compared
	against other producers in the international market.
\item Demand for our exports from international consumers will fall.
\item Demand for imports from domestic consumers will increase.
\item Net exports will decrease (falling exports and rising imports).
\item Reduces aggregate demand.
\end{itemize}


\subsection{Factors Affecting}

A change in any factor other than price level will cause a shift in the
aggregate demand curve:

\begin{description}
\item [Increase] Curve shifts to the right.
\item [Decrease] Curve shifts to the left.
\end{description}

Factors that can cause this shift are any of those which affect aggregate
expenditure:

\begin{itemize}
\item Consumption
\item Investment
\item Government spending
\item Net exports
\end{itemize}

For example, the exchange rate and terms of trade will affect net exports,
changing aggregate demand. Interest rates will affect investment and consumption
(through affecting the cost of borrowing).


\subsection{Effects}

An increase in aggregate demand will:

\begin{itemize}
\item Increase real GDP
\item Decrease unemployment
\end{itemize}




\section{Aggregate Supply}

The total supply of all goods and services an economy produces within a given
period of time.


\subsection{Long Run Aggregate Supply Curve}

Represents the maximum level of output (real GDP) when an economy fully utilises
all of its resources.

Since the maximum level of output does not change with price level, the long run
aggregate supply curve is a straight line:

% TODO: Diagram


\subsubsection{Natural Rate}

In reality, this maximum level will not occur when all resources are fully
utilised, but at the natrual rate of utilisation for the economy.  For example,
the natural rate of unemployment in Australia is around 5\%.

The natural rate of unemployment is also called the frictional and structural
rate of unemployment.


\subsubsection{Factors Affecting}

The position of the LRAS depends on the quantity of resources a country has.

Factors which affect this quantity of resources include all those that affect
the quantity of the factors of production available in the economy:

\begin{itemize}
\item Stock of physical capital
\item Productivity rate
\item Technological advances
\item Size of the labour force
\item Natural disasters
\end{itemize}

An improvement in any of these factors will shift the LRAS to the right.

Over time these factors improve naturally, causing the LRAS to shift to the
right at a rate equal to the growth rate of the labour force and productivity
(roughly 3.5\% annually).


\subsection{Short Run Aggregate Supply Curve}

Represents the relationship between price level and economic growth, where
price level rises as economic growth increases.

A positively sloping line on a graph of price level vs. real GDP:

% TODO: Diagram


\subsubsection{Positive Gradient}

The SRAS is positively sloped for 2 reasons.

% TODO: I made up the name cost effect. What is it really?

The first is the cost effect:

\begin{itemize}
\item As output increases (an increase in real GDP along the X axis).
\item Demand pull inflation causes the price of resources (such as the price of
	capital equipment or labour wages) to rise.
\item This increases the cost of production for firms, increasing cost push
	inflation.
\item This means that as output rises, the price level also rises.
\end{itemize}

The second is profit motivation:

\begin{itemize}
\item As price level rises, the level of output will rise.
\item Since higher prices for producers allow them to make larger profits.
\end{itemize}


\subsubsection{Factors Affecting}

Any change in a factor other than price level will cause the SRAS to shift.

Factors which can cause this include all those that affect the quantity of
factors of production available, as well as those that affect the price of
important inputs:

\begin{itemize}
\item Stock of physical capital
\item Labour force productivity
\item Size of the labour force
\item Technological advances
\item Natural disasters
\item Wage rates (the cost of labour)
\item Oil prices (through increasing transporation and fuel costs)
\item Company tax rates
\end{itemize}

Shifts in the aggregate supply curve are referred to as supply shocks.


\subsection{Three Stage Aggregate Supply Curve}

Combines the SRAS and LRAS into a single curve in a graph of price level vs.
real GDP.

The curve is drawn as a flat line, curving upwards into a vertical line:

% TODO: Diagram


\subsubsection{Ranges}

\begin{description}
\item [Keynesian Range] The horizontal part of the line.
\item [Intermediate Range] The part of the line that curves upwards.
\item [Classical Range] The vertical part of the line.
\end{description}

In the Keynesian range:

\begin{itemize}
\item The economy has underutilised resources (capital and labour).
\item An increase in output increasing the utilisation of resources does not
	severely tighten labour and resource markets.
\item Firms will only increase their usage of existing capital, meaning they
	only incur additional variable costs, which are covered by increased sales.
\item This does not drive up the price goods and services.
\item Thus price level increases very little with increased economic growth.
\end{itemize}

In the intermediate range:

\begin{itemize}
\item The economy is approaching its full employment level of output.
\item An increase in output increases demand for labour and other resources.
\item This increases their price (through demand pull inflation).
\item This drives up the price of produced goods (cost push inflation).
\item Causes a rise in the general price level.
\end{itemize}

In the classical range:

\begin{itemize}
\item The economy is at full employment level of output.
\item Increases in economy activity cannot produce any more output (as there is
	insufficient resources to do so).
\item But, an increase in economic activity drives up demand for goods and
	services in the economy, raising their price.
\item This increases the price level.
\end{itemize}




\section{Macroeconomic Equilibrium}

Macroeconomic equilibrium is where the quantity of aggregate demand equals the
quantity of aggregate supply in an economy.

We can combine the aggregate supply and aggregate demand curves on one graph to
show macroeconomic equilibrium.


\subsection{Short Run Equilibrium}

% TODO: Diagram

An economy is in short run equilibrium at the point where the aggregate demand
curve intersects the short run aggregate supply curve.

Short run equilibrium is the economy's actual level of production.


\subsection{Long Run Equilibrium}

% TODO: Diagram

An economy is in long run equilibrium when the aggregate demand, short
run, and long run aggregate supply curves all intersect at the same point.

This is where the level of output is equal to full employment GDP, and the
economy is at its natural rate of unemployment with a relatively low level of
inflation.

This is the most optimal position for the economy.


\subsection{Effects}

A movement in either of the aggregate supply or demand curves will shift
macroeconomic equilibrium.

An increase in aggregate demand, when the demand curve is in the intermediate
range:

\begin{itemize}
\item Increase in economic growth (advantage)
\item Increase in price level (disadvantage)
\end{itemize}

A decrease in aggregate demand (demand shock):

\begin{itemize}
\item Decrease in economic growth (disadvantage)
\item Decrease in price level (advantage)
\end{itemize}

The increase in economic growth that occurs when aggregate demand rises is far
more beneficial than the fall in price level from a decrease in aggregate
demand.

An increase in aggregate supply:

\begin{itemize}
\item Increase in economic growth (advantage)
\item Decrease in price level (advantage)
\end{itemize}

This is the best outcome for the economy, but the most difficult to achieve, as
firms act in their best interest and do not create new productive capacity when
it is unneeded.

This can be more readily achieved through microeconomic reform or improvements
to technology.

A decrease in aggregate supply (supply shock):

\begin{itemize}
\item Decrease in economic growth (disadvantage)
\item Increase in price level (disadvantage)
\end{itemize}


\subsection{Water Tank Analogy}

When stuck with an AD/AS question, consider the water tank analogy:

\begin{itemize}
\item The tank itself is aggregate supply.
\item The water level within the tank is aggregate demand.
\item Changes to aggregate demand or supply will change the size of the tank, or
	the water level within it.
\end{itemize}




\section{Expansionary and Contractionary Gaps}

\subsection{Expansionary Gap}

Equilibrium level of real GDP is above that of potential GDP.

Also called an inflationary gap, or positive GDP gap.

The term inflationary gap usually signifies the use of an aggregate expenditure
model, whereas expansionary gap implies an AD/AS model.

Features of an economy in an expansionary gap include:

\begin{itemize}
\item Low cyclical unemployment (below the natural rate)
\item Infrastructure bottlenecks and skill shortages
\item High inflation
\item High consumer and business confidence
\item High interest rates (due to increased demand for funds)
\item High import spending increasing the CAD
\item Decreased government spending on welfare and other forms of assistance
\end{itemize}


\subsection{Contractionary Gap}

Equilibrium level of real GDP is less than the potential level of GDP, where
aggregate demand is insufficient for the economy to fully utilise its resources.

Also called a deflationary gap, a recessionary gap, a negative GDP gap, or
excess capacity.

The term deflationary gap usually signifies the use of an aggregate expenditure
model, whereas contractionary gap implies an AD/AS model.

Features of an economy in a contractionary gap include:

\begin{itemize}
\item High cyclical unemployment (above the natural rate)
\item Low inflation
\item Low consumer and business confidence
\item Low interest rates (due to increased demand for funds)
\item Low import spending reducing the CAD
\item Increased government spending on welfare and other forms of assistance
\end{itemize}


\subsection{Leakages and Injections}

When leakages in the economy are greater than injections:

\begin{itemize}
\item Level of output is greater than the level of consumption.
\item Inventories will rise, prompting firms to cut output.
\item Reduces economic growth.
\item An expansionary gap (to the right of full employment output in an
	aggregate expenditure model).
\end{itemize}

When injections are greater than leakages in the economy:

\begin{itemize}
\item Level of output is less than the desired level of consumption (aggregate
	demand).
\item Inventories will fall, prompting firms to increase output.
\item Increases economic growth.
\item A contractionary gap (to the left of full employment output in an
	aggregate expenditure model).
\end{itemize}


\subsection{Aggregate Expenditure Model}

% TODO: Diagram

We can show expansionary and contractionary gaps using an aggregate expenditure
model:

\begin{itemize}
\item Two aggregate expenditure lines are drawn.
\item The equilibrium of one is labelled as full employment output along the X
	axis.
\item The equilibrium of the other is labelled as actual output ($Y_1$).
\item If actual output is beyond full employment output, this is an expansionary
	gap.
\item If actual output is below full employment output, this is a contractionary
	gap.
\item On the graph, the horizontal distance between actual and full employment
	output is labelled as the expansionary or contractionary gap.
\end{itemize}

To correct an expansionary or contractionary gap, a change in the MPC (changing
the gradient and thus closing the gap) is not sufficient to correct the gap.

An autonomous increase or decrease in consumption, investment, or government
spending it required to correct the gap.


\subsection{AD/AS Model}

% TODO: Diagram

We can show expansionary and contractionary gaps using an AD/AS model with
separate long and short run aggregate supply curves:

\begin{itemize}
\item If short run equilibrium is to the left of the LRAS, this is a
	contractionary gap.
\item If short run equilibrium is to the right of the LRAS, this is an
	expansionary gap.
\item The actual level of output in this model will be the point of intersection
	between the LRAS and the AD curve (ie. full employment GDP), not the level
	of output at shot run equilibrium.
\item On the graph, the horizontal distance between short run equilibrium and
	the LRAS is labelled as the expansionary or contractionary gap.
\end{itemize}


\subsection{Three Stage Aggregate Supply Curve}

% TODO: Diagram

An expansionary gap cannot be shown on an AD/AS model using a three stage
aggregate supply curve.

A contractionary gap is shown whenever the aggregate demand curve doesn't lie on
the classical range.

The horizontal distance between actual output and full employment output is the
contractionary gap.


\subsection{Correction Mechanism}

When the economy is in an expansionary gap:

\begin{itemize}
\item Tight resource and labour markets drive up the cost of resources like
	labour, causing cost push inflation for firms.
\item High levels of aggregate demand cause demand pull inflation.
\item These drive up the price of goods and services in the economy.
\item This decreases demand for these goods, reducing aggregate demand.
\item Rising wage rates shift the SRAS to the left.
\item This moves the economy back towards full employment equilibrium
\end{itemize}

When the economy is in a contractionary gap:

\begin{itemize}
\item Decreased demand for labour and resources decrease wage rates and the
	price of other resources, decreasing cost push inflation.
\item Lower levels of aggregate demand also reduce demand pull inflation.
\item This reduces the price level.
\item This increases demand for goods, increasing aggregate demand.
\item The reduction in wage rates shifts the SRAS to the right.
\item This moves the economy back towards full employment equilibrium.
\end{itemize}

Imports cannot be used to correct expansionary or contractionary gaps, since
aggregate demand represents only domestic demand, and imports represent money
leaving the economy.




\section{Stagflation}

A simultaneous increase in the level of inflation and unemployment.

The term is derived from a stagnant economy, experiencing inflation.

The result of a supply shock, reducing aggregate supply.

No policy solution to aid fixing the situation. We usually just have to wait
until the supply shock can be rectified (eg. for a drought to end).


\subsection{Causes}

Potential causes for stagflation include all those that could reduce aggregate
supply, such as:

\begin{itemize}
\item Natural disasters
\item Rising world oil prices
\item Rising wage rates
\item Decreased labour productivity
\end{itemize}




\section{Business Cycle}

The AD/AS model can demonstrate the phases of the business cycle.

During different stages of the business cycle, the aggregate demand curve will
intersect different ranges of a three stage aggregate supply curve.


\subsection{Boom}

In a boom, actual output lies at or above the full employment level of output.

The aggregate demand curve will intersect the aggregate supply curve in the
Classical range, either at long run equilibrium or beyond it, resulting in an
expansionary gap.

This demonstrates high economic growth and low unemployment.


\subsection{Downswing}

For an economy in long run equilibrium, a downswing (reduced economic growth)
can be caused by:

\begin{itemize}
\item Reduced aggregate demand
\item Reduced aggregate supply
\end{itemize}


\subsection{Trough}

In a trough, actual output lies below the level of full employment output.

The aggregate demand curve will intersect the aggregate supply curve in the
intermediate or Keynesian range, below long run equilibrium, resulting in a
contractionary gap.

This demonstrates low economic growth and high unemployment.


\subsection{Upswing}

For an economy in a trough, an upswing (increased economic growth) can be
caused by:

\begin{itemize}
\item Increased aggregate demand
\item Increased aggregate supply
\end{itemize}

\end{document}
