
\documentclass[a4paper,11pt]{article}

% Math symbols
\usepackage{amsmath}
\usepackage{amsfonts}
\usepackage{esvect}

% Hyperlink contents page
\usepackage{hyperref}
\hypersetup{
	colorlinks,
	citecolor=black,
	filecolor=black,
	linkcolor=black,
	urlcolor=black
}

% AI files
\usepackage{graphicx}
\DeclareGraphicsRule{.ai}{pdf}{.ai}{}

% No indent on new paragraphs
\setlength{\parindent}{0mm}
\setlength{\parskip}{0.3cm}

% Alias \boldsymbol to \bb for vectors
\newcommand{\bb}{\boldsymbol}


\begin{document}

\title{Terms of Trade}
\author{Ben Anderson}
\date{\today}
\maketitle
\pagebreak

\tableofcontents
\pagebreak


\section{Import and Export Price Index}

A weighted average of the prices of a representative group of imports or
exports for a country in a particular year.

All prices are in Australian dollar terms.


\subsection{Abbreviations}

\begin{description}
\item [XPI] Export price index.
\item [MPI] Import price index.
\end{description}


\subsection{Significance}

The value of the index is insignificant (whether it is above or below 100).

Whether there has been a change in the index from year to year is important.


\subsection{Control}

These indices are set by world markets.

Australia has little direct influence over their value.




\section{Terms of Trade}

An index that measures the relative movements in the price of exports and
imports for a country (weighted according to their importance to the country).

The quantity of imports a country can buy given a quantity of exports sold.

Gives no indication of volume or value of exports or imports traded.


\subsection{Formula}

$$
\frac{\text{XPI}}{\text{MPI}} \times 100
$$

Export price divided by import price multiplied by 100.


\subsection{Trends}

Large favourable movement (57 to 107) throughout mining boom (2011 to 2011).

Sharp unfavourable movement during GFC (2008 to 2009), but quick recovery.

Unfavourable movement since 2012.


\subsubsection{Mining Boom}

Caused by large growth in emerging markets like China and India, particularly
in their manufacturing sectors.

Lead to large global demand for commodities, increasing their prices,
increasing our XPI.

Cheap labour force and improvements in efficiency by China lead to decrease in
price for manufactures, decreasing our MPI.

Lead to large favourable movement in Australia's terms of trade.




\section{Movements}

Actual value of terms of trade insignificant.


\subsection{Favourable Movement}

Increase in the terms of trade between two years.

Caused by:

\begin{itemize}
\item XPI rise, MPI fall
\item XPI rise, MPI constant
\item XPI rise, MPI rise by less
\item XPI constant, MPI fall
\item XPI fall, MPI fall by more
\end{itemize}


\subsection{Unfavourable Movement}

Decrease in the terms of trade between two years.

Caused by:

\begin{itemize}
\item XPI fall, MPI rise
\item XPI fall, MPI constant
\item XPI fall, MPI fall by less
\item XPI constant, MPI rise
\item XPI rise, MPI rise by more
\end{itemize}




\section{Elasticity}

How price elastic a country's exports and imports are determines the
significance of the terms of trade to its economy.


\subsection{Inelastic}

If a country's exports and imports are price inelastic (like Australia):

\begin{itemize}
\item A favourable or unfavourable movement in the terms of trade will not
	affect volume of imports and exports traded.
\item A favourable movement will increase export income.
\item An unfavourable movement will decrease export income.
\end{itemize}


\subsection{Elastic}

If a country's exports and imports are highly price elastic:

\begin{itemize}
\item A favourable movement will decrease volume of exports by more than the
	rise in price compensates for.
\item Will result in a decrease in export income.
\end{itemize}

For an unfavourable movement:

\begin{itemize}
\item An unfavourable movement will increase volume of exports by more than the
	fall in price reduces income by.
\item Will result in an increase in export income.
\end{itemize}




\section{Factors Affecting Export Price Index}

Australia's XPI has (on average) fallen since 2012.


\subsection{World Growth}

\begin{itemize}
\item China is our largest trading partner.
\item China has had slowing growth over the past four years (10\% growth rate
	on average over the decade, predicted 6 - 7\% for 2016).
\item Decreases demand for our exports.
\item Decreases the price of our exports (iron ore went from \$160 AUD a tonne
	at the height of the boom to \$40 AUD a tonne recently in 2016).
\end{itemize}


\subsection{Productive Capacity}

\begin{itemize}
\item The commodities price boom caused other countries to invest heavily in
	their mining sectors, increasing their productive capacity.
\item Caused a surplus of base metals in the world market, reducing their
	price.
\item Australia's largest exports are raw commodities (iron ore, coal, and
	natural gas make up 39.5\% of our exports).
\item Reduced the price of our exports.
\end{itemize}


\subsection{Diversification}

\begin{itemize}
\item Australia has little diversity in its major exports, predominantly
	exporting raw minerals and agricultural products.
\item This is a consequence of a decade long mining boom and sustained heavy
	investment in the mining sector.
\item Amplifies the effect of falling commodity prices on our XPI.
\end{itemize}




\section{Factors Affecting Import Price Index}

Australia's MPI has (on average) slightly risen since 2012.


\subsection{Manufactures}

\begin{itemize}
\item Australia predominantly imports manufactured goods because of our
	competitive disadvantage in the industry.
\item The emergence of China as a cheap manufacturing powerhouse 15 - 20 years
	ago dramatically reduced prices of imported manufactures, decreasing our
	MPI.
\item This is no longer a new phenomenon in the market, remaining fairly stable
	the past few years.
\item No longer causing a decrease in our MPI.
\end{itemize}


\subsection{Depreciation of Exchange Rate}

Australia has seen a depreciation in its exchange rate since 2012:

\begin{itemize}
\item Increased the price of our imports in Australian dollar terms.
\item Increased our MPI.
\end{itemize}




\section{Effects}

The effects of an unfavourable movement in the terms of trade on our economy.


\subsection{Balance of Goods and Services}

For a decrease in the XPI:

\begin{itemize}
\item Decrease in export income due to inelastic nature of Australia's exports.
\item Decreases balance of goods and services.
\end{itemize}


\subsubsection{Extenuating Circumstances}

In theory, there should always be a direct, positive relationship between the
terms of trade and balance of goods and services, given the inelasticity of
Australia's exports.

Circumstances where this doesn't hold:

\begin{description}
\item [Natural Disasters] Causes a severe reduction in Australian export
	volumes, despite an increase in price on the world market.
\item [Export Capacity] If there is a cap on volume exported (eg. limited
	productive capacity), while also low import prices. May import much larger
	volume than we export leading to a trade deficit, while also having a
	favourable terms of trade movement.
\end{description}


\subsection{Current Account Deficit}

There is no direct link between the terms of trade and the income categories
of the current account.

Thus an unfavourable movement increases the current account deficit by
decreasing the balance of goods and services.


\subsection{Exchange Rate}

For a decrease in the XPI:

\begin{itemize}
\item Decreases export income due to inelastic nature of Australian imports.
\item Decreases demand for the \$AUD, depreciating the dollar.
\end{itemize}


\subsection{Inflation}

For a decrease in the XPI:

\begin{itemize}
\item Falling export income reduces aggregate demand.
\item Reduces demand pull inflation.
\end{itemize}

But, for an increase in the MPI:

\begin{itemize}
\item Increases cost of imported intermediate goods for domestic producers.
\item Increases cost push inflation.
\item Further compounded by depreciating dollar increasing the price of
	imports.
\end{itemize}


\subsection{Domestic Economic Growth}

For a decrease in the XPI:

\begin{itemize}
\item Decreased export income reduces aggregate demand.
\item Decreases domestic economic growth.
\end{itemize}

For an increase in the MPI:

\begin{itemize}
\item Bestows a competitive advantage on import competing industries over
	imported goods.
\item Increases demand for products produced by these industries, increasing
	their growth.
\item May partially offset the fall in economic growth.
\end{itemize}


\subsection{Unemployment}

Due to reduced economic growth:

\begin{itemize}
\item Reduced derived demand for labour.
\item Increased unemployment.
\item May be offset by an increase in growth and employment from import
	competing industries.
\item Strutural unemployment may occur in industries the falling XPI has
	impacted the most.
\end{itemize}


\subsection{Purchasing Power}

An unfavourable movement in the terms of trade:

\begin{itemize}
\item Reduces the volume of imports that can be bought given a volume of
	exports.
\item Decreases a nation's purchasing power.
\end{itemize}

\end{document}
