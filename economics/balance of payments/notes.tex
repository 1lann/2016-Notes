
\documentclass[a4paper,11pt]{article}

% Math symbols
\usepackage{amsmath}
\usepackage{amsfonts}
\usepackage{esvect}

% Hyperlink contents page
\usepackage{hyperref}
\hypersetup{
	colorlinks,
	citecolor=black,
	filecolor=black,
	linkcolor=black,
	urlcolor=black
}

% AI files
\usepackage{graphicx}
\DeclareGraphicsRule{.ai}{pdf}{.ai}{}

% No indent on new paragraphs
\setlength{\parindent}{0mm}
\setlength{\parskip}{0.3cm}

% Alias \boldsymbol to \bb for vectors
\newcommand{\bb}{\boldsymbol}


\begin{document}

\title{Balance of Payments}
\author{Ben Anderson}
\date{\today}
\maketitle
\pagebreak

\tableofcontents
\pagebreak


\section{Balance of Payments}

A systematic record of financial transactions between Australia and the rest of
the world.

Transactions that occur over the period of the last year.


\subsection{Credit}

A transaction that causes an inflow of money into Australia.

Recorded as a positive value.


\subsection{Debit}

A transaction that causes an outflow of money from Australia.

Recorded as a negative value.


\subsection{Balance}

All credits plus debits (where debits are negative).


\subsection{Surplus}

A positive balance.

Value of credits is greater than value of debits.


\subsection{Deficit}

A negative balance.

Value of debits is greater than value of credits.



\section{Current Account}

\subsection{Goods}

Contains all financial transactions that result from the sale or purchase of
goods.

All transactions associated with the export and import of goods.


\subsubsection{Size}

Largest category of current account.


\subsubsection{Nature}

Volatile in value.

Fluctuations caused by:

\begin{description}
\item [Marginal Propensity to Import] Australians have a high marginal
	propensity to import. Value of imports is primarily a function of economic
	growth.
\item [Exogenous Events] Natural disasters, terrorism, the exchange rate all
	affect the value of goods we export.
\end{description}


\subsection{Services}

Contains all financial transactions that result from the sale or purchase of
services.

All transactions associated with the export and import of services.


\subsubsection{Size}

About 25\% the size of goods.


\subsubsection{Exporting and Importing}

For example, Australia's niche market in education:

\begin{description}
\item [Exporting] International students paying for tutition in Australian
	universities.
\item [Importing] Australian students paying for tutition in international
	universities.
\end{description}


\subsubsection{Nature}

Relatively stable.

Trending towards deficit from high exchange rate after mining boom, reducing
our international competitiveness.


\subsection{Primary Income}

Payments incurred for the use of another country's capital.


\subsubsection{Size}

About 25\% the size of goods.


\subsubsection{Examples}

\begin{description}
\item [Interest Payments] Interest payments on foreign liabilities.
\item [Dividend Payments] Dividend payments for stocks.
\item [Capital Gains] Money earned through the sale of an asset at a higher
	price than what it was bought for.
\item [Profit Flows] Profits earned by foreign subsidiaries paid to the parent
	company in a different country.
\item [Labour Costs] Payments for the use of another country's labour. For
	example, salary payments to foreign residents working in Australia.
\end{description}


\subsubsection{Nature}

Large deficit, main cause of current account deficit.


\subsection{Secondary Income}

Contains one-sided transactions in which nothing of economic value is received
in return for a payment.


\subsubsection{Size}

Smallest section of the current account.


\subsubsection{Examples}

\begin{description}
\item [Emergency Relief] Capital paid to foreign countries for emergency
	relief. For example, tornado relief funds.
\item [Gifts and Donations] Money donated to organisations in foreign countries.
\item [Pensions] International citizens receiving pensions from their foreign
	government.
\end{description}


\subsection{Balance of Merchandise Trade}

All credits and debits in the goods category added together.

Exports of goods subtract imports of goods.


\subsection{Balance of Goods and Services}

Also called balance of trade.

All credits and debits in the goods and services categories added together.

Exports of goods and services subtract imports of goods and services.


\subsection{Net Invisible Trade}

Balance of services and income. Excludes goods.




\section{Capital and Financial Account}

Contains all financial transactions that are likely to result in future
financial flows in forthcoming accounting periods.


\subsection{Capital Account}

Contains transactions which are not commercial (not motivated by profit).


\subsubsection{Size}

Relatively small.


\subsubsection{Non-Producible Non-Financial Assets}

Transactions of intellectual property. Includes copyright, patents, trademarks,
and franchises.


\subsubsection{Examples}

\begin{itemize}
\item Transfers of intellectual property.
\item Debt forgiveness.
\item Foreign aid for fixed capital formation (eg. infrastructure construction).
\item Capital transferred by migrants.
\end{itemize}


\subsection{Financial Account}

Contains all transactions pertaining to foreign investment into Australia or
abroad.


\subsubsection{Size}

Largest section of capital and financial account.


\subsection{Capital and Financial Account Balance}

Balance on capital account added to the balance on the financial account.




\section{Balance}

A country's balance of payments always sums to zero under a floating exchange
rate.


\subsection{Net Errors and Omissions}

A balancing item added to ensure the balance of payments does indeed balance.


\subsection{Reason}

The demand for a floating currency represents credits in the balance of
payments.

The supply of a currency represents debits.

Since a floating exchange rate will always move to a new market equilibrium,
the demand for the currency (credits) will equal the supply of it (credits).



\section{Types of Effects}

\subsection{Structural}

Inherent features of the economy which cause consistent and long lasting
effects.


\subsection{Cyclical}

Effects that vary over time with the state of the economy.




\section{Investment and Savings Imbalance}

This is the predominant structural cause for the current account deficit and
capital and financial account surplus.

Australia has a fundamental imbalance in the levels of national saving and
investment:

\begin{itemize}
\item Small population results in a low level of national savings.
\item Many investment opportunities in the economy leads to a high level of
	investment.
\item The additional funds to sustain this investment and savings imbalance
	comes from the use of other country's financial capital by creating a
	foreign liability.
\end{itemize}


\subsection{Capital and Financial Account Surplus}

The creation of a foreign liability is represented by a credit in the financial
account.

Results in a higher value of credits than debits in the capital and financial
account, forming a surplus.


\subsection{Current Account Deficit}

Foreign investors require a return on their investment through dividend
payments, profit flows, interest repayments, etc.

Represented by a debit in the primary income category of the current account.

These repayments are usually stable and long term, causing little volatility
in the deficit on the primary incomes category.

Causes a consistent deficit on the current account.




\section{Cyclical Effects in Primary Income}

\subsection{Valuation Effect}

Roughly half of Australia's foreign debt is denominated in overseas currency.

When the \$AUD appreciates, the value of this debt in Australian dollars is
immediately less.

When the \$AUD depreciates, the value of this debt in Australian dollars is
immediately more.

This will increase or decrease the value of interest payments on the debt.


\subsection{Interest Rates}

A higher interest rate in an overseas country will result in larger interest
repayments on foreign debt.


\subsection{Profitability of Australian Businesses}

Affects size of dividend payments and profit flows to foreign investors.




\section{Cyclical Effects in Goods and Services}

Balance of goods and services is consistently a deficit, but very volatile.


\subsection{Structural Deficit}

Australia has a high marginal propensity to import, meaning the value of
imports is predominantly a function of domestic growth.

The price of our exports is determined in the world market (where Australia has
little individual influence).

Australia has had, on average, stronger domestic growth than the world average
over the past few years.

Caused the value of our imports to be larger than that of exports, causing a
goods and services deficit.


\subsection{Cyclical Volatility}

Australia's largest exports are all commodities (raw minerals and agricultural
products), with few substitutes.

Thus, demand for our exports is relatively inelastic.

Construction of new infrastructure to increase supply in response to any
increases in demand takes a vast amount of resources and time.

Thus, supply of our exports is relatively inelastic.

Any small movements in the supply or demand curves for our exports results in
a large change in price.

Causes significant volatility in the value of goods exported.


\subsection{Other Factors}

Other factors that have an effect on the balance of goods and services include:

\begin{description}
\item [Terms of Trade] Affects the price of our exports and imports.
\item [Exchange Rate] Affects the price of our exports relative to other
	countries, affecting our international competitiveness. Affects the demand
	for imports through price.
\item [International Competitiveness] Affects demand for our exports.
\end{description}




\section{Benefits of a Current Account Deficit}

The current account deficit is caused by the investment and savings imbalance.

We are saving more than other countries. For 2010 - 2012, Australia's savings
level was at 22.3\% of GDP. The average for the OECD countries was 20.4\%.

Implies we're investing heavily in our economy, growing our export markets,
financed by foreign investors.

Thus the current account deficit is financing our economic growth.


\subsection{Debt Trap}

This becomes an issue if the foreign debt is being used to service (pay
interest repayments on) existing debt, rather than grow our economy.

Eventually causes a country to default on their debt.




\section{Statistics}

Average current account deficit: -4.5\% of GDP over the last 30 years

Average balance of goods and services deficit: -1.0\% of GDP over the last 30
years

Average primary income deficit: -3.5\% of GDP over the last 30 years.

\end{document}
