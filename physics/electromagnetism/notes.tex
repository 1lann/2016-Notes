
\documentclass[a4paper,11pt]{article}

% Math symbols
\usepackage{amsmath}
\usepackage{amsfonts}
\usepackage{esvect}

% Hyperlink contents page
\usepackage{hyperref}
\hypersetup{
	colorlinks,
	citecolor=black,
	filecolor=black,
	linkcolor=black,
	urlcolor=black
}

% No indent on new paragraphs
\setlength{\parindent}{0mm}
\setlength{\parskip}{0.2cm}

% Alias \boldsymbol to \bb for vectors
\newcommand{\bb}{\boldsymbol}


\begin{document}

\title{Electromagnetism}
\author{Ben Anderson}
\date{\today}
\maketitle
\pagebreak

\tableofcontents
\pagebreak


\section{Charge}

Symbol: $q$

Units: Coulombs ($C$)

Charge is a fundamental property of a subatomic particle.


\subsection{Types}

There are two types of charge:

\begin{enumerate}
\item Positive
\item Negative
\end{enumerate}

A particle can also possess no charge (neutral).


\subsection{Quantised Nature}

Charge is quantised (exists in discrete units).

The charge on any particle is some integer multiple of the charge on an
electron.

1 electron has a charge of $1.60 \times 10^{-19}$.

1 Coulomb of charge contains $6.25 \times 10^{18}$ electrons.


\subsection{Repulsion and Attraction}

Like charges repel, and opposite charges attract.




\section{Electric Field Lines}

The requirements in drawing field lines around point charges:

\begin{itemize}
\item Symmetrical about every axis.
\item Point towards negative charge, away from positive charge.
\item Density of lines indicates the relative strength of the field at a point
	around the charge.
\end{itemize}


\subsection{Required Diagrams}

The required electric field line diagrams are:

\begin{itemize}
\item Single positive and negative point charges.
\item A positive and negative charge near each other.
\item Two positive and two negative charges near each other.
\item Two charged plates separated by a distance.
\end{itemize}




\section{Electricity}

\subsection{Electrical Potential Energy}

Work must be done to move two particles with like charges close together.

Potential energy stored by a particle with a charge when work is done in moving
it close to another particle with the same charge.


\subsection{Voltage}

$$
V = \frac{W}{q}
$$

Symbol: $V$

Units: Volts ($V$), equivalent to Joules per Coulomb ($JC^{-1}$)

The difference in potential energy between two points in a circuit.


\subsection{Current}

$$
I = \frac{q}{s}
$$

Symbol: $I$

Units: Amperes ($A$), equivalent to Coulombs per second ($Cs^{-1}$)

The rate of flow of charge through a certain point in a circuit.


\subsection{Conventional Current}

The direction of flow of positive charge in a circuit.

Use this unless stated otherwise.


\subsection{Electron Current}

The direction of flow of negative charge (electrons) in a circuit.


\subsection{Electric Field}

$$
\begin{aligned}
E & = \frac{F}{q} \\
E & = \frac{V}{d} \\
\end{aligned}
$$

Symbol: $E$

Units: Newtons per Coulomb ($NC^{-1}$)

An electric field exists at a point in space when a force is exerted on a
charge at that point.

The electric field strength at a point in space is defined as the force exerted
per unit of charge at that point.

The electric field strength around a point charge decays with the square of
the distance (inverse square law).


\subsection{Force on a Charged Particle}

$$
F = \frac{1}{4 \pi \epsilon_0} \times \frac{q_1 q_2}{r^2}
$$

The force on a particle with charge $q_1$ positioned $r$ metres away from
another particle with charge $q_2$ is $F$.

The direction of the force will depend on whether the charges are positive or
negative.


\subsection{Field Strength on a Test Charge}

$$
E = \frac{1}{4\pi \epsilon_0} \times \frac{q_2}{r^2}
$$

The force experienced per unit charge by a test charge at a certain radius from
the source of an electric field.




\section{Electricity Problems}

\subsection{Acceleration Across a Plate}

Given two metal plates separated by a distance with a potential difference
across them.

Calculate the electric field strength between the two plates.

Calculate the force on a positive charge positioned at the positive plate.

Calculate its acceleration by dividing by its mass.


\subsection{Forces Between Multiple Particles}

To calculate the resultant force on a set of more than 2 particles positioned
near each other.

For each particle, calculate the force exerted on it by every other particle,
then sum these vectors to find the resultant force.


\subsection{Projectile Motion Across Plates}

Where a particle enters horizontally between two plates with a potential
difference between them, where it experiences a force downwards due to an
electric field.

This downwards force will act similar to gravity in a projectile motion
question.




\section{Magnetism}

\subsection{Permanent Magnets}

A permanent magnet is formed when the spin on a significant number of
electrons within a region of a piece of solid metal align.

The regions of aligned electron spin are called domains.

Only ferromagnetic metals can form permanent magnets (eg. iron, nickel,
cobalt).


\subsection{Paramagnetism}

Where the domains in a metal when in the presence of an external magnetic field
temporarily align in the same direction as those of the external field.

Causes an attraction between the paramagnet and the external magnet.


\subsection{Diamagnetism}

Where the domains in a metal when in the presence of an external magnetic field
temporarily align in the opposite direction to that of the exernal field.

Causes a repulsion between the diamagnet and the external magnet.

Only occurs for certain metals below a critical temperature.




\section{Magnetic Field Lines}

\subsection{Convention}

Conditions for drawing magnetic field lines:

\begin{itemize}
\item Symmetrical about every axis.
\item Lines point from north to south.
\item Density of lines indicates the relative field strength.
\end{itemize}

The required magnetic field diagrams are:

\begin{itemize}
\item A single dipole magnet.
\item Two dipole magnets placed vertically side by side, either repelling or
	attracting each other.
\item Two dipole magnets placed horizontally side by side, either repelling or
	attracting each other.
\end{itemize}


\subsubsection{Convention}

\begin{description}
\item [Cross] Field lines pointing out of the page.
\item [Dot] Field lines pointing into the page.
\end{description}


\subsection{Magnetic Field Lines Around Earth}

The Earth acts as a large bar.

The south magnetic pole is at the geographic north pole.

The north magnetic pole is at the south geographic pole.


\subsubsection{Direction of Field Lines}

At the equator, the magnetic field lines have no vertical component. They point
directly towards the north geographic pole.

At the north geographic pole, the magnetic field lines point directly down.

At the south geographic pole, the magnetic field lines point directly up.


\subsubsection{Angle of Dip}

The angle at which, to a tangent at a point on Earth's surface, the magnetic
field lines point.




\section{Magnetic Fields}

\subsection{Magnetic Field Strength}

Symbol: $B$

Units: Tesla ($T$)

Requires a direction.

Exists wherever a permanent magnet would experience a force.


\subsection{Moving Charge}

A moving charge experiences a force when travelling through a magnetic field:

$$
F = qvB
$$

Only want the component of the magnetic field perpendicular to the velocity
of the particle.


\subsubsection{Direction}

The force acts in a direction as determined by the right hand rule (ie. use
your right hand):

\begin{itemize}
\item Point fingers in the direction of the magnetic field.
\item Point thumb in the direction of conventional current.
\item Palm points in direction of force.
\end{itemize}


\subsection{Current in a Wire}

A current flowing through a wire that is in the presence of a magnetic field
will cause the wire to experience a force:

$$
F = BIl
$$

Only the component of the magnetic field perpendicular to the wire is used.

The direction of the force is determined by the right hand rule above.




\section{Induced Magnetic Field}

A moving charge in a wire will induce a circular magnetic field around the
wire.


\subsection{Direction}

The direction of the field is determined by the right grip rule (ie. use your
right hand):

\begin{itemize}
\item Point thumb in direction of conventional current.
\item The direction your fingers move when curled inwards is the direction of
	the circular induced magnetic field around the wire.
\end{itemize}


\subsection{Induced Magnetic Field Strength}

The strength of an induced magnetic field in a wire depends on the current in
the wire (how fast charge is flowing), and the distance from the wire:

$$
B = \frac{\mu_0}{2\pi} \times \frac{I}{r}
$$

Where $\mu_0$ is the permeability of free space:

$$
\mu_0 = 4\pi \times 10^{-7}
$$

The direction of the magnetic field at a point can be found using the right
grip rule to determine the circular direction of the field, as above.


\subsection{In Two Wires}

In two wires side by side, the induced magnetic fields will either cause them
to repel or attract each other, depending on the direction of current.

For current travelling in the same direction (upwards):

\begin{itemize}
\item The induced magnetic field around each magnet will be anticlockwise when
	viewed from above (right grip rule).
\item For the wire on the left, the direction of the magnetic field induced by
	the current in the other wire will be out of the page.
\item This will exert a force towards the right on the wire (right hand rule).
\item Similarly, a force towards the left will be exerted on the wire on the
	right.
\item This will cause an attraction between the two wires.
\end{itemize}


\subsection{In a Solenoid}

A solenoid is a long circular coil of wire with a current running through it.

This creates a dipole electromagnet.

From the direction of flow of current, use the right grip rule to determine the
direction of the field lines around the solenoid.

Use the direction of field lines to determine the polarity of the solenoid
(which end is north and south).


\subsubsection{Soft Iron Core}

An iron core within a solenoid has the effect of greatly increasing the
strength of the magnetic field produced by the solenoid, as the magnetic
domains within the soft iron can easily align.




\section{Magnetic Flux}

\subsection{Magnetic Flux Density}

Equivalent to magnetic field strength.


\subsection{Magnetic Flux}

$$
\begin{aligned}
\phi & = BA_{\bot} \\
& = BA \cos{\theta} \\
\end{aligned}
$$

Symbol: $\phi$

Units: Tesla metres squared ($\text{Tm}^2$) or Webber ($\text{Wb}$)

The amount of magnetic field that passes through a given area perpendicular to
the magnetic field.

In the equation above, $\theta$ is the angle between the normal to the area,
and the magnetic field.


\subsubsection{Maximum Flux}

When the normal to the area is parallel to the magnetic field ($\theta$ is 0).


\subsection{Change in Magnetic Flux}

$$
\Delta \phi = \phi_f - \phi_i
$$

Ways to change magnetic flux:

\begin{itemize}
\item Change the area in the magnetic field.
\item Change the angle between the area and the magnetic field.
\item Change the magnetic field strength.
\item Move the area outside the magnetic field.
\end{itemize}


\subsubsection{Direction}

To find the direction of a change in magnetic flux:

\begin{itemize}
\item Find the initial direction and rough magnitude of the magnetic flux.
\item Find the final direction and rough magnitude of magnetic flux.
\item Find the direction which would cause such a change.
\end{itemize}

For example, in moving the area outside of the magnetic field which is directed
into the page:

\begin{itemize}
\item Initial magnetic flux is into the page.
\item Final magnetic flux is 0 (no magnetic field).
\item A change in out of the page.
\end{itemize}

For example, increasing the area in a magnetic field directed into the page:

\begin{itemize}
\item Initial magnetic flux is a small into the page.
\item Final magnetic flux is a large into the page
\item A change in into the page.
\end{itemize}




\section{Induced EMF}

$$
\varepsilon = -n \frac{\Delta \phi}{\Delta t}
$$

Symbol: $\varepsilon$

Units: Volts ($V$)

Induced EMF is where the movement of a wire through a magnetic field will induce
a potential difference across the wire.

Magnetic flux must change per unit time to induce an EMF.

The negative in the equation implies that the direction of the induced EMF will
be opposite to the change in flux that caused it.


\subsection{Reason}

The reason for induced EMF:

\begin{itemize}
\item A wire moving through a magnetic field represents the motion of charge
	through a magnetic field.
\item Thus a force is exerted on the electrons in the wire, moving a majority
	towards one end of the wire.
\item This induces a potential difference across the two ends of the wire.
\end{itemize}

For example, a wire moving left through a magnetic field that is coming out of
the page:

\begin{itemize}
\item Positive charges are moving left.
\item Will experience a force up.
\item Positive charge will accumulate at the top of the wire.
\item Negative charge will accumulate at the bottom of the wire.
\end{itemize}


\subsection{Faraday's Law}

States that the magnitude of the induced EMF is proportional to the change in
magnetic flux.


\subsection{Lenz's Law}

States that the direction of the induced EMF opposes that of the change in
magnetic flux.


\subsection{Wire}

For a wire moving with a constant velocity through a magnetic field, the induced
EMF is:

$$
\begin{aligned}
\varepsilon & = -\frac{\Delta \phi}{\Delta t} \\
& = -\frac{Blvt}{t} \\
& = Blv \\
\end{aligned}
$$

Where $v$ is perpendicular to the direction of $B$.




\section{Eddy Currents}

Where small circular spirals of moving charge are set up in metals that
experience a changing magnetic flux ($\Delta \phi$).

Only occurs in metals, as they have free moving electrons that are capable of
forming eddy currents.

Power lost to eddy currents:

$$
P = I^2 R
$$


\subsection{Energy Transformation}

Convert mechanical or kinetic energy in a moving magnetised object (which
causes a changing magnetic flux) into heat (increasing the kinetic energy of
particles in the metal).

As the velocity of the moving object increases, so does the changing magnetic
flux, increasing the magnitude of the eddy currents in the metal and the power
loss.


\subsection{Magnet Through Aluminium Pipe}

As the magnet falls through an aluminium pipe:

\begin{itemize}
\item Sections of the pipe experience a change in magnetic flux.
\item Lenz's law states that the direction of the induced EMF in the pipe will
	be opposite to that of the change in flux.
\item This produces spirals of moving charge (eddy currents) in the pipe.
\item This exerts a magnetic repulsion force on the falling magnet, slowing its
	descent.
\end{itemize}


\subsubsection{Direction of Eddy Current Spirals}

Consider a magnet falling through a pipe, with its north end closest to the
ground.

Just below the magnet:

\begin{itemize}
\item Magnetic field lines extend from below the magnet, then out of the page,
	then upwards parallel to the page.
\item Thus on the surface of the aluminium pipe, the magnetic field lines
	are out of the page.
\item This produces anticlockwise eddy currents (right screw rule).
\end{itemize}

Just above the magnet:

\begin{itemize}
\item On the surface of the aliminium pipe, the magnetic field lines are into
	the page.
\item This produces clockwise eddy currents (right screw rule).
\end{itemize}


\subsection{Induction Heating}

Induction heating works by:

\begin{itemize}
\item AC current is used in a solenoid to generate a changing magnetic field.
\item Metals placed near this changing magnetic field experience a changing
	magnetic flux.
\item This induces eddy currents in the metal, converting eletrical energy to
	heat, increasing the temperature of the metal.
\end{itemize}

Stovetops:

\begin{itemize}
\item Solenoid is placed below the stovetop, with a non conducting substance
	placed above it (where the pan sits on top of).
\item This non-conducting substance does not have free moving electrons,
	preventing eddy currents from forming in it due to the changing magnetic
	flux produced by the solenoid.
\item Thus the stovetop does not become hot.
\end{itemize}


\subsection{Reducing Eddy Currents}

Eddy currents can be reduced by laminating the metal:

\begin{itemize}
\item This involves cutting small slits in the metal.
\item This reduces the bulk of metal in which eddy currents can form.
\item Reduces loss of energy to heat as the formation of eddy currents is
	impeded.
\end{itemize}




\section{Basic DC Motor}

For a rectangular coil of wire in a permanent magnetic field, allowed to rotate
about a horizontal axis.

The two sides perpendicular to the magnetic field will experience opposing
forces (one up, one down).

The two sides parallel to the magnetic field will experience no force.

There is no net translational force, but there is an unbalanced rotational
force.


\subsection{Under Rotation}

The unbalanced rotational force exerts a torque on the motor, causing it to
rotate.

The direction and magnitude of the force exerted on the two sides does not
change.

The angle between the force and the line of action will increase, reaching
$180^\circ$ when the coil has rotated $90^\circ$.

This will cause the coil to remain vertical, not continuing its motion.


\subsection{Torque}

The torque on the coil when in its initial position (where the force is
perpendicular to the line of action):

$$
\begin{aligned}
\tau & = rF \\
\sum \tau_{\mbox{cw}} & = \sum \tau_{\mbox{acw}} \\
\tau & = 2rF \\
& = 2rBIl \\
\end{aligned}
$$

Since $A$ (the area of the coil) is equivalent to $2rl$:

$$
\tau = BAI
$$


\subsubsection{Multiple Coils}

For a wire that has been coiled in a rectanglar shape $n$ times, each coil will
have the same force exerted on it.

For the force exerted on one side of the motor:

$$
F = BIln
$$

For the total torque exerted on the motor:

$$
\begin{aligned}
\sum \tau_{\mbox{cw}} & = \sum \tau_{\mbox{acw}} \\
\tau & = rF \\
& = 2r BILn \\
& = BAIn \\
\end{aligned}
$$

When rotated by an angle of $\theta$:

$$
\tau = BAIn \sin{\theta} \\
$$


\subsubsection{RMS Torque}

The root mean squared torque (average torque) is:

$$
\tau_{\mbox{RMS}} = \frac{\tau}{\sqrt{2}}
$$




\section{DC Motor}

When a basic DC motor reaches $90^\circ$, it will become stuck since the force
is always directly upwards (it may wobble slightly).

This is solved by flipping the direction of current in the wire when the coil
reaches $90^\circ$, flipping the direction of the force, allowing it to
continue rotating.

This is done using a commutator.


\subsection{Commutator}

Two C-shaped semicircles connected to the rotating coil, which rotate as the
coil does.

Brushes fixed to the battery circuit come in contact with the commutators to
complete the circuit.

After $90^\circ$, the commutators swap the brushes they are in contact with,
reversing the direction of current.


\subsection{Torque Curve}

A graph of the magnitude of torque against time.

Appears as the graph of $\lvert \cos{\theta} \rvert$ (absolute value of a
cosine graph).

The maximum torque in this graph will be when the force exerted by the magnetic
field is perpendicular to the line of action ($\tau_{\text{max}}$).

The average torque will be less than this maximum torque (drawn as a dotted
line).


\subsection{Curving Permanent Magnets}

The permanent magnets are curved around the circular path of the coils.

This causes the applied force to act perpendicular to the line of action for
a larger rotation angle.

This increases the average torque applied, but leaves the maximum torque
constant.

The torque curve graph appears flatter at the peaks, more like a table.


\subsection{Multiple Coils}

Where another set of C-shaped commutators and another coil are added,
perpendicular to the existing coil, using a curved magnet.

Only the commutator and coil in the optimal location within the magnet for
maximum torque is in contact with the brushes at any time.

Torque curve graph appears as two or more overlapping cosine graphs that are
$90^\circ$ out of phase.

Maximum torque remains the same, but increases the average torque.


\subsection{Increase Maximum Torque}

Increase any variables in the equation:

$$
\tau = B A I n \sin{\theta}
$$

\begin{itemize}
\item Increase magnetic field strength
\item Increase number of coils
\item Increase current running through coils
\item Increase the width of the coil (radius from pivot)
\end{itemize}

Note for a constant voltage supply, increasing the length of the wire would
increase the resistance and decrease the current. This would result in no
change in the maximum torque.


\subsection{Increase Average Torque}

Ways to increase the average torque of a DC motor without changing the maximum
torque:

\begin{itemize}
\item Curve the permanent magnets to fit the circular rotation of the coils.
\item Add another commutator pair and coil.
\end{itemize}


\subsection{Servicing Motors}

Factors that could cause a decrease in the efficiency, average, or maxium
torque of a motor:

\begin{itemize}
\item Sparking when the brushes jump the small gap between commutators,
	causing carbon build up on the copper commutators, increasing the
	resistance and reducing the current through the circuit (due to a constant
	supplied voltage).
\item The carbon brushes may disintegrate, decreasing the time in one full
	revolution during which they are in contact with the commutators,
	decreasing the average current flow, decreasing maximum torque.
\end{itemize}




\section{AC Motors}

AC motors rotate at the same frenquecy as the oscillating current supplied.

Construction of the motor:

\begin{itemize}
\item Rectangular coil of wire free to rotate on an axis.
\item Stator coils (solenoids) are used instead of permanent magnets.
\item A split ring commutator pair is used at the end of the coil.
\end{itemize}

How it works:

\begin{itemize}
\item When the coil reaches $90^\circ$, the current in the wires swaps direction
	(since it is AC) for both the armature windings and stator coils.
\item This swaps the direction of the magnetic field produced by the stator
	coils.
\item The commutator pair swaps the direction of the current in the armature
	windings a second time.
\item This swaps the direction of the force, producing a continuous torque.
\end{itemize}


\subsection{Benefits Over DC Motors}

The benefits of using an AC motor over a DC one include:

\begin{itemize}
\item Using solenoids for magnets allows for much stronger magnetic fields than
	can be achieved with the permanent magnets in a DC motor. Allows for much
	higher maximum torque.
\item DC motors produce a rectified DC wave, with a lower average voltage.
	Average voltage for AC motors is not considered.
\end{itemize}




\section{Back EMF}

Back EMF can be explained as the following:

\begin{itemize}
\item As the coil in the motor is rotated, an EMF is induced in the coil
	opposite to the direction of charge from the DC power source.
\item As the coil's velocity increases, the back EMF increases, reducing the
	potential difference across the coil.
\item This reduces the current flowing in the coil, since the coil's resistance
	is constant.
\item The motor eventually reaches a constant RPM as the back EMF approaches
	the supplied voltage.
\end{itemize}

Back EMF applies to both DC and AC motors.


\subsection{Calculation}

After a period of time, the motor's back EMF will approach a constant:

$$
\begin{aligned}
V_{\mbox{effective}} = \varepsilon - \varepsilon_{\mbox{back}}
I = \frac{\varepsilon - \varepsilon_{\mbox{back}}}{R}
\end{aligned}
$$

The current that accounts for the back EMF is the operating current of the
motor.


\subsection{Current Drop}

Initially as the coils being to move:

\begin{itemize}
\item Frequency of rotation of the coils is low.
\item Results in a low back EMF induced in the coils.
\item High effective voltage.
\item High current.
\end{itemize}

As the coils spin faster, the motor reaches its operating current.

It takes little time for this operating current to be reached.


\subsection{Damage}

\begin{itemize}
\item The motor provides a near constant average torque.
\item If a high load is placed on the motor, the frequency of revolution of the
	motor will decrease.
\item This reduces the back EMF (since $\varepsilon = 2 \pi BAnf$) in the coils.
\item This increases the voltage in the coils.
\item Since the resistance of the wire is constant, this will increase the
	current in the coil.
\item If this current exceeds the maximum current for which the motor was
	designed for an extended period of time, this will damage the motor.
\end{itemize}

This explains why motors are designed for a maximum load.




\section{AC Generator}

Generators produce an AC current using an induced EMF in a wire.

There is a retarding force exerted on the coil when the generator is turned due
to induced EMF (similar to back EMF in a motor).


\subsection{Workings}

\begin{itemize}
\item Generators are a coil free to rotate on an axis that can be manually
	turned.
\item The coil sits in a permanent magnetic field.
\item Each end of the wire is attached to a slip ring which rotates in contact
	with a carbon brush attached to a wire.
\item As the coil is rotated, this produces a changing effective area.
\item Since the magnetic field is constant, this produces a changing magnetic
	flux proportional to this change in area.
\item This induces an EMF in the wire, producing an AC voltage.
\end{itemize}


\subsection{Voltage Graphs}

The graph of effective area against time is a sine curve offset depending on
the initial rotation of the coil.

The change in effective area against time (equivalent to the change in magnetic
flux since the magnetic field is constant) is the derivative of this graph.

The EMF graph is the negative of the change in flux graph (since Lenz's law,
where the direction of the induced EMF is opposite to that of the change in
flux).


\subsection{Maximum EMF}

When moving through the magnetic field, the coil traces out a changing area.
The change in magnetic flux over time is:

$$
\frac{\Delta \phi}{t} = 2vBl
$$

This is multiplied by 2 since there are 2 wires either side of the coil.

The induced EMF is:

$$
\varepsilon = 2nvBl
$$

Since the wire is moving in a circular path, $v = 2\pi r f$:

$$
\varepsilon = 2n 2 \pi r f B l
$$

Since the area of the coil is $A = 2 r l$:

$$
\varepsilon = 2\pi n BA f
$$


\subsection{Root Mean Squared EMF}

The root mean squared (RMS) EMF is:

$$
\varepsilon_{\mbox{RMS}} = \frac{\varepsilon}{\sqrt{2}}
$$




\section{DC Generator}

Same as an AC generator, but a split ring commutator pair as in a DC motor is
used, instead of slip rings.

This flips the direction of current at $90^\circ$, creating a recitified DC
wave.


\subsection{Increase Average Voltage}

We can increase the average output voltage the same way we increase the average
torque in a DC motor:

\begin{enumerate}
\item Curve magnets to apply a field perpendicular to the area of the coil for
	a greater portion of its rotation.
\item Increase the number of coils and commutator pairs.
\end{enumerate}

Both of these have the same effect on the voltage graph as they did on the
torque graph for a DC motor.


\subsection{Increase Maximum Voltage}

Increase any variables in the equation:

$$
\varepsilon = 2\pi n BA f
$$

\begin{itemize}
\item Increase number of coils.
\item Increase the strength of the magnetic field.
\item Increase the area of the coils.
\item Increase the frequency of rotation.
\end{itemize}




\section{Transformer}

Raises or lowers AC voltages in wires.

Since power must remain constant (due to conservation of energy), the current
in the wire is inversely raised or lowered as well.


\subsection{Construction}

The construction of a transformer:

\begin{itemize}
\item A square loop of soft iron.
\item One side has a number of coils of wire (the primary coils), through which
	the initial AC current is passed.
\item The other side has a different number of coils (the secondary coils),
	in which a potential difference is induced due to the changing magnetic
	field.
\end{itemize}

How it works:

\begin{itemize}
\item An AC current in the primary coils creates a changing magnetic field
	within the soft iron core.
\item The secondary coils experience a changing magnetic flux.
\item Faraday's law states that the induced EMF in the secondary coils will be
	proportional to this changing magnetic flux.
\item Since the rate of change of magnetic flux for both coils is the same,
	the ratio of the number of primary and secondary coils determines the
	factor by which the voltage is stepped up or down.
\end{itemize}


\subsection{Lamination}

Lamination is where:

\begin{itemize}
\item Soft iron core is cut into thin pieces and laminated.
\item This reduces the amount of bulk metal in which eddy currents can form.
\item Reduces power lost to resistive heating in the core
\end{itemize}


\subsection{Calculations}

$$
\frac{V_p}{V_s} = \frac{N_p}{N_s} = \frac{I_s}{I_p}
$$


\subsection{Power Loss}

$$
P_s = \mbox{efficiency} \times P_p
$$

Power will be lost due to heating of the primary or secondary wires, or eddy
currents in the soft iron core (also heating it).

The ratio of voltages will always remain constant no matter the efficiency of
the transformer.

Any loss in power will be expressed by a drop in current in the second coil
from what would exist in an ideal transformer.




\section{Electricity Transmission}

When electricity is travelling large distances, power is lost to heat in the
wires it travels through according to:

$$
\begin{aligned}
V & = IR \\
P & = IV \\
P & = I^2 R \\
\end{aligned}
$$

Electricity should be transmitted at high voltage as:

\begin{itemize}
\item There will be a voltage drop across the wires as the wire has a
	resistance, and power will also be lost due to resistive heating.
\item Since the power at the generator is constant, a large voltage will
	result in a low current ($P = IV$).
\item This decreases voltage drop across the wires ($V = IR$) and decreases
	power lost to resistive heating ($P = I^2R$)
\end{itemize}

Transmitted as AC because:

\begin{itemize}
\item Transformers are used to step up and down the voltage in the wires.
\item Transformers require AC voltage as they must produce a changing magnetic
	field (ie. changing magnetic flux) to induce an EMF in the secondary coils
	(Faraday's law).
\end{itemize}




\section{Questions}

\subsection{Field Lines}

Draw electric or magnetic field lines around:

\begin{enumerate}
\item A single positive charge
\item A single negative charge
\item A positive and negative charge placed apart
\item Two positive charges placed apart
\item Two negative charges placed apart
\item A single bar magnet
\item Two bar magnets placed side by side, where the north pole of each is at
	the top of the page
\item Two bar magnets placed side by side, where the north pole of one is placed
	next to the south pole of the other
\item Two bar magnets placed end to end, where the two north poles are closest
\item Two bar magnets placed end to end, where the north pole of one is closest
	to the south pole of the other
\end{enumerate}


\subsection{Electric Fields}

\begin{enumerate}
\item Find the electric field strength of two plates placed apart
\item Find the acceleration of a charged particle placed between these two
	plates
\item Find the vertical displacement of a charged particle fired horizontally
	thorugh a pair of charged plates
\item Find the resultant force on 3 charged particles placed near each other
\end{enumerate}


\subsection{Magnetic Fields}

\begin{enumerate}
\item Find the force experienced by a charge moving through a magnetic field
\item Find force experienced by a wire with a current flowing through it when
	in the presence of a magnetic field
\item Find the magnetic flux density at a point a certain distance away from a
	wire with a current running through it
\item Find the force exerted on each of two wires placed next to each other,
	both with current running through them
\item Find the polarity of a solenoid
\end{enumerate}


\subsection{Induced EMF}

\begin{enumerate}
\item Explain the direction of an induced EMF in a coil
\item Find the induced EMF in a wire moving through a magnetic field with a
	velocity
\item Explain why a magnet dropped through an aluminium pipe falls slower than
	in a plastic pipe
\item Explain how induction heating works
\item Explain why the ceramic above the stovetop on which the pan is placed
	does not heat up
\end{enumerate}


\subsection{DC Motors}

\begin{enumerate}
\item Explain how a DC motor works
\item Calculate the maximum and RMS torque for a DC motor
\item Explain how to increase the maximum and average torque for this motor
\item Graph a torque curve of a DC motor, with curved magnets, and multiple
	coils
\item Explain factors that could decrease the efficiency of a DC motor over
	time
\item Explain why a DC motor approaches a constant RPM
\item Explain why placing a large load on a motor could damage it
\end{enumerate}


\subsection{AC Motors}

\begin{enumerate}
\item Explain how an AC motor works
\item Explain the benefits of using an AC motor over DC
\end{enumerate}


\subsection{AC Generators}

\begin{enumerate}
\item Explain how an AC generator works
\item Draw a voltage graph of an AC generator
\item Explain why there is a resistive force felt when the generator is turned
\end{enumerate}


\subsection{DC Generators}

\begin{enumerate}
\item Explain how a DC generator works
\item Explain how to increase the average and maximum output voltage of a DC
	generator
\end{enumerate}


\subsection{Transformers}

\begin{enumerate}
\item Explain how a transformer works
\item Explain the features of a transformer that reduce power loss
\item Calculate voltage transformations in transformers
\item Apply efficiency of a transformer to calculations
\item Explain why electricity is transferred as high voltage AC
\end{enumerate}

\end{document}
