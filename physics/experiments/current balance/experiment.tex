\documentclass[a4paper,11pt]{article}

% Math symbols
\usepackage{amsmath}

% Tables
\usepackage{tabularx}
\usepackage{multirow}
\newcolumntype{Y}{>{\centering\arraybackslash}X}

% No indent on new paragraphs
\setlength{\parindent}{0mm}

\begin{document}

\title{Current Balance Experiment}
\author{Ben Anderson}
\date{\today}
\maketitle


\section{Results}

$l = 0.025\mbox{ m}$

6000 squares have a mass of 0.00201 kg.

\begin{center}
\begin{tabular}{c|c|c|c|c|c}
\multirow{2}{*}{I (A)} & \multicolumn{4}{c|}{Number of Squares} & \multirow{2}{*}{Mass ($\times 10^{-6}\mbox{ kg}$)} \\
\cline{2-5}
& Trial 1 & Trial 2 & Trial 3 & Average & \\
\hline
0.580 & 20  & 20  & 19  & 20  & 6.70 \\
1.00  & 47  & 52  & 49  & 50  & 16.8 \\
1.60  & 112 & 110 & 110 & 110 & 36.9 \\
2.12  & 180 & 179 & 180 & 180 & 60.3 \\
2.55  & 261 & 258 & 260 & 260 & 87.1 \\
2.70  & 280 & 278 & 281 & 280 & 93.8 \\
\end{tabular}
\end{center}

\begin{center}
\begin{tabular}{c|c}
Measured I (A) & Calculated I (A) \\
\hline
0.580 & 0.658 \\
1.00  & 1.04  \\
1.60  & 1.54  \\
2.12  & 1.97  \\
2.55  & 2.37  \\
2.70  & 2.46  \\
\end{tabular}
\end{center}


\section{Sample Calculations}

Mass of paper squares (using the number of squares for a measured current of
0.58 A as an example):

$$
\begin{aligned}
m & = n \times \frac{0.00201}{6000} \\
& = 20 \times \frac{0.00201}{6000} \\
& = 6.70 \times 10^{-6}\mbox{ kg} \\
\end{aligned}
$$

Derivation of the formula for calculated current:

$$
\begin{aligned}
	F & = BIl \\
	F & = mg \\
	B & = \mu_0 NI \\
	\frac{F}{Il} & = \mu_0 NI \\
	I^2 & = \frac{F}{l \mu_0 N} \\
	I & = \sqrt{\frac{mg}{l \mu_0 N}} \\
\end{aligned}
$$

Calculation of $N$:

$$
\begin{aligned}
	N & = \frac{\mbox{turns}}{\mbox{length of solenoid}} \\
	& = \frac{770}{0.16} \\
	& = 4810\mbox{ m}^{-1} \\
\end{aligned}
$$

Sample calculation for the first measured current value (of 0.58 A):

$$
\begin{aligned}
	I & = \sqrt{\frac{m g}{l \mu_0 N}} \\
	& = \sqrt{\frac{6.70 \times 10^{-6} \times 9.8}{0.025 \times 1.26 \times 10^{-6} \times 4810}} \\
	& = 0.658\mbox{ A} \\
\end{aligned}
$$


\section{Analysis}

\subsection{Gradient}

$$
\begin{aligned}
\mbox{gradient} & = \frac{\mbox{rise}}{\mbox{run}} \\
& = \frac{2.52 - 0.24}{2.32 - 0.4} \\
& = 1.19 \\
\end{aligned}
$$

The gradient of our graph is 1.19.


\subsection{Y Intercept}

The y intercept of our graph is -0.24 A.


\subsection{Calibration Factor}

Thus the equation for the line of best fit for our graph is:

$$
y = 1.19 x - 0.24
$$

This represents the calibration factor for the ammeter.


\section{Sources of Error}

\subsection{Varying Torque}

An assumption made in the experiment was that the distance the paper squares
were from the pivot was equal to the distance from the pivot to the point at
which the magnetic force due to the solenoid was exerted. This allowed us to
assume the torques exerted on the balance by the weight of the paper squares and
the force due to the magnetic field were equal (since $\tau = r F$, where $r$
was assumed to be constant), allowing us to equate the forces they exerted
(ie. $mg = BIl$). \\

In reality, each of the paper squares was not placed the exact same distance
from the pivot. This would have caused the torque exerted by some of the squares
to be greater than the others, as some would've been further away than others. \\

Let the distance between the pivot and the point at which the force due to the
magnetic field is exerted be $r$. If the average distance between the pivot and
the centre of mass of each square was less than $r$, then the torque exerted by
the weight force of the paper would be less than that exerted by the force due
to the magnetic field (since $\tau = r F$). Thus more paper squares would be
required to create equal torques and level the current balance, increasing the
mass of paper squares
measured, increasing our measured current values. This would decrease the
accuracy of our results. The reverse would happen if the average distance was
greater than $r$. \\

Quantitatively, the distance between the pivot and the point at which the force
due to the magnetic field was exerted was 0.11 m. If we assume that on average
the squares were placed a distance of 0.10 m from the pivot (rather than 0.11
m):

$$
\begin{aligned}
F & = BIl \\
F & = mg \\
\tau & = r F \\
\sum \tau_{\mbox{cw}} & = \sum \tau_{\mbox{acw}} \\
0.11 B I l & = 0.10 m g \\
BIl & = 0.909 mg \\
\end{aligned}
$$

$$
\begin{aligned}
B & = \mu_0 NI \\
\frac{F}{Il} & = \mu_0 NI \\
I^2 & = \frac{F}{l \mu_0 N} \\
I & = \sqrt{\frac{0.909 mg}{l \mu_0 N}} \\
I & = 0.953 \times \sqrt{\frac{mg}{l \mu_0 N}} \\
\end{aligned}
$$

As can be seen from the above equation, the final measured current value is
approximately 4.7\% less than what it would have been had the paper squares been
placed an equal distance from the pivot. \\

For example, our first result (a measured current of 0.58 A) recalculated
using the above equation results in a calculated current of 0.627 A,
rather than our actual result of 0.658 A.


\subsection{Inaccurate Cutting}

Another possible source of error includes inaccuracy when cutting out the small
paper squares. When conducting the experiment, we measured the mass of 6000
squares and used this to calculate the mass of the number of squares placed on
the end of the current balance. Due to inaccuracy in our cutting of the paper
squares, there may be more or less paper present on the balance than what was
measured - we had assumed that our cutting was infinitely accurate. \\

If we added more paper than measured, our measured number of paper squares would
be smaller than the actual value, causing the calculated mass of squares to be
less than the actual value, decreasing our calculated current values. The
reverse would happen if less paper was added than measured. This decreases the
precision of our results. \\

For example, if we had actually placed 20.2 squares instead of 20 (as recorded)
on the balance for our first result (a measured current of 0.58 A),
this would've increased the mass of squares from $6.70 \times 10^{-6}\mbox{ kg}$,
to $6.77 \times 10^{-6}\mbox{ kg}$. \\

The calculated current based on this mass is:

$$
\begin{aligned}
	I & = \sqrt{\frac{m g}{l \mu_0 N}} \\
	& = \sqrt{\frac{6.77 \times 10^{-6} \times 9.8}{0.025 \times 1.26 \times 10^{-6} \times 4810}} \\
	& = 0.662\mbox{ A} \\
\end{aligned}
$$

Thus the calculated current would increase from 0.658 A to 0.662 A. This
highlights the potential effect of this error on the precision of our results.


\section{Conclusion}

We have successfully determined the calibration factor of an ammeter by using a
simple current balance to 'weigh' an electric current. The graph of our
measured current values plotted against our calculated current had a gradient of
1.19, with a y intercept of -0.24 A. Thus the equation for the line of best fit
for our graph is $y = 1.19 x - 0.24$, which represents the calibration factor
for the ammeter.

\end{document}
