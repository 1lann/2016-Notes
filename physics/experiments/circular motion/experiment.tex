\documentclass[a4paper,11pt]{article}

% Math symbols
\usepackage{amsmath}

% Tables
\usepackage{tabularx}
\usepackage{multirow}
\newcolumntype{Y}{>{\centering\arraybackslash}X}

% No indent on new paragraphs
\setlength{\parindent}{0mm}

\begin{document}

\title{Circular Motion Experiment}
\author{Ben Anderson}
\date{\today}
\maketitle


\section{Results}

Mass of stopper: 0.0582 kg


\subsection{Varying Force}

Constant radius: 0.600 m

\begin{center}
\begin{tabularx}{\textwidth}{Y Y Y Y Y}
\multirow{2}{*}{Force (N)} & \multicolumn{4}{c}{Period For 20 Revolutions (s)} \\
& Trial 1 & Trial 2 & Trial 3 & Average \\
\hline
1.50 & 19.24 & 19.20 & 19.23 & 19.22 \\
2.00 & 16.72 & 17.02 & 16.88 & 16.87 \\
2.50 & 14.34 & 14.33 & 14.39 & 14.35 \\
3.00 & 12.90 & 12.65 & 12.79 & 12.78 \\
3.50 & 12.05 & 12.45 & 12.08 & 12.19 \\
4.00 & 9.92  & 9.84  & 10.05 & 9.94  \\
\end{tabularx}
\end{center}


\hspace*{-1.2cm}
\begin{tabularx}{1.2\textwidth}{Y Y Y Y}
Force (N) & Average Period (s) & Speed ($\mbox{ms}^{-1}$) &  $v^2$ ($\mbox{m}^{2}\mbox{s}^{-2}$) \\
\hline
1.50 & 0.961 & 3.92 & 15.4 \\
2.00 & 0.844 & 4.47 & 20.0 \\
2.50 & 0.718 & 5.25 & 27.6 \\
3.00 & 0.639 & 5.90 & 34.8 \\
3.50 & 0.610 & 6.19 & 38.3 \\
4.00 & 0.497 & 7.59 & 57.6 \\
\end{tabularx}
\hspace*{-1.2cm}


\subsection{Varying Radius}

Constant force: 3.00 N

\begin{center}
\begin{tabularx}{\textwidth}{Y Y Y Y Y}
\multirow{2}{*}{Radius (m)} & \multicolumn{4}{c}{Period For 20 Revolutions (s)} \\
& Trial 1 & Trial 2 & Trial 3 & Average \\
\hline
0.200 & 8.90  & 9.05  & 8.96  & 8.97  \\
0.400 & 11.62 & 11.78 & 11.30 & 11.53 \\
0.600 & 13.66 & 13.20 & 13.72 & 13.53 \\
0.800 & 16.25 & 16.48 & 16.10 & 16.28 \\
1.00  & 18.89 & 18.79 & 18.95 & 18.88 \\
\end{tabularx}
\end{center}


\hspace*{-1.2cm}
\begin{tabularx}{1.2\textwidth}{Y Y Y Y}
Radius (m) & Average Period (s) & Speed ($\mbox{ms}^{-1}$) & $v^2$ ($\mbox{m}^{2}\mbox{s}^{-2}$) \\
\hline
0.200 & 0.449 & 2.80 & 7.84 \\
0.400 & 0.577 & 4.36 & 19.0 \\
0.600 & 0.677 & 5.57 & 31.0 \\
0.800 & 0.814 & 6.18 & 38.2 \\
1.00  & 0.944 & 6.66 & 44.4 \\
\end{tabularx}
\hspace*{-1.2cm}


\section{Analysis}

\subsection{Gradient of $F_c$ vs $v^2$}

\addtolength{\jot}{0.5em}
\begin{align*}
	\mbox{gradient} & = \frac{\mbox{rise}}{\mbox{run}} \\
					& = \frac{6.97 - 1.0}{65 - 11} \\
					& = 0.111\mbox{ N s}^{2}\mbox{ m}^{-2} \\
\end{align*}


\subsection{Gradient of $r$ vs $v^2$}

\begin{align*}
	\mbox{gradient} & = \frac{\mbox{rise}}{\mbox{run}} \\
					& = \frac{0.97 - 0.1}{45 - 4.8} \\
					& = 0.0216\mbox{s}^{2}\mbox{ m}^{-1} \\
\end{align*}


\section{Questions}

\subsection{Question 1}

The centripetal force is provided by the tension in the string attached to the
rubber stopper. The tension is caused by the movement of the person's arm in a
repeating pattern.


\subsection{Question 2}

\begin{align*}
				F_c & = \frac{mv^2}{r} \\
	\frac{F_c}{v^2} & = \frac{m}{r} \\
	\mbox{gradient} & = \frac{m}{r} \\
\end{align*}

The gradient of the graph of $F_c$ vs $v^2$ is equivalent to the mass of the
stopper divided by the radius of its motion.

\begin{align*}
	\frac{m}{r} & = \frac{0.0582}{0.600} \\
				& = 0.0970\mbox{ kg m}^{-1} \\
\end{align*}

The gradient of our graph was $0.111\mbox{ N s}^{2}\mbox{ m}^{-2}$. The mass of
the stopper divided by the radius of its motion is $0.0970\mbox{ kg m}^{-1}$.
Our measured value is reasonably close to the theoretical value, with a
percentage difference of:

$$
\frac{0.111 - 0.0970}{0.0970} \times 100 = 14.4\%
$$


\subsection{Question 3}

\begin{align*}
				F_c & = \frac{mv^2}{r} \\
	  \frac{r}{v^2} & = \frac{m}{F_c} \\
	\mbox{gradient} & = \frac{m}{F_c} \\
\end{align*}

The gradient of the graph of $r$ vs $v^2$ is equivalent to the centripetal force
divded by the mass of the stopper, which is equivalent to the inverse of
centripetal acceleration.

\begin{align*}
	\frac{m}{F_c} & = \frac{0.0582}{3.00} \\
				  & = 0.0194\mbox{ kg N}^{-1} \\
\end{align*}

The gradient of our graph was $0.0216\mbox{ s}^{2}\mbox{ m}^{-1}$. The mass of
the stopper divided by the radius of its motion is $0.0194\mbox{ kg N}^{-1}$.
Our measured value is again reasonably close to the theoretical one, with a
percentage difference of:

$$
\frac{0.0216 - 0.0194}{0.0194} \times 100 = 11.3\%
$$


\subsection{Question 4}

No, the string not being perfectly horizontal does not influence the
relationship between $F_c$ and $v$. \\

In our experiment, we make the assumption that the value displayed by the force
guage is the centripetal force. This value is in fact the tension in the string.
This fact is demonstrated in diagram 1 below. The centripetal force is actually
$F_c = T \cos{\theta}$, where $\theta$ is the angle the string makes with the
horizontal. \\

Similarly, the radius of centripetal motion we measured was the distance between
the top of the glass tube and the stopper (labelled $l$ in diagram 2). Again,
this is not the actual radius of centripetal motion, which is instead
$r = l \cos{\theta}$. \\

The relationship between $F_c$ and $v$ can be represented by the ratio between
these two values:

\begin{align*}
	\frac{F_c}{v} & = \frac{F_c}{\sqrt{\frac{F_c r}{m}}} \\
	& = \frac{T \cos{\theta}}{\sqrt{\frac{(T \cos{\theta}) (l \cos{\theta})}{m}}} \\
	& = \frac{T \cos{\theta}}{\sqrt{\frac{T l \cos^2{\theta}}{m}}} \\
	& = \frac{T \cos{\theta}}{\sqrt{\frac{T l}{m}} \cos{\theta}} \\
	& = \frac{T}{\sqrt{\frac{T l}{m}}} \\
\end{align*}

It follows from the above equation that the relationship between $F_c$ and $v$
does not depend on $\theta$, the angle from the horizontal. Hence the
relationship between $F_c$ and $v$ doesn't change if the string is not exactly
horizontal (when $\theta = 0$).

% Yes, the relationship between $F_c$ and $v$ is affected by the string not being
% exactly horizontal. \\

% The stopper's circular motion is always in a single horizontal plane. The
% radius of this motion is the horizontal distance between the stopper and glass
% tube, labelled as $r$ on diagrams 1 and 2 below. When the stopper is not held
% exactly horizontal, it forms an angle $\theta$ with the horizontal (as seen in
% diagram 3). The radius of circular motion therefore varies with the cosine of
% $\theta$. If we let the length of string between the stopper and the top of the
% glass tube (the pivot point) be $l$, then the radius of circular motion is $r
% = l \cos{\theta}$. Hence substituting this into the equation for centripetal
% force:

% \begin{align*}
% 	F_c & = \frac{mv^2}{l \cos{\theta}} \\
% 	  v & = \sqrt{\frac{F_c l \cos{\theta}}{m}} \\
% \end{align*}

% Let the velocity of the stopper when the string is perfectly horizontal be
% $v_1$, and the centripetal force required to maintain this velocity be $F_{c1}$.
% It follows from the above two equations that when $\theta > 0$ (ie. the string
% isn't perfectly horizontal), the radius of circular motion will decrease, and
% the centripetal force required to maintain the velocity $v_1$ will be greater
% than $F_{c1}$. Hence the relationship between $F_c$ and $v$ has been affected.


\section{Sources of Error}

\subsection{One}

In our experiment, we assumed the stopper's motion was perfectly horizontal, and
made no downward angle with the horizontal. \\

In reality, it is impossible for the stopper to travel in a perfectly horizontal path around
the top of the glass tube, and will always make some downward angle $\theta$
with the horizontal. This is because the stopper has mass, and therefore a
downwards weight force due to gravity is exerted on it. Since the centripetal
force is the sum of the tension force in the string and the weight force, and
since the centripetal force acts horizontally (perpendicular to the
weight force), the tension force in the string must act at an angle, and cannot
be perfectly horizontal. \\

Let the distance between the stopper and top of the glass tube (the pivot point)
be $l$. In our experiment, we assumed this distance was the radius of circular
motion of the stopper. The actual radius of motion is instead
$r = l \cos{\theta}$, since the string must make some downward angle $\theta$
with the horizontal (as established above). \\

We calculated our experimental velocity values from our measured radius ($l$)
using the equation:

$$
v = \frac{2 \pi l}{T}
$$

Substituting $l$ from the equation above, we find:

$$
v = \frac{2 \pi r}{T \cos{\theta}}
$$

We see that our calculated velocities are larger than what they should have
been by a factor of $\frac{1}{\cos{\theta}}$. This is a systematic error in our
experiment, reducing the accuracy of our results. \\

For example, consider our velocity of $5.90\mbox{ ms}^{-1}$ for a force of
3.00 N and a constant radius of 0.600 m, which had an average period of 0.639 s.
If we assume that the string made an angle of $10^\circ$ with the horizontal,
the actual radius of circular motion would be
$r = l \cos{\theta} = 0.6 \cos{10^\circ} = 0.591\mbox{ m}$. If we recalculate
our velocity value from this:

\begin{align*}
	v & = \frac{2 \pi r}{T} \\
	  & = \frac{2 \pi 0.591}{0.639} \\
	  & = 5.81\mbox{ ms}^{-1} \\
\end{align*}

Hence the actual velocity, assuming the angle with the horizontal was
$10^\circ$, is $5.81\mbox{ ms}^{-1}$, less than our experimental value, as
expected.

% In addition, the force measured in the experiment was the tension in the string, and not the
% centripetal force. These are related similarly, $F_c = T \cos{\theta}$ \\

% The gradient of our graph was calculated

% If we rearrange the formula for centripetal force for velocity:

% $$
% v = \sqrt{\frac{F_c r}{m}}
% $$

% We can see that since both our measured force and radius were greater than their
% actual values, our calculated velocity is larger also. \\

% For example, if use the experimental values for a force of 3.00 N with a
% constant radius of 0.600 m, we found a velocity of $5.90\mbox{ ms}^{-1}$. If we
% assume that the string made an angle of 10 degrees below the horizontal, then we
% can calculate the expected velocity:

% \begin{align*}
% 	v & = \sqrt{\frac{F_c r}{m}} \\
% 	  & = \sqrt{\frac{3.00 \cos{10} \times 0.600 \cos{10}}{0.0582}} \\
% 	  & = 5.48\mbox{ ms}^{-1} \\
% \end{align*}

% Which is less than our measured value, as expected.


\subsection{Two}

A possible source of error in the experiment is that when the tension force in the
string increases, the spring the string is attached to inside the force guage
extends, causing the length of string to increase. This increases the radius of
circular motion of the stopper. Since the tension force in the string was kept
constant, this will increase the velocity of the stopper, causing our results to
be systematically greater than the correct values of velocity. \\

For example, consider our results for a force of 4.00 N with a constant radius
of 0.600 m. Our measured velocity was $7.59\mbox{ ms}^{-1}$, yet theoretically
it should be:

\begin{align*}
	v & = \sqrt{\frac{F_c r}{m}} \\
	  & = \sqrt{\frac{4.00 \times 0.6}{0.0582}} \\
	  & = 6.42\mbox{ ms}^{-1} \\
\end{align*}

Consider if the string extended 4 cm, making the length of string 0.64 m. If we
assume the string is perfectly horizontal, then the radius of circular motion
is also 0.64 m. This creates a new theoretical velocity value of:

\begin{align*}
	v & = \sqrt{\frac{F_c r}{m}} \\
	  & = \sqrt{\frac{4.00 \times 0.64}{0.0582}} \\
	  & = 6.63\mbox{ ms}^{-1} \\
\end{align*}

An increase of $0.210\mbox{ ms}^{-1}$ over the previous value. Hence it can be
seen that this source of error has caused our measured velocities to be larger
than their correct values.


\section{Conclusion}

By considering the relatively small difference between the gradients of our graphs and their
corresponding theoretical values, it can be seen that our results verify the theoretical
relationship between the variables in a system rotating in a circular path.
Hence we have achieved our aim. \\

For the relationship between $F_c$ and $v^2$, we arrived at a gradient of
$0.111\mbox{ N s}^{2}\mbox{ m}^{-2}$. The theoretical value for this gradient
was $0.0970\mbox{ kg m}^{-1}$, a percentage difference of 14.4\%. \\

For the relationship between $r$ and $v^2$, we arrived at a gradient of
$0.0216\mbox{ s}^{2}\mbox{ m}^{-1}$. This is compared to a theoretical value of
$0.0194\mbox{ kg N}^{-1}$, a percentage difference of 11.3\%.

\end{document}
