\documentclass[a4paper,11pt]{article}

% Math symbols
\usepackage{amsmath}

% Tables
\usepackage{tabularx}
\usepackage{multirow}
\newcolumntype{Y}{>{\centering\arraybackslash}X}

% No indent on new paragraphs
\setlength{\parindent}{0mm}

\begin{document}

\title{Projectile Motion Experiment}
\author{Ben Anderson}
\date{\today}
\maketitle

\section{Results}

\begin{tabularx}{\textwidth}{|Y|Y|Y|Y|Y|Y|Y|}
\hline
\multirow{2}{1cm}{Setting} & \multicolumn{6}{c|}{Height (m)} \\
\cline{2-7}
& Trial 1 & Trial 2 & Trial 3 & Trial 4 & Trial 5 & Average \\
\hline
Low & 0.447 & 0.450 & 0.435 & 0.434 & 0.435 & 0.440 \\
Medium & 1.06 & 1.07 & 1.06 & 1.06 & 1.04 & 1.06 \\
\hline
\end{tabularx}

\section{Calculations}

\subsection{Low Setting}

\begin{align*}
v_y^2 & = u_y^2 + 2 a_y s_y \\
0 & = u_y^2 + 2 (-9.8) (0.440) \\
u_y & = 2.94\mbox{ ms}^{-1} \\
\end{align*}

\subsection{Medium Setting}

\begin{align*}
v_y^2 & = u_y^2 + 2 a_y s_y \\
0 & = u_y^2 + 2 (-9.8) (1.06) \\
u_y & = 4.56\mbox{ ms}^{-1} \\
\end{align*}

\end{document}
