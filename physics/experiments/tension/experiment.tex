
\documentclass[a4paper,11pt]{article}

% Math symbols
\usepackage{amsmath}

% Tables
\usepackage{tabularx}
\usepackage{multirow}
\newcolumntype{Y}{>{\centering\arraybackslash}X}

% No indent on new paragraphs
\setlength{\parindent}{0mm}

\begin{document}

\title{Equilibrium Experiment}
\author{Ben Anderson}
\date{\today}
\maketitle


\section{Results}

Mass of aluminium beam: 0.0827 kg

\subsection{Part 1}

$$
\begin{aligned}
h & = 0.430\mbox{ m} \\
x & = 0.490\mbox{ m} \\
r & = 0.455\mbox{ m} \\
s & = 0.300\mbox{ m} \\
\end{aligned}
$$

\begin{center}
\begin{tabular}{c|c}
Mass (kg) & Tension (N) \\
\hline
0.00  & 0.7 \\
0.100 & 1.8 \\
0.200 & 3.7 \\
0.300 & 5.5 \\
0.400 & 7.4 \\
0.500 & 9.4 \\
0.600 & 11.2 \\
\end{tabular}
\end{center}


\subsection{Part 2}

Mass of load: 0.500 kg

Mass of counterbalance: 0.300 kg

$$
\begin{aligned}
h & = 0.430\mbox{ m} \\
d & = 0.045\mbox{ m} \\
\end{aligned}
$$


\subsubsection{Trial 1}

$$
\begin{aligned}
r & = 0.455\mbox{ m} \\
s & = 0.455\mbox{ m} \\
\end{aligned}
$$

\begin{center}
\begin{tabular}{c|c}
$x$ (m) & Tension (N) \\
\hline
0.145 & 1.9 \\
0.200 & 2.5 \\
0.290 & 3.6 \\
0.365 & 4.6 \\
0.455 & 5.7 \\
0.505 & 6.2 \\
0.550 & 6.7 \\
0.605 & 7.3 \\
0.702 & 8.4 \\
\end{tabular}
\end{center}


\subsubsection{Trial 2}

$$
\begin{aligned}
r & = 0.300\mbox{ m} \\
s & = 0.300\mbox{ m} \\
\end{aligned}
$$

\begin{center}
\begin{tabular}{c|c}
$x$ (m) & Tension (N) \\
\hline
0.160 & 2.6 \\
0.205 & 2.8 \\
0.232 & 3.0 \\
0.286 & 3.5 \\
0.345 & 4.3 \\
0.403 & 5.0 \\
0.448 & 5.4 \\
0.505 & 6.1 \\
0.577 & 6.5 \\
\end{tabular}
\end{center}


\section{Analysis}

\subsection{Part 1}

\subsubsection{Equation}

$$
\begin{aligned}
\sum \tau & = 0 \\
\sum \tau_{cw} & = \sum \tau_{acw} \\
\tau & = r F \sin{\theta} \\
\\
T s \sin{\theta} & = m_b g a + m_l g r \\
T s \frac{h}{x} & = m_b g a + m_l g r \\
\end{aligned}
$$

Where $m_b$ is the mass of the beam and $m_l$ is the mass of the load hung from the beam.


\subsubsection{Gradient}

$$
\begin{aligned}
\mbox{gradient} & = \frac{10.4 - 1}{0.567 - 0.04} \\
	& = 17.8\mbox{ N kg}^{-1} \\
\mbox{y-intercept} & = 0.3\mbox{ N} \\
\end{aligned}
$$

Rearranging the equation above for $T$:

$$
\begin{aligned}
T s \frac{h}{x} & = m_b g a + m_l g r \\
T & = \frac{m_b g a x + m_l g r x}{h s} \\
T & = \frac{g r x}{h s} m_l + \frac{m_b g a x}{h s} \\
\mbox{gradient} & = \frac{g r x}{h s} \\
\mbox{y-intercept} & = \frac{m_b g a x}{h s} \\
\end{aligned}
$$

Solving for $a$:

$$
\begin{aligned}
\mbox{y-intercept} & = \frac{m_b g a x}{h s} \\
0.3 & = \frac{0.0827 \times 9.8 \times a \times 0.49}{0.43 \times 0.3} \\
a_1 & = \frac{0.3 \times 0.43 \times 0.3}{0.0827 \times 9.8 \times 0.49}
a_1 & = 0.0975\mbox{ m from the pivot} \\
\end{aligned}
$$

Hence the centre of mass of the beam is 9.75 cm to the left of the pivot.


\subsection{Part 2}

\subsubsection{Equation}

$$
\begin{aligned}
\sum \tau & = 0 \\
\sum \tau_{cw} & = \sum \tau_{acw} \\
\tau & = r F \sin{\theta} \\
\\
T s \sin{\theta} + m_c g d \sin{\phi} & = m_b g a \sin{\phi} + m_l g r \sin{\phi} \\
T s \frac{\sin{\theta}}{\sin{\phi}} + m_c g d & = m_b g a + m_l g r \\
T s \frac{h}{x} + m_c g d & = m_b g a + m_l g r \\
\end{aligned}
$$

Where $m_b$ is the mass of the beam, $m_l$ is the mass of the load hung from the
beam, and $m_c$ is the mass of the counterbalance.


\subsubsection{Gradient for Trial 1}

$$
\begin{aligned}
\mbox{gradient} & = \frac{10.2 - 1}{0.85 - 0.065} \\
	& = 11.7\mbox{ N m}^{-1} \\
\mbox{y-intercept} & = 0.3\mbox{ N} \\
\end{aligned}
$$

Rearranging the above equation for $T$:

$$
\begin{aligned}
T s \frac{h}{x} + m_c g d & = m_b g a + m_l g r \\
T s \frac{h}{x} & = m_b g a + m_l g r - m_c g d \\
T & = \frac{m_b g a + m_l g r - m_c g d}{h s} x \\
\mbox{gradient} & = \frac{m_b g a + m_l g r - m_c g d}{h s} \\
\end{aligned}
$$

Although our line of best fit has a positive y intercept, we can see from the
above formula for $T$ that the y intercept should be zero. This could be due to
any number of errors in the experiment.

Solving for $a$:

$$
\begin{aligned}
\mbox{gradient} & = \frac{m_b g a + m_l g r - m_c g d}{h s} \\
11.7 & = \frac{0.0827 \times 9.8 \times a_2 + 0.5 \times 9.8 \times 0.455 - 0.3 \times 9.8 \times 0.045}{0.43 \times 0.455} \\
a_2 & = \frac{11.7 \times 0.43 \times 0.455 + 0.3 \times 9.8 \times 0.045 - 0.5 \times 9.8 \times 0.455}{9.8 \times 0.0827} \\
a_2 & = 0.237\mbox{ m} \\
\end{aligned}
$$

The centre of mass of the beam is 23.7 cm to the left of the pivot.


\subsubsection{Gradient for Trial 2}

$$
\begin{aligned}
\mbox{gradient} & = \frac{9.4 - 1}{0.85 - 0.2} \\
& = 12.9\mbox{ N m}^{-1} \\
\mbox{y-intercept} & = 0.84\mbox{ N} \\
\end{aligned}
$$

Similarly as above, we record a positive y intercept, when in fact it should be
zero in accordance with our equation for $T$.

Solving for $a$ using the same equation for $T$ and for the gradient as above:

$$
\begin{aligned}
\mbox{gradient} & = \frac{m_b g a + m_l g r - m_c g d}{h s} \\
12.9 & = \frac{0.0827 \times 9.8 \times a_3 + 0.5 \times 9.8 \times 0.3 - 0.3 \times 9.8 \times 0.045}{0.43 \times 0.3} \\
a_3 & = \frac{12.9 \times 0.43 \times 0.3 + 0.3 \times 9.8 \times 0.045 - 0.5 \times 9.8 \times 0.3}{9.8 \times 0.0827} \\
a_3 & = 0.403\mbox{ m} \\
\end{aligned}
$$

The centre of mass of the beam is 40.3 cm to the left of the pivot.


\subsection{Average Centre of Mass}

$$
\begin{aligned}
a & = \frac{a_1 + a_2 + a_3}{3} \\
& = \frac{0.403 + 0.237 + 0.0975}{3} \\
& = 0.246\mbox{ m} \\
\end{aligned}
$$

The average centre of mass of the beam from the three calculated values is
24.6 cm to the left of the pivot.

\section{Questions}

\subsection{Question 1}

\begin{itemize}
\item The force measuring device measures the net force acting in the vertical direction
\item There are two tension forces exerted on the device
\item In order to measure the tension in the string, we must only measure one of the two tension forces
	exerted on the force measuring device
\item This is possible if we hold the string parallel to the table, where the second tension thus has no
	component acting in the vertical direction (see diagram 1)
\end{itemize}

\subsubsection{Part A}

\begin{itemize}
\item If the string is held above the horizontal (ie. pulled upwards), there will be a vertical component
	of the second tension force acting upwards
\item This increases the net force in the vertical direction (see diagram 2)
\item Thus the force reading will increase
\end{itemize}

\subsubsection{Part B}

\begin{itemize}
\item If the string is held below the horizontal (ie. pulled downwards), the vertical component
of the second tension acts downwards
\item This reduces the net force in the vertical direction (see diagram 3)
\item Thus the force reading will decrease
\end{itemize}


\subsection{Question 2}

\subsubsection{Part A}

\begin{itemize}
\item As the mass of the load is increased, the weight force exerted on the beam by the load increases
\item The mangitude of the anticlockwise torque exerted on the beam will increase, since $\tau = r F \sin{\theta}$
\item Thus a larger clockwise torque is required to keep the beam in rotational equilibrium
\item The only force acting on the beam that provides a clockwise torque is the tension in the string
\item Thus the tension in the string must increase to increase the clockwise torque to maintain rotational equilibrium
\end{itemize}

\subsubsection{Part B}

\begin{itemize}
\item Moving the load closer to the pivot decreases its radius from the pivot
\item Since $\tau = r F \sin{\theta}$, this decreases the anticlockwise torque
exerted on the beam by the load, decreasing the total anticlockwise torque exerted on the beam
\item Thus a smaller clockwise torque is required to keep the beam in rotational equilibrium
\item The only force acting on the beam that provides the clockwise torque is the tension in the string
\item Thus the tension in the string will decrease to maintain rotational equilibrium
\end{itemize}


\subsection{Question 3}

As the beam is raised, $\theta$ increases and $\phi$ decreases.


\subsection{Question 4}

Consider our equation for tension, derived again below:

$$
\begin{aligned}
\sum \tau & = 0 \\
\sum \tau_{cw} & = \sum \tau_{acw} \\
\tau & = r F \sin{\theta} \\
\\
T s \sin{\theta} + m_c g d \sin{\phi} & =  m_b g a \sin{\phi} + m_l g r \sin{\phi} \\
T s \frac{h}{x} + m_c g d & = m_b g a + m_l g r \\
T s \frac{h}{x} & = m_b g a + m_l g r - m_c g d \\
T & = \frac{m_b g a + m_l g r - m_c g d}{h s} x \\
\end{aligned}
$$

\begin{itemize}
\item As the beam is raised, the distance between the point at which the string is attached to the beam and the
	point at which it makes contact with the pulley (ie. the distance $x$) decreases
\item In the above equation, the coefficient of $x$ is a constant
\item Hence $T$ is directly proportional to $x$ (ie. $T \propto x$)
\item Thus when raising the beam, decreasing $x$, the tension in the string must also decrease to maintain rotational equilibrium
\end{itemize}

% \begin{itemize}
% \item As the beam is raised, the angle between the string and the line of action ($\theta$) increases
% \item This increases the magnitude of the component of the tension force that is perpendicular to the line of action
% \item Thus less total tension force is required to provide the same amount of clockwise torque
% \item In addition, the angle $\phi$ between the weight force of the load and beam, and the line of action, decreases
% \item This decreases the component of both weight forces that is perpendicular to the line of action
% \item This decreases the total anticlockwise torque acting on the beam
% \item Since the beam is in rotational equilibrium, this should decrease the tension in the string
% \end{itemize}

% We must also consider the counterbalance on the beam:

% \begin{itemize}
% \item As the beam is raised, the angle between the counterbalance's weight force
% 	and the line of action ($\theta$) decreases
% \item This decreases the component of the counterbalance's weight force that is perpendicular to the line of action
% \item This decreases the clockwise torque exerted on the beam by the counterbalance
% \item Since the only other source of clockwise torque is the tension in the string,
% 	this will increase the torque that must be provided by the tension to maintain rotational equilibrium
% \item The magnitude of the counterbalance's weight force is $0.3 \times 9.8$ N, whereas the magnitude of
% 	the beam's and load's weight force is $(0.5 + 0.0827) \times 9.8$ N
% \item Since they both decrease by the same factor as the beam is raised ($\sin{\theta}$), and since the
% 	counterbalance's weight force is less than that of the beam's and load's, the amount by which the anticlockwise
% 	torque produced by the beam and load decreases is greater than the amount by which the counterbalance's clockwise
% 	torque decreases, as $\tau = r F \sin{\theta}$
% EXPAND HERE
% \item Thus tension in the string will still decrease
% \end{itemize}


\subsection{Question 5}

\begin{itemize}
\item The total anticlockwise torque exerted on the beam is composed of the anticlockwise torque
	produced by the weight force of the load attached to the beam, and the weight force of the beam itself
\item To maintain rotational equilibrium, this anticlockwise torque must be opposed by an equal clockwise torque
\item This clockwise torque is composed of the clockwise torques produced by the tension in the string, and the weight force of the included counterbalance
\item Thus part of this total clockwise torque is provided by the counterbalance
% \item Since the torque produced by the counterbalance's weight force and the tension force act in the same direction,
% 	the torque produced by the counterbalance provides part of this total clockwise torque
\item This reduces the required clockwise torque the tension in the string must produce
\item Since $\tau = r F \sin{\theta}$, this decreases the required tension in the string
\item Thus the tension in the string decreases with the inclusion of the counterbalance
% \item The inclusion of the counterbalance decreases the amount of torque that the tension in the string must produce in order to maintain rotational equilibrium
% \item This is because the torque produced by the counterbalance and by the tension in the string act in the same direction
% \item Thus the tension in the string required to raise the beam is lower when the counterbalance is included
\end{itemize}


\section{Conclusion}

We have successfully achieved our aim of investigating the tension in a string
attached to a metal bar as the bar is raised, and locating the centre of mass of
this bar. The three values for the centre of mass of the bar we calculated were
9.75 cm, 23.7 cm, and 40.3 cm to the left of the pivot. The average of these
three values, which constitutes our final caculated centre of mass of the beam,
was 24.6 cm to the left of the pivot.

\end{document}
