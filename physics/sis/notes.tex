\documentclass[a4paper,11pt]{article}

% Math symbols
\usepackage{amsmath}
\usepackage{amsfonts}
\usepackage{esvect}

% No indent on new paragraphs
\setlength{\parindent}{0mm}
\setlength{\parskip}{0.2cm}

% Alias \boldsymbol to \bb for vectors
\newcommand{\bb}{\boldsymbol}


\begin{document}

\title{Science Inquiry Skills}
\author{Ben Anderson}
\date{\today}
\maketitle

\pagebreak



\section{Errors}

\subsection{Accuracy}

Accuracy is how close a value is to the accepted value.


\subsection{Precision}

Precision is how spread out a set of values are. Values with lower uncertainty
are more precise.

If a set of results is precise, it does not necessarily mean they are close to
the accepted value.


\subsection{Reliability}

A set of results is more reliable if it can be readily reproduced. More trials
and lower percentage error make a set of results more reliable.


\subsection{Systematic Error}

Error intrinsic to measurement devices and instruments, which cause results to
be consistently higher or lower than the accepted value by a near constant
amount.


\subsection{Random Error}

Error that causes the results to appear more scattered, to vary more
significantly between trials. Doesn't cause values to be consistenly higher or
lower than the accepted value.




\section{Uncertainties}

Uncertainty gives a range that a value may be between, for example:

$$
20.0 \pm 0.1
$$

The uncertainty must be rounded so that it is in the last significant figure of
the value.


\subsection{For a Measurement}

The uncertainty for a value read off a measuring device is half the smallest
scale step.

We must read a value on a measuring device to 1 significant figure more than its
smallest scale step.

For example, if we measure 20.00 cm using a ruler with 1 mm increments, the
value with its uncertainty is:

$$
20.00 \pm 0.05\text{ cm}
$$


\subsection{Absolute Uncertainty}

The uncertainty associated with a value in the same units as the value.


\subsection{Relative Uncertainty}

Also called fractional uncertainty.

Calculated as:

$$
\frac{\text{uncertainty}}{\text{value}}
$$


\subsection{Percentage Uncertainty}

The fractional uncertainty multiplied by 100.

Calculated as:

$$
\frac{\text{uncertainty}}{\text{value}} \times 100\%
$$


\subsection{Percentage Difference}

The difference between a value and the associated accepted value measured as a
percentage of the accepted value.

Calculated as:

$$
\frac{\text{measured value} - \text{accepted value}}{\text{accepted value}} \times 100\%
$$




\section{Significant Figures}

\subsection{Addition and Subtraction}

Give the answer to the least number of decimal places used in one of the values
in the operation.


\subsection{Multiplication and Division}

Give the answer to the least number of significant figures used in one of the
values in the operation.




\section{Operations with Uncertainties}

\subsection{With a Constant}

The uncertainty does not change when adding, subtracting, multiplying, or
dividing by a constant.


\subsection{Over a Number of Repetitions}

If a value is measured over a number of repetitions (eg. the time for 20 swings
of a pendulum), then the error associated with the result is divided by the
number of repetitions.


\subsection{Addition and Subtraction}

Add the absolute uncertainties.


\subsection{Multiplication and Division}

Add the fractional uncertainties and convert back to an absolute uncertainty by
multiplying by the final rounded value.




\section{Linearising Graphs}

Manipulate the given data to give an equation in the form $y = mx + c$, allowing
us to plot a linear graph.


\subsection{Polynomials}

For the relation $A = mB^2 + C$, plot a graph of $A$ vs. $B^2$, which has a
gradient of $m$ and y intercept of $C$.

For the relation $A^3 = mB^4 + C$, plot a graph of $A^3$ vs. $B^4$, which has a
gradient of $m$ and y intercept of $C$.


\subsection{Hyperbolas}

For the relation $A = m \frac{1}{B} + C$, plot a graph of $A$ vs. $\frac{1}{B}$.


\subsection{Square Roots}

For the relation $A = m \sqrt{B} + C$, plot a graph of $A$ vs. $\sqrt{B}$.


\subsection{Units of Gradient}

The units of the gradient of a graph is the units on the y axis divided by the
units on the x axis.

\end{document}
