
\documentclass[a4paper,11pt]{report}

% Math symbols
\usepackage{amsmath}
\usepackage{amsfonts}
\usepackage{esvect}

% Hyperlink contents page
\usepackage{hyperref}
\hypersetup{
	colorlinks,
	citecolor=black,
	filecolor=black,
	linkcolor=black,
	urlcolor=black
}

% No indent on new paragraphs
\setlength{\parindent}{0mm}
\setlength{\parskip}{0.2cm}

% Alias \boldsymbol to \bb for vectors
\newcommand{\bb}{\boldsymbol}


\begin{document}

\title{Physics Notes}
\author{Ben Anderson}
\date{\today}
\maketitle
\pagebreak

\tableofcontents
\pagebreak




\chapter{Motion}

\section{Vectors}

A value with both magnitude and direction.

Diagramatically represented as an arrow.

\subsection{Components}

A vector can be split into its horizontal or vertical components:

$$
\begin{aligned}
v_x & = v \cos{\theta} \\
v_y & = v \sin{\theta} \\
\end{aligned}
$$

\subsection{Negation}

Reverses the direction of the vector.

Negate each component separately.

\subsection{Addition}

Place two vectors end to end and determine magnitude and direction of
resultant.

To solve, either:

\begin{enumerate}
\item Use the cosine rule to find the magnitude, and sine rule for the
	direction.
\item Split both vectors into horizontal and vertical components, then add
	corresponding components.
\end{enumerate}

\subsection{Subtraction}

Negate the second vector, then add them as above.


\section{Motion}

\subsection{Displacement}

Symbol: $s$

Units: metres ($m$)

Requires a direction.

A vector from the origin to an object's position.

\subsection{Velocity}

Symbol: $v$

Units: $\mbox{ms}^{-1}$

Requires a direction.

The rate of change of displacement with respect to time:

$$
v = \frac{\Delta s}{\Delta t}
$$

\subsection{Change in Velocity}

The final velocity of an object, subtract its initial velocity:

$$
\Delta v = v_f - v_i
$$

Use vector subtraction (as above) to find the correct direction.

\subsection{Relative Velocity}

The velocity of $A$ relative to a stationary observer $B$ is:

$$
v_{a/b}
$$

This can be calculated using the velocities of $A$ and $B$ relative to a third,
stationary observer $C$:

$$
v_{a/b} = v{a/c} + v{c/b}
$$

\subsubsection{River Problem}

Description:

\begin{itemize}
\item You stand at $A$ on the side of a river 30 m wide.
\item You must reach $B$ on the opposite side of the river, 10 m downstream.
\item You can swim with a velocity of $2\mbox{ ms}^{-1}$.
\item The river has a current with velocity of $3\mbox{ ms}^{-1}$
\end{itemize}

Questions:

\begin{itemize}
\item At what angle with the shore should you swim to reach $B$?
\item What is your speed relative to the Earth?
\item How long will it take to cross the river?
\end{itemize}

Solution:

\begin{itemize}
\item The direction of your resultant velocity must be towards $B$.
\item Your velocity added to the velocity of the river's current forms a
	triangle.
\item Solve for the required sides and angles using various trigonometric rules.
\end{itemize}

\subsection{Acceleration}

Symbol: $a$

Units: $\mbox{ms}^{-2}$

Requires a direction.

The rate of change of velocity with respect to time:

$$
a = \frac{\Delta v}{\Delta t}
$$


\section{Equations of Motion}

$$
\begin{aligned}
s & = vt \\
v & = u + at \\
v^2 & = u^2 + 2as \\
s & = ut + \frac{1}{2}at^2 \\
\end{aligned}
$$

For a constant acceleration.


\section{Momentum}

$$
p = mv
$$

Symbol: $p$

Units: $\mbox{kg m s}^{-1}$

Requires a direction.

\subsection{Conservation of Momentum}

Assuming no energy loss to the surroundings, the total momentum of a system is
conserved before and after a collision:

$$
\sum p_i = \sum p_f
$$


\section{Energy}

Symbol: $E$

Units: Joules ($J$)

Scalar quantity.

\subsection{Kinetic Energy}

$$
E_k = \frac{1}{2}mv^2
$$

Energy due to an object's motion.

\subsection{Gravitational Potential Energy}

$$
E_p = mgh
$$

Energy due to an object's height above a fixed reference point (usually the
ground).

\subsection{Conservation of Energy}

The total energy in a system is conserved:

$$
\sum E_i = \sum E_f
$$


\section{Force}

Symbol: $F$

Units: Newtons ($N$)

Requires a direction.

Anything that changes an object's acceleration.

\subsection{Weight}

Symbol: $W$ or $mg$

Force due to gravity.

Acts from the centre of mass towards the centre of the Earth (usually straight
down).

\subsection{Normal}

Symbol: $F_N$

When two objects come into contact.

\subsection{Tension}

Symbol: $T$

A restoring force due to intermolecular attractions between particles that make
up an object.

Acts parallel to an object.

\subsection{Friction}

Symbol: $F_f$

A force that retards the motion of an object.

Acts along a surface, opposite to the direction of the object's motion.

Static friction is the force required to start moving an object from rest:

$$
F_f = \mu F_N
$$

Where $\mu$ is the coefficient of static friction for the surface on which the
object is moving across.

Kinetic friction is the force required to overcome the frictional force while
an object is moving, to maintain a constant velocity.

\subsection{Newton's Laws}

\subsubsection{First Law}

Every object will continue in its current state of rest or straight line,
uniform motion unless acted upon by a net, non-zero, external force.

Uniform means the magnitude of the object's velocity isn't changing.

Straight line means the direction of the object's velocity isn't changing.

\subsubsection{Second Law}

$$
\sum F = ma
$$

The net force exerted on an object is equal to the object's mass multiplied by
its acceleration.

\subsubsection{Third Law}

If one object exerts a force on another, the second will also exert a force on
the first equal in magnitude but opposite in direction.


\section{Dynamics}

For objects undergoing a constant acceleration.

\subsection{Method}

The general approach to solving dynamics problems:

\begin{enumerate}
\item Split the forces acting on a body into horizontal and vertical components.
\item Equate each component to $ma$, using the acceleration of the object along
	each axis.
\item Solve for unknowns.
\end{enumerate}


\section{Statics}

For objects at a constant velocity or at rest.

\subsection{Method}

The general approach to solving statics problems:

\begin{enumerate}
\item Split the forces acting on a body into horizontal and vertical components.
\item Equate each component to 0.
\item Solve for unknowns.
\end{enumerate}


\section{Projectile Motion}

For an object fired at an angle to the ground into the air.

Assume:

\begin{itemize}
\item Gravity acts directly downwards.
\item No air resistance (horizontal velocity is constant).
\end{itemize}

\subsection{Variables}

Potential variables that may be given, or asked to be solved:

\begin{itemize}
\item Horizontal range
\item Flight time
\item Maximum height
\item Initial velocity
\item Final velocity when the object hits the ground
\item Velocity after $t$ seconds
\item Angle to the horizontal at which the object was fired
\item Height above the ground of the surface from which the object was fired
\item Height above the ground of the surface on which the object landed
\end{itemize}

\subsection{Flight Time}

To find the time the object is in the air before hitting the ground again, use:

$$
s_y = u_y t + \frac{1}{2} a_y t^2
$$

Set $s_y$ to the vertical displacement of the object relative to its initial
starting height at the point where it lands.

Solve using the quadratic formula.

\subsection{Range}

The range the projectile travels is equivalent to its horizontal displacement:

$$
s_x = u_x t
$$

\subsection{Maximum Height}

The vertical velocity of the projectile will be 0 at its maximum height, so
solve for $s_y$ using:

$$
u_y^2 + 2 a_y s_y = 0
$$

\subsection{Final Velocity}

Solve for each component of the final velocity separately, then combine them.

The horizontal component is the same as the initial horizontal velocity (since
there's no acceleration along the horizontal axis):

$$
v_x = u_x
$$

The vertical component is:

$$
v_y = u_y + a_y t
$$

\subsection{Air Resistance}

Effects of air resistance on the path of the projectile:

\begin{itemize}
\item Reduced range.
\item Reduced maximum height.
\item Asymmetrical path, where the downwards part of its flight is more stunted.
\end{itemize}

These effects are more pronounced for faster moving objects, since the
frictional force is proportional to the object's velocity.

The upwards part of the projectile's flight takes less time than the downwards
because:

\begin{itemize}
\item Air resistance and gravity both act downwards on the upwards part of the
	flight, instead of in opposite directions on the downwards part.
\item This causes the upwards part to take less time for the projectile to
	complete.
\end{itemize}


\section{Circular Motion}

For an object travelling in a circular path, at a radius from a central point.

\subsection{Period}

Symbol: $T$

Units: seconds ($s$)

Time taken for the object to complete one revolution.

\subsection{Frequency}

Symbol: $f$

Units: Hertz (Hz) or $\mbox{s}^{-1}$

The number of revolutions completed in 1 second.

Related to the period by:

$$
f = \frac{1}{T}
$$

\subsection{Velocity}

$$
\begin{aligned}
v & = \frac{2\pi r}{T} \\
& = 2\pi r f \\
\end{aligned}
$$

Tangential to the circular path, perpendicular to the radius.

Constant magnitude, but continuously changing direction.

\subsection{Centripetal Acceleration}

$$
\begin{aligned}
a & = \frac{v^2}{r} \\
& = \frac{4 \pi^2 r}{T^2} \\
\end{aligned}
$$

Directed towards the centre of the circle.

Constant magnitude, but continuously changing direction.

\subsection{Centripetal Force}

$$
F_c = \frac{mv^2}{r}
$$

Directed towards the centre of the circle.

The centripetal force isn't a new type of force, but is the resultant unbalanced
force acting on an object travelling in a circular path.

It is caused by other forces, such as a tension force in a string, or the normal
force for a car travelling around a corner.

\subsection{Horizontal Circular Motion}

For an object travelling in a horizontal circle, parallel to the ground, where
the weight force is perpendicular to the object's path.

The centripetal force will be supplied solely by the tension force in the
string.

\subsection{Vertical Circular Motion}

For an object travelling in a vertical circle, perpendicular to the ground. We
assume the average velocity of such an object to be constant.

The centripetal force will be provided by a combination of the tension force
in the string, and the object's weight force:

$$
F_c = T + W
$$

At the top of the object's path:

\begin{itemize}
\item Tension, weight, and the resultant centripetal force act down.
\item Tension is at a minimum, as the weight force is providing part of the
	required centripetal force.
\end{itemize}

At the bottom of the object's path:

\begin{itemize}
\item Tension and the resultant centripetal force act upwards, weight downwards.
\item Tension is at a maximum, as it must overcome the weight force and provide
	the required centripetal force.
\end{itemize}

Can also consider a rollercoaster or plane moving in a vertical circle, where
the centripetal force is provided by the normal force, instead of tension.

Applies when a car moves over a speed hump (decrease in normal force) and when
a car moves through a ditch (increase in normal force).

\subsubsection{Apparent Weightlessness}

Apparent weightlessness occurs when the normal force exerted on a person is 0.

TODO: Questions about spaceship

\subsubsection{Minimum Velocity}

There exists a certain velocity for a rollercoaster in a loop such that a
person would experience apparent weightlessness at the top of the loop, where
the normal force would be 0.

At the top of the loop, taking up to be positive:

$$
-F_c = -F_N - mg
$$

$F_N$ is 0, so:

$$
\begin{aligned}
\frac{mv^2}{r} & = mg \\
v^2 & = rg \\
v & = \sqrt{rg} \\
\end{aligned}
$$

\subsection{Conical Circular Motion}

When an object moves in a horizontal circle, making an angle with the vertical.

For example, a mass on the end of a string:

\begin{itemize}
\item Forms a cone when swung, with the string at an angle to the vertical.
\item The tension force is at an angle to the vertical, and has a horizontal
	component.
\item This horizontal component provides the centripetal force for the
	horizontal circular motion.
\end{itemize}

For example, a plane banking around a corner, travelling in a horizontal circle:

\begin{itemize}
\item The lift force, perpendicular to the wings, is at an angle to the
	vertical.
\item The horizontal component of the lift force provides the centripetal force
	for the circle.
\end{itemize}

For example, a car on a cambered (angled) track:

\begin{itemize}
\item The normal force, perpendicular to the cambered track, is at an angle to
	the vertical.
\item The horizontal component of the normal force provdies the centripetal
	force.
\end{itemize}

The upwards vertical component of these forces will counteract the weight force
of the object.

\subsubsection{Velocity}

For a specific angle with the vertical, the mass will have a certain velocity.

For a car on a cambered track, a certain velocity will allow it to travel
around the corner without any friction acting on the tyres (where $F_f$ along
the slope of the road is 0):

$$
\begin{aligned}
\sum F & = ma \\
\sum F_v & = 0 \\
F_N \cos{\theta} - mg & = 0 \\
F_N \cos{\theta} & = mg \\
\sum F_h & = \frac{mv^2}{r} \\
F_N \sin{\theta} & = \frac{mv^2}{r} \\
\end{aligned}
$$

Dividing the one equation by the other:

$$
\begin{aligned}
\frac{F_N \sin{\theta}}{F_N \cos{\theta}} & = \frac{mv^2}{rmg} \\
\tan{\theta} & = \frac{v^2}{rg} \\
\end{aligned}
$$

Solving for the velocity:

$$
v = \sqrt{rg \tan{\theta}}
$$


\section{Gravity}

\subsection{Gravitational Field}

A gravitational field exists at a point if a force is exerted on a particle
with mass when it is placed at the point.

Any particle with mass has a gravitational field around it.

\subsection{Gravitational Field Lines}

The conditions for drawing gravitational field lines around an object with mass:

\begin{itemize}
\item Symmetrical about every axis.
\item Arrows point towards centre of mass.
\item All arrows the same length.
\item Density of lines indicates the relative strength of the gravitational
	field.
\end{itemize}

\subsection{Universal Law of Gravitation}

$$
F_g = \frac{G m_1 m_2}{r^2}
$$

The force exerted on an object with mass $m_1$ by another object with mass
$m_2$, when they are distance of $r$ metres apart.

$G$ is the universal gravitation constant:

$$
G = 6.67 \times 10^{-11}\mbox{ N m}^2\mbox{ kg}^{-2}
$$

The force requires a direction. For the force exerted by $m_2$ on $m_1$, the
direction will be towards $m_2$.

\subsection{Gravitational Field Strength}

$$
a_g = \frac{G m}{r^2}
$$

Units: Newtons per metre ($\mbox{N m}^{-1}$), equivalent to $\mbox{ms}^{-1}$

The force exerted on an object per unit of mass by an object with mass $m_2$,
a distance of $r$ metres away.

Also called acceleration due to gravity ($g$).

\subsection{Centripetal Force from Planet's Rotation}

An object on Earth experiences a centripetal force from the Earth's rotation.

TODO: What was I going on about??

\subsection{Circular Orbit}

Assuming the path of an object orbiting a planet is circular, the centripetal
force is due solely to the planet's gravity:

$$
\begin{aligned}
F_g & = F_c \\
\frac{G m_1 m_2}{r^2} & = \frac{m_1 v^2}{r} \\
v^2 = \frac{G m_2}{r} \\
\end{aligned}
$$

The velocity of an orbiting object is constant for a particular altitude.

When the velocity of an orbiting object is increased:

\begin{itemize}
\item Increase in tangential velocity of the object increases the required
	centripetal force to sustain the current radius.
\item Only gravity provides the centripetal force, and it can't increase to
	compensate for the required centripetal force.
\item Object will increase in altitude, increasing the radius of orbit,
	decreasing the required centripetal force.
\item Increasing its altitude increases the object's gravitational potential
	energy.
\item Since energy must be conserved, the object's kinetic energy must
	decrease, decreasing its velocity.
\end{itemize}

\subsection{Escape Velocity}

The minimum velocity required by an object launched from an altitude of $r$
above a body to completely escape its gravitational field, ignoring any other
energy loss:

$$
v = \sqrt{\frac{2 G m_2}{r}}
$$

\subsection{Lagrangian Points}

A point in between two bodies where the gravitational force exerted by each is
equal in magnitude and opposite in direction.

An object at this point will not fall towards either body.

For two bodies of mass $m_1$ and $m_2$ separated by a distance $r$, where $x$
is the distance from body 1 to the Lagrangian point:

$$
\begin{aligned}
\frac{G m_1 m_b}{x^2} & = \frac{G m_2 m_b}{(r - x)^2} \\
\frac{m_1}{m_2} & = \frac{x^2}{(r - x)^2} \\
x & = \frac{r\sqrt{\frac{m_E}{m_m}}}{1 + \sqrt{\frac{m_E}{m_m}}} \\
\end{aligned}
$$

\subsection{Kepler's Laws}

\subsubsection{First Law}

The path of the planets around the sun are elliptical.

One foci of the ellipse is the sun.

\subsubsection{Second Law}

All planets orbiting the sun sweep out equal areas in equal times.

\subsubsection{Third Law}

Derives a relationship between an objects's period of orbit ($T$) and orbital
radius ($r$) around a planet with mass $m$:

$$
\begin{aligned}
F_g & = F_c \\
\frac{G m m_b}{r^2} & = \frac{m_b v^2}{r} \\
& = \frac{m_b 4 \pi^2 r}{T^2} \\
\frac{r^3}{T^2} & = \frac{G m}{4 \pi^2} \\
\end{aligned}
$$

\subsection{Geosynchronous Orbit}

Where the period of rotation of an object around a planet is the same as the
planet's rotational period (24 hours for Earth).

The altitude of this orbit can be derived using Kepler's third law (above),
since the mass of the Earth and the orbital period (24 hours) is known.

\subsection{Geostationary Orbit}

Where the period of rotation of an object is equal to the planet's rotational
period, and the object orbits around the equator.

This means the object is always in the same position in the sky relative to an
observer on Earth.

This is used for communication satellites, where radio waves are always aimed
at the same point in the sky.

\subsection{Barycentre}

Two planets will orbit each other around their centre of mass.

Most commonly:

\begin{itemize}
\item The mass of the planet is much larger than that of the orbiting object.
\item The centre of mass will be very close to the centre of the planet, making
	the difference negligible.
\item Only significant when the masses of both objects are similar.
\end{itemize}

For two objects with masses $m_1$ and $m_2$ a distance $r$ apart, the barycentre
is located $x$ metres from object 1:

$$
x = \frac{m_2}{m_1 + m_2} r
$$

\subsubsection{Kepler's Third Law}

Kepler's third law cannot be used as we assumed the radius of orbit was equal
to the distance the planets were apart.


\section{Torque}

Symbol: $\tau$

Units: Newton metres ($N m$)

Requires a direction (commonly clockwise or anticlockwise).

A rotational force about a pivot point.

For a force $F$ applied $r$ metres away from the pivot, at an angle of $\theta$
to the line of action, this produces the torque:

$$
\tau = rF \sin{\theta}
$$

\subsection{Centre of Mass}

Assume all objects are of uniform density, unless otherwise stated.

Thus an object's weight force acts from its centre of mass.

\subsection{Stability}

How prone an object is to falling over.

If an object's centre of mass moves outside it's base, it will fall over.

To increase stability:

\begin{itemize}
\item Lower the centre of mass.
\item Widen the object's base.
\end{itemize}

Both options increase the distance the centre of mass must be moved for it to
be outside the object's base, increasing its stability.

\subsection{Equilibrium}

Translational equilibrium is where:

$$
\begin{aligned}
\sum F_x & = 0 \\
\sum F_y & = 0 \\
\end{aligned}
$$

There is no net force acting on the object, and its acceleration remains
constant.

Rotational equilibrium is where:

$$
\sum \tau_{cw} = \sum \tau_{acw}
$$

Where $cw$ and $acw$ stand for clockwise and anticlockwise respectively.

There is no net change on an object's rotational acceleration.

\subsubsection{Types of Equilibrium}

The types of equilibrium can be demonstrated with a cylinder:

\begin{description}
\item [Stable] Cylinder stands on its circular base. Will not move if pushed.
\item [Neutral] Cylinder stands still on its side. Will move if pushed.
\item [Unstable] Cylinder balances on the edge of its circular base. Will fall
	over if pushed.
\end{description}

\subsection{Solving Torque Problems}

In torque problems, all objects will be in rotational and translational
equilibrium, so:

\begin{enumerate}
\item Set $\sum \tau_{cw} = \sum \tau_{acw}$ and solve for required unknowns.
\item Set $\sum F_y = 0$ and solve for vertical reaction force.
\item Set $\sum F_x = 0$ and solve for horizontal reaction force.
\end{enumerate}

\subsubsection{Perpendicular Distance}

Instead of finding the angle with the line of action, we can use the
perpendicular distance from where the force is acting to the pivot point.

There will be 2 perpendicular distances, one representing $\cos{\theta}$ and
the other $\sin{\theta}$.
Use the one representing $\sin{\theta}$.

\subsection{See Saw Problem}

Description:

\begin{itemize}
\item Multiple objects on a see saw at different distances from a pivot point.
\item Find the location of a missing mass in order to balance the system.
\item Find the mass of an object at a location in order to balance the system.
\end{itemize}

Variables:

\begin{itemize}
\item Mass of objects.
\item Distance of objects from pivot.
\item Location of pivot.
\item Reaction force at pivot.
\end{itemize}

Method:

\begin{itemize}
\item Set $\sum \tau = 0$ and solve for unknown mass/radius.
\item Set $\sum F_y = 0$ and solve for reaction force at pivot.
\end{itemize}

\subsection{Painter's Scaffolding Problem}

Description:

\begin{itemize}
\item Platform suspended by multiple cables.
\item Multiple masses placed on platform.
\item Solve for unknown masses, distances, or tension in cables.
\end{itemize}

Variables:

\begin{itemize}
\item Mass of objects.
\item Distances of objects from a side of the plank.
\item Length of plank (ie. distance of an ending cable from the left pivot).
\item Tension in cables.
\end{itemize}

Method:

\begin{itemize}
\item Chose one cable to act as the pivot point.
\item Set $\sum \tau = 0$ and solve for unknown tension.
\item Set $\sum F_y = 0$ and solve for other tension.
\end{itemize}

\subsection{Ladder Problem}

Description:

\begin{itemize}
\item Ladder resting against a wall at an angle.
\item Weight force acts down, exerting a torque on the ladder.
\item Normal force provided by wall counteracts the torque produced by the
	weight.
\item Normal force acts perpendicular to wall.
\end{itemize}

Variables:

\begin{itemize}
\item Mass of ladder.
\item Length of ladder.
\item Angle made with wall.
\item Normal force produced by wall.
\item Friction force exerted by ground on base of ladder (horizontal component
	of reaction force).
\item Normal force produced by ground (vertical component of reaction force).
\end{itemize}

Method:

\begin{itemize}
\item Take the ladder's point of contact with ground as the pivot point.
\item Set $\sum \tau = 0$ and solve for the wall's normal force.
\item Set $\sum F_y = 0$ and solve for the normal force exerted by ground.
\item Set $\sum F_x = 0$ and solve for the friction force between the ladder
	and the ground.
\item Find the resultant reaction force.
\end{itemize}

\subsection{Crane Problem}

Description:

\begin{itemize}
\item Crane arm suspended at an angle above the horizontal.
\item Supported by a cable fixed to a vertical tower.
\item A load is attached at the end of the arm.
\end{itemize}

Variables:

\begin{itemize}
\item Mass of arm.
\item Mass of load.
\item Angle the arm makes with the horizontal.
\item Tension in the supporting cable.
\end{itemize}

Method:

\begin{itemize}
\item Set $\sum \tau = 0$ and solve for the unknown tension in the supporting
	cable.
\item Set $\sum F_y = 0$ and solve for the vertical reaction force.
\item Set $\sum F_x = 0$ and solve for the horizontal reaction force.
\end{itemize}

\subsection{Door Hinge Problem}

Description:

\begin{itemize}
\item Door attached to a door frame at two hinges.
\item Centre of mass acting downwards in the middle of the door.
\item Door attempts to rotate clockwise around lower hinge.
\item There is a reaction force directed left exerted on the upper hinge, and
	one exerted right on the lower hinge.
\end{itemize}

Variables:

\begin{itemize}
\item Mass of door
\item Dimensions of door
\item Location of hinges
\end{itemize}

Method:

TODO: I don't know how to do this one




\chapter{Electromagnetism}

\section{Charge}

Symbol: $q$

Units: Coulombs ($C$)

Charge is a fundamental property of a subatomic particle.

\subsection{Types}

There are two types of charge:

\begin{enumerate}
\item Positive
\item Negative
\end{enumerate}

A particle can also possess no charge (neutral).

\subsection{Quantised Nature}

Charge is quantised (exists in discrete units).

The charge on any particle is some integer multiple of the charge on an
electron.

1 electron has a charge of $1.60 \times 10^{-19}\mbox{ C}$.

1 Coulomb of charge contains $6.25 \times 10^{18}$ electrons.

\subsection{Repulsion and Attraction}

Like charges repel, and opposite charges attract.


\section{Electric Field Lines}

The requirements in drawing field lines around point charges:

\begin{itemize}
\item Symmetrical about every axis.
\item Arrows point from positive towards negative.
\item Density of lines around the charge indicates the relative strength of the
	electric field.
\end{itemize}

\subsection{Required Diagrams}

The required electric field line diagrams are:

\begin{itemize}
\item Single positive and negative point charges.
\item A positive and negative charge near each other.
\item Two positive and two negative charges near each other.
\item Two charged plates separated by a distance.
\end{itemize}


\section{Electricity}

\subsection{Electrical Potential Energy}

Work must be done to move two particles with like charges close together.

A charge gains electrical potential energy when work is done in moving it close
to another particle with like charge.

\subsection{Voltage}

$$
V = \frac{W}{q}
$$

Symbol: $V$

Units: Volts ($V$), equivalent to Joules per Coulomb ($JC^{-1}$)

The difference in potential energy between two points in a circuit.

\subsection{Current}

$$
I = \frac{q}{s}
$$

Symbol: $I$

Units: Amperes ($A$), equivalent to Coulombs per second ($Cs^{-1}$)

The rate of flow of charge through a certain point in a circuit.

\subsubsection{Conventional Current}

The direction of flow of positive charge in a circuit.

Assume all given current directions are conventional unless stated otherwise.

\subsubsection{Electron Current}

The direction of flow of negative charge (electrons) in a circuit.

Opposite to conventional current.

\subsection{Resistance}

Symbol: $R$

Units: Ohms ($\Omega$)

A measure of by how much a circuit or element in a circuit impedes the flow of
charge.

\subsection{Voltage Drop}

Voltage drop is how much energy per Coulomb of charge is lost over a segment of
a circuit.

This means that the electrons just before the entering the component have a
higher electrical potential energy than after they leave.

This difference in potential energy is equivalent to a drop in voltage across
the component.

\subsection{Ohms Law}

$$
V = IR
$$

Where $V$ is the voltage drop across a component of a circuit with resistance
$R$, when a current $I$ flows through it.


\section{Electric Fields}

\subsection{Eletric Field Strength}

$$
E = \frac{F}{q}
$$

Symbol: $E$

Units: Newtons per Coulomb ($NC^{-1}$)

Requires a direction.

An electric field exists at a certain point in space only when a point charge
$q$ experiences a force $F$ at this point.

The electric field strength at a point in space is defined as the force exerted
per unit of charge at that point.

\subsection{Electric Field Between Two Plates}

$$
E = \frac{V}{d}
$$

The eletric field strength between two metal plates separated by a distance of
$d$ metres, with a potential difference of $V$ across them.

\subsection{Force on a Charged Particle}

$$
F = \frac{1}{4 \pi \epsilon_0} \times \frac{q_1 q_2}{r^2}
$$

The force on a particle with charge $q_1$ positioned $r$ metres away from
another particle with charge $q_2$ is $F$.

The direction of this force will either be ``attraction" or ``repulsion", for
unlike or like charges respectively.

The electric field strength around a point charge decays with the square of
the distance from the point (inverse square law).

\subsection{Field Strength due to a Point Charge}

$$
E = \frac{1}{4\pi \epsilon_0} \times \frac{q}{r^2}
$$

The electric field strength a distance of $r$ metres away from a point charge
with a charge of $q$.

Derived from the equation for the force on a charged particle, when divided by
$q_2$.


\section{Magnetism}

\subsection{Permanent Magnets}

A permanent magnet is formed when the spin on a significant number of
electrons within a region of a piece of solid metal align.

The regions of aligned electron spin are called domains.

Only ferromagnetic metals can form permanent magnets (eg. iron, nickel,
cobalt).

Ferromagnetic metals can have their domains aligned when exposed to an external
magnetic field, becoming permanent magnets.

\subsection{Paramagnetism}

Where the domains in a metal temporarily align when in the presence of an
external magnetic field.
The domains align in the same direction as those of the external field.

Causes an attraction between the paramagnet and the external magnet.

\subsection{Diamagnetism}

Where the domains in a metal temporarily align when in the presence of an
external magnetic field.
The domains align in the opposite direction as those of the external field.

Causes a repulsion between the diamagnet and the external magnet.

Only occurs for certain metals below a critical temperature.


\section{Magnetic Field Lines}

\subsection{Convention}

Conventions for drawing magnetic field lines:

\begin{itemize}
\item Symmetrical about every axis.
\item Arrows point from North to South.
\item Density of lines indicates the relative field strength.
\end{itemize}

In addition to arrows, we can represent magnetic fields travelling into and
out of the page using the symbols:

\begin{description}
\item [Cross] Field lines pointing out of the page.
\item [Dot] Field lines pointing into the page.
\end{description}

\subsection{Required Diagrams}

The required magnetic field diagrams are:

\begin{itemize}
\item A single dipole magnet.
\item A single bar magnet
\item Two bar magnets placed side by side, where the North pole of each is at
	the top of the page
\item Two bar magnets placed side by side, where the North pole of one is placed
	next to the South pole of the other
\item Two bar magnets placed end to end, where the two North poles are closest
\item Two bar magnets placed end to end, where the North pole of one is closest
	to the South pole of the other
\item Magnetic field around a wire.
\item Magnetic field around two wires with current flowing in either the same
	or opposite directions (showing an attraction or repulsion between the
	wires).
\end{itemize}

\subsection{Magnetic Field Lines Around Earth}

The Earth acts as a large bar.

The South magnetic pole is at the geographic North pole.

The North magnetic pole is at the South geographic pole.

\subsubsection{Direction of Field Lines}

Above the equator, the Earth's magnetic field lines have no vertical component.
They point directly towards the North geographic pole.

At the North geographic pole, the magnetic field lines point directly down.

At the South geographic pole, the magnetic field lines point directly up.

\subsubsection{Angle of Dip}

The angle above a tangent at a point on Earth's surface which the magnetic
field lines point.

\subsubsection{Angle of Declination}

The angle between the North-South magnetic and geographic axes.


\section{Magnetic Fields}

\subsection{Magnetic Field Strength}

Symbol: $B$

Units: Tesla ($T$)

Requires a direction.

A magnetic field exists at a certain point in space only when a permanent
magnet would experience a force at this point.

Also called magnetic flux density.

\subsection{Charge Moving Through a Magnetic Field}

A moving charge experiences a force when travelling through a magnetic field:

$$
F = qvB
$$

Use only the component of the magnetic field perpendicular to the particle's
velocity.

\subsubsection{Direction}

The force acts in a direction as determined by the right hand rule (ie. use
your right hand):

\begin{itemize}
\item Point fingers in the direction of the magnetic field.
\item Point thumb in the direction of conventional current.
\item Palm points in direction of force.
\end{itemize}

\subsection{Current in a Wire}

A current flowing through a wire in the presence of a magnetic field will
cause the wire to experience a force:

$$
F = BIl
$$

Only use the component of the magnetic field perpendicular to the wire.

The direction of the force is determined by the right hand rule.


\section{Induced Magnetic Field}

A moving charge in a wire will induce a circular magnetic field around the
wire.

\subsection{Direction}

The direction of the field is determined by the right grip rule (ie. use your
right hand):

\begin{itemize}
\item Point thumb in direction of conventional current.
\item The direction your fingers move when curled inwards is the direction of
	the circular induced magnetic field around the wire.
\end{itemize}

\subsection{Induced Magnetic Field Strength}

$$
B = \frac{\mu_0}{2\pi} \times \frac{I}{r}
$$

The magnetic field strength of an induced field around a wire with a current
$I$ flowing through it, a distance of $r$ metres away.

$\mu_0$ is the permeability of free space:

$$
\mu_0 = 4\pi \times 10^{-7}
$$

The direction of the magnetic field at a point can be found using the right
grip rule to determine the circular direction of the field, as above.

\subsection{In Two Wires}

In two wires side by side, the induced magnetic fields will either cause them
to repel or attract each other, depending on the direction of current.

For current travelling in the same direction (upwards):

\begin{itemize}
\item The induced magnetic field around each magnet will be anticlockwise when
	viewed from above (right grip rule).
\item For the wire on the left, the direction of the magnetic field induced by
	the current in the other wire will be out of the page.
\item This will exert a force towards the right on the wire (right hand rule).
\item Similarly, a force towards the left will be exerted on the wire on the
	right.
\item This will cause an attraction between the two wires.
\end{itemize}

\subsection{In a Solenoid}

A solenoid is a long circular coil of wire with a current running through it.

This creates a dipole electromagnet.

From the direction of flow of current, use the right grip rule to determine the
direction of the field lines around the solenoid.

Use the direction of field lines to determine the polarity of the solenoid
(which end is North and South).

\subsubsection{Soft Iron Core}

An iron core within a solenoid has the effect of greatly increasing the
strength of the magnetic field produced by the solenoid, as the magnetic
domains within the soft iron can easily align.


\section{Magnetic Flux}

\subsection{Magnetic Flux Density}

Equivalent to magnetic field strength.

\subsection{Magnetic Flux}

$$
\begin{aligned}
\phi & = BA_{\bot} \\
& = BA \cos{\theta} \\
\end{aligned}
$$

Symbol: $\phi$

Units: Tesla metres squared ($\text{Tm}^2$) or Webber ($\text{Wb}$)

The amount of magnetic field that passes through a given area perpendicular to
the magnetic field.

In the equation above, $\theta$ is the angle between the normal to the area
and the magnetic field.

\subsubsection{Maximum Flux}

When the area is perpendicular to the magnetic field (the normal is parallel).

\subsection{Change in Magnetic Flux}

$$
\Delta \phi = \phi_f - \phi_i
$$

Ways to change magnetic flux:

\begin{itemize}
\item Change the area in the magnetic field.
\item Move the area outside the magnetic field.
\item Change the angle between the area and the magnetic field.
\item Change the magnetic field strength.
\end{itemize}

\subsubsection{Direction}

To find the direction of a change in magnetic flux:

\begin{itemize}
\item Find the initial direction and rough magnitude of the magnetic flux.
\item Find the final direction and rough magnitude of magnetic flux.
\item Find the direction which would cause such a change.
\end{itemize}

For example, in moving the area outside of the magnetic field which is directed
into the page:

\begin{itemize}
\item Initial magnetic flux is into the page.
\item Final magnetic flux is 0 (no magnetic field).
\item A change in out of the page.
\end{itemize}

For example, increasing the area inside a magnetic field directed into the
page:

\begin{itemize}
\item Initial magnetic flux is a small into the page.
\item Final magnetic flux is a large into the page
\item A change in into the page.
\end{itemize}


\section{Induced EMF}

$$
\varepsilon = -n \frac{\Delta \phi}{\Delta t}
$$

Symbol: $\varepsilon$

Units: Volts ($V$)

Induced EMF is where the movement of a wire through a magnetic field will induce
a potential difference across the wire.

Magnetic flux must change per unit time to induce an EMF.

The negative in the equation implies that the direction of the induced EMF will
be opposite to the change in flux that caused it.

\subsection{Reason}

Induced EMF occurs because:

\begin{itemize}
\item A wire moving through a magnetic field represents the motion of charge
	through a magnetic field.
\item Thus a force is exerted on the electrons in the wire, moving a majority
	towards one end of the wire.
\item This induces a potential difference across the two ends of the wire.
\end{itemize}

For example, a wire moving left through a magnetic field directed into the
page:

\begin{itemize}
\item Positive charges are moving left.
\item Will experience a force up.
\item Positive charge will accumulate at the top of the wire.
\item Negative charge will accumulate at the bottom of the wire.
\item Induces a potential difference across the wire.
\end{itemize}

In a circuit with a changing magnetic flux, this potential difference will
induce a current in the wire.

\subsection{Faraday's Law}

States that the magnitude of the induced EMF is proportional to the change in
magnetic flux.

\subsection{Lenz's Law}

States that the direction of the induced EMF opposes that of the change in
magnetic flux.

\subsection{Wire}

For a wire moving with a constant velocity through a magnetic field, the induced
EMF is:

$$
\begin{aligned}
\phi & = BA \\
\varepsilon & = -\frac{\Delta \phi}{\Delta t} \\
& = -\frac{Blvt}{t} \\
& = Blv \\
\end{aligned}
$$

Where $v$ is perpendicular to the direction of $B$.


\section{Eddy Currents}

Where small circular spirals of moving charge are set up in metals that
experience a changing magnetic flux ($\Delta \phi$).

Only occurs in metals, as they have free moving electrons that are capable of
forming eddy currents.

\subsection{Energy Transformation}

Converts mechanical or kinetic energy in a moving magnetised object (which
causes a changing magnetic flux) into heat (increasing the kinetic energy of
particles in the metal).

As the velocity of the moving object increases, so does the changing magnetic
flux, increasing the magnitude of the eddy currents in the metal and the energy
loss.

\subsection{Magnet Through Aluminium Pipe}

As the magnet falls through an aluminium pipe:

\begin{itemize}
\item Sections of the pipe experience a changing magnetic flux.
\item Faraday's law states that the magnitude of the induced EMF in the pipe
	will be proportional to the changing magnetic flux.
\item Lenz's law states that the direction of the induced EMF in the pipe will
	be opposite to that of the change in flux.
\item This produces spirals of moving charge (eddy currents) in the pipe.
\item This exerts a magnetic repulsion force on the falling magnet opposite to
	its direction of motion.
\item This slows its descent and causes it to take longer to fall through the
	pipe.
\end{itemize}

\subsubsection{Direction of Eddy Current Spirals}

Consider a magnet falling through a pipe, with its North end closest to the
ground.

Just below the magnet:

\begin{itemize}
\item Magnetic field lines extend from below the magnet, then out of the page,
	then upwards parallel to the page.
\item Thus on the surface of the aluminium pipe, the magnetic field lines
	are out of the page.
\item This produces anticlockwise eddy currents (right screw rule).
\end{itemize}

Just above the magnet:

\begin{itemize}
\item On the surface of the aliminium pipe, the magnetic field lines are into
	the page.
\item This produces clockwise eddy currents (right screw rule).
\end{itemize}

\subsection{Induction Heating}

Induction heating works by:

\begin{itemize}
\item AC current is used in a solenoid to generate a changing magnetic field.
\item Metals placed near this changing magnetic field experience a changing
	magnetic flux.
\item This induces eddy currents in the metal, converting eletrical energy to
	heat, increasing the temperature of the metal.
\end{itemize}

Stovetops:

\begin{itemize}
\item Solenoid is placed below the stovetop, with a non conducting substance
	placed above it (which the pan sits on top of).
\item This non-conducting substance does not have free moving electrons,
	preventing eddy currents from forming in it due to the changing magnetic
	flux produced by the solenoid.
\item Thus the stovetop does not become hot.
\end{itemize}

\subsection{Reducing Eddy Currents}

Eddy currents can be reduced by laminating the metal:

\begin{itemize}
\item This involves cutting small slits in the metal.
\item This reduces the bulk of metal in which eddy currents can form.
\item Reduces loss of energy to heat as the formation of eddy currents is
	impeded.
\end{itemize}


\section{Basic DC Motor}

For a rectangular coil of wire in a permanent magnetic field, allowed to rotate
about a horizontal axis.

The two sides perpendicular to the magnetic field will experience opposing
forces (one up, one down).

The two sides parallel to the magnetic field will experience no force.

There is no net translational force, but there is an unbalanced rotational
torque.

\subsection{Torque}

The torque on the coil when in its initial position (where the force is
perpendicular to the line of action):

$$
\begin{aligned}
\tau & = rF \\
\sum \tau_{cw} & = \sum \tau_{acw} \\
\tau & = 2rF \\
& = 2rBIl \\
\end{aligned}
$$

Since $A$ (the area of the coil) is equivalent to $2rl$:

$$
\tau = BAI
$$

\subsubsection{Multiple Coils}

For a wire that has been coiled in a rectanglar shape $n$ times, each coil will
have the same force exerted on it.

For the force exerted on one side of the motor:

$$
F = BIln
$$

For the total torque exerted on the motor:

$$
\begin{aligned}
\sum \tau_{cw} & = \sum \tau_{acw} \\
\tau & = rF \\
& = 2r BILn \\
& = BAIn \\
\end{aligned}
$$

When rotated by an angle of $\theta$:

$$
\tau = BAIn \sin{\theta} \\
$$

\subsubsection{RMS Torque}

The root mean squared torque (average torque) is:

$$
= \frac{\tau}{\sqrt{2}}
$$

\subsection{Design Limitations}

The unbalanced rotational force exerts a torque on the motor, causing it to
rotate.

Once the coil passes $90^\circ$, the direction of applied torque will reverse
since the direction of force applied on each side of the coil doesn't change.

This torque opposes the angular velocity of the coil, causing it to rotate in
the opposite direction.

This stops the motor from continuing its motion.


\section{DC Motor}

The design limitations of the basic DC motor are resolved by flipping the
direction of current in the wire at $90^\circ$.

This swaps the direction of the forces exerted on the sides of the coil,
maintaining a constant direction of torque, allowing it to continue to rotate.

This is achieved using a split ring commutator.

\subsection{Split Ring Commutator}

Two C-shaped semicircles connected to the rotating coil and rotate with it.

Brushes fixed to the battery circuit come in contact with the moving commutators
to close the circuit.

At $90^\circ$, the brushes change which commutator they are in contact with,
reversing the current direction.

\subsection{Torque Curve}

A graph of torque against time.

Appears as the graph of $\lvert \cos{\theta} \rvert$ (absolute value of a
cosine graph).

The maximum torque in this graph will be when the forces exerted on the sides
of the coil is perpendicular to the line of action.

The average torque will be less than this maximum torque (drawn as a dotted,
horizontal line).

\subsection{Curving Permanent Magnets}

The permanent magnets supplying the magnetic field can be curved around the
circular path of the coils.

This causes the applied force to act perpendicular to the line of action for
a larger part of the coil's rotation.

This increases the average torque of the motor, but doesn't affect the maximum
torque.

The torque curve graph appears flatter at the peaks, more like a table.

\subsection{Multiple Coils}

Another split ring commutator pair and coil can be added, perpendicular to the
existing coil.

Only the commutator pair and coil at the optimal rotation angle will form a
circuit with the brushes.

Torque curve graph appears as two overlapping cosine graphs $90^\circ$ out of
phase.

Maximum torque remains the same, but increases the average torque.

\subsection{Increase Maximum Torque}

Increase any variables in the equation to increase the maximum torque of a DC
motor:

$$
\tau = B A I n \sin{\theta}
$$

\begin{itemize}
\item Increase magnetic field strength
\item Increase number of coils
\item Increase current in the circuit
\item Increase the width of the coil (radius from pivot)
\end{itemize}

Note that, for a constant voltage supply, increasing the length of the wire
would increase the resistance and decrease the current. This would result in
little change in the maximum torque.

\subsection{Increase Average Torque}

Ways to increase the average torque of a DC motor without changing the maximum
torque:

\begin{itemize}
\item Curve the permanent magnets to fit the circular rotation of the coils.
\item Add another commutator pair and coil.
\end{itemize}

\subsection{Servicing Motors}

Factors that could cause a decrease in the efficiency, average, or maxium
torque of a motor:

\begin{itemize}
\item Sparking when the brushes jump the small gap between commutators,
	causing carbon build up on the copper commutators, increasing the
	resistance and reducing the current through the circuit.
\item The carbon brushes may disintegrate, decreasing the time in one full
	revolution during which they are in contact with the commutators,
	decreasing the average current flow, decreasing maximum torque.
\end{itemize}


\section{AC Motors}

AC motors rotate at the same frenquecy as the oscillating current supplied.

Construction of the motor:

\begin{itemize}
\item Rectangular coil of wire free to rotate on an axis.
\item Stator coils (solenoids) are used instead of permanent magnets.
\item A split ring commutator pair is used at the end of the coil.
\end{itemize}

How it works:

\begin{itemize}
\item When the coil reaches $90^\circ$, the current in the wires swaps direction
	(since it is AC) for both the armature windings and stator coils.
\item This swaps the direction of the magnetic field produced by the stator
	coils.
\item The commutator pair swaps the direction of the current in the armature
	windings a second time.
\item This swaps the direction of the force, producing a continuous torque.
\end{itemize}

\subsection{Benefits Over DC Motors}

The benefits of using an AC motor over a DC one include:

\begin{itemize}
\item Using stator coils allows for much stronger magnetic fields than can be
	achieved with the permanent magnets in a DC motor. Allows for a higher
	maximum torque.
\item DC motors produce a rectified DC wave, with a lower average voltage.
	Average voltage for AC motors is not considered.
\end{itemize}


\section{Back EMF}

Back EMF:

\begin{itemize}
\item The rotation of the coil exposes the coil to a changing magnetic flux.
\item Faraday's law states that the magnitude of the induced EMF in the coil
	is proportional to this changing magnetic flux.
\item Lenz's law states that the direction of this induced EMF in the coil
	opposes that of the magnetic flux that induced it.
\item This reduces the net EMF in the coil, reducing the current flowing through
	the wire.
\item This decreases the torque of the motor.
\item As the angular velocity of the coil increases over time, so does the
	changing magnetic flux, further reducing the net EMF in the wire, causing
	the motor to eventually reach a constant RPM.
\end{itemize}

Back EMF applies to both DC and AC motors.

\subsection{Calculation}

After a period of time, the motor's back EMF will approach a constant:

$$
\begin{aligned}
V = \varepsilon - \varepsilon_b \\
I = \frac{\varepsilon - \varepsilon_b}{R} \\
\end{aligned}
$$

Where $V$ is the operating current, and $\varepsilon_b$ is the back EMF
induced in the coil.

This calculated current, incorporating the back EMF, is the operating current
of the motor.

\subsection{Damage}

Motors have a maximum load limit because:

\begin{itemize}
\item If a high load is placed on the motor, the velocity of the coil will
	decrease.
\item This reduces the magnitude of the changing magnetic flux, reducing the
	back EMF in the coil.
\item This increases the net voltage in the coil, increasing the current.
\item If this current exceeds the maximum current for which the motor was
	designed, the motor may be damaged.
\end{itemize}


\section{AC Generator}

Generators produce an AC current using induced EMF in a wire.

There is a retarding force exerted on the coil when the generator is turned due
to induced EMF (similar to back EMF in a motor).

\subsection{Workings}

\begin{itemize}
\item A coil, free to rotate on an axis, is manually turned.
\item The coil sits in a permanent magnetic field.
\item Each end of the coil is attached to a slip ring which rotates in contact
	with a carbon brush attached to a wire.
\item As the coil is rotated, this produces a changing effective area and thus
	a changing magnetic flux.
\item Faraday's law states that the induced EMF in the wire is proportional to
	this changing magnetic flux.
\item This produces an AC voltage in the wire.
\end{itemize}

\subsection{Voltage Graphs}

The graph of effective area against time is a sine curve offset depending on
the initial rotation of the coil.

The change in effective area against time (equivalent to the graph of change in
magnetic flux) is the derivative of the effective area graph.

The EMF graph is the negative of the change in flux graph (since Lenz's law,
where the direction of the induced EMF is opposite to that of the change in
flux).

\subsection{Maximum EMF}

When moving through the magnetic field, the coil traces out a changing area.
The change in magnetic flux over time is:

$$
\frac{\Delta \phi}{t} = 2vBl
$$

This is multiplied by 2 since there are 2 wires either side of the coil.

The induced EMF is:

$$
\varepsilon = 2nvBl
$$

Since the wire is moving in a circular path, $v = 2\pi r f$:

$$
\varepsilon = 2n 2 \pi r f B l
$$

Since the area of the coil is $A = 2 r l$:

$$
\varepsilon = 2\pi n BA f
$$

\subsection{Root Mean Squared EMF}

The root mean squared (RMS) EMF is:

$$
= \frac{\varepsilon}{\sqrt{2}}
$$


\section{DC Generator}

Same as an AC generator, but a split ring commutator pair as in a DC motor is
used, instead of slip rings.

This flips the direction of current at $90^\circ$, creating a recitified DC
wave.

\subsection{Increase Average Voltage}

We can increase the average output voltage the same way we increase the average
torque in a DC motor:

\begin{enumerate}
\item Curve magnets to apply a field perpendicular to the area of the coil for
	a greater portion of its rotation.
\item Increase the number of coils and commutator pairs.
\end{enumerate}

Both of these have the same effect on the voltage graph as they did on the
torque graph for a DC motor.

\subsection{Increase Maximum Voltage}

Increase any variables in the equation:

$$
\varepsilon = 2\pi n BA f
$$

\begin{itemize}
\item Increase number of coils.
\item Increase the strength of the magnetic field.
\item Increase the area of the coils.
\item Increase the frequency of rotation.
\end{itemize}


\section{Transformer}

Raises or lowers AC voltages in wires.

\subsection{Construction}

The construction of a transformer:

\begin{itemize}
\item A square loop of soft iron.
\item One side has a number of coils of wire (the primary coils), through which
	the initial AC current is passed.
\item The other side has a different number of coils (the secondary coils),
	in which an EMF is induced due to a changing magnetic field.
\end{itemize}

How it works:

\begin{itemize}
\item An AC current in the primary coils creates a changing magnetic field
	within the soft iron core.
\item The secondary coils experience a changing magnetic flux.
\item Faraday's law states that the induced EMF in the secondary coils will be
	proportional to this changing magnetic flux.
\item Since the rate of change of magnetic flux for both coils is the same,
	the ratio of the number of primary and secondary coils determines the
	factor by which the voltage is stepped up or down.
\end{itemize}

\subsection{Lamination}

Lamination is where:

\begin{itemize}
\item Soft iron core is cut into thin strips and laminated.
\item This reduces the amount of bulk metal in which eddy currents can form.
\item Reduces energy lost to resistive heating in the core
\end{itemize}

\subsection{Calculations}

$$
\frac{V_p}{V_s} = \frac{N_p}{N_s} = \frac{I_s}{I_p}
$$

\subsection{Power Loss}

$$
P_s = \mbox{efficiency} \times P_p
$$

Power will be lost due to heating of the primary or secondary wires, or eddy
currents in the soft iron core (also heating it).

The ratio of voltages will always remain constant no matter the efficiency of
the transformer.

Any loss in power will be expressed by a drop in current in the second coil
from what would exist in an ideal transformer.


\section{Electricity Transmission}

When electricity is travelling large distances, power is lost to heat in the
wires it travels through, according to:

$$
\begin{aligned}
V & = IR \\
P & = IV \\
P & = I^2 R \\
\end{aligned}
$$

Electricity should be transmitted at high voltage as:

\begin{itemize}
\item There will be a voltage drop across the wires as the wire has a
	resistance, and power will also be lost due to resistive heating.
\item Since the power at the generator is constant, a large voltage will
	result in a low current ($P = IV$).
\item This decreases voltage drop across the wires ($V = IR$) and decreases
	power lost to resistive heating ($P = I^2R$).
\end{itemize}

Transmitted as AC because:

\begin{itemize}
\item Transformers are used to step up and down the voltage in the wires.
\item Transformers require AC voltage as they must produce a changing magnetic
	field to induce an EMF in the secondary coils.
\end{itemize}


\section{Questions}

\subsection{Field Lines}

Draw electric or magnetic field lines around:

\begin{enumerate}
\item A single positive charge
\item A single negative charge
\item A positive and negative charge placed apart
\item Two positive charges placed apart
\item Two negative charges placed apart
\item A single bar magnet
\item Two bar magnets placed side by side, where the North pole of each is at
	the top of the page
\item Two bar magnets placed side by side, where the North pole of one is placed
	next to the South pole of the other
\item Two bar magnets placed end to end, where the two North poles are closest
\item Two bar magnets placed end to end, where the North pole of one is closest
	to the South pole of the other
\end{enumerate}

\subsection{Electric Fields}

\begin{enumerate}
\item Find the electric field strength of two plates placed apart
\item Find the acceleration of a charged particle placed between these two
	plates
\item Find the vertical displacement of a charged particle fired horizontally
	thorugh a pair of charged plates
\item Find the resultant force on 3 charged particles placed near each other
\end{enumerate}

\subsection{Magnetic Fields}

\begin{enumerate}
\item Find the force experienced by a charge moving through a magnetic field
\item Find force experienced by a wire with a current flowing through it when
	in the presence of a magnetic field
\item Find the magnetic flux density at a point a certain distance away from a
	wire with a current running through it
\item Find the force exerted on each of two wires placed next to each other,
	both with current running through them
\item Find the polarity of a solenoid
\end{enumerate}

\subsection{Induced EMF}

\begin{enumerate}
\item Explain the direction of an induced EMF in a coil
\item Find the induced EMF in a wire moving through a magnetic field with a
	velocity
\item Explain why a magnet dropped through an aluminium pipe falls slower than
	in a plastic pipe
\item Explain how induction heating works
\item Explain why the ceramic above the stovetop on which the pan is placed
	does not heat up
\end{enumerate}

\subsection{DC Motors}

\begin{enumerate}
\item Explain how a DC motor works
\item Calculate the maximum and RMS torque for a DC motor
\item Explain how to increase the maximum and average torque for this motor
\item Graph a torque curve of a DC motor, with curved magnets, and multiple
	coils
\item Explain factors that could decrease the efficiency of a DC motor over
	time
\item Explain why a DC motor approaches a constant RPM
\item Explain why placing a large load on a motor could damage it
\end{enumerate}

\subsection{AC Motors}

\begin{enumerate}
\item Explain how an AC motor works
\item Explain the benefits of using an AC motor over DC
\end{enumerate}

\subsection{AC Generators}

\begin{enumerate}
\item Explain how an AC generator works
\item Draw a voltage graph of an AC generator
\item Explain why there is a resistive force felt when the generator is turned
\end{enumerate}

\subsection{DC Generators}

\begin{enumerate}
\item Explain how a DC generator works
\item Explain how to increase the average and maximum output voltage of a DC
	generator
\end{enumerate}

\subsection{Transformers}

\begin{enumerate}
\item Explain how a transformer works
\item Explain the features of a transformer that reduce power loss
\item Calculate voltage transformations in transformers
\item Apply efficiency of a transformer to calculations
\item Explain why electricity is transferred as high voltage AC
\end{enumerate}




\chapter{Atomic Structure}

\section{Thompson}

The \textbf{Thompson} model of an atom postulated that:

\begin{itemize}
\item Each atom was a spherical particle that contained positive and negative
	charge.
\item These charges were distributed evenly throughout the particle such that
	the overall charge was neutral.
\item It was known that positive charge made up most of the mass of an atom.
\end{itemize}

\subsection{Flaw}

Rutherford performed an experiment that contradicted this model:

\begin{itemize}
\item Alpha particles ($\alpha^{2+}$) were fired at gold foil.
\item About 1 in every 20000 of these particles were deflected backwards
	towards the source from where they were fired.
\item This phenomena was impossible with the Thompson model.
\item Thus Rutherford concluded there was a nucleus consisting of dense positive
	charge with electrons orbiting it.
\end{itemize}


\section{Rutherford}

The \textbf{Rutherford} model postulated:

\begin{itemize}
\item An atom consists of a dense nucleus containing all positive charge.
\item The negatively charged electrons then orbit this nucleus in circular
	paths, under the influence of electrostatic attraction.
\end{itemize}

\subsection{Flaw}

\begin{itemize}
\item The electrons are undergoing centripetal acceleration around the nucelus,
	meaning they should be emitting energy in the form of electromagnetic waves.
\item This would cause the electrons to lose kinetic energy and spiral into the
	nucelus, rather than maintain stable orbits.
\end{itemize}


\section{Bohr}

The \textbf{Bohr} model postulated:

\begin{itemize}
\item Electrons move in a circular orbit about the nucelus under the influence
	of electrostatic attraction.
\item Electrons are only capable of moving in certain discrete orbits, where
	their angular momentum is an integer multiple of $h$.
\item The accelerating electrons do not emit electromagnetic waves. Their
	total energy is constant.
\item Electromagnetic radiation is emitted if an electron discontinuously
	changes the orbital it exists in.
\end{itemize}

\subsection{Orbital Transitions}

% TODO: Diagram of Bohr atom showing transitions between orbitals

When an electron moves from a higher orbital to a lower one, it emits an
electromagnetic wave with energy equal to the difference in energies between
each of the orbitals:

$$
E = hf = E_{n + 1} - E_n
$$

Certain transitions are more likely to occur than others.


\section{Emission and Absorption Spectra}

An \textbf{emission spectra} is where the electrons in an atom of a particular
element are excited to higher level energy states, then allowed to transition
back to lower ones. The emitted electromagnetic waves are then plotted as
colourful lines on a continuous spectrum (only for visible light).

An \textbf{absorption spectra} is where white light (consisting of all
frequencies of light) is passed through a cold gas of an element. This excites
electrons up to higher orbital levels, absorbing specific wavelengths of light.
This results in black bars on a continuous colourful spectra (only for visible
light).

Although these excited electrons will decay again, releasing these wavelengths
of light, it is very unlikely that the electron will emit the photon in the
direction of the detector, since there is an equal probability assigned to
every possible direction in which to emit the photon.

The black bars in an absorption spectra correspond exactly to the colourful
bars on an emission spectra.

The relative intensity of these bars corresponds to how likely a specific
transition is.

\subsection{Types of Spectra}

A \textbf{continuous spectrum} is one where all colours are outputted, with
no discrete elements.

A \textbf{band absorption spectrum} is a continuous spectrum with discrete bands
missing from the spectrum.

A \textbf{line emission spectrum} is a non-continuous spectrum with colourful
bands on a black background.

\subsection{Energy Level Diagram}

% TODO: Diagram of energy levels for Hydrogen

Each downwards transition corresponds to the emission of an electromagnetic
wave with a particular energy:

$$
E = hf = E_{n + 1} - E_n
$$

The frequency and wavelength of the emitted light can then be calculated.

Any emitted waves that fall in the visible part of the electromagnetic spectrum
will correspond to lines on an absorption/emission spectra.

\subsection{Fraunhofer Lines}

\textbf{Fraunhofer lines} are absorption/emission spectra for molecules, rather
than pure elements. They consist of bands spread across a range of frequencies,
rather than discrete lines.

\subsection{Photon Excitation}

If a photon is to excite an electron to a higher orbital level, it must have
exactly the amount of energy required by the electron to reach that higher
orbital level in order to be absorbed. The photon cannot impart only some of
its energy to an electron.

If a photon is to ionise an electron, it can have any amount of energy greater
than the ionisation energy. Any additional energy will be transferred to the
electron as kinetic energy.

\subsection{Electron Excitation}

\textbf{Electron excitation} is where another electron excites an orbital
electron within an atom to a higher energy state, rather than a photon.

Unlike a photon, an electron can impart a fraction of its energy to an orbital
electron, exciting it up to a higher state, while the incident electron
continues on with a lower $E_K$.

Similar to a photon, an electron can ionise an orbital electron, imparting
part of its energy. The incident electron will recoil with a lower $E_K$, and
the orbital electron will be ejected with some $E_K$.


\section{Fluorescence}

% TODO: Energy level diagram of cascading effect

\textbf{Fluorescence} is the emission of visible light by certain substances
after the excitation of orbital electrons from the absorption of radiation of
a higher frequency than visible light.

For a material to be able to fluoresce:

\begin{itemize}
\item Capable of absorbing radiation with a higher frequency than visible light.
\item Must have energy transitions that emit radiation with frequencies within
	the visible light part of the electromagnetic spectrum.
\end{itemize}

This is called a ``cascading" effect.

\subsection{Fluroescent Light}

% TODO: Diagram of light

A \textbf{fluorescent light} works by:

\begin{itemize}
\item A tube contains a low pressure gas.
\item Electrons are accelerated across a voltage through the tube.
\item The electrons collide with particles in the gas, exciting their orbital
	electrons, producing UV light.
\item This UV light is absorbed by a fluorescent coating on the outside of the
	tube.
\item The coating absorbs the UV light and emits visible light through the
	cascading effect.
\end{itemize}

Some UV light will still be produced.

\subsection{Fluorescing Dyes}

Laundry detergents and paper contain \textbf{fluorescing dyes}, which make
clothes and paper appear brighter (or cleaner) as they absorb ambient UV light
produced by artificial light and fluoresce.

LED lights do not emit UV light and thus fail to create this effect.

\subsection{Phosphorescence}

\textbf{Phosphorescence} is similar to fluorescence, but the subsequent emission
of radiation after the initial excitation occurs over a much longer time frame,
allowing a substance to glow in the dark after being exposed to incident
radiation.


\section{X-Rays}

\textbf{Soft} X-rays have an energy roughly less than 10000 eV.

\textbf{Hard} X-rays have an energy roughly less than 50000 eV.

\subsection{Bremsstrahlung Radiation}

% TODO: Diagram of electron bending around an atom, emitting radiation

% TODO: Diagram of Bremsstrahlung (no spikes, just the curve)

\textbf{Bremsstrahlung radiation} is emitted when an electron bends around an
atom due to an electrostatic attraction to the nucelus (not colliding with it
or any of its orbital electrons), losing kinetic energy and emitting radiation.

This is usually accomplished by firing electrons at a metal plate after
accelerating them across a high potential.

The maximum possible energy for a produced X-ray would be if all the kinetic
energy of an electron accelerated across the potential difference were
transferred to this X-ray:

$$
E = qV = hf
$$

Not all of the electron's kinetic energy is likely to be transferred to the
emitted radiation. A spectrum of radiation is produced (see Figure ??).

We can graph intensity of X-rays with a given wavelength against wavelength.
There will be a minimum wavelength on the graph, corresponding to the case when
all an electron's kinetic energy is transferred to the radiation.

\subsection{K-Shell Emission}

% TODO: Diagram of K-Shell emission (just vertical spikes at certain
% wavelengths)

\textbf{K-Shell emission} occurs when an incident electron ionises an orbital
electron from an inner shell, causing an electron in a higher orbital to
transition down, replacing the ionised electron to stabilise the atom. This
produces high energy photons in the process.

X-rays with discrete wavelengths/frequencies will be measured (no spectrum is
produced).

The transition from level 2 to 1 is most likely (higher intensity), and has
the least energy emitted (higher wavelength), than the transition from 3 to 1,
etc.

There is a gap along the x axis between transitions to level 1 and transitions
to level 2. A transition from 3 to 2 produces less energy than any transition
to level 1. We cannot comment on the relative intensity of these transitions
compared to transitions to level 1.

Since some orbital electron transitions are more likely than others, the
measured intensity of some produced X-rays will be higher than others.

\subsection{Bremsstrahlung and K-Shell Emission}

% TODO: Combined diagram of Bremsstrahlung and K-Shell emission
% TODO: Diagram of V_2 > V_1 (two voltages)

Both Bremsstrahlung and K-Shell emission are combined on the same graph.

If the accelerating voltage for the electrons is increased (see Figure ??):

\begin{itemize}
\item The minimum wavelength will decrease.
\item The intensity level at every measured wavelength will increase.
\item The spikes from K-Shell emission will remain at the same wavelengths
	(since they depend on the metal, not on the accelerating voltage).
\end{itemize}

If the metal is changed:

\begin{itemize}
\item The minimum wavelength and shape of the curve will remain the same.
\item The spikes from K-Shell emission will change locations.
\end{itemize}

A minimum wavelength exists because:

\begin{itemize}
\item The minimum wavelength corresponds to a maximum energy value.
\item This maximum energy radiation occurs when an incident electron imparts all
	of its kinetic energy to the emitted radiation.
\item Since an electron cannot have negative kinetic energy in this context,
	this is the maximum possible energy for the emitted photon.
\end{itemize}


\section{De Broglie Wavelength}

The \textbf{De Broglie wavelength} is the wavelength of a photon with a
particular momentum:

$$
\lambda = \frac{h}{p}
$$

\subsection{Electron Standing Waves}

% TODO: Diagram of electrons forming standing waves

For an electron to exist in an orbital, it must have a wavelength that is an
integer multiple of $h$?

\subsection{Electron Diffraction}

Electrons also show interference patterns when used instead of light in a
double slit experiment consistent with wave behaviour.

For an electron accelerated across a voltage $V$:

$$
E = qV
$$

The speed of the electron is:

$$
\begin{aligned}
qV & = \frac{1}{2}mv^2 \\
v & = \sqrt{\frac{2qV}{m}} \\
\end{aligned}
$$

The associated De Broglie wavelength is:

$$
\begin{aligned}
\lambda & = \frac{h}{p} \\
& = \frac{h}{mv} \\
& = \frac{h}{\sqrt{2qVm}} \\
\end{aligned}
$$




\chapter{Light}

\section{Electrons}

\subsection{Electron Volts}

An \textbf{electron volt} (symbol: eV) is a unit of energy, where 1 eV is
equivalent to the increase in energy one electron incurs when travelling across
a potential difference of 1 V.

It is \textbf{not} a measure of voltage.

\subsection{Kinetic Energy from Voltage}

For an electron accelerated across a potential difference of $V$ volts, the
imparted kinetic energy on the electron is:

$$
E = qV
$$


\section{Waves}

\textbf{Waves} are the regular oscillation of particles, propagated without the
net movement of particles.

\textbf{Mechanical waves} require a medium to travel through.

\textbf{Electromagnetic waves} do not require a medium to travel through, and
are self propagating, due to self-sustained oscillations between a magentic and
electric field.

A \textbf{cycle} of a wave is one complete oscillation of a particle, where it
returns to an identical position and velocity it had previously.

The \textbf{amplitude} of a wave is the distance from the mean point of
oscillation to the maximum point.

The \textbf{period} ($T$) is the time taken for the wave to complete one cycle.

The \textbf{frequency} ($f$, in Hz or $s^{-1}$) is the number of cycles
completed in 1 second.

The \textbf{wavelength} ($\lambda$) is the distance between two consecutive,
identical points on the wave.

The \textbf{wave speed} ($v$) is the rate at which the wave propagates.

\subsection{Wave Equation}

$$
v = f \lambda
$$

The speed of the wave is the product of its frequency and wavelength.


\section{Wave Fronts}

A \textbf{ray} is a straight line drawn in the direction of propagation of a
wave.

A \textbf{wave front} is a straight line drawn perpendicular to a ray, where
consecutive wave fronts are separated by a distance of $\lambda$ (the
wavelength).

Although a source emitting waves will likely emit them in a circular or
spherical fashion, waves far away from their source can be considered planar
(parallel to each other).

\subsection{Reflection}

% TODO: Diagram

\textbf{Reflection} is where a wave bounces of a surface.

The angle of incidence of a reflected ray is equal to the angle of reflection,
where angles are taken between the ray and the normal to the surface.

Changes:

\begin{itemize}
\item No change in the speed of the wave, since there is no change in the
	medium.
\item No change in frequency, since the frequency is set by the source.
\item No change in wavelength, since no change in freuqnecy or speed.
\end{itemize}

\subsection{Refraction}

% TODO: Diagram

\textbf{Refraction} is where a wave bends towards or away from the normal as it
enters a substance of a different refractive index.

For light:

\begin{itemize}
\item Refracts towards the normal when entering a more dense substance (with
	a higher refractive index), since light slows down.
\item Refracts away from the normal when entering a less dense substance (with
	a lower refractive index), since light speeds up.
\end{itemize}

\subsection{Diffraction}

% TODO: Diagram (for obstacles and slits)

\textbf{Diffraction} is the bending of light around obstacles.

For obstacles:

\begin{itemize}
\item For a wavelength much larger than the obstacle, the wave passes through
	the obstacle without any change.
\item For a given obstacle, a wave with a larger wavelength will diffract more
	around the obstacle than one with a smaller wavelength.
\end{itemize}

For slits:

\begin{itemize}
\item For a given wavelength, a wave will diffract more through a small slit
	than a large slit.
\item For a given slit size, a wave with a larger wavelength will diffract more
	than one with a smaller wavelength.
\end{itemize}

Maximum diffraction occurs when the slit size equals the wavelength.


\section{Light}

\textbf{Light} is an electromagnetic wave produced by a changing magnetic and
electric field.

The \textbf{electromagnetic spectrum} is the range of wavelengths an
electromagnetic wave can assume.

\subsection{Speed of Light}

$$
c = 3.00 \times 10^8
$$

The speed of light in a vacuum is constant.


\section{Wave Nature}

Light can be demonstrated to act as a wave in certain experiments.

\subsection{Wave Equation}

$$
c = f \lambda
$$

$v$ in the wave equation is constant when dealing with light, and is equal to
$c$.

\subsection{Young's Double Slit Experiment}

% TODO: Diagram

Light is passed through two small slits separated by distance $d$.

When light rays pass through the slits, they are assumed to be parallel.

A screen is set up parallel to the slits, separated by a distance of $L$.

Bright and dark areas on the screen are observed, corresponding to areas of
destructive and constructive interference from waves of light.

To reach point $p$ on the screen, which is a distance of $y$ above the midpoint
of the two slits, the light from each slit would have to travel different
distances, $r_1$ and $r_2$.

The \textbf{path difference} ($\delta$) is the difference in these distances
(ie. $r_2 - r_1$):

$$
\delta = r_2 - r_1
$$

The angle between the midpoint of the two slits and $p$ is $\theta$, which is
equal to the angle between the normal to $r_1$ and $d$, if we assume $r_1$ and
$r_2$ are parallel.

Thus:

$$
\begin{aligned}
\sin{\theta} & = \frac{\delta}{d} \\
\tan{\theta} & = \frac{y}{L} \\
\end{aligned}
$$

For small $\theta$, $\sin{\theta} \approx \tan{\theta}$:

$$
\frac{\delta}{d} = \frac{y}{L}
$$

Thus:

$$
y = \frac{L\delta}{d}
$$

For bright spots, the path difference is some multiple of the wavelength of the
light $\lambda$:

$$
y = \frac{L n \lambda}{d} \qquad n \in \mathbb{Z}
$$

For dark spots, the light waves from each slit are $90^\circ$ out of phase:

$$
y = \frac{L (n + \frac{1}{2}) \lambda}{d} \qquad n \in \mathbb{Z}
$$


\section{Particle Nature}

Packets of light can only exist in discrete \textbf{quanta} (called a photon).

A quanta of light with frequency $f$ has the energy:

$$
E = hf
$$

Where $h$ is Plank's constant ($6.63 \times 10^{-34}$ Js).

Given the wavelength of light $\lambda$ instead of the frequency:

$$
E = \frac{hc}{\lambda}
$$

\subsection{Black Body Radiation}

% TODO: Diagram of intensity vs. wavelength

\textbf{Black body radiation} is the constant emission of electromagnetic waves
by all objects.

The distribution and intensity of these waves depends on the object's
temperature.

As an object's temperature increases, the distribution of wavelengths emitted
changes:

\begin{itemize}
\item The peak wavelength (which has the highest intensity) decreases, moving
	more into the visible light part of the spectrum, and eventually the
	ultraviolet part.
\item The peak intensity increases, making any emitted light brighter.
\end{itemize}

The wavelength at the peak intensity is given by Wien's law:

$$
\lambda = \frac{b}{T}
$$

Where $b$ is Wien's constant ($2.90 \times 10^{-3}$) and $T$ is the temperature
in Kelvin.

The wave model of light predicts no lower bound on the intensity of short
wavelengths of light, suggesting there will be an infinite amount of short
wavelength radiation, contradicting the black body curve observed
experimentally.

\subsection{Photoelectric Effect}

Light is shone at a polished metal surface connected to a circuit that is
capable of measuring current.

Each photon, when it collides with an electron on the metal surface, imparts
energy, causing a flow of \textbf{photoelectrons}, causing the ammeter to show
a current reading.

Photoelectrons are only produced when the light is above a certain frequency
(shorter than a certain wavelength).

A certain number of photoelectrons are produced per second. This is dependent
on how many photons strike the plate per second. This is equivalent to the
current in the circuit (since current is the rate of flow of charge).

Thus the current is proportional to the intensity, not the frequency of light.

When a photon strikes an electron, it imparts a certain amount of energy to
the electron. This is equivalent to the voltage in the circuit (since voltage
is the amount of energy per Coulomb of charge).

The \textbf{stopping voltage} is a measure of how much energy each Coulomb of
photoelectrons has. It is calculated by creating a potential difference in the
circuit that opposes the photocurrent.

Thus the stopping voltage is proportional to the frequency of light, not the
intensity.

\subsubsection{Einstein's Explanation}

When a photon strikes an electron on the plate, it imparts all its energy to
the electron. The energy of the photon is given by:

$$
E = hf
$$

The \textbf{work function} ($W$) for the metal plate is equal to the energy
required to liberate an electron from the metal, turning it into a
photoelectron.

More strongly bound electrons (closer to the nucleus) require more energy to
liberate (in accordance with their ionisation energies).

Different metals will have different work functions. Metals with more strongly
bound electrons will require more energy to liberate, and have higher work
functions.

Any additional energy supplied by the photon after that will turn into the
electron's kinetic energy.

The energy of the incident photon is (where $V_s$ is the stopping voltage):

$$
E_{ph} = qV_s + W
$$

The kinetic energy of the photoelectron is:

$$
E_K = hf - W
$$

\subsubsection{Photocurrent Graph}

% TODO: Diagram

A graph of kinetic energy of photoelectron against frequency of light.

The line represents how much kinetic energy each photoelectron will have if it
collides with a photon of the given frequency:

$$
E_K = hf - W
$$

Negative energy values represent photons with not enough energy to liberate the
electron from the metal to produce a photocurrent.

The x intercept represents the minimum frequency of light required to liberate
an electron from the metal.

Positive energy values represent photons with energy greater than the metal's
work function, creating a photocurrent with the additional energy.

The gradient of the line (in accordance with the above equation) is Plank's
constant, $h$.

Different metals (with different work functions) will all be parallel lines,
but with different y intercepts.

\subsection{Compton Scattering}

\textbf{Compton scattering} is where incident radiation is fired at an atom
with sufficient energy to ionise an electron.

By measuring the angle at which the electron and the incident photon move after
the collision, Compton could determine the momentum of a photon:

$$
p = \frac{h}{\lambda}
$$



\chapter{Special Relativity}

\section{Michelson-Morley Experiment}

% TODO: Diagram of equipment setup

The \textbf{Michelson-Morley experiment} demonstrated that the speed of light
was constant in all inertial reference frames.

A light was shone from a source onto a half silvered mirror, where half of the
light was reflected to an upper mirror, and half of the light was allowed to
pass through onto another mirror. The two paths for light were of the same
distance, but in different directions.

If Earth was moving relative to an aether, this would produce fringe effects
in the observed interference patterns as the speed of light relative to Earth
changed over a 6 month period.

No such fringe effects were observed, proving that the speed of light was
a constant in all inertial reference frames.


\section{Reference Frames}

A \textbf{reference frame} specifies the motion of an observer, able to make
measurements of the world around them.

An \textbf{inertial reference frame} is one that has a constant velocity (no
acceleration).

\textbf{Gallileo's principle of relativity} states that all motion is relative
to a particular reference frame. There is no reference frame that has an
absolute zero velocity.


\section{Classical Relativity}

An object travelling at \textbf{relativistic speeds} has a velocity greater
than $0.1c$ (for calculations done to 3 significant figures).

\textbf{Classical relativity} is derived from Newton's laws of motion. It is
applicable for all objects travelling under relativistic speeds. Above this,
relativistic effects become significant in calculations (to 3 significant
figures).

Classical relativity is merely an approximation of special relativity, which
only holds for velocities less than relativistic speeds.


\section{Einstein's Postulates}

The \textbf{first postulate} states that the laws of physics are the same in
any inertial reference frame.

The \textbf{second postulate} states that the speed of light is the same
relative to any observer, and therefore independent of the motion of the
observer.


\section{Effects of Special Relativity}

Since no object can have a velocity greater than the speed of light, time,
distance, and mass change depending on velocity to ensure this.

\subsection{Time Dilation}

\textbf{Time dilation} occurs for objects travelling at relativistic speeds,
where their perception of time slows (moving clocks tick slower):

$$
\begin{aligned}
t & = \frac{t_0}{\sqrt{1 - \frac{v^2}{c^2}}} \\
& = \gamma t_0 \\
\end{aligned}
$$

Where $\gamma$ is the \textbf{Lorentz factor}:

$$
\gamma = \frac{1}{\sqrt{1 - \frac{v^2}{c^2}}}
$$

For $v \to c$, $t \to \infty$. Thus an object travelling at the speed of light
experiences no time.

\subsubsection{Proof}

Consider a train moving with velocity $v$. On the floor of the train is a
photon emitter, firing a photon vertically so that it hits the roof of the
train.

An observer on the train watches the photon travel up vertically, hitting the
roof in time $t_0$ (the \textbf{proper time}).

A stationary observer sees the light travel diagonally (with a horizontal
component of $v$ due to the train's movement), hitting the roof in a time $t$.

$ct_0$ is the vertical distance from the floor to the roof for the observer on
the train.

$vt$ is the distance the train moves forward in time $t$ for the stationary
observer.

$ct$ is the diagonal distance the light travels in time $t$ for the stationary
observer.

By Pythagoras:

$$
\begin{aligned}
(ct_0)^2 + (vt)^2 & = (ct)^2 \\
t^2 & = \frac{c^2t_0^2}{c^2 - v^2} \\
& = \frac{t_0^2}{1 - \frac{v^2}{c^2}} \\
t & = \frac{t_0}{\sqrt{1 - \frac{v^2}{c^2}}} \\
& = \gamma t_0 \\
\end{aligned}
$$

\subsection{Simultaneity}

\textbf{Simultaneity} is where two events occur at the same moment in time, as
measured by two observers in different inertial reference frames.

Simultaneity does not necessarily apply when considering observers travelling
at relativistic speeds.

\subsection{Length Contraction}

\textbf{Length contraction} occurs for objects travelling at relativistic
speeds, where their perception of distance in the direction of their velocity
vector contracts:

$$
\begin{aligned}
L & = L_0 \sqrt{1 - \frac{v^2}{c^2}} \\
& = \frac{L_0}{\gamma} \\
\end{aligned}
$$

\subsection{Mass Increase}

As $v$ asymptotes at $c$, the force required to produce a given acceleration
approaches infinity. This can be expressed by an increase in the mass of the
object.

Objects travelling at relativistic speeds experience an \textbf{increase in
mass}:

$$
\begin{aligned}
m & = \frac{m_0}{\sqrt{1 - \frac{v^2}{c^2}}} \\
& = \gamma m_0 \\
\end{aligned}
$$

\subsection{Kinetic Energy}

An object's \textbf{kinetic energy} when travelling at relativistic speeds is
given by:

$$
E_K = m_0 c^2 (\gamma - 1)
$$

\subsection{Rest Energy}

An object's \textbf{rest energy} is the amount of energy produced if all its
mass were converted into energy:

$$
E_0 = m_0 c^2
$$

\subsection{Total Energy}

An object's \textbf{total energy} is the sum of its rest and kinetic energies:

$$
\begin{aligned}
E & = m_0 c^2 + m_0 c^2 (\gamma - 1)
& = \gamma m_0 c^2
\end{aligned}
$$

Total energy is \textbf{conserved} across interations.

\subsection{Momentum}

An object's \textbf{momentum} when travelling at relativistic speeds is given
by:

$$
p = \gamma m_0 v
$$

\subsubsection{Photons}

For a massless photon, where $v = c$:

$$
\begin{aligned}
p & = \gamma m_0 c \\
E & = \gamma m_0 c^2 \\
& = pc \\
\end{aligned}
$$

Thus for photons:

$$
\begin{aligned}
p & = \frac{E}{c} \\
& = \frac{hf}{c} \\
& = \frac{hc}{\lambda c} \\
& = \frac{h}{\lambda} \\
\end{aligned}
$$

Proving that photons have a momentum that depends upon their wavelength,
despite having no mass.


\section{Relative Velocities}

Since no object can have $v > c$, expressing the velocity of an object from a
different reference frame can no longer be calculated through simple addition.

For an object travelling with velocity $u'$ relative to a moving observer
travelling with velocity $v$, the velocity $u$ of the object relative to a
second, stationary observer is:

$$
u = \frac{v + u'}{1 + \frac{vu'}{c^2}}
$$

This can be applied to objects travelling in opposite directions through
manipulating the signs of the velocities.

\end{document}
