
\documentclass[a4paper,11pt]{article}

% Math symbols
\usepackage{amsmath}
\usepackage{amsfonts}
\usepackage{esvect}

% Hyperlink contents page
\usepackage{hyperref}
\hypersetup{
	colorlinks,
	citecolor=black,
	filecolor=black,
	linkcolor=black,
	urlcolor=black
}

% No indent on new paragraphs
\setlength{\parindent}{0mm}
\setlength{\parskip}{0.2cm}

% Alias \boldsymbol to \bb for vectors
\newcommand{\bb}{\boldsymbol}


\begin{document}

\title{Motion and Forces}
\author{Ben Anderson}
\date{\today}
\maketitle
\pagebreak

\tableofcontents
\pagebreak


\section{Vectors}

A value with both magnitude and direction.

Diagramatically represented as an arrow.


\subsection{Components}

A vector can be split into its horizontal or vertical components:

$$
\begin{aligned}
v_x & = v \cos{\theta} \\
v_y & = v \sin{\theta} \\
\end{aligned}
$$


\subsection{Arithmetic}

\subsubsection{Negation}

Reverses the direction of the vector.

Negate each component separately.


\subsubsection{Addition}

Place two vectors end to end and determine magnitude and direction of
resultant.

To solve, either:

\begin{enumerate}
\item Use the cosine rule to find the magnitude, and sine rule for the
	direction.
\item Split both vectors into horizontal and vertical components, then add
	corresponding components.
\end{enumerate}


\subsubsection{Subtraction}

Negate the second vector, then add them as above.




\section{Motion}

\subsection{Displacement}

Symbol: $s$

Units: metres ($m$)

Requires a direction.

A vector from the origin to an object's position.


\subsection{Velocity}

Symbol: $v$

Units: $\mbox{ms}^{-1}$

Requires a direction.

The rate of change of displacement with respect to time:

$$
v = \frac{\Delta s}{\Delta t}
$$


\subsection{Change in Velocity}

The final velocity of an object, subtract its initial velocity:

$$
\Delta v = v_f - v_i
$$

Use vector subtraction (as above) to find the correct direction.


\subsection{Relative Velocity}

The velocity of $A$ relative to a stationary observer $B$ is:

$$
v_{a/b}
$$

This can be calculated using the velocities of $A$ and $B$ relative to a third,
stationary observer $C$:

$$
v_{a/b} = v{a/c} + v{c/b}
$$


\subsubsection{River Problem}

Description:

\begin{itemize}
\item You stand at $A$ on the side of a river 30 m wide.
\item You must reach $B$ on the opposite side of the river, 10 m downstream.
\item You can swim with a velocity of $2\mbox{ ms}^{-1}$.
\item The river has a current with velocity of $3\mbox{ ms}^{-1}$
\end{itemize}

Questions:

\begin{itemize}
\item At what angle with the shore should you swim to reach $B$?
\item What is your speed relative to the Earth?
\item How long will it take to cross the river?
\end{itemize}

Solution:

\begin{itemize}
\item The direction of your resultant velocity must be towards $B$.
\item Your velocity added to the velocity of the river's current forms a
	triangle.
\item Solve for the required sides and angles using various trigonometric rules.
\end{itemize}


\subsection{Acceleration}

Symbol: $a$

Units: $\mbox{ms}^{-2}$

Requires a direction.

The rate of change of velocity with respect to time:

$$
a = \frac{\Delta v}{\Delta t}
$$




\section{Equations of Motion}

$$
\begin{aligned}
s & = vt \\
v & = u + at \\
v^2 & = u^2 + 2as \\
s & = ut + \frac{1}{2}at^2 \\
\end{aligned}
$$

For a constant acceleration.




\section{Momentum}

$$
p = mv
$$

Symbol: $p$

Units: $\mbox{kg m s}^{-1}$

Requires a direction.


\subsection{Conservation of Momentum}

Assuming no energy loss to the surroundings, the total momentum of a system is
conserved before and after a collision:

$$
\sum p_i = \sum p_f
$$




\section{Energy}

Symbol: $E$

Units: Joules ($J$)

Scalar quantity.


\subsection{Kinetic Energy}

$$
E_k = \frac{1}{2}mv^2
$$

Energy due to an object's motion.


\subsection{Gravitational Potential Energy}

$$
E_p = mgh
$$

Energy due to an object's height above a fixed reference point (usually the
ground).


\subsection{Conservation of Energy}

The total energy in a system is conserved:

$$
\sum E_i = \sum E_f
$$




\section{Force}

Symbol: $F$

Units: Newtons ($N$)

Requires a direction.

Anything that changes an object's acceleration.


\subsection{Types of Forces}

\subsubsection{Weight Force}

Symbol: $W$ or $mg$

Force due to gravity.

Acts from the centre of mass towards the centre of the Earth (usually straight
down).


\subsubsection{Normal Force}

Symbol: $F_N$

When two objects come into contact.


\subsubsection{Tension Force}

Symbol: $T$

A restoring force due to intermolecular attractions between particles that make
up an object.

Acts parallel to an object.


\subsubsection{Friction Force}

Symbol: $F_f$

A force that retards the motion of an object.

Acts along a surface, opposite to the direction of the object's motion.

Static friction is the force required to start moving an object from rest:

$$
F_f = \mu F_N
$$

Where $\mu$ is the coefficient of static friction for the surface on which the
object is moving across.

Kinetic friction is the force required to overcome the frictional force while
an object is moving, to maintain a constant velocity.


\subsection{Newton's Laws}

\subsubsection{First Law}

Every object will continue in its current state of rest or straight line,
uniform motion unless acted upon by a net, non-zero, external force.

Uniform means the magnitude of the object's velocity isn't changing.

Straight line means the direction of the object's velocity isn't changing.


\subsubsection{Second Law}

$$
\sum F = ma
$$

The net force exerted on an object is equal to the object's mass multiplied by
its acceleration.


\subsubsection{Third Law}

If one object exerts a force on another, the second will also exert a force on
the first equal in magnitude but opposite in direction.




\section{Dynamics}

For objects undergoing a constant acceleration.


\subsection{Method}

The general approach to solving dynamics problems:

\begin{enumerate}
\item Split the forces acting on a body into horizontal and vertical components.
\item Equate each component to $ma$, using the acceleration of the object along
	each axis.
\item Solve for unknowns.
\end{enumerate}



\section{Statics}

For objects at a constant velocity or at rest.


\subsection{Method}

The general approach to solving statics problems:

\begin{enumerate}
\item Split the forces acting on a body into horizontal and vertical components.
\item Equate each component to 0.
\item Solve for unknowns.
\end{enumerate}




\section{Projectile Motion}

For an object fired at an angle to the ground into the air.

Assume:

\begin{itemize}
\item Gravity acts directly downwards.
\item No air resistance (horizontal velocity is constant).
\end{itemize}


\subsection{Variables}

Potential variables that may be given, or asked to be solved:

\begin{itemize}
\item Horizontal range
\item Flight time
\item Maximum height
\item Initial velocity
\item Final velocity when the object hits the ground
\item Velocity after $t$ seconds
\item Angle to the horizontal at which the object was fired
\item Height above the ground of the surface from which the object was fired
\item Height above the ground of the surface on which the object landed
\end{itemize}


\subsection{Flight Time}

To find the time the object is in the air before hitting the ground again, use:

$$
s_y = u_y t + \frac{1}{2} a_y t^2
$$

Set $s_y$ to the vertical displacement of the object relative to its initial
starting height at the point where it lands.

Solve using the quadratic formula.


\subsection{Range}

The range the projectile travels is equivalent to its horizontal displacement:

$$
s_x = u_x t
$$


\subsection{Maximum Height}

The vertical velocity of the projectile will be 0 at its maximum height, so
solve for $s_y$ using:

$$
u_y^2 + 2 a_y s_y = 0
$$


\subsection{Final Velocity}

Solve for each component of the final velocity separately, then combine them.

The horizontal component is the same as the initial horizontal velocity (since
there's no acceleration along the horizontal axis):

$$
v_x = u_x
$$

The vertical component is:

$$
v_y = u_y + a_y t
$$


\subsection{Air Resistance}

Effects of air resistance on the path of the projectile:

\begin{itemize}
\item Reduced range.
\item Reduced maximum height.
\item Asymmetrical path, where the downwards part of its flight is more stunted.
\end{itemize}

These effects are more pronounced for faster moving objects, since the
frictional force is proportional to the object's velocity.

The upwards part of the projectile's flight takes less time than the downwards
because:

\begin{itemize}
\item Air resistance and gravity both act downwards on the upwards part of the
	flight, instead of in opposite directions on the downwards part.
\item This causes the upwards part to take less time for the projectile to
	complete.
\end{itemize}




\section{Circular Motion}

For an object travelling in a circular path, at a radius from a central point.


\subsection{Period}

Symbol: $T$

Units: seconds ($s$)

Time taken for the object to complete one revolution.


\subsection{Frequency}

Symbol: $f$

Units: Hertz (Hz) or $\mbox{s}^{-1}$

The number of revolutions completed in 1 second.

Related to the period by:

$$
f = \frac{1}{T}
$$


\subsection{Velocity}

$$
\begin{aligned}
v & = \frac{2\pi r}{T} \\
& = 2\pi r f \\
\end{aligned}
$$

Tangential to the circular path, perpendicular to the radius.

Constant magnitude, but continuously changing direction.


\subsection{Centripetal Acceleration}

$$
\begin{aligned}
a & = \frac{v^2}{r} \\
& = \frac{4 \pi^2 r}{T^2} \\
\end{aligned}
$$

Directed towards the centre of the circle.

Constant magnitude, but continuously changing direction.


\subsection{Centripetal Force}

$$
F_c = \frac{mv^2}{r}
$$

Directed towards the centre of the circle.

The centripetal force isn't a new type of force, but is the resultant unbalanced
force acting on an object travelling in a circular path.

It is caused by other forces, such as a tension force in a string, or the normal
force for a car travelling around a corner.


\subsection{Horizontal Circular Motion}

For an object travelling in a horizontal circle, parallel to the ground, where
the weight force is perpendicular to the object's path.

The centripetal force will be supplied solely by the tension force in the
string.


\subsection{Vertical Circular Motion}

For an object travelling in a vertical circle, perpendicular to the ground. We
assume the average velocity of such an object to be constant.

The centripetal force will be provided by a combination of the tension force
in the string, and the object's weight force:

$$
F_c = T + W
$$

At the top of the object's path:

\begin{itemize}
\item Tension, weight, and the resultant centripetal force act down.
\item Tension is at a minimum, as the weight force is providing part of the
	required centripetal force.
\end{itemize}

At the bottom of the object's path:

\begin{itemize}
\item Tension and the resultant centripetal force act upwards, weight downwards.
\item Tension is at a maximum, as it must overcome the weight force and provide
	the required centripetal force.
\end{itemize}

Can also consider a rollercoaster or plane moving in a vertical circle, where
the centripetal force is provided by the normal force, instead of tension.

Applies when a car moves over a speed hump (decrease in normal force) and when
a car moves through a ditch (increase in normal force).


\subsubsection{Apparent Weightlessness}

Apparent weightlessness occurs when the normal force exerted on a person is 0.

TODO: Questions about spaceship


\subsubsection{Minimum Velocity}

There exists a certain velocity for a rollercoaster in a loop such that a
person would experience apparent weightlessness at the top of the loop, where
the normal force would be 0.

At the top of the loop, taking up to be positive:

$$
-F_c = -F_N - mg
$$

$F_N$ is 0, so:

$$
\begin{aligned}
\frac{mv^2}{r} & = mg \\
v^2 & = rg \\
v & = \sqrt{rg} \\
\end{aligned}
$$


\subsection{Conical Circular Motion}

When an object moves in a horizontal circle, making an angle with the vertical.

For example, a mass on the end of a string:

\begin{itemize}
\item Forms a cone when swung, with the string at an angle to the vertical.
\item The tension force is at an angle to the vertical, and has a horizontal
	component.
\item This horizontal component provides the centripetal force for the
	horizontal circular motion.
\end{itemize}

For example, a plane banking around a corner, travelling in a horizontal circle:

\begin{itemize}
\item The lift force, perpendicular to the wings, is at an angle to the
	vertical.
\item The horizontal component of the lift force provides the centripetal force
	for the circle.
\end{itemize}

For example, a car on a cambered (angled) track:

\begin{itemize}
\item The normal force, perpendicular to the cambered track, is at an angle to
	the vertical.
\item The horizontal component of the normal force provdies the centripetal
	force.
\end{itemize}

The upwards vertical component of these forces will counteract the weight force
of the object.


\subsubsection{Velocity}

For a specific angle with the vertical, the mass will have a certain velocity.

For a car on a cambered track, a certain velocity will allow it to travel
around the corner without any friction acting on the tyres (where $F_f$ along
the slope of the road is 0):

$$
\begin{aligned}
\sum F & = ma \\
\sum F_v & = 0 \\
F_N \cos{\theta} - mg & = 0 \\
F_N \cos{\theta} & = mg \\
\sum F_h & = \frac{mv^2}{r} \\
F_N \sin{\theta} & = \frac{mv^2}{r} \\
\end{aligned}
$$

Dividing the one equation by the other:

$$
\begin{aligned}
\frac{F_N \sin{\theta}}{F_N \cos{\theta}} & = \frac{mv^2}{rmg} \\
\tan{\theta} & = \frac{v^2}{rg} \\
\end{aligned}
$$

Solving for the velocity:

$$
v = \sqrt{rg \tan{\theta}}
$$




\section{Gravity}

\subsection{Gravitational Field}

A gravitational field exists at a point if a force is exerted on a particle
with mass when it is placed at the point.

Any particle with mass has a gravitational field around it.


\subsection{Gravitational Field Lines}

The conditions for drawing gravitational field lines around an object with mass:

\begin{itemize}
\item Symmetrical about every axis.
\item Arrows point towards centre of mass.
\item All arrows the same length.
\item Density of lines indicates the relative strength of the gravitational
	field.
\end{itemize}


\subsection{Universal Law of Gravitation}

$$
F_g = \frac{G m_1 m_2}{r^2}
$$

The force exerted on an object with mass $m_1$ by another object with mass
$m_2$, when they are distance of $r$ metres apart.

$G$ is the universal gravitation constant:

$$
G = 6.67 \times 10^{-11}\mbox{ N m}^2\mbox{ kg}^{-2}
$$

The force requires a direction. For the force exerted by $m_2$ on $m_1$, the
direction will be towards $m_2$.


\subsection{Gravitational Field Strength}

$$
a_g = \frac{G m}{r^2}
$$

Units: Newtons per metre ($\mbox{N m}^{-1}$), equivalent to $\mbox{ms}^{-1}$

The force exerted on an object per unit of mass by an object with mass $m_2$,
a distance of $r$ metres away.

Also called acceleration due to gravity ($g$).


\subsection{Centripetal Force from Planet's Rotation}

An object on Earth experiences a centripetal force from the Earth's rotation.

TODO: What was I going on about??


\subsection{Circular Orbit}

Assuming the path of an object orbiting a planet is circular, the centripetal
force is due solely to the planet's gravity:

$$
\begin{aligned}
F_g & = F_c \\
\frac{G m_1 m_2}{r^2} & = \frac{m_1 v^2}{r} \\
v^2 = \frac{G m_2}{r} \\
\end{aligned}
$$

The velocity of an orbiting object is constant for a particular altitude.

When the velocity of an orbiting object is increased:

\begin{itemize}
\item Increase in tangential velocity of the object increases the required
	centripetal force to sustain the current radius.
\item Only gravity provides the centripetal force, and it can't increase to
	compensate for the required centripetal force.
\item Object will increase in altitude, increasing the radius of orbit,
	decreasing the required centripetal force.
\item Increasing its altitude increases the object's gravitational potential
	energy.
\item Since energy must be conserved, the object's kinetic energy must
	decrease, decreasing its velocity.
\end{itemize}


\subsection{Escape Velocity}

The minimum velocity required by an object launched from an altitude of $r$
above a body to completely escape its gravitational field, ignoring any other
energy loss:

$$
v = \sqrt{\frac{2 G m_2}{r}}
$$


\subsection{Lagrangian Points}

A point in between two bodies where the gravitational force exerted by each is
equal in magnitude and opposite in direction.

An object at this point will not fall towards either body.

For two bodies of mass $m_1$ and $m_2$ separated by a distance $r$, where $x$
is the distance from body 1 to the Lagrangian point:

$$
\begin{aligned}
\frac{G m_1 m_b}{x^2} & = \frac{G m_2 m_b}{(r - x)^2} \\
\frac{m_1}{m_2} & = \frac{x^2}{(r - x)^2} \\
x & = \frac{r\sqrt{\frac{m_E}{m_m}}}{1 + \sqrt{\frac{m_E}{m_m}}} \\
\end{aligned}
$$


\subsection{Kepler's Laws}

\subsubsection{First Law}

The path of the planets around the sun are elliptical.

One foci of the ellipse is the sun.


\subsubsection{Second Law}

All planets orbiting the sun sweep out equal areas in equal times.


\subsubsection{Third Law}

Derives a relationship between an objects's period of orbit ($T$) and orbital
radius ($r$) around a planet with mass $m$:

$$
\begin{aligned}
F_g & = F_c \\
\frac{G m m_b}{r^2} & = \frac{m_b v^2}{r} \\
& = \frac{m_b 4 \pi^2 r}{T^2} \\
\frac{r^3}{T^2} & = \frac{G m}{4 \pi^2} \\
\end{aligned}
$$


\subsection{Geosynchronous Orbit}

Where the period of rotation of an object around a planet is the same as the
planet's rotational period (24 hours for Earth).

The altitude of this orbit can be derived using Kepler's third law (above),
since the mass of the Earth and the orbital period (24 hours) is known.


\subsection{Geostationary Orbit}

Where the period of rotation of an object is equal to the planet's rotational
period, and the object orbits around the equator.

This means the object is always in the same position in the sky relative to an
observer on Earth.

This is used for communication satellites, where radio waves are always aimed
at the same point in the sky.


\subsection{Barycentre}

Two planets will orbit each other around their centre of mass.

Most commonly:

\begin{itemize}
\item The mass of the planet is much larger than that of the orbiting object.
\item The centre of mass will be very close to the centre of the planet, making
	the difference negligible.
\item Only significant when the masses of both objects are similar.
\end{itemize}

For two objects with masses $m_1$ and $m_2$ a distance $r$ apart, the barycentre
is located $x$ metres from object 1:

$$
x = \frac{m_2}{m_1 + m_2} r
$$


\subsubsection{Kepler's Third Law}

Kepler's third law cannot be used as we assumed the radius of orbit was equal
to the distance the planets were apart.




\section{Torque}

Symbol: $\tau$

Units: Newton metres ($N m$)

Requires a direction (commonly clockwise or anticlockwise).

A rotational force about a pivot point.

For a force $F$ applied $r$ metres away from the pivot, at an angle of $\theta$
to the line of action, this produces the torque:

$$
\tau = rF \sin{\theta}
$$


\subsection{Centre of Mass}

Assume all objects are of uniform density, unless otherwise stated.

Thus an object's weight force acts from its centre of mass.


\subsection{Stability}

How prone an object is to falling over.

If an object's centre of mass moves outside it's base, it will fall over.

To increase stability:

\begin{itemize}
\item Lower the centre of mass.
\item Widen the object's base.
\end{itemize}

Both options increase the distance the centre of mass must be moved for it to
be outside the object's base, increasing its stability.


\subsection{Equilibrium}

Translational equilibrium is where:

$$
\begin{aligned}
\sum F_x & = 0 \\
\sum F_y & = 0 \\
\end{aligned}
$$

There is no net force acting on the object, and its acceleration remains
constant.

Rotational equilibrium is where:

$$
\sum \tau_{cw} = \sum \tau_{acw}
$$

Where $cw$ and $acw$ stand for clockwise and anticlockwise respectively.

There is no net change on an object's rotational acceleration.


\subsubsection{Types of Equilibrium}

The types of equilibrium can be demonstrated with a cylinder:

\begin{description}
\item [Stable] Cylinder stands on its circular base. Will not move if pushed.
\item [Neutral] Cylinder stands still on its side. Will move if pushed.
\item [Unstable] Cylinder balances on the edge of its circular base. Will fall
	over if pushed.
\end{description}


\subsection{Solving Torque Problems}

In torque problems, all objects will be in rotational and translational
equilibrium, so:

\begin{enumerate}
\item Set $\sum \tau_{cw} = \sum \tau_{acw}$ and solve for required unknowns.
\item Set $\sum F_y = 0$ and solve for vertical reaction force.
\item Set $\sum F_x = 0$ and solve for horizontal reaction force.
\end{enumerate}


\subsubsection{Perpendicular Distance}

Instead of finding the angle with the line of action, we can use the
perpendicular distance from where the force is acting to the pivot point.

There will be 2 perpendicular distances, one representing $\cos{\theta}$ and
the other $\sin{\theta}$.
Use the one representing $\sin{\theta}$.


\subsection{See Saw Problem}

Description:

\begin{itemize}
\item Multiple objects on a see saw at different distances from a pivot point.
\item Find the location of a missing mass in order to balance the system.
\item Find the mass of an object at a location in order to balance the system.
\end{itemize}

Variables:

\begin{itemize}
\item Mass of objects.
\item Distance of objects from pivot.
\item Location of pivot.
\item Reaction force at pivot.
\end{itemize}

Method:

\begin{itemize}
\item Set $\sum \tau = 0$ and solve for unknown mass/radius.
\item Set $\sum F_y = 0$ and solve for reaction force at pivot.
\end{itemize}


\subsection{Painter's Scaffolding Problem}

Description:

\begin{itemize}
\item Platform suspended by multiple cables.
\item Multiple masses placed on platform.
\item Solve for unknown masses, distances, or tension in cables.
\end{itemize}

Variables:

\begin{itemize}
\item Mass of objects.
\item Distances of objects from a side of the plank.
\item Length of plank (ie. distance of an ending cable from the left pivot).
\item Tension in cables.
\end{itemize}

Method:

\begin{itemize}
\item Chose one cable to act as the pivot point.
\item Set $\sum \tau = 0$ and solve for unknown tension.
\item Set $\sum F_y = 0$ and solve for other tension.
\end{itemize}


\subsection{Ladder Problem}

Description:

\begin{itemize}
\item Ladder resting against a wall at an angle.
\item Weight force acts down, exerting a torque on the ladder.
\item Normal force provided by wall counteracts the torque produced by the
	weight.
\item Normal force acts perpendicular to wall.
\end{itemize}

Variables:

\begin{itemize}
\item Mass of ladder.
\item Length of ladder.
\item Angle made with wall.
\item Normal force produced by wall.
\item Friction force exerted by ground on base of ladder (horizontal component
	of reaction force).
\item Normal force produced by ground (vertical component of reaction force).
\end{itemize}

Method:

\begin{itemize}
\item Take the ladder's point of contact with ground as the pivot point.
\item Set $\sum \tau = 0$ and solve for the wall's normal force.
\item Set $\sum F_y = 0$ and solve for the normal force exerted by ground.
\item Set $\sum F_x = 0$ and solve for the friction force between the ladder
	and the ground.
\item Find the resultant reaction force.
\end{itemize}


\subsection{Crane Problem}

Description:

\begin{itemize}
\item Crane arm suspended at an angle above the horizontal.
\item Supported by a cable fixed to a vertical tower.
\item A load is attached at the end of the arm.
\end{itemize}

Variables:

\begin{itemize}
\item Mass of arm.
\item Mass of load.
\item Angle the arm makes with the horizontal.
\item Tension in the supporting cable.
\end{itemize}

Method:

\begin{itemize}
\item Set $\sum \tau = 0$ and solve for the unknown tension in the supporting
	cable.
\item Set $\sum F_y = 0$ and solve for the vertical reaction force.
\item Set $\sum F_x = 0$ and solve for the horizontal reaction force.
\end{itemize}


\subsection{Door Hinge Problem}

Description:

\begin{itemize}
\item Door attached to a door frame at two hinges.
\item Centre of mass acting downwards in the middle of the door.
\item Door attempts to rotate clockwise around lower hinge.
\item There is a reaction force directed left exerted on the upper hinge, and
	one exerted right on the lower hinge.
\end{itemize}

Variables:

\begin{itemize}
\item Mass of door
\item Dimensions of door
\item Location of hinges
\end{itemize}

Method:

TODO: I don't know how to do this one

\end{document}
