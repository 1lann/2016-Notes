
\documentclass[a4paper,11pt]{article}

% Math symbols
\usepackage{amsmath}
\usepackage{amsfonts}
\usepackage{esvect}

% Hyperlink contents page
\usepackage{hyperref}
\hypersetup{
	colorlinks,
	citecolor=black,
	filecolor=black,
	linkcolor=black,
	urlcolor=black
}

% No indent on new paragraphs
\setlength{\parindent}{0mm}
\setlength{\parskip}{0.2cm}

% Alias \boldsymbol to \bb for vectors
\newcommand{\bb}{\boldsymbol}


\begin{document}

\title{Systems of Linear Equations}
\author{Ben Anderson}
\date{\today}
\maketitle
\pagebreak

\tableofcontents
\pagebreak


\section{Systems of Equations}

Multiple variables to be solved simultaneously:

$$
\begin{cases}
\begin{aligned}
3x + 4y - z & = 4 \\
2x + 9y + 5z & = 10 \\
-x - 5y + z & = 12 \\
\end{aligned}
\end{cases}
$$

We require as many unique equations as variables in order to fully solve the
system.


\subsection{Matrix Form}

Each column represents a variable.

The final column represents the constant on the right side of each of the
equations.

Each row represents the coefficients of the variables in one equation.

For example, the system of equations:

$$
\begin{cases}
\begin{aligned}
 3x + 4y - z & = 4  \\
2x + 9y + 5z & = 10 \\
 -x - 5y + z & = 12 \\
\end{aligned}
\end{cases}
$$

In matrix form is:

$$
\begin{bmatrix}
 3 &  4 & -1 & 4  \\
 2 &  9 &  5 & 10 \\
-1 & -5 &  1 & 12 \\
\end{bmatrix}
$$


\subsection{Elementary Row Operations}

Elementary row operations on a system of equations in matrix form include:

\begin{itemize}
\item Reordering rows.
\item Adding or subtracting two rows from each other.
\item Multiplying or dividing a row by a constant.
\end{itemize}

To perform a series of elementary row operations, we assign a row to the desired
combination of these operations.


\subsection{Row Echelon Form}

Where all coefficients below the leading diagonal for a system of equations in
matrix form are 0.

For example:

$$
\begin{bmatrix}
8 & 1 & -3 & -2 \\
0 & 2 &  2 &  4 \\
0 & 0 & -1 &  5 \\
\end{bmatrix}
$$

Manipulate the matrix into row echelon form using the elementary row operations.


\subsection{Solving}

Once a matrix is in row echelon form, we can solve the system of equations.

In the example above:

$$
\begin{aligned}
z & = -5 \\
2y + 2 \times -5 & = 4 \\
y & = 7 \\
8x + 1 \times 7 - 3 \times -5 & = -2 \\
x & = -3 \\
\end{aligned}
$$

Using this process to solve simultaneous equations is called Gaussian
Elimination.



\section{No Solutions}

If any row in a row echelon form matrix has coefficients of 0 for all variables,
with a non-zero constant in the final column:

$$
\begin{bmatrix}
a & b & c & d \\
0 & e & f & g \\
0 & 0 & 0 & h \\
\end{bmatrix}
$$

This implies $0x + 0y + 0z = h$.

This system has no solutions.




\section{Infinite Solutions}

If any row in a row echelon form matrix has coefficients of 0 for all variables,
and 0 in the final column:

$$
\begin{bmatrix}
a & b & c & d \\
0 & e & f & g \\
0 & 0 & 0 & 0 \\
\end{bmatrix}
$$

This implies $0x + 0y + 0z = 0$.

There is not enough information to uniquely solve the system.

The system has infinitely many solutions.


\subsection{Recognition}

If any of the equations is a scalar multiple of another, the system has an
infinite number of solutions.




\section{Conditions for Infinite or No Solutions}

Given the system of equations:

$$
\begin{cases}
\begin{aligned}
x - y + 2z & = 1 \\
2x - 5y + 5z & = 9 \\
3x + 3y + pz & = q \\
\end{aligned}
\end{cases}
$$

The row echelon form matrix for this system is:

$$
\begin{bmatrix}
1 & -1 & 2 & 1 \\
0 & -3 & 1 & 7 \\
0 & 0 & p - 4 & q + 11 \\
\end{bmatrix}
$$

The system has a unique solution when the coefficient of $z$ in the final row
is non-zero:

$$
\begin{aligned}
p - 4 \neq 0 \\
p \neq 4 \\
\end{aligned}
$$

Written as: $p, q \in \mathbb{R}, p \neq 4$

The system has no solutions when $p = 4$ and $q \in \mathbb{R}, q \neq -11$.

The system has infinite solutions when $p = 4$ and $q = -11$.

\end{document}
