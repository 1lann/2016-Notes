
\documentclass[a4paper,11pt]{article}

% Math symbols
\usepackage{amsmath}
\usepackage{amsfonts}
\usepackage{esvect}

% Hyperlink contents page
\usepackage{hyperref}
\hypersetup{
	colorlinks,
	citecolor=black,
	filecolor=black,
	linkcolor=black,
	urlcolor=black
}

% No indent on new paragraphs
\setlength{\parindent}{0mm}
\setlength{\parskip}{0.2cm}

% Alias \boldsymbol to \bb for vectors
\newcommand{\bb}{\boldsymbol}


\begin{document}

\title{Systems of Linear Equations}
\author{Ben Anderson}
\date{\today}
\maketitle
\pagebreak

\tableofcontents
\pagebreak


\section{Systems of Equations}

Multiple variables to be solved simultaneously:

$$
\begin{cases}
\begin{aligned}
3x + 4y - z & = 4 \\
2x + 9y + 5z & = 10 \\
-x - 5y + z & = 12 \\
\end{aligned}
\end{cases}
$$

We require as many unique equations as variables in order to fully solve the
system.


\subsection{Matrix Form}

Each column represents a variable.

The final column represents the constant on the right hand side of each of the
three equations.

Each row represents the coefficients of the variables in one equation.

For example, the system of equations:

$$
\begin{cases}
\begin{aligned}
 3x + 4y - z & = 4  \\
2x + 9y + 5z & = 10 \\
 -x - 5y + z & = 12 \\
\end{aligned}
\end{cases}
$$

Can be written in matrix form as:

$$
\begin{bmatrix}
 3 &  4 & -1 & 4  \\
 2 &  9 &  5 & 10 \\
-1 & -5 &  1 & 12 \\
\end{bmatrix}
$$

The leading diagonal is the diagonal line of coefficients starting at the top
left corner, moving down and to the right.


\subsection{Elementary Row Operations}

Elementary row operations on a system of equations in matrix form include:

\begin{itemize}
\item Reordering rows.
\item Multiplying or dividing a row by a constant.
\item Adding or subtracting two rows from each other.
\end{itemize}

To perform a series of elementary row operations, we assign a row to the desired
combination of these operations.


\subsection{Row Echelon Form}

Where all coefficients below the leading diagonal of a system of equations in
matrix form are 0.

For example:

$$
\begin{bmatrix}
8 & 1 & -3 & -2 \\
0 & 2 &  2 &  4 \\
0 & 0 & -1 &  5 \\
\end{bmatrix}
$$

Matrices can be manipulated using elementary row operations into row echelon
form.


\subsection{Solving}

To solve a system of equations, write them in matrix form, manipulate the matrix
into row echelon form, and solve for each variable.

In the example above:

$$
\begin{aligned}
z & = -5 \\
2y + 2 \times -5 & = 4 \\
y & = 7 \\
8x + 1 \times 7 - 3 \times -5 & = -2 \\
x & = -3 \\
\end{aligned}
$$

Using this process to solve simultaneous equations is called Gaussian
Elimination.



\section{No Solutions}

Once a system of equations in matrix form is reduced to row echelon form, the
system has no solutions if all variables in one row have coefficients of 0,
with a non-zero constant in the final column.

For example:

$$
\begin{bmatrix}
a & b & c & d \\
0 & e & f & g \\
0 & 0 & 0 & h \\
\end{bmatrix}
$$

The final row implies $0x + 0y + 0z = h$, where $h \neq 0$. This equation has
no solutions.

Thus the system itself has no solutions.




\section{Infinite Solutions}

Once a system of equations in matrix form is reduced to row echelon form, the
system has no solutions if all variables in one row have coefficients of 0,
with an additional 0 in the final column.

For example:

$$
\begin{bmatrix}
a & b & c & d \\
0 & e & f & g \\
0 & 0 & 0 & 0 \\
\end{bmatrix}
$$

The final row implies $0x + 0y + 0z = 0$.

There is not enough information to uniquely solve the system, thus the system
has infinitely many solutions.

If any row in the system is a multiple of another, the system has infinite
solutions.




\section{Conditions for Infinite or No Solutions}

Given the system of equations, where $p, q \in \mathbb{R}$:

$$
\begin{cases}
\begin{aligned}
x - y + 2z & = 1 \\
2x - 5y + 5z & = 9 \\
3x + 3y + pz & = q \\
\end{aligned}
\end{cases}
$$

The row echelon form matrix is:

$$
\begin{bmatrix}
1 & -1 & 2 & 1 \\
0 & -3 & 1 & 7 \\
0 & 0 & p - 4 & q + 11 \\
\end{bmatrix}
$$

The system has a unique solution when the coefficient of $z$ in the final row
is non-zero:

$$
\begin{aligned}
p - 4 \neq 0 \\
p \neq 4 \\
\end{aligned}
$$

This is written as: $p, q \in \mathbb{R}, p \neq 4$

The system has no solutions when $p = 4$ and $q \in \mathbb{R}, q \neq -11$.

The system has infinite solutions when $p = 4$ and $q = -11$.




\section{Graphical Representation}

Each equation in the system can be represented by a 3D plane:

\begin{description}
\item [Unique Solution] When the 3 planes intersect at a single point.
\item [Infinite Solutions] When the 3 planes intersect in a single line.
\item [No Solutions] When the 3 planes intersect in 3 separate lines.
\end{description}

\end{document}
