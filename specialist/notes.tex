
\documentclass[a4paper,11pt]{report}

% Math symbols
\usepackage{amsmath}
\usepackage{amsfonts}
\usepackage{esvect}

% Hyperlink contents page
\usepackage{hyperref}
\hypersetup{
	colorlinks,
	citecolor=black,
	filecolor=black,
	linkcolor=black,
	urlcolor=black
}

% \cis and \Arg
\DeclareMathOperator\cis{cis}
\DeclareMathOperator\Arg{Arg}

% No indent on new paragraphs
\setlength{\parindent}{0mm}
\setlength{\parskip}{0.2cm}

% Alias \boldsymbol to \bb for vectors
\newcommand{\bb}{\boldsymbol}


\begin{document}

\title{Mathematics Specialist Notes}
\author{Ben Anderson}
\date{\today}
\maketitle
\pagebreak

\tableofcontents
\pagebreak




\chapter{Complex Numbers}

\section{Number Sets}

\subsection{Natural Numbers}

Symbol: $\mathbb{N}$

All positive integers, and 0.

\subsection{Integers}

Symbol: $\mathbb{Z}$

All positive integers, negative integers, and 0.

Contains all natural numbers.

\subsection{Rational Numbers}

Symbol: $\mathbb{Q}$ (meaning quotient)

All numbers, positive and negative, that can be represented in the form
$\frac{a}{b}$ where $a, b \in \mathbb{Z}$.

Contains all integers.

\subsection{Irrational Numbers}

All numbers which cannot be represented in the form $\frac{a}{b}$.

Contains all real numbers not in the set of all rational numbers.

\subsection{Real Numbers}

Symbol: $\mathbb{R}$

The set of all rational and irrational numbers.

\subsection{Complex Numbers}

Symbol: $\mathbb{C}$

The set of all numbers with a real and imaginary component.

Contains all real numbers.


\section{Complex Numbers}

\subsection{i}

$i$ is defined as the solution to the equation:

$$
i^2 = -1
$$

\subsection{Representation}

A complex number is a number with both a real and imaginary component.

Written as:

$$
z = a + bi
$$

This representation is called cartesian or rectangular form.

\subsection{Components}

Given the complex number $z = a + bi$.

The real component is $Re(z) = a$.

The imaginary component is $Im(z) = b$.

\subsection{Conjugate}

The conjugate of $z = a + bi$ is $\overline{z} = a - bi$.

Found by negating the imaginary component.

\subsubsection{Properties}

The conjugate of a complex number has the following properties:

$$
\begin{aligned}
\overline{z_1 \pm z_2} & = \overline{z_1} \pm \overline{z_2} \\
\overline{z_1 \times z_2} & = \overline{z_1} \times \overline{z_2} \\
\overline{(\frac{z_1}{z_2})} & = \frac{\overline{z_1}}{\overline{z_2}} \\
\end{aligned}
$$


\section{Arithmetic}

\subsection{Surd Simplification}

Examples of simplifying complex surds:

$$
\begin{aligned}
\sqrt{-64} & = \sqrt{-1} \times \sqrt{64} \\
& = 8i \\
\end{aligned}
$$

$$
\begin{aligned}
\sqrt{-8} & = \sqrt{-1} \times \sqrt{8} \\
& = 2 \sqrt{2} i \\
\end{aligned}
$$

\subsection{Addition and Subtraction}

Add both real and complex components separately.

Given $z_1 = a + bi$ and $z_2 = c + di$:

$$
z_1 + z_2 = a + c + (b + d)i
$$

\subsection{Multiplication or Division by Scalar}

Multiply or divide each component by the scalar.

Given $z = a + bi$:

$$
\begin{aligned}
cz & = c(a + bi) \\
& = ca + cbi \\
\end{aligned}
$$

\subsection{Multiplication by Complex Number}

Distribute the multiplication across each component.

Given $z_1 = a + bi$ and $z_2 = c + di$:

$$
\begin{aligned}
z_1 z_2 & = (a + bi)(c + di) \\
& = ac + adi + bci + bdi^2 \\
& = ac - bd + (ad + bc)i \\
\end{aligned}
$$

\subsection{Multiplication by Conjugate}

Given $z = a + bi$ and its conjugate $\overline{z} = a - bi$:

$$
\begin{aligned}
z \overline{z} & = (a + bi)(a - bi) \\
& = a^2 + b^2 \\
\end{aligned}
$$

\subsection{Division by Complex Number}

Multiply the numerator and denominator by the conjugate of the denominator.

$$
\begin{aligned}
\frac{4 + 5i}{2 + i} & = \frac{4 + 5i}{2 + i} \times \frac{2 - i}{2 - i} \\
& = \frac{13 + 6i}{3} \\
& = \frac{13}{3} + 2i \\
\end{aligned}
$$

\subsection{Equality}

Two complex numbers are equal if and only if both their real and imaginary
components are equal.

Given $z_1 = z_2$, then $Re(z_1) = Re(z_2)$ and $Im(z_1) = Im(z_2)$.


\section{Quadratics}

\subsection{Roots}

Given the quadratic $ax^2 + bx + c$.

If the discriminant $b^2 - 4ac < 0$, then the quadratic will have 2 complex
roots.

Complex roots for any polynomial with real coefficients appear in conjugate
pairs.

\subsection{Solving}

\subsubsection{Quadratic Formula}

Apply the formula:

$$
x = \frac{-b \pm \sqrt{b^2 - 4ac}}{2a}
$$

To solve for the 2 complex conjugate roots.

\subsubsection{Completing the Square}

Complete the square:

$$
\begin{aligned}
x^2 + 4x + 10 & = 0 \\
(x + 2)^2 + 6 & = 0 \\
(x + 2)^2 & = -6 \\
x & = -2 \pm \sqrt{6}i
\end{aligned}
$$

\subsection{Expression From Roots}

Given a complex root for a quadratic $z = a + bi$.

The other root is its conjugate $\overline{z} = a - bi$.

The quadratic expression can be found by expanding:

$$
(x - z)(x - \overline{z}) = (x - (a + bi))(x - (a - bi))
$$

\subsection{Coefficients From Roots}

Given the quadratic $ax^2 + bx + c$ with roots $x_1$ and $x_2$ (real or
complex):

$$
\begin{aligned}
-\frac{b}{a} & = x_1 + x_2 \\
\frac{c}{a} & = x_1 x_2 \\
\end{aligned}
$$


\section{Argand Plane}

A cartesian plane, where the y axis is the imaginary axis, and the x axis the
real axis.

Complex numbers can be plot as vectors on this plane.

\subsection{Magnitude}

The length of the vector formed by the complex number when placed on the plane.

Given $z = a + bi$, the magnitude $\lvert z \rvert$ is:

$$
\lvert z \rvert = \sqrt{a^2 + b^2}
$$

\subsubsection{Properties}

The magnitude of a complex number has the following properties:

$$
\begin{aligned}
\lvert z_1 z_2 \rvert & = \lvert z_1 \rvert \lvert z_2 \rvert \\
\lvert \frac{z_1}{z_2} \rvert & = \frac{\lvert z_1 \rvert}{\lvert z_2 \rvert} \\
\end{aligned}
$$

\subsection{Argument}

The angle the complex number makes with the positive real axis.

Given $z = a + bi$, the argument $\Arg(z)$ is:

$$
\Arg(z) = \tan^{-1}(\frac{b}{a})
$$

Usually given in radians in the range $-\pi < \Arg(z) \leq \pi$.

\subsection{Transformations}

\subsubsection{Conjugate}

Reflection about the x axis.

\subsubsection{Multiplication by i}

Rotation anti-clockwise by $90^\circ$.

Expontents of $i$ represent multiples of this rotation:

\begin{center}
\begin{tabular}{c|c}
$n$ & Rotation for $i^n$ \\
\hline
-1 & $90^\circ$ clockwise \\
0 & No rotation \\
1 & $90^\circ$ anti-clockwise \\
2 & $180^\circ$ \\
3 & $90^\circ$ clockwise \\
4 & No rotation \\
\end{tabular}
\end{center}


\section{Polar Form}

Representation of a complex number by a magnitude and angle, measured
anti-clockwise from the positive real axis.

For a complex number of magnitude $r$, making an angle of $\theta$ with the
positive real axis:

$$
z = r (\cos{\theta} + i\sin{\theta})
$$

Also written:

$$
z = r \cis{\theta}
$$

Where $r > 0$ and $-\pi < \theta \leq \pi$.

\subsection{Simplifcation}

Add or subtract multiples of $2\pi$ to restrict $\theta$ to the range
$-\pi < \theta \leq \pi$.

If $r < 0$, take the absolute value of $r$, and add $\pi$ to $\theta$ to rotate
by $180^\circ$.

\subsection{Conversion}

\subsubsection{From Cartesian Form}

$r$ is the magnitude of the complex number.

$\theta$ is its argument.

Given the complex number $z = a + bi$, the polar form of this number will be:

$$
\begin{aligned}
z & = r \cis{\theta} \\
r & = \sqrt{a^2 + b^2} \\
\theta & = \tan^{-1}(\frac{b}{a}) \\
\end{aligned}
$$

\subsubsection{To Cartesian Form}

Expand the definition of $r \cis{\theta}$ to:

$$
z = r \cos{\theta} + ri\sin{\theta}
$$

And calculate the cosine and sine values.

\subsection{Arithmetic}

\subsubsection{Conjugate}

Negate the argument.

Given $z = r \cis{\theta}$:

$$
\overline{z} = r \cis(-\theta)
$$

Thus $\cos{-\theta} + i\sin{-\theta}$ can be written as
$\cos{\theta} - i\sin{\theta}$ (important for proofs).

\subsubsection{Negation}

Subtract or add $\pi$ to the argument.

Given $z = r \cis{\theta}$:

$$
-z = r \cis(\theta \pm \pi)
$$

Chose positive or negative depending on which will result in an argument within
the correct principle range.

\subsubsection{Addition and Subtraction}

Convert to cartesian form and perform the addition or subtraction.

\subsubsection{Multiplication}

Multiply radii and add arguments.

Given $z_1 = r_1 \cis{\theta_1}$ and $z_2 = r_2 \cis{\theta_2}$:

$$
z_1 z_2 = r_1 r_2 \cis(\theta_1 + \theta_2)
$$

\subsubsection{Division}

Divide radii and subtract arguments.

Given $z_1 = r_1 \cis{\theta_1}$ and $z_2 = r_2 \cis{\theta_2}$:

$$
\frac{z_1}{z_2} = \frac{r_1}{r_2} \cis(\theta_1 - \theta_2)
$$


\section{Regions}

\subsection{Lines Perpendicular to Axes}

\subsubsection{Real Axis}

For a line perpendicular to the real axis at $k$, where $k \in \mathbb{R}$:

$$
Re(z) = k
$$

\subsubsection{Imaginary Axis}

For a line perpendicular to the imaginary axis at $k$, where $k \in \mathbb{R}$:

$$
Im(z) = k
$$

\subsection{Circles}

For a circle centred at the origin, with radius $r$:

$$
\lvert z \rvert = r
$$

For a circle centred at $a + bi$, with radius $r$:

$$
\lvert z - (a + bi) \rvert = r
$$

For a region containing all points within a circle at center $a + bi$, with
radius $r$, including all points on the circumference:

$$
\lvert z - (a + bi) \rvert \leq r
$$

Excluding the circumference:

$$
\lvert z - (a + bi) \rvert < r
$$

\subsubsection{Maximum Modulus}

Given the equation of a circle:

$$
\lvert z - (a + bi) \rvert = r
$$

The maximum value of $\lvert z \rvert$ will be the distance to the centre of
the circle, plus the radius:

$$
= \sqrt{a^2 + b^2} + r
$$

\subsubsection{Minimum or Maximum Argument}

Given the equation of a circle:

$$
\lvert z - (a + bi) \rvert = r
$$

The minimum or maximum value of $\Arg(z)$ is a tangent to the circle,
perpendicular to the radius.

Find the angle between the tangent and the vector from the origin to the centre
of the circle using tan.

The minimum or maximum argument is then the argument to the centre of the
circle, plus or minus this angle.

\subsection{Perpendicular Bisector}

For a perpendicular bisector between $a + bi$ and $c + di$:

$$
\lvert z - (a + bi) \rvert = \lvert z - (c + di) \rvert
$$

\subsection{Rays}

For a ray extending from the origin in direction $\theta$:

$$
\Arg(z) = \theta
$$

Excludes the origin as the argument of 0 is undefined.

For a ray starting at $a + bi$ in direction $\theta$:

$$
\Arg(z - (a + bi)) = \theta
$$

Excludes the starting point $a + bi$, as the argument of 0 is undefined.

\subsection{Cartesian Equation}

Substitude $x + yi$ for $z$ in the complex equation and simplify.

\subsubsection{Circle}

$$
\begin{aligned}
\lvert (x + yi) - (a + bi) \rvert & = r \\
\lvert (x - a) + i(y - b) \rvert & = r \\
(x - a)^2 + (y - b)^2 & = r^2 \\
\end{aligned}
$$

\subsubsection{Perpendicular Bisector}

A perpendicular bisector equation will simplify to a straight line.

\subsubsection{Rays}

Rays can be represented by a line with a restriction on the $x$ value.

The gradient of the line will be $\tan{\theta}$ (given $\theta$ is in the first
quadrant):

$$
\begin{aligned}
\Arg(z) & = \theta \\
y & = \tan{\theta} x \quad x > 0 \\
\end{aligned}
$$

Must be $x > 0$ (not $x \geq 0$) since the ray is undefined at the origin.

If $\theta$ isn't in the first quadrant, then modify the restriction on $x$
accordingly.


\section{De Moivre's Theorem}

For the exponent of a complex number in polar form:

$$
(r \cis{\theta})^n = r^n \cis{n \theta}
$$

The theorem is valid for positive, negative, and fractional indices.

\subsection{Proof}

For $n = 1$:

$$
\begin{aligned}
(r \cis{\theta})^1 = r^1 \cis(1 \theta) \\
r \cis{\theta} = r \cis{\theta} \\
\end{aligned}
$$

Thus the theorem holds for $n = 1$.

Assume $n = k$:

$$
(r \cis{\theta})^k = r^k \cis(k \theta) \\
$$

Let $n = k + 1$:

$$
\begin{aligned}
(r \cis{\theta})^{k + 1} & = (r \cis(\theta))(r \cis(\theta))^k \\
& = r^{k + 1} (\cis(\theta) \cis(k \theta)) \\
& = r^{k + 1} (\cos(\theta) + i \sin(\theta))(\cos(k \theta) + i \sin(k \theta)) \\
& = r^{k + 1} (\cos(\theta) \cos(k \theta) - \sin(\theta) \sin(k \theta) + i(\sin(\theta) \cos(k \theta) + \cos(\theta) \sin(k \theta))) \\
& = r^{k + 1} (\cos((k + 1) \theta) + i \sin((k + 1) \theta)) \\
\end{aligned}
$$

\subsubsection{For Negative Integers}

$$
\begin{aligned}
(r \cis{\theta})^{-1} & = \frac{1}{r \cis{\theta}} \\
& = \frac{\cis{0}}{r \cis{\theta}} \\
& = \frac{1}{r} \cis{0 - \theta} \\
& = \frac{1}{r} \cis{-\theta} \\
\end{aligned}
$$

\subsection{For Cartesian Form}

To raise a complex number in cartesian form to a certain power:

\begin{enumerate}
\item Convert it to polar form and use De Moivre's Theorem.
\item Expand using binomial expansion and simplify.
\end{enumerate}

\subsection{Expansion of Trigonometric Functions}

Express $\cos{n \theta}$ or $\sin{n \theta}$ in terms of only $\cos{\theta}$ or
$\sin{\theta}$:

\begin{enumerate}
\item Express as $\cis{n \theta}$.
\item Expand to $\cos{n \theta} + i \sin{n \theta}$.
\item Reverse De Moivre's Theorem $(\cos{\theta} + i \sin{\theta})^n$.
\item Expand using binomial expansion.
\item Take only the real or complex component.
\end{enumerate}


\section{Roots of Real Number}

The equation:

$$
z^n = k
$$

Will have $n$ complex and real solutions.

The first solution will be the $n$th root of $k$ (ie. $\sqrt[n]{k}$).

The roots will evenly divide the complex plane into $n$ regions (evenly spaced
around a unit circle).

The roots will appear in conjugate pairs.

\subsection{Example}

For the equation:

$$
z^5 = 1
$$

The first solution is $\sqrt[5]{1} = 1$, equivalent to $\cis{0}$.

The angle between adjacent roots will be $\frac{2\pi}{5}$.

Thus all roots are:

\begin{itemize}
\item $\cis{0}$
\item $\cis{\frac{2\pi}{5}}$
\item $\cis{\frac{4\pi}{5}}$
\item $\cis{-\frac{4\pi}{5}}$
\item $\cis{-\frac{2\pi}{5}}$
\end{itemize}

Ensure all roots are given with $r > 0$ and $-\pi < \theta \leq \pi$.


\section{Roots of Complex Number}

The equation:

$$
z^n = a + bi
$$

Will have $n$ complex and real solutions.

The roots will evenly divide the complex plane into $n$ regions (evenly spaced
around a unit circle).

The roots do not necessarily appear in conjugate pairs.

\subsection{Given a Solution}

Given the equation:

$$
z^6 = -8i
$$

Has a solution $1 + i$, find the other solutions.

Implies $(1 + i)^6 = -8i$.

Writing $1 + i$ in polar form, $\sqrt{2} \cis{\frac{\pi}{4}}$.

Angle between roots will be $\frac{2\pi}{6} = \frac{\pi}{3}$.

Thus solutions are:

\begin{itemize}
\item $\sqrt{2} \cis{\frac{\pi}{4}}$ (given)
\item $\sqrt{2} \cis{\frac{7\pi}{12}}$
\item $\sqrt{2} \cis{\frac{11\pi}{12}}$
\item $\sqrt{2} \cis{-\frac{3\pi}{4}}$
\item $\sqrt{2} \cis{-\frac{5\pi}{12}}$
\item $\sqrt{2} \cis{-\frac{\pi}{12}}$
\end{itemize}

\subsection{Given No Solutions}

Given the equation:

$$
z^n = a + bi
$$

Write $a + bi$ in polar form with an additional constant:

$$
a + bi = r \cis{\theta + 2k\pi} \quad k \in \mathbb{Z}
$$

This implies:

$$
\begin{aligned}
z^n & = r \cis{\theta + 2k\pi} \\
(z^n)^{\frac{1}{n}} & = (r \cis{\theta + 2k\pi})^{\frac{1}{n}} \\
z & = \sqrt[n]{r} \cis{\frac{\theta}{n} + \frac{2k\pi}{n}} \\
\end{aligned}
$$

Substitute $n$ values of $k$ centred around 0 to find $n$ solutions for $z$.

Centre the values of $k$ around 0 to ensure the arguments are within the range
$-\pi < \theta \leq \pi$.


\section{Polynomials}

\subsection{Remainder Theorem}

The remainder when $f(x)$ is divided by $x - a$ is $f(a)$.

\subsubsection{Proof}

Consider the division identity:

$$
f(x) = D(x) Q(x) + R(x)
$$

When $f(x)$ is divided by $x - a$, the remainder must be at least 1 order less
than the divisor, thus $R(x)$ is a constant:

$$
f(x) = (x - a) Q(x) + k \quad k \in \mathbb{R}
$$

Consider $f(a)$:

$$
\begin{aligned}
f(a) & = (a - a) Q(a) + k \\
& = k \\
\end{aligned}
$$

Thus $f(a)$ is the remainder when $f(x)$ is divided by $x - a$.

\subsection{Factor Theorem}

$x - a$ is a factor of $f(x)$ if and only if $f(a) = 0$.

\subsubsection{Proof}

If $f(a) = 0$, then $f(x)$ will have a remainder of 0 when divided by $x - a$,
by the remainder theorem proved above.

Thus $x - a$ is a factor of $f(x)$.

\subsection{Factorising}

Find factors of $f(x)$ using the factor theorem by guessing values of $a$.

Try values of $a$ that are factors of the constant term.

Find as many real factors as possible.

\subsubsection{Inspection}

If sufficient real factors can be found such that we are left with a series of
linear factors and 1 complex quadratic factor, like so:

$$
\begin{aligned}
f(x) & = 8x^4 + 8x^3 - 4x^2 - 3x - 9 \\
& = (x - 1)(2x + 3)(ax^2 + bx + c) \\
\end{aligned}
$$

For $a$:

$$
\begin{aligned}
1 \times 2 \times a & = 8 \\
a & = 4 \\
\end{aligned}
$$

For $c$:

$$
\begin{aligned}
-1 \times 3 \times c & = -9 \\
c & = 3 \\
\end{aligned}
$$

For $b$ (using the cubed term):

$$
\begin{aligned}
1 \times 2 \times b + 1 \times 3 \times a - 1 \times 2 \times a & = 8 \\
2b + 12 - 8 & = 8 \\
b & = 2 \\
\end{aligned}
$$

Thus $f(x) = (x - 1)(2x + 3)(4x^2 + 2x + 3)$.

\subsubsection{Algebraic Juggling}

For example, $x^2 + 3x + 4$ divided by $x + 4$:

$$
\begin{aligned}
\frac{x^2 + 3x + 4}{x + 4} & = \frac{x(x + 4) - x + 4}{x + 4} \\
& = x + \frac{-x + 4}{x + 4} \\
& = x + \frac{-(x + 4) + 8}{x + 4} \\
& = x - 1 + \frac{8}{x + 4} \\
\end{aligned}
$$

Thus the quotient is $x - 1$, with a remainder of $8$.

\subsection{Solving}

Factorise the polynomial, leaving only linear or quadratic factors.

Apply the null factor law to find solutions.

\subsubsection{Real Coefficients}

All complex roots will appear in conjugate pairs.

\subsubsection{Complex Coefficients}

Complex roots do not necessarily appear in conjugate pairs.

Also try simple complex factors of the constant term when guessing values for
$a$.




\chapter{Calculus}

\section{Vector Calculus}

\subsection{Derivative}

Given the vector function:

$$
\bb{r} = \binom{2 + 3t}{3t + t^2}
$$

Derive each component of the vector function to find its derivative:

$$
\frac{d\bb{r}}{dt} = \binom{3}{3 + 2t}
$$

\subsection{Indefinite Integral}

Given the vector function:

$$
\bb{r} = \binom{2 + 4t}{8t + 3t^2}
$$

Integrate each component to find its indefinite integral:

$$
\int \bb{r} dt = \binom{2t + 2t^2 + c_1}{4t^2 + t^3 + c_2}
$$

\subsection{Definite Integral}

Same as normal calculus:

$$
\int_a^b \frac{d}{dt}\bb{r}(t) dt = \bb{r}(b) - \bb{r}(a)
$$


\section{Rectilinear Motion}

\subsection{Circular or Elliptical Motion}

For a body travelling in a circle or an ellipse around a fixed point.

The velocity is perpendicular to the acceleration and to the direction of
motion:

$$
\begin{aligned}
\bb{v} \cdot \bb{r} & = 0
\bb{v} \cdot \bb{a} & = 0
\end{aligned}
$$

\subsection{Cartesian Equation of Path}

Convert $\bb{r}(t)$ to a cartesian equation.

\subsection{Graphing Circular Motion}

For increasing values of $t$, plot the corresponding value of $\bb{r}(t)$ on a
cartesian plane.

To indicate the direction of motion, find $\bb{r}(0)$ to find the initial
position of the body, then find $\bb{r}(1)$ to find where it moves to next,
indicating its direction of motion.

\subsection{Projectile Motion}

To find the maximum height reached, solve for when the $\bb{j}$ component of
the velocity is 0.


\section{Differentiation Techniques}

\subsection{Implicit Differentiation}

Given some implicitly defined function $y$:

$$
y - 4x = 2
$$

We are able to differentiate each term separately to obtain an expression for
the derivative:

$$
\frac{dy}{dx} - 4 = 0
$$

\subsubsection{Product Rule}

The implicit function:

$$
3xy = 2
$$

Can be derived using the product rule:

$$
3y + 3x \frac{dy}{dx} = 0
$$

\subsubsection{Chain Rule}

Given the implicit function:

$$
y^3 - 10x = 4
$$

The first term $y^3$ is derived using the chain rule:

$$
\begin{aligned}
\frac{d}{dx} (y^3) - 10 & = 0 \\
\frac{d}{dy} (y^3) \times \frac{dy}{dx} - 10 & = 0 \\
3y^2 \frac{dy}{dx} - 10 & = 0 \\
\end{aligned}
$$

\subsubsection{Second Derivative}

Given the implicitly defined function:

$$
y^2 + 3x = 2x^3
$$

The first derivative is determined by implicit differentiation:

$$
2y \frac{dy}{dx} + 3 = 6x
$$

The second derivative is determined by deriving again:

$$
2y \frac{d^2y}{dx^2} + 2 (\frac{dy}{dx})^2 = 6
$$

\subsection{Parametric Function Differentiation}

Given the parametrical equations:

$$
x = 3t + 1 \qquad y = t^2
$$

Finding the derivative of each:

$$
\begin{aligned}
\frac{dx}{dt} & = 3 \\
\frac{dy}{dt} & = 2t \\
\end{aligned}
$$

Using the chain rule:

$$
\begin{aligned}
\frac{dy}{dx} & = \frac{dy}{dt} \frac{dt}{dx} \\
& = \frac{2t}{3} \\
\end{aligned}
$$

\subsection{Related Rates}

Given a function $y$ in terms of $x$:

$$
y = 5x^2 + 2x - 3
$$

And knowing $\frac{dx}{dt} = a$, we can find $\frac{dy}{dt}$ using the chain
rule:

$$
\begin{aligned}
\frac{dy}{dt} & = \frac{dy}{dx} \frac{dx}{dt} \\
& = a(10x + 2) \\
\end{aligned}
$$

\subsubsection{Rectilinear Motion}

Given the velocity function $v$ in terms of $x$ (the displacement):

$$
v = 3x + 4
$$

To determine the acceleration function:

$$
\begin{aligned}
a & = \frac{dv}{dt} \\
& = \frac{dv}{dx} \frac{dx}{dt} \\
& = \frac{dv}{dx} v \\
& = 3v \\
& = 3(3x + 4) \\
\end{aligned}
$$

This makes use of the fact that $\frac{dx}{dt} = v$.

This fact can also be done when integrating. Given the acceleration $a$:

$$
a = 3x^2 + 1
$$

The velocity can be found by:

$$
\begin{aligned}
\frac{dv}{dt} & = 3x^2 + 1 \\
\frac{dv}{dx} \frac{dx}{dt} & = 3x^2 + 1 \\
v \frac{dv}{dx} & = 3x^2 + 1 \\
\int v \frac{dv}{dx} dx & = \int 3x^2 + 1 dx \\
\frac{1}{2} v^2 & = x^2 + x + c \\
\end{aligned}
$$

\subsubsection{Word Problems}

The information that must be determined includes:

\begin{itemize}
\item An equation relating two variables.
\item The rate of change of one variable with respect to a third.
\end{itemize}

For example, consider a ladder of length $l$ leaning on a wall, with its base
$x$ meters from the wall and its top $y$ meters above the wall. An equation
relating $x$ and $y$ is Pythagoras:

$$
x^2 + y^2 = l^2
$$

If the base slips away from the wall at a rate of $r$ meters per second, then
$\frac{dx}{dt} = r$ (a constant rate of change). The rate at which the top
of the ladder falls towards the ground is $\frac{dy}{dt}$.

This is determined by implicitly differentiating the above equation.

\subsection{Logarithmic Differentiation}

Given the function:

$$
y = 3^x
$$

We can find the derivative by taking the log of both sides:

$$
\ln{y} = x \ln{3}
$$

Then implicitly differentiating:

$$
$$

Thus the derivative is:

$$
\begin{aligned}
\frac{1}{y} \frac{dy}{dx} & = \ln{3} \\
\frac{dy}{dx} & = y\ln{3} \\
& = 3^x \ln{3} \\
\end{aligned}
$$


\section{Integration Techniques}

\subsection{Basic}

If given an integral in the form:

$$
\int f'(x) f(x) dx
$$

Then use standard integration techniques to solve it.

\subsection{Substitution}

Involves changing the variable used in the integration.

Given:

$$
\int 56x(2x - 3)^5 dx
$$

Find an appropriate substitution. Chose a part of the function which is raised
to a power.

Let $u = 2x - 3$:

$$
\int 56xu^5 dx
$$

Convert $dx$ to $du$:

$$
\begin{aligned}
\frac{du}{dx} & = 2 \\
dx & = \frac{du}{2} \\
\end{aligned}
$$

Thus:

$$
\int 56xu^5 \frac{1}{2} du
$$

Convert $x$ to $u$ by rearranging $u = 2x - 3$ for $x$:

$$
\int 14u^5(u + 3) du
$$

Expand the brackets:

$$
\int 14u^6 + 42u^5 du
$$

Perform the integration:

$$
2u^7 + 7u^6 + c
$$

Substitute in $u = 2x - 3$:

$$
2(2x - 3)^7 + 7(2x - 3)^6 + c
$$

\subsection{Substitution with Definite Integrals}

When solving definite integrals using substitution, convert the limits of the
integral to be in terms of $u$.

Given:

$$
\int_1^2 56x(2x - 3)^5 dx
$$

Let $u = 2x - 3$.

$u$ given $x = 1$ is -1, and given $x = 2$ is 1.

From above:

$$
\begin{aligned}
& = \int_{-1}^1 14u^6 + 42u^5 du \\
& = \big[2u^7 + 7u^6\big]_{-1}^1 \\
& = 2 + 7 - (-2 + 7) \\
& = 4 \\
\end{aligned}
$$

\subsection{Trigonometric Identities}

\subsubsection{Odd Power}

For $\sin^n{x}$ or $\cos^n{x}$, where $n$ is odd, pull out a $\sin{x}$ or
$\cos{x}$ to make the power even, then substitute $1 - \cos^2{x}$ or
$1 - \sin^2{x}$ from the Pythagorean identity:

$$
\begin{aligned}
\int \sin^3{x} dx & = \int \sin{x}\sin^2{x} dx \\
& = \int \sin{x}(1 - \cos^2{x}) dx \\
& = \int \sin{x} - \sin{x}\cos^2{x} dx \\
& = -\cos{x} + \frac{1}{3} \cos^3{x} + c \\
\end{aligned}
$$

\subsubsection{Even Power}

For $\sin^n{x}$ or $\cos^n{x}$, where $n$ is even, use the double angle formula
for $\cos{2x}$:

$$
\begin{aligned}
\cos{2x} & = 1 - 2\sin^2{x} \\
& = 2\cos^2{x} - 1 \\
\end{aligned}
$$

Rearrange the equation in terms of $\cos{2x}$:

$$
\begin{aligned}
\int \sin^2{x} dx & = \int \frac{1}{2} - \frac{1}{2}\cos{2x} dx \\
& = \frac{x}{2} - \frac{1}{4}\sin{2x} + c \\
\end{aligned}
$$

\subsubsection{Secant}

For $\tan^2{x}$, substitute $\sec^2{x} - 1$ from the identity:

$$
\sec^2{x} = \tan^2{x} + 1
$$

\subsubsection{Product to Sum}

For:

$$
\int \sin{ax}\cos{bx} dx
$$

Use the product to sum identities (these should be given in the question).

\subsection{Logarithmic Integration}

$$
\begin{aligned}
\int \frac{1}{x} dx & = \ln{|x|} + c \qquad x \neq 0 \\
\int \frac{f'(x)}{f(x)} dx & = \ln(|f(x)|) + c \qquad f(x) \neq 0 \\
\end{aligned}
$$
\subsubsection{Improper Fractions}

Given an improper fraction to integrate:

$$
\int \frac{2x}{x + 1} dx
$$

Convert the fraction to a proper one (use algebraic juggling):

$$
\int 2 - \frac{2}{x + 1} dx
$$

And integrate:

$$
2x - 2\ln{|x + 1|} + c
$$

\subsection{Logarithmic Definite Integrals}

Given:

$$
\int_a^b \frac{f'(x)}{f(x)} dx
$$

If $f(x) = 0$ for any $x$ in the interval $a \leq x \leq b$, then the integral
is undefined.

\subsection{Partial Fractions}

For proper fractions with a series of factors in the denominator.

The number of unknowns in the numerator will be the same as the degree of the
denominator.

\subsubsection{Linear Factors}

Given a proper fraction with a series of linear factors in the denominator:

$$
\int \frac{4x - 3}{(x + 3)(2x + 1)} dx
$$

Consider the inner fraction:

$$
\frac{4x - 3}{(x + 3)(2x + 1)}
$$

Split the fraction into parts, where the denominator of each is one of the
linear factors, and the numerator is some polynomial of one degree less than
the denominator:

$$
\frac{4x - 3}{(x + 3)(2x + 1)} = \frac{A}{x + 3} + \frac{B}{2x + 1}
$$

Multiply by $(x + 3)(2x + 1)$:

$$
4x - 3 = A(2x + 1) + B(x + 3)
$$

Compare coefficients to solve for $A$ and $B$:

$$
\begin{aligned}
4 & = 2A + B \\
-3 & = A + 3B \\
\end{aligned}
$$

Solve the system of equations by elimination:

$$
\begin{aligned}
10 & = -5B \\
B & = -2 \\
A & = 3 \\
\end{aligned}
$$

Thus the partial fractions are:

$$
\frac{3}{x + 3} - \frac{2}{2x + 1}
$$

Use logarithms to solve the integral:

$$
\begin{aligned}
\int \frac{4x - 3}{(x + 3)(2x + 1)} dx & = \int \frac{3}{x + 3} - \frac{2}{2x + 1} dx \\
& = 3\ln{|x + 3|} - \ln{|2x + 1|} \\
\end{aligned}
$$

\subsubsection{Cover Up Method}

A method to easily find the required partial fractions.
Only applies to fractions with linear factors.

Again, consider the fraction:

$$
\frac{4x - 3}{(x + 3)(2x + 1)}
$$

We require an expression in the form:

$$
\frac{A}{x + 3} + \frac{B}{2x + 1}
$$

For each linear factor, solve for $x$ when the factor equals 0.

Then cover up this linear factor in the original fraction, and substitute in
the value for $x$ to find the corresponding constant in the partial fractions.

To find $A$, solve $x + 3 = 0$, hence $x = -3$. Cover up the factor $x + 3$ in
the original fraction to give:

$$
\frac{4x - 3}{2x + 1}
$$

And substitute $x = -3$ to give $A = -3$.

To find $B$, solve $2x + 1 = 0$, hence $x = -0.5$. Substitute this into:

$$
\frac{4x - 3}{x + 3}
$$

To give $B = -2$.

\subsubsection{Quadratic Factors}

To find the integral:

$$
\int \frac{7x^2 - 2x + 5}{(x - 1)(x^2 + 1)} dx
$$

Use the partial fractions:

$$
\frac{A}{x - 1} + \frac{Bx + C}{x^2 + 1}
$$

Solve for the constants by equating this to the original fraction:

$$
\frac{7x^2 - 2x + 5}{(x - 1)(x^2 + 1)}  = \frac{A}{x - 1} + \frac{Bx + C}{x^2 + 1}
$$

Multiply through by $(x - 1)(x^2 + 1)$:

$$
7x^2 - 2x + 5  = A(x^2 + 1) + (Bx + C)(x - 1)
$$

Solve for $A$, $B$, and $C$ by comparing coefficients:

$$
\begin{aligned}
7 & = A + B \\
5 & = A - C \\
-2 & = -B + C \\
\end{aligned}
$$

Solve the system of equations by elimination:

$$
\begin{aligned}
5 & = A + C \\
10 & = 2A \\
A & = 5 \\
B & = 2 \\
C & = 0 \\
\end{aligned}
$$

Substitute these values into the partial fractions and solve the integral using
logarithms.

\subsubsection{Repeated Factors}

Given the integral:

$$
\int \frac{6x^2 + 6x - 20}{(x + 3)(x - 1)^2} dx
$$

Use the partial fractions:

$$
\frac{A}{x + 3} + \frac{B}{x - 1} + \frac{C}{(x - 1)^2}
$$

Solve for the constants by equating the two:

$$
\frac{6x^2 + 6x - 20}{(x + 3)(x - 1)^2} = \frac{A}{x + 3} + \frac{B}{x - 1} + \frac{C}{(x - 1)^2}
$$

Multiply by the denominator:

$$
6x^2 + 6x - 20 = A(x - 1)^2 + B(x + 3)(x - 1) + C(x + 3)
$$

Solve by comparing coefficients:

$$
\begin{aligned}
6 & = A + B \\
-20 & = A + 2B + 3C \\
6 & = -2A + B + C \\
\end{aligned}
$$

Solve the system of equations:

$$
\begin{aligned}
A & = -4 \\
B & = 10 \\
C & = -12 \\
\end{aligned}
$$

Perform the integration using logarithms.

\subsection{Integration by Parts}

Used to integrate expressions with two functions multiplied together.

The formula is derived from the product rule:

$$
\begin{aligned}
(f(x)g(x))' & = f'(x)g(x) + g'(x)f(x) \\
f(x)g(x) & = \int f'(x)g(x) dx + \int g'(x)f(x) dx \\
\int f'(x)g(x) dx & = f(x)g(x) - \int g'(x)f(x) dx \\
\end{aligned}
$$

Given the integral:

$$
\int e^x x dx
$$

Let $f'(x) = e^x$ and $g(x) = x$.

Hence $f(x) = e^x$ (ignore the integrating constant) and $g'(x) = 1$.

Using the formula from above:

$$
\begin{aligned}
\int e^x x dx & = e^x x - \int 1 \times e^x dx \\
& = x e^x - e^x + c \\
\end{aligned}
$$

\subsection{Summary}

Given a certain function in different forms, use several different methods for
integration:

Basic:

$$
\int f'(x) f(x) dx
$$

Substitution (let $u = f(x)$):

$$
\int g(x) (f(x))^n dx
$$

Use trigonometric identities for an integration involving some combination of
trigonometric functions.

Logarithms:

$$
\int \frac{f'(x)}{f(x)} dx
$$

Partial fractions, using the cover up method:

$$
\int \frac{f(x)}{(x + a)(x + b)}
$$

Partial fractions with a quadratic factor:

$$
\int \frac{f(x)}{(x^2 + bx + c)(x + d)}
$$

Partial fractions with a repeated factor:

$$
\int \frac{f(x)}{(x + a)^2 (x + b)}
$$

Integration by parts:

$$
\int f(x)g(x) dx
$$


\section{Numerical Integration}

Several functions cannot be algebraically integrated.
Numerical integration approximates a definite integral of such functions.

Given the definite integral:

$$
\int^2_0 e^{x^2} dx
$$

Different rules express the integral as an infinite sum of the areas of adjacent
shapes.

The integral can be approximated by taking a finite number of terms from such
sequences.

Summing more terms improves the accuracy of the approximation.

\subsection{Trapezium Rule}

The area of a trapezium with parallel side lengths $y_0$ and $y_1$, and
perpendicular height $h$ is:

$$
h\frac{y_0 + y_1}{2}
$$

Summing the areas of these trapeziums:

$$
\begin{aligned}
& = \sum_{i = 1}^n h\frac{y_{i - 1} + y_i}{2} \\
& = \frac{h}{2} \sum_{i = 1}^n y_{i - 1} + y_i \\
& = \frac{h}{2} (y_0 + 2\sum_{i = 1}^{n - 1} y_i + y_n) \\
\end{aligned}
$$

Given some definite integral:

$$
\int^a_b f(x) dx
$$

We have:

$$
\begin{aligned}
h & = \frac{a - b}{n} \\
y_0 & = f(b) \\
y_n & = f(a) \\
y_i & = f(b + ih) \\
\end{aligned}
$$

\subsection{Simpson's Rule}

Models the tops of the strips as parabolas.

Only applies for an even number of strips.

$$
\frac{h}{3}(y_0 + \sum_{i = 1}^{n - 2} (4y_i + 2y_{i + 2}) + 4y_{n - 1} + y_n)
$$

$h$ and $y_i$ are calculated the same as when applying the Trapezium rule.


\section{Volumes of Revolution}

The volume of the solid formed when an area under a curve is rotated in 3D
around a line (usually an axis).

\subsection{Around X Axis}

Split the area into vertical disks of thickness $\delta x$.

The volume of one of these disks is:

$$
\pi y^2 \delta x
$$

Summing the volume of an infinite number of these disks between $x = a$ and
$x = b$ gives:

$$
\lim_{\delta x \to 0} \sum_{x = a}^{b} \pi y^2 \delta x
$$

Using the definition of integration:

$$
\pi \int^b_a y^2 dx
$$

\subsubsection{Area with the Y Axis Rotated About the X Axis}

The volume of a solid formed when an area bounded by a function $f(x)$, the x
axis, and the lines $x = a$ and $x = b$ is rotated about the y axis.

Consider the inner surface area of a cylinder of radius $x$ and height $y$:

$$
2 \pi x y
$$

The volume of the solid is the sum of an infinite number of these cylinders
between $x = a$ and $x = b$:

$$
2 \pi \int^b_a x y dx
$$

\subsection{Around Y Axis}

Split the area into horizontal disks of thickness $\delta y$.

The volume of one disk is:

$$
\pi x^2 \delta y
$$

Summing the volumes of these disks between $y = a$ and $y = b$ gives:

$$
\lim_{\delta y \to 0} \sum_{y = a}^{b} \pi x^2 \delta y
$$

Using the definition of integration:

$$
\pi \int^b_a x^2 dy
$$

\subsubsection{Area with the X Axis Rotated About the Y Axis}

The volume of a solid formed when an area bounded by a function $f(x)$, the y
axis, and the lines $y = a$ and $y = b$ is rotated about the x axis.

Consider the inner surface area of a cylinder of radius $y$ and height $x$:

$$
2 \pi x y
$$

The volume of the solid is the sum of an infinite number of these cylinders
between $y = a$ and $y = b$:

$$
2 \pi \int^b_a x y dy
$$

\subsection{Around a Vertical Line}

Perform a change in origin for the function. To rotate around the line:

$$
x = a
$$

Use the function $f(x - a)$ to make the function centred around $a$.

\subsection{Area Between Two Curves}

The volume of a solid formed when the area between the curves $f(x)$ and $g(x)$
is rotated about one of the axes.

Given $f(x)$ is above $g(x)$ for the interval over which we're rotating,
subtract the volume of the solid formed using $g(x)$ from that of $f(x)$:

For example, rotating about the $x$ axis and bound by the lines $x = a$ and
$x = b$:

$$
\pi \int^b_a (f(x))^2 dx - \pi \int^b_a (g(x))^2 dx
$$

Use a graph of the two functions to determine the required areas in more
complicated scenarios.


\section{Differential Equations}

A \textbf{differential equation} is where the defining expression for a function
$y$ involves one of its derivatives.

The \textbf{order} of a differential equation is equal to the highest order of
any derivative involed in the function.

To \textbf{solve} a differential equation, we must find a relationship between the
variables that does not involve a derivative.

\subsection{Separating the Variables}

This involves manipulating the expression into the form:

$$
f(y) \frac{dy}{dx} = g(x)
$$

We can then integrate both sides with respect to $x$:

$$
\begin{aligned}
\int f(y) \frac{dy}{dx} dx & = \int g(x) dx \\
\int f(y) dy & = \int g(x) dx \\
\end{aligned}
$$

This will produce two constants of integration, although these can be written
as a single one:

$$
F(y) = G(x) + c
$$

Where $F(y)$ and $G(x)$ are the anti-derivatives of $f(y)$ and $g(x)$
respectively.

\subsection{Logistic Growth Model}

A model for growth which initially rapidly increases, but plateaus off at a
certain $y$ value (a horizontal asymptote):

$$
\frac{dy}{dx} = ay - by^2
$$

Where $a > 0$ and $b > 0$.

It has the general solution:

$$
y = \frac{a}{b + ce^{-at}}
$$

\subsubsection{Proof}

% TODO. Involves partial fractions and logarithmic integration

\subsection{Slope Field}

A \textbf{slope} or \textbf{gradient field} shows the gradient of a function at
each point on a Cartesian plane, through a short line segment pointing in a
certain direction.

To determine what a slope field looks like for a differential equation, consider
for what $x$ or $y$ values the derivative is 0, positive, and negative.

\subsection{Euler's Approximation}

\textbf{Euler's method} is used to determine approximate solutions to
differential equations, using the small changes formula:

$$
\delta y \approx \frac{dy}{dx} \delta x
$$

Given an expression involving the derivative, and a known $x$ and $y$
coordinate, the procedure is:

\begin{itemize}
\item Compute the derivative at the starting coordinate $(x, y)$.
\item Chose an increment for $\delta x$.
\item Calculate the corresponding $\delta y$ using the small changes formula.
\item This results in a new coordinate to plot $(x + \delta x, y + \delta y)$.
\item Repeat this procedure for the new coordinate.
\end{itemize}


\section{Simple Harmonic Motion}

\textbf{Simple harmonic motion} is where an object's acceleration is negatively
proportional to its position from its mean point:

$$
\frac{d^2x}{dt^2} = -k^2 x
$$

This causes an object to \textbf{oscillate} about a mean point.

The \textbf{amplitude} ($a$) of the motion is the distance between the object's
mean point of oscillation and its maximum displacement from this mean point.

The \textbf{period} ($T$) of the motion is the time the object takes to
complete one full oscillation.

The \textbf{phase angle} ($\alpha$) of the motion is where the object starts
within a cycle.

\subsection{Solving}

To \textbf{solve} a differential equation in this form:

$$
\begin{aligned}
\frac{dv}{dt} & = -k^2 x \\
\frac{dv}{dx} \frac{dx}{dt} & = -k^2 x \\
v \frac{dv}{dx} & = -k^2 x \\
\int v dv & = \int -k^2 x dx \\
\frac{v^2}{2} & = -\frac{k^2 x^2}{2} + c \\
\end{aligned}
$$

Given the object's amplitude $a$, it is known that when the displacement is at
its maximum (ie. $x = a$), the velocity is 0 ($v = 0$):

$$
\begin{aligned}
0 & = -\frac{k^2 a^2}{2} + c \\
c & = \frac{k^2 a^2}{2} \\
\end{aligned}
$$

Substituting into the original velocity equation, we have:

$$
v^2 = k^2(a^2 - x^2)
$$

Continuing further to find an equation for displacement:

$$
\begin{aligned}
\frac{dx}{dt} & = k\sqrt{a^2 - x^2} \\
\int \frac{1}{\sqrt{a^2 - x^2}} dx & = \int k dt \\
\end{aligned}
$$

Using the substitution $x = a \sin{u}$, which has the derivative
$\frac{dx}{du} = a \cos{u}$:

$$
\begin{aligned}
\int du & = kt + \alpha \\
u & = kt + \alpha \\
\end{aligned}
$$

From the earlier substitution equation:

$$
x = a \sin(kt + \alpha)
$$

\subsection{Relationships}

An equation for an object's acceleration is:

$$
a = -k^2 x
$$

An equation for its velocity is:

$$
v^2 = k^2(a^2 - x^2)
$$

An equation for its displacement is:

$$
x = a \sin(kt + \alpha)
$$

Another equation for its velocity is thus:

$$
v = ka \cos(kt + \alpha)
$$

The period is (identical to the period of the sin function in the displacement
equation):

$$
k = \frac{2\pi}{T}
$$

The phase angle is $\alpha$.

If an object starts its motion at its mean displacement (the origin), then:

$$
x = a \sin(kt)
$$

If an object starts its motion at its maximum or minimum displacement, then:

$$
x = a \sin(kt + \frac{\pi}{2}) = a \cos(kt)
$$

An object's maximum velocity is:

$$
|ka|
$$

When an object is at its mean displacement:

\begin{itemize}
\item $x = 0$
\item $v$ is at its maximum or minimum value
\end{itemize}

When an object is at its maximum or minimum displacement:

\begin{itemize}
\item $x$ is at its maximum or minimum value
\item $v = 0$
\end{itemize}

\subsection{Demonstrating Simple Harmonic Motion}

Given an equation for $x$:

$$
x = a \sin(kt + \alpha)
$$

We can show that this motion is simple harmonic motion by demonstrating that
the equation satisfies the rule:

$$
\frac{d^2x}{dt^2} = -k^2 x
$$

Differentiating once:

$$
\frac{dx}{dt} = ka \cos(kt + \alpha)
$$

Differentiating again:

$$
\begin{aligned}
\frac{d^2x}{dt^2} & = -k^2 a \sin(kt + \alpha) \\
& = -k^2 x \\
\end{aligned}
$$




\chapter{Functions}

\section{Function and Relations}

A function is a series of operations that maps one or more input values to one
output value.

A relation is a series of operations that maps one or more input values to one
or more output values.

\subsection{Types}

A one-to-one function means each input value has one possible output value.
For example, $f(x) = x$.

A many-to-one function means multiple input values will map to the same output
value. For example, $f(x) = x^2$.

A one-to-many relation means one input value will map to multiple output values.
For example, $f(x) = \pm \sqrt{x}$.

A many-to-many relation means multiple input values will map to multiple output
values. For example, $x^2 + y^2 = 9$.

\subsection{Vertical Line Test}

Where a vertical line is moved from left to right over a graph.

If it intersects the curve more than once at any point, the graph is a relation,
not a function.

\subsection{Horizontal Line Test}

Where a horizontal line is moved from top to bottom over a graph.

If it intersects the curve more than once at any point, the function is
one-to-many (not one-to-one).


\section{Domain}

The set of all possible input values for a function.

\subsection{Polynomials}

All polynomials have a domain of $x \in \mathbb{R}$.

\subsection{Fractional Polynomials}

For any function with $x$ in the denominator of a fraction, the entire
denominator cannot equal 0.

For example:

$$
f(x) = \frac{g(x)}{h(x)}
$$

Solve $h(x) = 0$ to determine the values of $x$ which are not in the domain.

\subsection{Square Root}

The contents of the square root must be greater than or equal to 0 (positive).

For example:

$$
f(x) = \sqrt{g(x)}
$$

Solve $g(x) > 0$ to determine the values of $x$ in the domain.

\subsection{Sine and Cosine}

The domain for $\sin{x}$ and $\cos{x}$ is $x \in \mathbb{R}$.

\subsection{Tangent}

The domain for $\tan{x}$ excludes all vertical asymptotes,
$x \neq \frac{\pi}{2} + n\pi, n \in \mathbb{Z}$.

\subsection{Logarithm}

The domain for $\ln{x}$ is $x > 0$, for $\ln{|x|}$ its $x \neq 0$.


\section{Range}

The set of all output values of a function.

The range is a subset of the codomain.

\subsection{Codomain}

The set of all possible output values of a function.

The codomain is part of the definition of the function. It is usually assumed
to be all real numbers, but can be defined as something else.

\subsubsection{Example}

Define the function $f(x) = x^2$ with a domain $x \in \mathbb{Z}$ and codomain
$y \in \mathbb{R}$.

The range of the function is ${1, 4, 9, 16, \ldots}$, yet the codomain was
defined to be all real numbers.

Thus the range is only a subset of the codomain.

\subsection{Calculating Range}

\subsubsection{Polynomials}

Polynomials with an odd degree have a domain and range across all real numbers.

Polynomials with an even degree have a domain across all real numbers and a
range above or below the lowest or highest turning point (depending if the
function has a maximum or minimum turning point).

\subsubsection{Hyperbolas}

Draw a rough sketch of the graph to determine the range of $y$ values which the
function never outputs.

\subsubsection{Square Root}

The result of a square root is greater than or equal to 0.


\section{Asymptotes}

To determine asymptotes for the function:

$$
f(x) = \frac{g(x)}{h(x)}
$$

Where $g(x)$ and $h(x)$ are polynomials.

\subsection{Vertical Asymptotes}

Occur when the denominator is 0.

Solve $h(x) = 0$.

\subsubsection{Behaviour Around Vertical Asymptotes}

Given there is a vertical asymptote at $x = a$.

The function will approach either positive or negative infinity either side of
$x = a$.

For the behaviour of the function as it approaches the asymptote from the left:

\begin{itemize}
\item If $f(a^-)$ is negative, the function approaches negative infinity.
\item If $f(a^-)$ is postiive, the function approaches positive infinity.
\end{itemize}

For the behaviour of the function as it approaches the asymptote from the right:

\begin{itemize}
\item If $f(a^+)$ is negative, the function approaches negative infinity.
\item If $f(a^+)$ is postiive, the function approaches positive infinity.
\end{itemize}

\subsubsection{Example}

Given the function:

$$
f(x) = \frac{2x - 1}{x^2 + 4x}
$$

Solving the denominator equal to 0 for vertical asymptotes:

$$
\begin{aligned}
x^2 + 4x & = 0 \\
x(x + 4) & = 0 \\
x & = 0, -4 \\
\end{aligned}
$$

Thus there are two vertical asymptotes with the equations $x = 0$ and $x = -4$.

Approaching the asymptote $x = -4$ from the left. Consider $f(-4^-)$:

\begin{itemize}
\item $2x$ is negative.
\item $2x - 1$ is still negative.
\item Consider the denominator as $x(x + 4)$.
\item $x + 4$ is still slightly negative.
\item $x(x + 4)$ is therefore two negatives multiplied together, forming a
	positive.
\item Thus overall the fraction is negative, so it approaches negative infinity.
\end{itemize}

Approaching the asymptote $x = -4$ from the right. Consider $f(-4^+)$:

\begin{itemize}
\item $2x - 1$ is negative.
\item $x + 4$ is slightly positive.
\item $x(x + 4)$ is therefore negative.
\item Thus overall the fraction is positive, so it approaches positive infinity.
\end{itemize}

\subsection{Oblique Asymptote}

A function will have at most 1 oblique asymptote.

It only occurs when the degree of the numerator is 1 larger than the denominator.

The linear equation for the asymptote can be found by simplifying the fraction
using algebraic juggling.

\subsection{Horizontal Asymptote}

A function will have at most 1 horizontal asymptote.

If the degree of the numerator is larger, there is no horizontal asymptote.

If the degree of numerator and denominator are the same, then there is a
horizontal asymptote at:

$$
\frac{\text{leading coefficient of $g(x)$}}{\text{leading coefficient of $h(x)$}}
$$

If the degree of the denominator is larger, then there is a horizontal
asymptote at $y = 0$.

\subsubsection{Behaviour Approaching Infinity}

The horizontal asymptote is apparent due to the behaviour of the function as
$x$ approaches $\pm \infty$.

It is important whether the function approaches the horizontal asymptote from
above or below.

Consider:

$$
\frac{\text{leading term of $g(x)$}}{\text{leading term of $h(x)$}}
$$

If the fraction is positive for $x = \infty$, then the function approaches the
horizontal asymptote from above for large positive values of $x$.

If the fraction is negative for $x = \infty$, then the function approaches the
horizontal asymptote from below for large positive values of $x$.

If the fraction is positive for $x = -\infty$, then the function approaches the
horizontal asymptote from above for large negative values of $x$.

If the fraction is negative for $x = -\infty$, then the function approaches the
horizontal asymptote from below for large negative values of $x$.


\section{Graphing Hyperbolas}

To graph the function, where $g(x)$ and $h(x)$ are polynomials:

$$
f(x) = \frac{g(x)}{h(x)}
$$

First attempt to simplify the function to a set of transformations on a normal
hyperbola:

$$
f(x) = \frac{a}{b(x + c)} + d
$$

Otherwise:

\begin{description}
\item [Asymptotes] Find all vertical, horizontal, and oblique asymptotes and
	the behaviour of the function either side of them.
\item [Axis Intercepts] Place a point at all axis intercepts.
\item [Stationary Points] Place a point at all stationary points (using
	$f'(x) = 0$).
\end{description}

\subsection{Simplification}

Given:

$$
f(x) = \frac{x + 3}{x + 3}
$$

This is equivalent to a graph of $f(x) = 1$, but with a point of discontinuity
at $x = -3$ (an open circle).

\subsection{Range}

For some $f(x)$ there is a gap in the middle of the graph for which no $y$ value
is defined.

Given:

$$
f(x) = \frac{3x - 2}{x(x + 2)}
$$

To solve for this gap algebraically:

$$
\begin{aligned}
y & = \frac{3x - 2}{x(x + 2)} \\
yx^2 + 2yx & = 3x - 2 \qquad x \neq 0, -2 \\
yx^2 + (2y - 3)x + 2 & = 0 \\
\end{aligned}
$$

This is a quadratic equation in terms of $x$.

For real valued $y$, the discriminant must be non-negative:

$$
\begin{aligned}
b^2 - 4ac & \geq 0 \\
(2y - 3)^2 - 8y & \geq 0 \\
4y^2 - 20y + 9 & \geq 0 \\
(2y - 9)(2y - 1) & \geq 0 \\
\end{aligned}
$$

Consider the sign of each factor between the roots of this quadratic:

\begin{center}
\begin{tabular}{c|c|c|c}
& $y < 0.5$ & $0.5 < y < 4.5$ & $y > 4.5$ \\
\hline
$2y - 9$ & - & - & + \\
$2y - 1$ & - & + & + \\
$(2y - 9)(2y - 1)$ & + & - & + \\
\end{tabular}
\end{center}

Thus the range for $f(x)$ is $y < 0.5$ and $y > 4.5$.


\section{Composite Functions}

Using the output of one function as the input for another:

$$
f(g(x))
$$

\subsection{Domain and Range}

\subsubsection{Method 1}

Find the resulting expression for the composite function, and determine the
domain and range of it.

Beware of simplifying (especially with square roots), as it can lead to a wrong
answer.

\subsubsection{Method 2}

Given the composite function $f(g(x))$.

$f(x)$ must be able to cope with the output of $g(x)$ for the composite
function to be defined.

Thus the range of $g(x)$ must be a subset of the domain of $f(x)$. It must not
output values $f(x)$ is not defined for.

Thus the domain of $g(x)$ must be restricted to exclude values from its range
for which $f(x)$ is not defined.

For example, given $f(x) = \frac{1}{x - 1}$ and $g(x) = x - 5$.

$f(x)$ has the domain $x \in \mathbb{R}, x \neq 1$ and range
$y \in mathbb{R}, y \neq 0$.

$g(x)$ has a domain and range $x, y \in \mathbb{R}$

For the domain of $f(g(x))$:

$$
\begin{aligned}
g(x) & \neq 1 \\
x - 5 & \neq 1 \\
x \neq 6 \\
\end{aligned}
$$

Thus the domain for $f(g(x))$ is $x \in \mathbb{R}, x \neq 6$.

The range for $f(g(x))$ will be the same as $f(x)$:
$y \in \mathbb{R}, y \neq 0$.


\section{Inverse Functions}

Reverses the operations performed by a function such that:

$$
f^{-1}(f(x)) = x
$$

\subsection{Conditions for an Inverse to Exist}

In order for the inverse of a function to exist as a function (and not a
relation), $f(x)$ must be a one-to-one function.

$f(x)$ must pass the horizontal line test.

\subsection{Domain and Range}

The domain of $f^{-1}(x)$ is the range of $f(x)$.

The range of $f^{-1}(x)$ is the domain of $f(x)$.

\subsection{Graphical Relation}

The inverse of a function represents a reflection about the line $y = x$.

The points at which $f(x)$ will intersect with its inverse are the same as the
intersection between $f(x)$ and the line $y = x$.

\subsection{Calculating}

Given the function $y$:

$$
y = 3x + 4
$$

The inverse function can be found by rearranging for $x$:

$$
\begin{aligned}
3x & = y - 4 \\
x & = \frac{y - 4}{3} \\
f^{-1}(x) & = \frac{x - 4}{3} \\
\end{aligned}
$$

\subsection{Many-To-One Functions}

The domain or range of many-to-one (or many-to-many) functions can be
restricted to make them one-to-one functions.

An inverse function can then be defined for the restricted domain or range.

\subsubsection{Quadratics}

We must decide whether to chose the positive or negative side of the quadratic
depending on the original function.

The inverse of $f(x) = x^2$ for $x < 0$ is $f^{-1}(x) = -\sqrt{x}$. We must
chose the negative square root (rather than using $\pm\sqrt{x}$) due to the
original restriction on the domain of $f(x)$.

The inverse of $f(x) = \sqrt{x - 5}$ is $f^{-1}(x) = x^2 + 5$. We must restrict
the domain of this inverse to accurately reflect $f(x)$.

The range of $f(x)$ is the domain of $f^{-1}(x)$, thus
$x \in \mathbb{R}, x \geq 0$.

\subsubsection{Example}

The function $f(x) = (x - 1)^2$ does not have an inverse function for its
natural domain.

$f(x)$ has a range $y \in \mathbb{R}, y \geq 0$.

The domain can be restricted to $x \geq 1$ or $x \leq 1$. The function is
one-to-one for this domain, and has an inverse.

The inverse is $f^{-1}(x) = 1 \pm \sqrt{x}$. We must chose either the positive
or negative square root based on our original domain restriction.

For the restriction $x \geq 1$ on $f(x)$, the inverse has a domain $x \geq 0$
and range $x \geq 1$.

To satisfy this range, we must chose the positive. The inverse is then
$f^{-1}(x) = 1 + \sqrt{x}$.


\section{Absolute Value Function}

The absolute value function $f(x) = \lvert x \rvert$ is defined as:

$$
\lvert x \rvert = \begin{cases}
\begin{aligned}
	x \qquad & x \geq 0, \\
	-x \qquad & x < 0, \\
\end{aligned}
\end{cases}
$$

It can also be defined as:

$$
\lvert x \rvert = \sqrt{x^2}
$$

The order of the square root and square is important.

All parts of the graph of any $f(x)$ that are below the x axis are reflected
about the x axis.

\subsection{Solving Equations}

$\lvert x \rvert$ represents the distance of $x$ from the origin.

If $\lvert x \rvert = k$, then either $x = k$ or $x = -k$.

Each solution must be checked by substituting it back into the original
equation, as false solutions can arise.

\subsection{Solving Complicated Equations}

Given an equation such as $2 - \lvert x + 1 \rvert = \lvert 3x + 2 \rvert$.

There are four possible equations that can be formed:

\begin{itemize}
\item $2 - (x + 1) = 3x + 2$
\item $2 - (x + 1) = -(3x + 2)$
\item $2 + (x + 1) = 3x + 2$
\item $2 + (x + 1) = -(3x + 2)$
\end{itemize}

These represent the intersection of each branch of the left hand function with
each branch of the right hand one.

Two of these equations will give invalid solutions, so substitute each result
back into the original equation to check its validity.

\subsubsection{Given a Graph}

Given a graph of the left and right hand sides of the equation, we can eliminate
the two equations which give invalid solutions.

For each intersection between the two functions in the graph, determine on which
branch of each function the intersection point lies.

Equate the equation of each of these branches to solve for $x$.

\subsubsection{Squaring}

Given the equation $\lvert x + 1 \rvert = \lvert 3x + 2 \rvert$.

We can also use the definition of the absolute value as $\sqrt{x^2}$ to solve
the equation.

Square both sides:

$$
(x + 1)^2 = (3x + 2)^2
$$

Expand the brackets and solve the resulting quadratic.

Substitute each solution back into the original inequality to ensure their
validity.

\subsection{Solving Inequalities}

$\lvert x \rvert < a$ implies the distance from $x$ to the origin is less than
$a$. This means:

$$
-a < x < a
$$

$\lvert x \rvert > a$ implies the distance from $x$ to the origin is greater
than $a$. This means:

$$
x < -a \qquad x > a
$$

$\lvert x - b \rvert < a$ implies the distance from $x$ to $b$ on a number line
is less than $a$.

$\lvert x - b \rvert > a$ implies the distance from $x$ to $b$ on a number line
is greater than $a$.

Since $\lvert x \rvert$ is always positive, $\lvert x \rvert > -a$ implies
$x \in \mathbb{R}$.

Also, $\lvert x \rvert < -a$ has no solutions for $x$.

\subsection{Solving Complicated Inequalities}

Given an inequality such as $2 - \lvert x + 1 \rvert > \lvert 3x + 2 \rvert$.

Solve for each intersection point by equating the left and right hand sides.

Given two solutions, $x_1$ and $x_2$, there are 2 possible cases:

\begin{itemize}
\item $x_1 < x < x_2$
\item $x < x_1$ and $x > x_2$
\end{itemize}

Find a test value of $x$ for each case, substitute into the original inequality,
and eliminate the incorrect case.

Given there is only one solution, $x_1$, to the equation, there are also 2
possible cases:

\begin{itemize}
\item $x < x_1$
\item $x > x_1$
\end{itemize}

Again, find a test value of $x$ for each case, substitute into the original
inequality, and eliminate the incorrect case.

\subsubsection{Given a Graph}

A graph of the two functions ($2 - \lvert x + 1 \rvert$ and
$\lvert 3x + 2 \rvert$) allows us to determine which of the two possible cases
is correct by inspection.

Solve the inequality for 1 or 2 solutions, $x_1$ and $x_2$, using the graph to
chose the correct branches to equate.

For each case, observe which function is on top of the other, and select the
correct range of $x$ values.


\section{Reciprocal Function}

The reciprocal graph of $f(x)$ is:

$$
\frac{1}{f(x)}
$$

\subsection{Graphing}

Given a graph of $f(x)$ and asked to find the reciprocal:

\begin{itemize}
\item Roots of $f(x)$ will be vertical asymptotes on the reciprocal.
\item The $x$ coordinate of stationary points on $f(x)$ will also be the $x$
	coordinate of a stationary point on the reciprocal.
\item Positive and negative points on $f(x)$ will also be positive or negative
	on the reciprocal.
\item $f(x)$ and its reciprocal will intersect wherever $f(x) = 1$.
\end{itemize}

\subsubsection{Behaviour Around Asymptotes}

Given $f(x)$ has a root at $x_1$, the reciprocal will have a vertical asymptote
with the equation $x = x_1$.
We must determine the behaviour of the reciprocal around this asymptote.

When approaching the asymptote from the left:

\begin{itemize}
\item If $f(x)$ is positive, the asymptote will approach positive infinity.
\item If $f(x)$ is negative, the asymptote will approach negative infinity.
\end{itemize}

Use the same method for approaching the asymptote from the right.


\section{Continuity}

\subsection{Limit}

The limit of a function at $x = a$ is defined when:

$$
\lim_{x \to a^+} f(x) = \lim_{x \to a^-} f(x)
$$

The value obtained when approaching $a$ from the left is equal to that obtained
when approaching $a$ from the right.

\subsection{Continuous}

A function $f(x)$ is continuous at a point $x = a$ if:

\begin{enumerate}
\item $f(a)$ is defined
\item $\lim_{x \to a} f(x)$ is defined
\item $\lim_{x \to a} f(x) = f(a)$
\end{enumerate}

A function is continuous if it satisifes the above conditions for all $x$ in
the function's domain.

Informally, a function is continuous if it can be drawn without lifting the pen
from the page.

\subsection{Differentiable}

A function $f(x)$ is differentiable at point $x = a$ if it is continuous at $a$,
and:

$$
\lim_{h \to 0^+} \frac{f(a + h) - f(a)}{h} = \lim_{h \to 0^-} \frac{f(a + h) - f(a)}{h}
$$

A function is differentiable if it is continuous and satisfies the above
equation for all $x$ in its domain.




\chapter{Vectors}

\section{Definition}

A vector is a value with both magnitude and direction.

A unit vector has a magnitude of 1.

$\bb{i}$, $\bb{j}$, $\bb{k}$ are unit vectors in the direction of the x, y, and z axes respectively.

\subsection{Magnitude}

$|\bb{a}|$ is the magnitude of vector $\bb{a}$, which is defined as:

$$
\begin{aligned}
\lvert \begin{pmatrix} x \\ y \end{pmatrix} \rvert & = \sqrt{x^2 + y^2} \\
\lvert \begin{pmatrix} x \\ y \\ z \end{pmatrix} \rvert & = \sqrt{x^2 + y^2 + z^2} \\
\end{aligned}
$$

\subsection{Equality}

Two vectors are equal if and only if the magnitude and direction of both vectors
are equal.

For vectors in component form, all components must be equal for the vectors to
be equal.

Two or three equations can be formed form equating two vectors:

$$
\begin{pmatrix} 3a \\ 4 \end{pmatrix} = \begin{pmatrix} 6 \\ 2b \end{pmatrix}
$$

This implies $3a = 6$ and $4 = 2b$.

\subsection{Parallel}

Two vectors are parallel if one is a scalar multiple of the other:

$$
\bb{a} = k \bb{b}
$$

If $k$ is positive, the vectors are like parallel vectors. They point in the
same direction.

If $k$ is negative, the vectors are unlike parallel vectors. They point in
opposite directions.

If $h \bb{a} = k \bb{b}$ where $\bb{a}$ and $\bb{b}$ are not parallel, then
$h = k = 0$.

\subsection{Position Vectors}

The position of a point P in space given as a direction and distance from the
origin O:

$$
\bb{p} = \vv{OP}
$$

\subsection{Relative Vectors}

Given the position vectors of two points A and B.

Point B relative to A is given as $\vv{AB} = _B\bb{r}_A = \vv{OB} - \vv{OA}$.

Point A relative to B is given as $\vv{BA} = _A\bb{r}_B = \vv{OA} - \vv{OB}$.

\subsection{Dot Product}

The dot product is defined as:

$$
\bb{a} \cdot \bb{b} = \lvert \bb{a} \rvert \lvert \bb{b} \rvert \cos{\theta}
$$

Where $\theta$ is the angle between the two vectors, when they are both pointing
away from a common point.

It can also be calculated as:

$$
\begin{pmatrix} x_1 \\ y_1 \end{pmatrix} \cdot \begin{pmatrix} x_2 \\ y_2 \end{pmatrix} = x_1 x_2 + y_1 y_2
$$

If $\bb{a}$ and $\bb{b}$ are perpendicular, then:

$$
\bb{a} \cdot \bb{b}\cdot = 0
$$

\subsubsection{Properties}

The dot product has the following properties:

$$
\begin{aligned}
\bb{a} \cdot \bb{b} & = \bb{b} \cdot \bb{a} \\
\bb{a} \cdot (k \bb{b}) & = (k \bb{a}) \cdot \bb{b} = k (\bb{a} \cdot \bb{b}) \\
\bb{a} \cdot \bb{a} & = \lvert \bb{a} \rvert^2 \\
\bb{a} \cdot (\bb{b} + \bb{c}) & = \bb{a} \cdot \bb{b} + \bb{a} \cdot \bb{c} \\
\end{aligned}
$$

\subsection{Cross Product}

The cross product is defined as:

$$
\bb{a} \times \bb{b} = \lvert \bb{a} \rvert \lvert \bb{b} \rvert \sin{\theta} \hat{\bb{n}}
$$

Where $\hat{\bb{n}}$ is a unit vector perpendicular to both $\bb{a}$ and
$\bb{b}$, called the normal.

The cross product of two parallel vectors (like or unlike parallel) is 0 (as
the angle between them is 0).

The cross product of a vector with itself is 0:

$$
\bb{a} \times \bb{a} = 0
$$

\subsubsection{Area of Parallelogram}

The magnitude of the cross product of two vectors $\bb{a}$ and $\bb{b}$:

$$
\lvert \bb{a} \times \bb{b} \rvert
$$

is equivalent to the area of the parallelogram bounded by $\bb{a}$ and $\bb{b}$.

Thus the area of the triangle bounded by $\bb{a}$, $\bb{b}$, and
$\bb{b} - \bb{a}$ is:

$$
\frac{1}{2} \lvert \bb{a} \times \bb{b} \rvert
$$

Proved using the definition of the cross product, relating it to to the area of
a triangle calculated by $\frac{1}{2} a b \sin{\theta}$.

\subsubsection{Angle Between Vectors}

Given two vectors $\bb{a}$ and $\bb{b}$ in component form.

The angle between them using the dot product is:

$$
\cos{\theta} = \frac{\bb{a} \cdot \bb{b}}{\lvert \bb{a} \rvert \lvert \bb{b} \rvert}
$$

The angle between them using the cross product is:

$$
\sin{\theta} = \frac{\lvert \bb{a} \times \bb{b} \rvert}{\lvert \bb{a} \rvert \lvert \bb{b} \rvert}
$$

\subsection{Scalar Projection}

The scalar projection of a vector onto another is the distance it extends along
it.

The scalar projection of $\bb{a}$ onto $\bb{b}$ is:

$$
\lvert \bb{a} \rvert \cos{\theta} = \bb{a} \cdot \hat{\bb{b}}
$$

Where $\theta$ is the angle between the two vectors.

\subsection{Vector Projection}

The vector projection is a vector parallel to one vector with magnitude equal to
the scalar projection of another vector onto that vector.

The vector projection of $\bb{a}$ onto $\bb{b}$ is:

$$
\lvert \bb{a} \rvert \cos{\theta} \hat{\bb{b}} = (\bb{a} \cdot \hat{\bb{b}}) \hat{\bb{b}}
$$

\subsection{Ratio Theorem}

Given points A and B. Point P divides the line $\vv{AB}$ in the ratio
$\vv{AP} : \vv{PB} = a : b$.

$\vv{OP}$ is given by:

$$
\vv{OP} = \frac{b\vv{OA} + a\vv{OB}}{a + b}
$$

If P lies outside the line segment $\vv{AB}$, then use point B as the central
point and rearrange the above equation for $\vv{OP}$.


\section{Lines}

\subsection{Vector Equation}

Vector equation for a line in direction $\bb{d}$ that passes through a point
with position vector $\bb{a}$:

$$
\bb{r} = \bb{a} + \lambda \bb{d} \qquad \lambda \in \mathbb{R}
$$

\subsection{Dot Product Equation}

2D only.

Dot product form of a line perpendicular to $\bb{n}$ (the normal) that passes
through a point with position vector $\bb{a}$ is:

$$
\bb{r} \cdot \bb{n} = \bb{a} \cdot \bb{n}
$$

The right hand side evaluates to a constant, leaving:

$$
\bb{r} \cdot \bb{n} = c \qquad c \in \mathbb{R}
$$

\subsection{Parametric Equation}

Given the vector equation of a line $\bb{r} = \bb{a} + \lambda \bb{b}$.

The parametric equation is:

$$
\begin{cases}
x & = \bb{a}_x + \lambda \bb{b}_x \\
y & = \bb{a}_y + \lambda \bb{b}_y \\
\end{cases}
$$

\subsection{Cartesian Equation}

The cartesian equation of a line in 2D is:

$$
ax + by = c
$$

The cartesian equation of a line in 3D is:

$$
ax + b = cy + d = ez + f
$$

\subsection{Conversion}

\subsubsection{Vector to Dot Product in 2D}

Given the vector equation of a line:

$$
\bb{r} = \bb{a} + \lambda \bb{d} \qquad \lambda \in \mathbb{R}
$$

Find a vector perpendicular to the direction for the normal.

$\bb{a}$ is a point on the line.

Substitute these values into the dot product equation of a line.

\subsubsection{Dot Product to Vector in 2D}

Given the dot product equation of a line:

$$
\bb{r} \cdot \bb{n} = c
$$

Find a vector perpendicular to the normal for the direction of the line.

Find a point $\bb{a}$ that satisfies the dot product equation as a point on the
line.

Substitute these values into the vector equation of a line.

\subsubsection{Vector to Parametric}

Let $\bb{r} = \begin{pmatrix} x \\ y \end{pmatrix}$:

$$
\begin{pmatrix} x \\ y \end{pmatrix} = \bb{a} + \lambda \bb{d} \qquad \lambda \in \mathbb{R}
$$

Gives the parametric equation:

$$
\begin{cases}
\begin{aligned}
x & = \bb{a}_x + \lambda \bb{d}_x \\
y & = \bb{a}_y + \lambda \bb{d}_y \\
\end{aligned}
\end{cases}
$$

\subsubsection{Parametric to Vector}

Perform by inspection from the coefficient of $\lambda$ and the constant
term.

\subsubsection{Vector to Cartesian}

Find the two parametric equations:

$$
\begin{aligned}
x & = \bb{a}_x + \lambda \bb{d}_x \\
y & = \bb{a}_y + \lambda \bb{d}_y \\
\end{aligned}
$$

Rearrange for $\lambda$ (assuming both components of $\bb{d}$ are non-zero):

$$
\begin{aligned}
\lambda & = \frac{x - \bb{a}_x}{\bb{d}_x} \\
\lambda & = \frac{y - \bb{a}_y}{\bb{d}_y} \\
\end{aligned}
$$

Equate the two equations:

$$
\frac{x - \bb{a}_x}{\bb{d}_x} = \frac{y - \bb{a}_y}{\bb{d}_y}
$$

If one of the components of $\bb{d}$ is 0, the line will be perpendicular to
one of the axes.

For example, if $\bb{d}_x = 0$, the parametric equations are:

$$
\begin{aligned}
x & = \bb{a}_x \\
y & = \bb{a}_y + \lambda \bb{d}_y \\
\end{aligned}
$$

Gives the cartesian equation:

$$
x = \bb{a}_x
$$

Which is a line parallel to the y axis.

\subsubsection{Dot Product to Cartesian in 2D}

Let $\bb{r} = \begin{pmatrix} x \\ y \end{pmatrix}$:

$$
\begin{pmatrix} x \\ y \end{pmatrix} \cdot \bb{n} = c
$$

Gives the cartesian equation:

$$
\bb{n}_x x + \bb{n}_y y = c
$$

\subsubsection{Cartesian to Dot Product in 2D}

Use the coefficients of $x$ and $y$ to determine the components of the normal.

The constant on the right hand side remains the same.

\subsection{Construction}

\subsection{From Two Points}

Given the points $\bb{a}$ and $\bb{b}$, the direction of the line is:

$$
\bb{d} = \bb{a} - \bb{b}
$$

The vector equation is:

$$
\bb{r} = \bb{a} + \lambda (\bb{a} - \bb{b})
$$

\subsection{Parallel Lines}

Two lines are parallel if:

\begin{itemize}
\item The two direction vectors are parallel.
\item The direction vector and the normal are perpendicular.
\item The two normals are parallel.
\end{itemize}

\subsection{Perpendicular Lines}

Two lines are perpendicular if:

\begin{itemize}
\item The two direction vectors are perpendicular.
\item The direction vector and normal are parallel.
\item The two normals are perpendicular.
\end{itemize}

\subsection{Collisions}

Two objects travelling with constant velocity must be in the same location at
the same time to collide:

$$
\bb{r}_1 + t_1 \bb{v}_1 = \bb{r}_2 + t_2 \bb{v}_2
$$

The objects collide if $t_1 = t_2$ and $t_1 \geq 0$.


\section{Circles and Spheres}

The vector equation for a circle or sphere with centre at position vector
$\bb{d}$ and radius $r$ is:

$$
\lvert \bb{r} - \bb{d} \rvert = r
$$

\subsection{Cartesian Equation}

Expanding the vector equation:

$$
\begin{aligned}
\lvert \begin{pmatrix} x \\ y \end{pmatrix} - \begin{pmatrix} a \\ b \end{pmatrix} \rvert & = r \\
(x - a)^2 + (y - b)^2 & = r^2 \\
\end{aligned}
$$

\subsection{Parametric Equation}

The parametric equation for a sphere is unnecessary.

A circle with centre $(a, b)$ and radius $r$ can be written in parametric form
as:

$$
\begin{cases}
x = r \cos{\theta} + a \\
y = r \sin{\theta} + b \\
\end{cases}
$$

Using the Pythagorean trigonometric identity to eliminate $\theta$ to arrive at
the cartesian equation:

$$
\begin{cases}
(x - a)^2 = r^2 \cos^2{\theta} \\
(y - b)^2 = r^2 \sin^2{\theta} \\
\end{cases}
$$

Add both equations together:

$$
\begin{aligned}
(x - a)^2 + (y - b)^2 & = r^2 \cos^2{\theta} + r^2 \sin^2{\theta} \\
(x - a)^2 + (y - b)^2 & = r^2 \\
\end{aligned}
$$


\section{Ellipses}

2D only.

\subsection{Parametric Equation}

An ellipse with centre $(a, b)$, horizontal radius $r_h$ and vertical radius
$r_v$ has the parametric equation:

$$
\begin{cases}
\begin{aligned}
x & = r_h \cos{\lambda} + a \\
y & = r_v \sin{\lambda} + b \\
\end{aligned}
\end{cases}
$$

\subsection{Cartesian Equation}

An ellipse with centre $(a, b)$, horizontal radius $r_h$ and vertical radius
$r_v$ has the Cartesian equation:

$$
(\frac{x - a}{r_h})^2 + (\frac{y - b}{r_v})^2 = 1
$$

\subsection{Conversions}

\subsubsection{Parametric to Cartesian}

Given the parametric equation for an ellipse:

$$
\begin{cases}
\begin{aligned}
x & = r_h \cos{\lambda} + a \\
y & = r_v \sin{\lambda} + b \\
\end{aligned}
\end{cases}
$$

Move $a$ and $b$ to the left hand side of the equations and divide by the
radii:

$$
\begin{cases}
\begin{aligned}
\frac{x - a}{r_h} & = \cos{\lambda} \\
\frac{y - b}{r_v} & = \sin{\lambda} \\
\end{aligned}
\end{cases}
$$

Square each equation and add them together:

$$
(\frac{x - a}{r_h})^2 + (\frac{y - b}{r_v})^2 = \cos^2{\lambda} + \sin^2{\lambda}
$$

Use the Pythagorean identity to eliminate $\lambda$:

$$
(\frac{x - a}{r_h})^2 + (\frac{y - b}{r_v})^2 = 1
$$


\section{Planes}

3D only.

\subsection{Vector Equation}

The vector equation for a plane that passes through a point with position
vector $\bb{a}$ and extends in directions $\bb{b}$ and $\bb{c}$.

$$
\bb{r} = \bb{a} + \lambda \bb{b} + \mu \bb{c}
$$

$\bb{b}$ and $\bb{c}$ are two vectors parallel to the plane.

\subsection{Dot Product Form}

The equation for a plane in dot product form with normal $\bb{n}$ that passes
through a point with position vector $\bb{a}$:

$$
\bb{r} \cdot \bb{n} = \bb{n} \cdot \bb{a}
$$

Since the right hand side evaluates to a constant, this is equivalent to:

$$
\bb{r} \cdot \bb{n} = c \qquad c \in \mathbb{R}
$$

\subsection{Parametric Form}

The parametric form of a plane from its vector equation is:

$$
\begin{cases}
\begin{aligned}
x & = \bb{a}_x + \lambda \bb{b}_x + \mu \bb{c}_x \\
y & = \bb{a}_y + \lambda \bb{b}_y + \mu \bb{c}_y \\
z & = \bb{a}_z + \lambda \bb{b}_z + \mu \bb{c}_z \\
\end{aligned}
\end{cases}
$$

\subsection{Cartesian Form}

The cartesian form of a plane is:

$$
ax + by + cz = d
$$

\subsection{Conversions}

\subsubsection{Vector to Dot Product}

Given the equation of a plane in vector form:

$$
\bb{r} = \bb{a} + \lambda \bb{b} + \mu \bb{c}
$$

The normal is the cross product of $\bb{b}$ and $\bb{c}$:

$$
\bb{n} = \bb{b} \times \bb{c}
$$

$\bb{a}$ is a point on the plane.

Substitute these values into the dot product equation for a plane.

\subsubsection{Dot Product to Vector}

Find any two non-parallel vectors that are perpendicular to the normal (dot
product is 0).

Find a point that satisfies the dot product equation for the plane.

Substitute these values into the vector equation for a plane.

\subsubsection{Dot Product to Cartesian}

Expand $\bb{r}$ into its components:

$$
\begin{pmatrix} x \\ y \\ z \end{pmatrix} \cdot \bb{n} = c
$$

Simplify the dot product:

$$
\bb{n}_x x + \bb{n}_y y + \bb{n}_z z = c
$$

\subsubsection{Cartesian to Dot Product}

Use the coefficients of $x$, $y$, and $z$ to find the components of the normal.

\subsubsection{Vector to Cartesian}

Convert the vector equation into dot product form, then the dot product form
to cartesian.

\subsubsection{Cartesian to Vector}

Convert the cartesian into dot product form, then the dot product form to
vector.

\subsection{Construction}

\subsubsection{From 2 Points and Parallel Vector}

Given two points at position vectors $\bb{a}$ and $\bb{b}$ that lie on the
plane, and a vector $\bb{c}$ parallel to the plane.

Find a second direction from the two points:

$$
\bb{d} = \bb{a} - \bb{b}
$$

Substitute into the vector equation for a plane:

$$
\bb{r} = \bb{a} + \lambda \bb{c} + \mu (\bb{a} - \bb{b})
$$

\subsubsection{From 3 Points}

Given three points at position vectors $\bb{a}$, $\bb{b}$ and $\bb{c}$ that lie
on the plane.

Find 2 direction vectors from the 3 points and substitute into the vector
equation of a plane:

$$
\bb{r} = \bb{a} + \lambda (\bb{a} - \bb{b}) + \mu (\bb{a} - \bb{c})
$$

\subsubsection{From 2 Lines}

Use the directions of the two lines as two vectors parallel to the plane, and
any point on either of the lines as a point on the plane.


\section{Angle Between}

Usually, a question will ask for the acute angle. If the angle found is obtuse,
use $180^\circ - \theta$.

\subsection{Two Lines}

Find the angle between the two direction vectors of the lines.

\subsection{Line and Plane}

\subsubsection{Plane in Dot Product Form}

Find the angle $\theta$ between the direction of the line and the plane's normal.

The angle between the plane and line is then $\frac{\pi}{2} - \theta$.

Convert to an acute angle.

\subsubsection{Plane in Vector Form}

Convert the plane to dot product form and use the method above.

\subsection{Two Planes}

Convert both planes to dot product form and find the angle between the two
normals.


\section{Point On Surface Tests}

Given a point with position vector $\bb{p}$.

\subsection{Line}

Equate the point and the equation of the line:

$$
\bb{p} = \bb{a} + \lambda \bb{b}
$$

Form 2 or 3 equations for $\lambda$:

$$
\begin{aligned}
\bb{p}_x & = \bb{a}_x + \lambda \bb{b}_x \\
\bb{p}_y & = \bb{a}_y + \lambda \bb{b}_y \\
\end{aligned}
$$

Solve for $\lambda$ using each equation.

If the values of $\lambda$ are consistent, the point lies on the line.

\subsection{Plane}

\subsubsection{Vector Form}

Equate the point and the equation of the plane:

$$
\bb{p} = \bb{a} + \lambda \bb{b} + \mu \bb{c}
$$

Solve for $\lambda$ and $\mu$ using equations formed from the x and y
components.

Substitute these values into the equation formed from the z components. If the
values are consistent, the point lies on the plane.

\subsubsection{Dot Product Form}

Substitute the point into the equation of the plane:

$$
\bb{p} \cdot \bb{n} = c
$$

Calculate the dot product and if it is consistent with the constant $c$, the
point lies on the plane.

\subsection{Sphere}

Substitute the point into the equation for the sphere:

$$
\lvert \bb{p} - \bb{d} \rvert = r
$$

If the left hand side is less than $r$, the point lies inside the sphere.

If the left hand side equals $r$, the point lies on the sphere.

If the left hand side is greater than $r$, the point lies outside the sphere.


\section{Line On Surface Tests}

\subsection{Plane}

A line lies on a plane if its direction is parallel to the plane, and the line
and plane have a common point.

Calculate the normal to the plane, and check it is perpendicular to the line's
direction.

Check a point on the line satisfies the equation for the plane.


\section{Intersections}

\subsection{Line and Line}

Given the equations of two lines:

$$
\begin{aligned}
\bb{r}_1 & = \bb{a}_1 + \lambda \bb{d}_1 \\
\bb{r}_2 & = \bb{a}_2 + \mu \bb{d}_2 \\
\end{aligned}
$$

Equate them:

$$
\bb{a}_1 + \lambda \bb{d}_1 = \bb{a}_2 + \mu \bb{d}_2
$$

Solve for $\lambda$ and $\mu$ using equations formed from the x and y
components.

Substitute the two values into the equation formed from the z component. If it
is consistent, the lines intersect.

Substitute $\lambda$ or $\mu$ back into one of the original line equations to
find the point of intersection.

\subsection{Line and Plane}

\subsubsection{Vector Equation}

Given the equation of a line and plane:

$$
\begin{aligned}
\bb{r}_1 & = \bb{a} + \lambda \bb{b} \\
\bb{r}_2 & = \bb{c} + \mu \bb{d} + \alpha \bb{e} \\
\end{aligned}
$$

Equate them:

$$
\bb{a} + \lambda \bb{b}  = \bb{c} + \mu \bb{d} + \alpha \bb{e}
$$

Solve for $\lambda$, $\mu$, and $\alpha$.

Substitute $\lambda$ back into the original line equation to find the point of
intersection.

\subsubsection{Dot Product Equation}

Given the equation of a line and plane:

$$
\begin{aligned}
\bb{r}_1 & = \bb{a} + \lambda \bb{b} \\
\bb{r}_2 \cdot \bb{n} & = c \\
\end{aligned}
$$

Substitute the line equation into the plane equation:

$$
(\bb{a} + \lambda \bb{b}) \cdot \bb{n} = c
$$

Expand the dot product and solve for $\lambda$.

Substitute $\lambda$ back into the original line equation to find the point of
intersection.

\subsection{Line and Circle}

Given the equation of a line and circle:

$$
\begin{aligned}
\bb{r}_1 & = \bb{a} + \lambda \bb{b} \\
\lvert \bb{r}_2 - \bb{c} \rvert & = r \\
\end{aligned}
$$

Substitute the line equation into the circle equation:

$$
\lvert \bb{a} + \lambda \bb{b} - \bb{c} \rvert = r
$$

This will simplify to a quadratic in terms of $\lambda$.

If there are two solutions, the line intersects the circle and is not
tangential to it.

If there is one solution, the line is tangent to the circle.

If there are no solutions, the line does not intersect the circle.

\subsection{Plane and Plane}

Two planes intersect along a line.

The line has a direction perpendicular to the two normals of the planes.

A point on the line is a point common to both planes (satisfies both plane
equations).

\subsection{Circle and Axis Plane}

For the xy-plane, $z = 0$.

For the xz-plane, $y = 0$.

For the yz-plane, $x = 0$.

Find the cartesian equation of a sphere:

$$
(x - a)^2 + (y - b)^2 + (z - c)^2 = r^2
$$

Substitute either $x = 0$, $y = 0$, or $z = 0$ depending on the axis plane.

The equation is the cartesian equation for the intersecting circle on this
plane.


\section{Minimum Distance}

The minimum distance between two vector surfaces.

\subsection{Point and Line}

Given a point at position vector $\bb{p}$ and line:

$$
\bb{r} = \bb{a} + \lambda \bb{d}
$$

\subsubsection{Dot Product}

The closest distance is when the vector between an arbitrary point on the line
and $\bb{p}$ and the direction of the line are perpendicular.

The vector between $\bb{p}$ and an aribtrary point on the line is:

$$
\bb{p} - \bb{r} = \bb{p} - \bb{a} - \lambda \bb{d}
$$

This must be perpendicular to the direction of the line:

$$
(\bb{p} - \bb{a} - \lambda \bb{d}) \cdot \bb{d} = 0
$$

Solve for $\lambda$.

Substitute $\lambda$ back into the line equation to get the closest point on
the line (point $\bb{c}$)

Find the magnitude $\lvert \bb{p} - \bb{c} \rvert$ for distance between the
point and the closest point on the line ($\bb{c}$).

\subsubsection{Cross Product}

Chose some aribtrary point on the line Q.

Let the point closest to P on the line be C.

Form a right angled triangle QPC.

Let $\theta$ be the angle at Q (ie. the angle between $\vv{QP}$ and
$\vv{QC}$).

The perpendicular distance is thus $\lvert \vv{QP} \rvert \sin{\theta}$.

This can be written as:

$$
\begin{aligned}
& = \frac{\lvert \vv{QP} \rvert \lvert \vv{QC} \rvert \sin{\theta}}{\lvert \vv{QC} \rvert} \\
& = \frac{\lvert \vv{QP} \times \vv{QC} \rvert}{\lvert \vv{QC} \rvert} \\
\end{aligned}
$$

Which is equivalent to:

$$
\frac{\lvert \bb{p} - \bb{a} \times \bb{d} \rvert}{\lvert \bb{d} \rvert}
$$

\subsection{Point and Plane}

Given the point $\bb{p}$.

\subsubsection{Vector Equation of Plane}

Given the vector equation of a plane:

$$
\bb{r} = \bb{a} + \lambda \bb{b} + \mu \bb{c}
$$

The vector between an arbitrary point on the plane and $\bb{p}$ will be
perpendicular to the directions of the plane:

$$
\begin{aligned}
(\bb{p} - \bb{a} - \lambda \bb{b} - \mu \bb{c}) \cdot \bb{b} & = 0 \\
(\bb{p} - \bb{a} - \lambda \bb{b} - \mu \bb{c}) \cdot \bb{c} & = 0 \\
\end{aligned}
$$

Solve simultaneously for $\lambda$ and $\mu$.

Substitute back into the equation for the plane to find the closest point and
distance.

Alternatively, determine the normal and use the below method.

\subsubsection{Dot Product Equation of Plane}

Given the dot product equation of a plane:

$$
\bb{r} \cdot \bb{n} = c
$$

Find a point on the plane $\bb{a}$.

Find the vector between $\bb{a}$ and $\bb{p}$, $\vv{AP}$.

Find the scalar projection of this onto the normal $\bb{n}$.

\subsection{Point and Sphere}

Equivalent to the distance between the point and the centre of the sphere,
subtract the radius.

\subsection{Line and Line}

\subsubsection{Not Parallel}

Find a vector $\bb{n}$ perpendicular to the directions of both lines.

Find the vector between two points on the lines.

Find the scalar projection of this onto $\bb{n}$.

\subsubsection{Parallel}

Pick a point on one of the lines.

Find the closest distance between this point and the other line using the
method above.

\subsection{Line and Parallel Plane}

Pick a point on the line.

Find the closest distance between this point and the plane using the method
above.

\subsection{Line and Sphere}

Find the closest distance between the line and the centre of the sphere, then
subtract the radius.

\subsection{Plane and Parallel Plane}

Pick a point on one of the planes.

Find the closest distance between this point and the other plane using the
method above.

\subsection{Plane and Sphere}

Find closest distance between plane and centre of sphere, then subtract the
radius.

\subsection{Sphere and Sphere}

Find the distance between the two centres, then subtract both radii.


\section{Closest Approach}

The closest distance between two objects travelling in defined paths. Cannot
use above methods, as the position of the objects depends on the time.

For two objects travelling in straight lines, this can be solved using relative
velocities and the methods above.

For objects not travelling in straight lines, requires calculus to solve.
Find an expression for the distance between the two objects at any time, then
minimise this function.

\subsection{Lines}

Given two objects travelling in the lines:

$$
\begin{aligned}
\bb{r}_1 & = \bb{a}_1 + t \bb{v}_1 \\
\bb{r}_2 & = \bb{a}_2 + t \bb{v}_2 \\
\end{aligned}
$$

Find the position of object 2 relative to object 1, $_2\bb{r}_1$.

Find the velocity of object 2 relative to object 1, $_2\bb{v}_1$.

Form an equation of motion for object 2 relative to object 1:

$$
\bb{r} = _2\bb{r}_1 + _2\bb{v}_1 t
$$

Minimise the distance between this line and the origin (the location of object
1) to find the closest time and distance they are apart.




\chapter{Systems of Equations}

\section{Definition}

Multiple variables to be solved simultaneously:

$$
\begin{cases}
\begin{aligned}
3x + 4y - z & = 4 \\
2x + 9y + 5z & = 10 \\
-x - 5y + z & = 12 \\
\end{aligned}
\end{cases}
$$

We require as many unique equations as variables in order to fully solve the
system.

\subsection{Matrix Form}

Each column represents a variable.

The final column represents the constant on the right hand side of each of the
three equations.

Each row represents the coefficients of the variables in one equation.

For example, the system of equations:

$$
\begin{cases}
\begin{aligned}
 3x + 4y - z & = 4  \\
2x + 9y + 5z & = 10 \\
 -x - 5y + z & = 12 \\
\end{aligned}
\end{cases}
$$

Can be written in matrix form as:

$$
\begin{bmatrix}
 3 &  4 & -1 & 4  \\
 2 &  9 &  5 & 10 \\
-1 & -5 &  1 & 12 \\
\end{bmatrix}
$$

The leading diagonal is the diagonal line of coefficients starting at the top
left corner, moving down and to the right.

\subsection{Elementary Row Operations}

Elementary row operations on a system of equations in matrix form include:

\begin{itemize}
\item Reordering rows.
\item Multiplying or dividing a row by a constant.
\item Adding or subtracting two rows from each other.
\end{itemize}

To perform a series of elementary row operations, we assign a row to the desired
combination of these operations.

\subsection{Row Echelon Form}

Where all coefficients below the leading diagonal of a system of equations in
matrix form are 0.

For example:

$$
\begin{bmatrix}
8 & 1 & -3 & -2 \\
0 & 2 &  2 &  4 \\
0 & 0 & -1 &  5 \\
\end{bmatrix}
$$

Matrices can be manipulated using elementary row operations into row echelon
form.

\subsection{Solving}

To solve a system of equations, write them in matrix form, manipulate the matrix
into row echelon form, and solve for each variable.

In the example above:

$$
\begin{aligned}
z & = -5 \\
2y + 2 \times -5 & = 4 \\
y & = 7 \\
8x + 1 \times 7 - 3 \times -5 & = -2 \\
x & = -3 \\
\end{aligned}
$$

Using this process to solve simultaneous equations is called Gaussian
Elimination.


\section{No Solutions}

Once a system of equations in matrix form is reduced to row echelon form, the
system has no solutions if all variables in one row have coefficients of 0,
with a non-zero constant in the final column.

For example:

$$
\begin{bmatrix}
a & b & c & d \\
0 & e & f & g \\
0 & 0 & 0 & h \\
\end{bmatrix}
$$

The final row implies $0x + 0y + 0z = h$, where $h \neq 0$. This equation has
no solutions.

Thus the system itself has no solutions.


\section{Infinite Solutions}

Once a system of equations in matrix form is reduced to row echelon form, the
system has no solutions if all variables in one row have coefficients of 0,
with an additional 0 in the final column.

For example:

$$
\begin{bmatrix}
a & b & c & d \\
0 & e & f & g \\
0 & 0 & 0 & 0 \\
\end{bmatrix}
$$

The final row implies $0x + 0y + 0z = 0$.

There is not enough information to uniquely solve the system, thus the system
has infinitely many solutions.

If any row in the system is a multiple of another, the system has infinite
solutions.


\section{Conditions for Infinite or No Solutions}

Given the system of equations, where $p, q \in \mathbb{R}$:

$$
\begin{cases}
\begin{aligned}
x - y + 2z & = 1 \\
2x - 5y + 5z & = 9 \\
3x + 3y + pz & = q \\
\end{aligned}
\end{cases}
$$

The row echelon form matrix is:

$$
\begin{bmatrix}
1 & -1 & 2 & 1 \\
0 & -3 & 1 & 7 \\
0 & 0 & p - 4 & q + 11 \\
\end{bmatrix}
$$

The system has a unique solution when the coefficient of $z$ in the final row
is non-zero:

$$
\begin{aligned}
p - 4 \neq 0 \\
p \neq 4 \\
\end{aligned}
$$

This is written as: $p, q \in \mathbb{R}, p \neq 4$

The system has no solutions when $p = 4$ and $q \in \mathbb{R}, q \neq -11$.

The system has infinite solutions when $p = 4$ and $q = -11$.


\section{Graphical Representation}

Each equation in the system can be represented by a 3D plane:

\begin{description}
\item [Unique Solution] When the 3 planes intersect at a single point.
\item [Infinite Solutions] When the 3 planes intersect in a single line.
\item [No Solutions] When the 3 planes intersect in 3 separate lines.
\end{description}




\chapter{Probability}

\textbf{Population parameters} or \textbf{population statistics} refer to
pieces of information derived in consideration of the entire population.

\textbf{Sample statistics} refer to pieces of information derived in
consideration of only a sample from the population.

The \textbf{standard error} is equivalent to the standard deviation.

Sample statistics approximate population statistics if the sample is unbiased.

Sample statistics usually are notated with a ``hat":

$$
\hat{\mu} \quad \hat{\sigma} \quad \hat{p}
$$


\section{Distribution of Sample Means}

Given a random variable $X$ with an unknown distribution. $X$ has a mean $\mu$
and standard deviation $\sigma$ (population statistics).

A \textbf{sample mean} is the mean of a sample taken from $X$, written as
$\hat{\mu}$.

If we take a number of samples from $X$ and calculate their sample means, we
can find a \textbf{distribution of sample means}.

\subsection{Central Limit Theorem}

The \textbf{Central Limit Theorem} states that the distribution of sample means
approximates a normal distribution with mean:

$$
\hat{\mu} = \mu
$$

and standard deviation:

$$
\hat{\sigma} = \frac{\sigma}{\sqrt{n}}
$$

Where $n$ is the sample size.

The mean of all sample means ($\hat{\mu}$) approaches the population mean
($\mu$) as the sample size increases.

The standard deviation of the distribution of sample means decreases as the
sample size increases.

For a population distributed according to a non-uniform or unknown distribution,
a sample size of $n \geq 30$ is required for the distribution of sample means to
adequately approximate a normal distribution.

For a population that is normally distributed, there is no restriction on the
sample size required for the distribution of sample means to approximate a
normal distribution.

\subsection{Unknown Population Statistics}

If the population mean $\mu$ and standard deviation $\sigma$ are unknown, then
we can approximate these using a point estimate from a single sample.

Given a single sample from the population with a mean $\hat{\mu}$ and standard
deviation $\hat{\sigma}$, the distribution of sample means can be approximated
by a normal distribution with mean $\hat{\mu}$ and standard deviation:

$$
\frac{\hat{\sigma}}{\sqrt{n}}
$$

\subsection{Sample Mean Probabilities}

Knowing that sample means are normally distributed allows us to:

\begin{itemize}
\item Estimate the probability of a sample having a mean that lies within a
	given interval.
\item Determine if a sample mean is unexpectedly high or low.
\end{itemize}

\subsection{Significant Difference}

A \textbf{significant difference} at the $a$\% level indicates that the sample
mean lies outside of $a$\% of all sample means.

It is either in the top or bottom $\frac{a}{2}$\% of all sample means.

\subsection{Sample Mean Confidence Intervals}

Similar to sample proportions, we can apply the Central Limit Theorem to infer
confidence intervals for the population mean from sample statistics.

Given a sample of size $n$ ($n \geq 30$) for a population of an unknown
distribution. The sample has a mean $\hat{\mu}$ and standard deviation
$\hat{\sigma}$.

The distributions of sample means will be a normal distribution with mean
$\hat{\mu}$ and standard deviation:

$$
\frac{\hat{\sigma}}{\sqrt{n}}
$$

We can be $a$\% confident that the population mean will lie within the interval:

$$
\hat{\mu} - k \frac{\hat{\sigma}}{\sqrt{n}} \leq \mu \leq \hat{\mu} + k \frac{\hat{\sigma}}{\sqrt{n}}
$$

Where $k$ is the corresponding number of standard deviations for an $a$\%
confidence interval:

\begin{itemize}
\item 90\% confidence interval has $k = 1.645$
\item 95\% confidence interval has $k = 1.96$
\item 99\% confidence interval has $k = 2.576$
\end{itemize}

Similar to sample proportions, in a set of $m$ samples, we would expect $a$\%
of samples to have a mean that lies within an $a$\% confidence interval.

\end{document}
