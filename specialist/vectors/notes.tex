
\documentclass[a4paper,11pt]{article}

% Math symbols
\usepackage{amsmath}
\usepackage{amsfonts}
\usepackage{esvect}

% No indent on new paragraphs
\setlength{\parindent}{0mm}
\setlength{\parskip}{0.2cm}

% Alias \boldsymbol to \bb for vectors
\newcommand{\bb}{\boldsymbol}


\begin{document}

\title{Vectors}
\author{Ben Anderson}
\date{\today}
\maketitle
\pagebreak

\tableofcontents
\pagebreak




\section{Vectors}

A vector is a value with both magnitude and direction.

A unit vector has a magnitude of 1.

$\bb{i}$, $\bb{j}$, $\bb{k}$ are unit vectors in the direction of the x, y, and z axes respectively.


\subsection{Magnitude}

$|\bb{a}|$ is the magnitude of vector $\bb{a}$, which is defined as:

$$
\begin{aligned}
\lvert \begin{pmatrix} x\\ y \end{pmatrix} \rvert & = \sqrt{x^2 + y^2} \\
\lvert \begin{pmatrix} x\\ y\\ z \end{pmatrix} \rvert & = \sqrt{x^2 + y^2 + z^2} \\
\end{aligned}
$$


\subsection{Equality}

Two vectors are equal if and only if the magnitude and direction of both vectors
are equal.


\subsection{Parallel}

Two vectors are parallel if one is a scalar multiple of the other:

$$
\bb{a} = k \bb{b}
$$

If $k$ is positive, the vectors are said to be like parallel vectors, as they
point in the same direction.

If $k$ is negative, the vectors are said to be unlike parallel vectors, as they
point in opposite directions.

If $h \bb{a} = k \bb{b}$ given that $\bb{a}$ and $\bb{b}$ are not parallel, then
$h = k = 0$.


\subsection{Position Vectors}

The position of a point P in space given as a direction and distance from the
origin O.

$\bb{p} = \vv{OP}$


\subsection{Relative Vectors}

Given the position vectors of two points A and B.

Point B relative to A is given as $\vv{AB} = _B\bb{r}_A = \vv{OB} - \vv{OA}$.

Point A relative to B is given as $\vv{BA} = _A\bb{r}_B = \vv{OA} - \vv{OB}$.


\subsection{Dot Product}

The dot product is defined as:

$$
\bb{a} \cdot \bb{b} = \lvert \bb{a} \rvert \lvert \bb{b} \rvert \cos{\theta}
$$

Where $\theta$ is the angle between the two vectors, when they are both pointing
away from a common point.

It can also be calculated as:

$$
\begin{pmatrix} x_1\\ y_1 \end{pmatrix} \cdot \begin{pmatrix} x_2\\ y_2 \end{pmatrix} = x_1 x_2 + y_1 y_2
$$

If $\bb{a}$ and $\bb{b}$ are perpendicular, then:

$$
\bb{a} . \bb{b} = 0
$$


\subsubsection{Properties}

The dot product has the following properties:

$$
\begin{aligned}
	\bb{a} \cdot \bb{b} & = \bb{b} \cdot \bb{a} \\
	\bb{a} \cdot (k \bb{b}) & = (k \bb{a}) \cdot \bb{b} = k (\bb{a} \cdot \bb{b}) \\
	\bb{a} \cdot \bb{a} & = \lvert \bb{a} \rvert^2 \\
	\bb{a} \cdot (\bb{b} + \bb{c}) & = \bb{a} \cdot \bb{b} + \bb{a} \cdot \bb{c} \\
\end{aligned}
$$


\subsection{Cross Product}

The cross product is defined as:

$$
\bb{a} \times \bb{b} = \lvert \bb{a} \rvert \lvert \bb{b} \rvert \sin{\theta} \hat{\bb{n}}
$$

Where $\hat{\bb{n}}$ is a unit vector perpendicular to both $\bb{a}$ and
$\bb{b}$, called the normal.

The cross product of two parallel vectors (like or unlike parallel) is 0 (as
the angle between them is 0).


\subsubsection{Area of Parallelogram}

The magnitude of the cross product of two vectors $\bb{a}$ and $\bb{b}$:

$$
\lvert \bb{a} \times \bb{b} \rvert
$$

is equivalent to the area of the parallelogram bounded by $\bb{a}$ and $\bb{b}$.

Thus the area of the triangle bounded by $\bb{a}$, $\bb{b}$, and
$\bb{b} - \bb{a}$ is:

$$
\frac{1}{2} \lvert \bb{a} \times \bb{b} \rvert
$$

Proved using the definition of the cross product, relating it to to the area of
a triangle calculated by $\frac{1}{2} a b \sin{\theta}$.


\subsubsection{Angle Between Vectors}

Given two vectors $\bb{a}$ and $\bb{b}$ in component form.

The angle between them using the dot product is:

$$
\cos{\theta} = \frac{\bb{a} \cdot \bb{b}}{\lvert \bb{a} \rvert \lvert \bb{b} \rvert}
$$

The angle between them using the cross product is:

$$
\sin{\theta} = \frac{\lvert \bb{a} \times \bb{b} \rvert}{\lvert \bb{a} \rvert \lvert \bb{b} \rvert}
$$


\subsection{Scalar Projection}

The scalar projection is the distance along one vector another vector reaches if
projected onto it.

The scalar projection of $\bb{a}$ onto $\bb{b}$ is:

$$
\lvert \bb{a} \rvert \cos{\theta} = \bb{a} \cdot \hat{\bb{b}}
$$

Where $\theta$ is the angle between the two vectors.


\subsection{Vector Projection}

The vector projection is a vector parallel to one vector with magnitude equal to
the scalar projection of another vector onto that vector.

The vector projection of $\bb{a}$ onto $\bb{b}$ is:

$$
\lvert \bb{a} \rvert \cos{\theta} \hat{\bb{b}} = \bb{a} \cdot \hat{\bb{b}} \hat{\bb{b}}
$$


\subsection{Ratio Theorem}

Given points A and B. Point P divides the line $\vv{AB}$ in the ratio
$\vv{AP} : \vv{PB} = a : b$.

$\vv{OP}$ is given by:

$$
\vv{OP} = \frac{b\vv{OA} + a\vv{OB}}{a + b}
$$

If P lies outside the line segment $\vv{AB}$, then use point B as the central
one and rearrange the above equation for $\vv{OP}$.


\section{Lines}

\subsection{Vector Equation}

The general equation for a line in direction $\bb{d}$ that passes through a
point with position vector $\bb{a}$ is:

$$
\bb{r} = \bb{a} + \lambda \bb{d}, \lambda \in \mathbb{R}
$$


\subsubsection{From 2 Points}

Given the two points A and B at position vectors $\bb{a}$ and $\bb{b}$.

The direction of the line is $\vv{AB} = \bb{b} - \bb{a}$.

The equation of the line is thus $\bb{r} = \bb{a} + \lambda (\bb{b} - \bb{a})$


\subsection{Dot Product Form}

This applies to 2D only.

The dot product form of a line perpendicular to $\bb{n}$ (the normal) and
passing through a point with position vector $\bb{a}$ is:

$$
\bb{r} \cdot \bb{n} = \bb{a} \cdot \bb{n}
$$

The right hand side evaluates to a constant, leaving:

$$
\bb{r} \cdot \bb{n} = c, c \in \mathbb{R}
$$


\subsection{Parametric Equation}

Given the standard equation of a line $\bb{r} = \bb{a} + \lambda \bb{b}$, the
parametric equation is:

$$
\begin{cases}
	x & = a_x + \lambda b_x \\
	y & = a_y + \lambda b_y \\
\end{cases}
$$


\subsection{Cartesian Equation}

Write each part of the parametric equation in terms of $\lambda$:

$$
\begin{cases}
	\lambda & = \frac{x - a_1}{b_1} \\
	\lambda & = \frac{y - a_2}{b_2} \\
\end{cases}
$$

Equate each part and simplify:

$$
\frac{x - a_1}{b_1} = \frac{y - a_2}{b_2}
$$

We arrive at a linear equation - the standard cartesian form for a line.


\subsubsection{From Dot Product Form}

Given the dot product form of a line $\bb{r} \cdot \bb{n} = c$, the cartesian
equation can be found by evaluating the dot product:

$$
\begin{aligned}
	\bb{r} \cdot \bb{n} & = c \\
	\begin{pmatrix} x\\ y \end{pmatrix} \cdot \bb{n} & = c \\
	x \bb{n}_x + y \bb{n}_y & = c
\end{aligned}
$$


\subsection{Point Lies on Line}

Given the point C at position vector $\bb{c}$, and the line
$\bb{r} = \bb{a} + \lambda \bb{b}$.

Equate $\bb{c}$ and the line:

$$
\bb{c} = \bb{a} + \lambda \bb{b}
$$

Form 2 equations for $\lambda$:

$$
\begin{aligned}
	c_x & = a_x + \lambda b_x \\
	c_y & = a_y + \lambda b_y \\
\end{aligned}
$$

Solve for $\lambda$ twice using each equation. If the two values of $\lambda$
are consistent, the point lies on the line.


\subsection{Intersection With Another Line}

Given the two lines $\bb{r}_1 = \bb{a} + \lambda \bb{b}$ and
$\bb{r}_2 = \bb{c} + \mu \bb{d}$.

Equate them and solve for $\lambda$ and $\mu$ simultaneously:

$$
\bb{a} + \lambda \bb{b} = \bb{c} + \mu \bb{d}
$$


\subsection{Collisions}

Two objects travelling with constant velocity must be in the same location at
the same time in order to collide. Thus:

$$
\bb{r}_1 + t_1 \bb{v}_1 = \bb{r}_2 + t_2 \bb{v}_2
$$

If $t_1 = t_2$, then the objects collide.

If $t_1 < 0$ or $t_2 < 0$, then the objects don't collide.



\section{Circles}

The vector equation for a 2D circle is:

$$
\lvert \bb{r} - \bb{d} \rvert = r
$$

Where $\bb{d}$ is the position vector of the circle's centre, and $r$ is its
radius.


\subsection{Cartesian Equation}

Expanding the vector equation in component form:

$$
\begin{aligned}
	\lvert \begin{pmatrix} x\\ y \end{pmatrix} - \begin{pmatrix} a\\ b \end{pmatrix} \rvert & = r \\
	(x - a)^2 + (y - b)^2 & = r^2 \\
\end{aligned}
$$

Where $(a, b)$ is the circle's centre and $r$ its radius.


\subsection{Parametric Equation}

A circle with centre $(a, b)$ and radius $r$ can be written in parametric form
as:

$$
\begin{cases}
	x = r \cos{\theta} + a \\
	y = r \sin{\theta} + b \\
\end{cases}
$$

Using the Pythagorean trigonometric identity to eliminate $\theta$ to arrive at
the cartesian equation:

$$
\begin{cases}
	(x - a)^2 = r^2 \cos^2{\theta} \\
	(y - b)^2 = r^2 \sin^2{\theta} \\
\end{cases}
$$

$$
\begin{aligned}
	(x - a)^2 + (y - b)^2 & = r^2 \cos^2{\theta} + r^2 \sin^2{\theta} \\
	(x - a)^2 + (y - b)^2 & = r^2 \\
\end{aligned}
$$


\subsection{Intersection With Line}

Given the line $\bb{r} = \bb{a} + \lambda \bb{b}$ and circle
$\lvert \bb{r} - \bb{c} \rvert = r$.

Substitute the equation of the line for $\bb{r}$ in the circle equation:

$$
\lvert \bb{a} + \lambda \bb{b} - \bb{c} \rvert = r
$$

Expand the magnitude to a quadratic and solve for $\lambda$.

Two solutions means the line passes right through the circle.

One solution implies the line touches the circle tangentially.

No solutions implies the line doesn't intersect the circle.




\section{Closest Approach}

The closest distance between a point and a line is always the perpendicular
distance.


\subsection{Calculus}

Given a point at position vector $\bb{p}$ and line
$\bb{r} = \bb{a} + \lambda \bb{b}$.

Express the distance between the point and the line as a function:

$$
f(\lambda) = \lvert \bb{p} - (\bb{a} + \lambda \bb{b}) \rvert
$$

Minimise the function using calculus, or the $\text{fMin}$ function on the
ClassPad.


\subsection{Dot Product}

Given a point at position vector $\bb{p}$ and line
$\bb{r} = \bb{a} + \lambda \bb{b}$.

The closest distance between the point and line is the perpendicular distance.

The vector between the point and the line is
$\bb{p} - (\bb{a} + \lambda \bb{b})$.

This will be perpendicular to the direction of the line:

$$
\bb{b} \cdot (\bb{p} - \bb{a} - \lambda \bb{b}) = 0
$$

Expand the dot product and solve for $\lambda$, then substitute it back into the
equation for the line to find the closest point on the line.


\subsubsection{Line in Dot Product Form}

Given a point at position vector $\bb{p}$ and line $\bb{r} \cdot \bb{n} = c$.

Form another line passing through $\bb{p}$ in the direction of the normal
$\bb{n}$, $\bb{r} = \bb{p} + \lambda \bb{n}$.

Substitute this into the original line and solve for $\lambda$:

$$
(\bb{p} + \lambda \bb{n}) \cdot \bb{n} = c
$$

Substitute $\lambda$ back into the original equation for the line to find the
closest point.


\subsection{Cross Product}




\section{Planes}

\subsection{Vector Equation}

The general equation for a plane is:

$$
\bb{r} = \bb{a} + \lambda \bb{b} + \mu \bb{c}
$$

Where $\bb{a}$ is a point on the plane, and $\bb{b}$ and $\bb{c}$ are two
vectors parallel to the plane.


\subsection{Dot Product Form}

The equation for a plane in dot product form is:

$$
\bb{r} \cdot \bb{n} = \bb{n} \cdot \bb{a}
$$

Where $\bb{n}$ is a vector perpendicular to the plane (the normal), and $\bb{a}$
is a point on the plane.

This is also written as $\bb{r} \cdot \bb{n} = c, c \in \mathbb{R}$, as the
right hand side evaluates to a constant.


\subsection{Parametric Form}


\subsection{Cartesian Form}




\section{Circles and Spheres}

\end{document}
