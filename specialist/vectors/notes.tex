
\documentclass[a4paper,11pt]{article}

% Math symbols
\usepackage{amsmath}
\usepackage{amsfonts}
\usepackage{esvect}

% Hyperlink contents page
\usepackage{hyperref}
\hypersetup{
	colorlinks,
	citecolor=black,
	filecolor=black,
	linkcolor=black,
	urlcolor=black
}

% No indent on new paragraphs
\setlength{\parindent}{0mm}
\setlength{\parskip}{0.2cm}

% Alias \boldsymbol to \bb for vectors
\newcommand{\bb}{\boldsymbol}


\begin{document}

\title{Vectors}
\author{Ben Anderson}
\date{\today}
\maketitle
\pagebreak

\tableofcontents
\pagebreak




\section{Vectors}

A vector is a value with both magnitude and direction.

A unit vector has a magnitude of 1.

$\bb{i}$, $\bb{j}$, $\bb{k}$ are unit vectors in the direction of the x, y, and z axes respectively.


\subsection{Magnitude}

$|\bb{a}|$ is the magnitude of vector $\bb{a}$, which is defined as:

$$
\begin{aligned}
\lvert \begin{pmatrix} x \\ y \end{pmatrix} \rvert & = \sqrt{x^2 + y^2} \\
\lvert \begin{pmatrix} x \\ y \\ z \end{pmatrix} \rvert & = \sqrt{x^2 + y^2 + z^2} \\
\end{aligned}
$$


\subsection{Equality}

Two vectors are equal if and only if the magnitude and direction of both vectors
are equal.

For vectors in component form, we can form multiple equations from each
component:

$$
\begin{pmatrix} 3a \\ 4 \end{pmatrix} = \begin{pmatrix} 6 \\ 2b \end{pmatrix}
$$

This implies $3a = 6$ and $4 = 2b$.


\subsection{Parallel}

Two vectors are parallel if one is a scalar multiple of the other:

$$
\bb{a} = k \bb{b}
$$

If $k$ is positive, the vectors are like parallel vectors. They point in the
same direction.

If $k$ is negative, the vectors are unlike parallel vectors. They point in
opposite directions.

If $h \bb{a} = k \bb{b}$ given that $\bb{a}$ and $\bb{b}$ are not parallel, then
$h = k = 0$.


\subsection{Position Vectors}

The position of a point P in space given as a direction and distance from the
origin O:

$$
\bb{p} = \vv{OP}
$$


\subsection{Relative Vectors}

Given the position vectors of two points A and B.

Point B relative to A is given as $\vv{AB} = _B\bb{r}_A = \vv{OB} - \vv{OA}$.

Point A relative to B is given as $\vv{BA} = _A\bb{r}_B = \vv{OA} - \vv{OB}$.


\subsection{Dot Product}

The dot product is defined as:

$$
\bb{a} \cdot \bb{b} = \lvert \bb{a} \rvert \lvert \bb{b} \rvert \cos{\theta}
$$

Where $\theta$ is the angle between the two vectors, when they are both pointing
away from a common point.

It can also be calculated as:

$$
\begin{pmatrix} x_1 \\ y_1 \end{pmatrix} \cdot \begin{pmatrix} x_2 \\ y_2 \end{pmatrix} = x_1 x_2 + y_1 y_2
$$

If $\bb{a}$ and $\bb{b}$ are perpendicular, then:

$$
\bb{a} . \bb{b} = 0
$$


\subsubsection{Properties}

The dot product has the following properties:

$$
\begin{aligned}
	\bb{a} \cdot \bb{b} & = \bb{b} \cdot \bb{a} \\
	\bb{a} \cdot (k \bb{b}) & = (k \bb{a}) \cdot \bb{b} = k (\bb{a} \cdot \bb{b}) \\
	\bb{a} \cdot \bb{a} & = \lvert \bb{a} \rvert^2 \\
	\bb{a} \cdot (\bb{b} + \bb{c}) & = \bb{a} \cdot \bb{b} + \bb{a} \cdot \bb{c} \\
\end{aligned}
$$


\subsection{Cross Product}

The cross product is defined as:

$$
\bb{a} \times \bb{b} = \lvert \bb{a} \rvert \lvert \bb{b} \rvert \sin{\theta} \hat{\bb{n}}
$$

Where $\hat{\bb{n}}$ is a unit vector perpendicular to both $\bb{a}$ and
$\bb{b}$, called the normal.

The cross product of two parallel vectors (like or unlike parallel) is 0 (as
the angle between them is 0).

The cross product of a vector with itself is 0.


\subsubsection{Area of Parallelogram}

The magnitude of the cross product of two vectors $\bb{a}$ and $\bb{b}$:

$$
\lvert \bb{a} \times \bb{b} \rvert
$$

is equivalent to the area of the parallelogram bounded by $\bb{a}$ and $\bb{b}$.

Thus the area of the triangle bounded by $\bb{a}$, $\bb{b}$, and
$\bb{b} - \bb{a}$ is:

$$
\frac{1}{2} \lvert \bb{a} \times \bb{b} \rvert
$$

Proved using the definition of the cross product, relating it to to the area of
a triangle calculated by $\frac{1}{2} a b \sin{\theta}$.


\subsubsection{Angle Between Vectors}

Given two vectors $\bb{a}$ and $\bb{b}$ in component form.

The angle between them using the dot product is:

$$
\cos{\theta} = \frac{\bb{a} \cdot \bb{b}}{\lvert \bb{a} \rvert \lvert \bb{b} \rvert}
$$

The angle between them using the cross product is:

$$
\sin{\theta} = \frac{\lvert \bb{a} \times \bb{b} \rvert}{\lvert \bb{a} \rvert \lvert \bb{b} \rvert}
$$


\subsection{Scalar Projection}

The scalar projection is the distance along one vector another vector reaches if
projected onto it.

The scalar projection of $\bb{a}$ onto $\bb{b}$ is:

$$
\lvert \bb{a} \rvert \cos{\theta} = \bb{a} \cdot \hat{\bb{b}}
$$

Where $\theta$ is the angle between the two vectors.


\subsection{Vector Projection}

The vector projection is a vector parallel to one vector with magnitude equal to
the scalar projection of another vector onto that vector.

The vector projection of $\bb{a}$ onto $\bb{b}$ is:

$$
\lvert \bb{a} \rvert \cos{\theta} \hat{\bb{b}} = \bb{a} \cdot \hat{\bb{b}} \hat{\bb{b}}
$$


\subsection{Ratio Theorem}

Given points A and B. Point P divides the line $\vv{AB}$ in the ratio
$\vv{AP} : \vv{PB} = a : b$.

$\vv{OP}$ is given by:

$$
\vv{OP} = \frac{b\vv{OA} + a\vv{OB}}{a + b}
$$

If P lies outside the line segment $\vv{AB}$, then use point B as the central
one and rearrange the above equation for $\vv{OP}$.




\section{Lines}

\subsection{Vector Equation}

Vector equation for a line in direction $\bb{d}$ that passes through a point
with position vector $\bb{a}$:

$$
\bb{r} = \bb{a} + \lambda \bb{d} \qquad \lambda \in \mathbb{R}
$$


\subsection{Dot Product Equation}

2D only.

Dot product form of a line perpendicular to $\bb{n}$ (the normal) that passes
through a point with position vector $\bb{a}$ is:

$$
\bb{r} \cdot \bb{n} = \bb{a} \cdot \bb{n}
$$

The right hand side evaluates to a constant, leaving:

$$
\bb{r} \cdot \bb{n} = c \qquad c \in \mathbb{R}
$$


\subsection{Parametric Equation}

Given the vector equation of a line $\bb{r} = \bb{a} + \lambda \bb{b}$.

The parametric equation is:

$$
\begin{cases}
x & = \bb{a}_x + \lambda \bb{b}_x \\
y & = \bb{a}_y + \lambda \bb{b}_y \\
\end{cases}
$$


\subsection{Cartesian Equation}

The cartesian equation of a line in 2D is:

$$
ax + by = c
$$

The cartesian equation of a line in 3D is:

$$
ax + b = cy + d = ez + f
$$


\subsection{Conversion}

\subsubsection{Vector to Dot Product in 2D}

Given the vector equation of a line:

$$
\bb{r} = \bb{a} + \lambda \bb{d} \qquad \lambda \in \mathbb{R}
$$

Find a vector perpendicular to the direction for the normal.

$\bb{a}$ is a point on the line.

Substitute these values into the dot product equation of a line.


\subsubsection{Dot Product to Vector in 2D}

Given the dot product equation of a line:

$$
\bb{r} \cdot \bb{n} = c
$$

Find a vector perpendicular to the normal for the direction of the line.

Find a point $\bb{a}$ that satisfies the dot product equation as a point on the
line.

Substitute these values into the vector equation of a line.


\subsubsection{Vector to Parametric}

Let $\bb{r} = \begin{pmatrix} x \\ y \end{pmatrix}$ and write two equations
from this:

$$
\begin{pmatrix} x \\ y \end{pmatrix} = \bb{a} + \lambda \bb{d} \qquad \lambda \in \mathbb{R}
$$

Gives the parametric equation:

$$
\begin{cases}
\begin{aligned}
x & = \bb{a}_x + \lambda \bb{d}_x \\
y & = \bb{a}_y + \lambda \bb{d}_y \\
\end{aligned}
\end{cases}
$$


\subsubsection{Parametric to Vector}

Perform by inspection from the coefficient of $\lambda$ and the additional
term.


\subsubsection{Vector to Cartesian}

Find the two parametric equations:

$$
\begin{aligned}
x & = \bb{a}_x + \lambda \bb{d}_x \\
y & = \bb{a}_y + \lambda \bb{d}_y \\
\end{aligned}
$$

Rearrange for $\lambda$ (assuming both components of $\bb{d}$ are non-zero):

$$
\begin{aligned}
\lambda & = \frac{x - \bb{a}_x}{\bb{d}_x} \\
\lambda & = \frac{y - \bb{a}_y}{\bb{d}_y} \\
\end{aligned}
$$

Equate the two equations:

$$
\frac{x - \bb{a}_x}{\bb{d}_x} = \frac{y - \bb{a}_y}{\bb{d}_y}
$$

If either component of $\bb{d}$ is 0, we ignore the equation formed using the
other component.  This results in a line that is perpendicular to one of the
axes.

For example, if $\bb{d}_x = 0$:

$$
\begin{aligned}
x & = \bb{a}_x \\
y & = \bb{a}_y + \lambda \bb{d}_y \\
\end{aligned}
$$

Gives the cartesian equation:

$$
x = \bb{a}_x
$$

A line parallel to the y axis.


\subsubsection{Dot Product to Cartesian in 2D}

Expand $\bb{r}$:

$$
\begin{pmatrix} x \\ y \end{pmatrix} \cdot \bb{n} = c
$$

Gives the cartesian equation:

$$
\bb{n}_x x + \bb{n}_y y = c
$$


\subsubsection{Cartesian to Dot Product in 2D}

Use the coefficients of $x$ and $y$ to determine the components of the normal.


\subsection{Construction}

\subsection{From Two Points}

Given the points $\bb{a}$ and $\bb{b}$, the direction of the line is:

$$
\bb{d} = \bb{a} - \bb{b}
$$

The vector equation is:

$$
\bb{r} = \bb{a} + \lambda (\bb{a} - \bb{b})
$$


\subsection{Parallel Lines}

Two lines are parallel if:

\begin{itemize}
\item The two direction vectors are parallel.
\item The direction vector and the normal are perpendicular.
\item The two normals are parallel.
\end{itemize}


\subsection{Perpendicular Lines}

Two lines are perpendicular if:

\begin{itemize}
\item The two direction vectors are perpendicular.
\item The direction vector and normal are parallel.
\item The two normals are perpendicular.
\end{itemize}


\subsection{Collisions}

Two objects travelling with constant velocity must be in the same location at
the same time to collide:

$$
\bb{r}_1 + t_1 \bb{v}_1 = \bb{r}_2 + t_2 \bb{v}_2
$$

The objects collide if $t_1 = t_2$ and $t_1 \geq 0$.




\section{Circles and Spheres}

The vector equation for a circle or sphere with centre at position vector
$\bb{d}$ and radius $r$ is:

$$
\lvert \bb{r} - \bb{d} \rvert = r
$$


\subsection{Cartesian Equation}

Expanding the vector equation:

$$
\begin{aligned}
\lvert \begin{pmatrix} x \\ y \end{pmatrix} - \begin{pmatrix} a \\ b \end{pmatrix} \rvert & = r \\
(x - a)^2 + (y - b)^2 & = r^2 \\
\end{aligned}
$$


\subsection{Parametric Equation}

The parametric equation for a sphere is unnecessary.

A circle with centre $(a, b)$ and radius $r$ can be written in parametric form
as:

$$
\begin{cases}
x = r \cos{\theta} + a \\
y = r \sin{\theta} + b \\
\end{cases}
$$

Using the Pythagorean trigonometric identity to eliminate $\theta$ to arrive at
the cartesian equation:

$$
\begin{cases}
(x - a)^2 = r^2 \cos^2{\theta} \\
(y - b)^2 = r^2 \sin^2{\theta} \\
\end{cases}
$$

Add both equations together:

$$
\begin{aligned}
(x - a)^2 + (y - b)^2 & = r^2 \cos^2{\theta} + r^2 \sin^2{\theta} \\
(x - a)^2 + (y - b)^2 & = r^2 \\
\end{aligned}
$$




\section{Ellipses}

2D only.


\subsection{Parametric Equation}

An ellipse with centre $(a, b)$, horizontal radius $r_h$ and vertical radius
$r_v$ has the parametric equation:

$$
\begin{cases}
\begin{aligned}
x & = r_h \cos{\lambda} + a \\
y & = r_v \sin{\lambda} + b \\
\end{aligned}
\end{cases}
$$


\subsection{Cartesian Equation}

An ellipse with centre $(a, b)$, horizontal radius $r_h$ and vertical radius
$r_v$ has the Cartesian equation:

$$
(\frac{x - a}{r_h})^2 + (\frac{y - b}{r_v})^2 = 1
$$


\subsection{Conversions}

\subsubsection{Parametric to Cartesian}

Given the parametric equation for an ellipse:

$$
\begin{cases}
\begin{aligned}
x & = r_h \cos{\lambda} + a \\
y & = r_v \sin{\lambda} + b \\
\end{aligned}
\end{cases}
$$

Move $a$ and $b$ to the left hand side of the equations and divide by the
radii:

$$
\begin{cases}
\begin{aligned}
\frac{x - a}{r_h} & = \cos{\lambda} \\
\frac{y - b}{r_v} & = \sin{\lambda} \\
\end{aligned}
\end{cases}
$$

Square each equation and add them together:

$$
(\frac{x - a}{r_h})^2 + (\frac{y - b}{r_v})^2 = \cos^2{\lambda} + \sin^2{\lambda}
$$

Use the Pythagorean identity to eliminate $\lambda$:

$$
(\frac{x - a}{r_h})^2 + (\frac{y - b}{r_v})^2 = 1
$$




\section{Planes}

3D only.


\subsection{Vector Equation}

The vector equation for a plane that passes through a point with position
vector $\bb{a}$ and extends in directions $\bb{b}$ and $\bb{c}$.

$$
\bb{r} = \bb{a} + \lambda \bb{b} + \mu \bb{c}
$$

$\bb{b}$ and $\bb{c}$ are two vectors parallel to the plane.


\subsection{Dot Product Form}

The equation for a plane in dot product form with normal $\bb{n}$ that passes
through a point with position vector $\bb{a}$:

$$
\bb{r} \cdot \bb{n} = \bb{n} \cdot \bb{a}
$$

Since the right hand side evaluates to a constant, this is equivalent to:

$$
\bb{r} \cdot \bb{n} = c \qquad c \in \mathbb{R}
$$


\subsection{Parametric Form}

The parametric form of a plane from its vector equation is:

$$
\begin{cases}
\begin{aligned}
x & = \bb{a}_x + \lambda \bb{b}_x + \mu \bb{c}_x \\
y & = \bb{a}_y + \lambda \bb{b}_y + \mu \bb{c}_y \\
z & = \bb{a}_z + \lambda \bb{b}_z + \mu \bb{c}_z \\
\end{aligned}
\end{cases}
$$


\subsection{Cartesian Form}

The cartesian form of a plane is:

$$
ax + by + cz = d
$$


\subsection{Conversions}

\subsubsection{Vector to Dot Product}

Given the equation of a plane in vector form:

$$
\bb{r} = \bb{a} + \lambda \bb{b} + \mu \bb{c}
$$

The normal is the cross product of $\bb{b}$ and $\bb{c}$:

$$
\bb{n} = \bb{b} \times \bb{c}
$$

$\bb{a}$ is a point on the plane.

Substitute these values into the dot product equation for a plane.


\subsubsection{Dot Product to Vector}

Find any two non-parallel vectors that are perpendicular to the normal (dot
product is 0).

Find a point that satisfies the dot product equation for the plane.

Substitute these values into the vector equation for a plane.


\subsubsection{Dot Product to Cartesian}

Expand $\bb{r}$ into its components:

$$
\begin{pmatrix} x \\ y \\ z \end{pmatrix} \cdot \bb{n} = c
$$

Simplify the dot product:

$$
\bb{n}_x x + \bb{n}_y y + \bb{n}_z z = c
$$


\subsubsection{Cartesian to Dot Product}

Use the coefficients of $x$, $y$, and $z$ to find the components of the normal.


\subsubsection{Vector to Cartesian}

Convert the vector equation into dot product form.


\subsubsection{Cartesian to Vector}

Convert the cartesian into dot product form, then convert the dot product form
to vector form.


\subsection{Construction}

\subsubsection{From 2 Points and Parallel Vector}

Given two points at position vectors $\bb{a}$ and $\bb{b}$ that lie on the
plane, and a vector $\bb{c}$ parallel to the plane.

Find a second direction from the two points:

$$
\bb{d} = \bb{a} - \bb{b}
$$

Substitute into the vector equation for a plane:

$$
\bb{r} = \bb{a} + \lambda \bb{c} + \mu (\bb{a} - \bb{b})
$$


\subsubsection{From 3 Points}

Given three points at position vectors $\bb{a}$, $\bb{b}$ and $\bb{c}$ that lie
on the plane.

Find 2 direction vectors from the 3 points and substitute into the vector
equation of a plane:

$$
\bb{r} = \bb{a} + \lambda (\bb{a} - \bb{b}) + \mu (\bb{a} - \bb{c})
$$


\subsubsection{From 2 Lines}

Use the directions of the two lines as two vectors parallel to the plane, and
any point on either of the lines as a point on the plane.



\section{Angle Between}

\subsection{Two Lines}

Find the angle between the two direction vectors of the lines.


\subsection{Line and Plane}

\subsubsection{Plane in Dot Product Form}

Find the angle $\theta$ between the direction of the line and the plane's normal.

The angle between the plane and line is then $\frac{\pi}{2} - \theta$.

Convert to an acute angle.


\subsubsection{Plane in Vector Form}

Convert the plane to dot product form and use the method above.


\subsection{Two Planes}

Convert both planes to dot product form and find the angle between the two
normals.



\section{Point On Surface}

Given the point with position vector $\bb{p}$.


\subsection{Line}

Equate the point and the equation of the line:

$$
\bb{p} = \bb{a} + \lambda \bb{b}
$$

Form 2 or 3 equations for $\lambda$:

$$
\begin{aligned}
\bb{p}_x & = \bb{a}_x + \lambda \bb{b}_x \\
\bb{p}_y & = \bb{a}_y + \lambda \bb{b}_y \\
\end{aligned}
$$

Solve for $\lambda$ using each equation.

If the values of $\lambda$ are consistent, the point lies on the line.


\subsection{Plane}

\subsubsection{Vector Form}

Equate the point and the equation of the plane:

$$
\bb{p} = \bb{a} + \lambda \bb{b} + \mu \bb{c}
$$

Solve for $\lambda$ and $\mu$ using equations formed from the x and y
components.

Substitute these values into the equation formed from the z components. If the
values are consistent, the point lies on the plane.


\subsubsection{Dot Product Form}

Substitute the point into the equation of the plane:

$$
\bb{p} \cdot \bb{n} = c
$$

Calculate the dot product and if it is consistent with the constant $c$, the
point lies on the plane.


\subsection{Sphere}

Substitute the point into the equation for the sphere:

$$
\lvert \bb{p} - \bb{d} \rvert = r
$$

If the left hand side is less than $r$, the point lies inside the sphere.

If the left hand side equals $r$, the point lies on the sphere.

If the left hand side is greater than $r$, the point lies outside the sphere.




\section{Line On Surface}

\subsection{Plane}

A line lies on a plane if its direction is parallel to the plane, and the line
and plane have a common point.

If the normal of the plane is perpendicular to the direction of the line, and
any point on the line satisfies the equation for the plane, the line lies on
the plane.




\section{Intersections}

\subsection{Line and Line}

Given the equations of two lines:

$$
\begin{aligned}
\bb{r}_1 & = \bb{a}_1 + \lambda \bb{d}_1 \\
\bb{r}_2 & = \bb{a}_2 + \mu \bb{d}_2 \\
\end{aligned}
$$

Equate them:

$$
\bb{a}_1 + \lambda \bb{d}_1 = \bb{a}_2 + \mu \bb{d}_2
$$

Solve for $\lambda$ and $\mu$ using equations formed from the x and y
components.

Substitute the two values into the equation formed from the z component. If it
is consistent, the lines intersect.

Substitute $\lambda$ or $\mu$ back into one of the original line equations to
find the point of intersection.


\subsection{Line and Plane}

\subsubsection{Vector Equation}

Given the equation of a line and plane:

$$
\begin{aligned}
\bb{r}_1 & = \bb{a} + \lambda \bb{b} \\
\bb{r}_2 & = \bb{c} + \mu \bb{d} + \alpha \bb{e} \\
\end{aligned}
$$

Equate them:

$$
\bb{a} + \lambda \bb{b}  = \bb{c} + \mu \bb{d} + \alpha \bb{e}
$$

Solve for $\lambda$, $\mu$, and $\alpha$.

Substitute $\lambda$ back into the original line equation to find the point of
intersection.


\subsubsection{Dot Product Equation}

Given the equation of a line and plane:

$$
\begin{aligned}
\bb{r}_1 & = \bb{a} + \lambda \bb{b} \\
\bb{r}_2 \cdot \bb{n} & = c \\
\end{aligned}
$$

Substitute the line equation into the plane equation:

$$
(\bb{a} + \lambda \bb{b}) \cdot \bb{n} = c
$$

Expand the dot product and solve for $\lambda$.

Substitute $\lambda$ back into the original line equation to find the point of
intersection.


\subsection{Line and Circle}

Given the equation of a line and circle:

$$
\begin{aligned}
\bb{r}_1 & = \bb{a} + \lambda \bb{b} \\
\lvert \bb{r}_2 - \bb{c} \rvert & = r \\
\end{aligned}
$$

Substitute the line equation into the circle equation:

$$
\lvert \bb{a} + \lambda \bb{b} - \bb{c} \rvert = r
$$

This will simplify to a quadratic in terms of $\lambda$.

If there are two solutions, the line intersects the circle and is not
tangential to it.

If there is one solution, the line is tangent to the circle.

If there are no solutions, the line does not intersect the circle.


\subsection{Plane and Plane}

Two planes intersect along a line.

The line has a direction perpendicular to the two normals of the planes.

A point on the line is a point common to both planes (satisfies both plane
equations).


\subsection{Circle and Axis Plane}

For the xy-plane, $z = 0$.

For the xz-plane, $y = 0$.

For the yz-plane, $x = 0$.

Find the cartesian equation of a sphere:

$$
(x - a)^2 + (y - b)^2 + (z - c)^2 = r^2
$$

Substitute either $x = 0$, $y = 0$, or $z = 0$ depending on the axis plane.

The equation is the cartesian equation for the intersecting circle on this
plane.




\section{Minimum Distance}

The minimum distance between two vector surfaces.


\subsection{Point and Line}

Given a point at position vector $\bb{p}$ and line:

$$
\bb{r} = \bb{a} + \lambda \bb{d}
$$


\subsubsection{Dot Product}

The closest distance is when the vector between an arbitrary point on the line
and $\bb{p}$ and the direction of the line are perpendicular.

The vector between $\bb{p}$ and an aribtrary point on the line is:

$$
\bb{p} - \bb{r} = \bb{p} - \bb{a} - \lambda \bb{d}
$$

This must be perpendicular to the direction of the line:

$$
(\bb{p} - \bb{a} - \lambda \bb{d}) \cdot \bb{d} = 0
$$

Solve for $\lambda$.

Substitute $\lambda$ back into the line equation to get the closest point on
the line (point $\bb{c}$)

Find the magnitude $\lvert \bb{p} - \bb{c} \rvert$ for distance between the
point and the closest point on the line ($\bb{c}$).


\subsubsection{Cross Product}

Chose some aribtrary point on the line Q.

Let the point closest to P on the line be C.

Form a right angled triangle QPC.

Let $\theta$ be the angle at Q (ie. the angle between $\vv{QP}$ and
$\vv{QC}$).

The perpendicular distance is thus $\lvert \vv{QP} \rvert \sin{\theta}$.

This can be written as:

$$
\begin{aligned}
& = \frac{\lvert \vv{QP} \rvert \lvert \vv{QC} \rvert \sin{\theta}}{\lvert \vv{QC} \rvert} \\
& = \frac{\lvert \vv{QP} \times \vv{QC} \rvert}{\lvert \vv{QC} \rvert} \\
\end{aligned}
$$

Which is equivalent to:

$$
\frac{\lvert \bb{p} - \bb{a} \times \bb{d} \rvert}{\lvert \bb{d} \rvert}
$$


\subsection{Point and Plane}

Given the point $\bb{p}$.


\subsubsection{Vector Equation of Plane}

Given the vector equation of a plane:

$$
\bb{r} = \bb{a} + \lambda \bb{b} + \mu \bb{c}
$$

The vector between an arbitrary point on the plane and $\bb{p}$ will be
perpendicular to the directions of the plane:

$$
\begin{aligned}
(\bb{p} - \bb{a} - \lambda \bb{b} - \mu \bb{c}) \cdot \bb{b} & = 0 \\
(\bb{p} - \bb{a} - \lambda \bb{b} - \mu \bb{c}) \cdot \bb{c} & = 0 \\
\end{aligned}
$$

Solve simultaneously for $\lambda$ and $\mu$.

Substitute back into the equation for the plane to find the closest point and
distance.

Alternatively, determine the normal and use the below method.


\subsubsection{Dot Product Equation of Plane}

Given the dot product equation of a plane:

$$
\bb{r} \cdot \bb{n} = c
$$

Find a point on the plane $\bb{a}$.

Find the vector between $\bb{a}$ and $\bb{p}$, $\vv{AP}$.

Find the scalar projection of this onto the normal $\bb{n}$.


\subsection{Point and Sphere}

Equivalent to the distance between the point and the centre of the sphere,
subtract the radius.


\subsection{Line and Line}

\subsubsection{Not Parallel}

Find a vector $\bb{n}$ perpendicular to the directions of both lines.

Find the vector between two points on the lines.

Find the scalar projection of this onto $\bb{n}$.


\subsubsection{Parallel}

Pick a point on one of the lines.

Find the closest distance between this point and the other line using the
method above.


\subsection{Line and Parallel Plane}

Pick a point on the line.

Find the closest distance between this point and the plane using the method
above.


\subsection{Line and Sphere}

Find the closest distance between the line and the centre of the sphere, then
subtract the radius.


\subsection{Plane and Parallel Plane}

Pick a point on one of the planes.

Find the closest distance between this point and the other plane using the
method above.


\subsection{Plane and Sphere}

Find closest distance between plane and centre of sphere, then subtract the
radius.


\subsection{Sphere and Sphere}

Find the distance between the two centres, then subtract both radii.




\section{Closest Approach}

The closest distance between two objects travelling in defined paths. Cannot
use above methods, as the position of the objects depends on the time.

For two objects travelling in straight lines, this can be solved using relative
velocities and the methods above.

For objects not travelling in straight lines, requires calculus to solve.
Find an expression for the distance between the two objects at any time, then
minimise this function.


\subsection{Lines}

Given two objects travelling in the lines:

$$
\begin{aligned}
\bb{r}_1 & = \bb{a}_1 + t \bb{v}_1 \\
\bb{r}_2 & = \bb{a}_2 + t \bb{v}_2 \\
\end{aligned}
$$

Find the position of object 2 relative to object 1, $_2\bb{r}_1$.

Find the velocity of object 2 relative to object 1, $_2\bb{v}_1$.

Form an equation of motion for object 2 relative to object 1:

$$
\bb{r} = \'_2\bb{r}_1 + \'_2\bb{v}_1
$$

Minimise the distance between this line and the origin (the location of object
1) to find the closest time and distance they are apart.

\end{document}
