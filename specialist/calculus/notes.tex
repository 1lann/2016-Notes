
\documentclass[a4paper,11pt]{article}

% Math symbols
\usepackage{amsmath}
\usepackage{amsfonts}
\usepackage{esvect}

% Hyperlink contents page
\usepackage{hyperref}
\hypersetup{
	colorlinks,
	citecolor=black,
	filecolor=black,
	linkcolor=black,
	urlcolor=black
}

% No indent on new paragraphs
\setlength{\parindent}{0mm}
\setlength{\parskip}{0.2cm}

% Alias \boldsymbol to \bb for vectors
\newcommand{\bb}{\boldsymbol}


\begin{document}

\title{Calculus}
\author{Ben Anderson}
\date{\today}
\maketitle
\pagebreak

\tableofcontents
\pagebreak


\section{Vector Calculus}

\subsection{Derivative}

Given the vector function:

$$
\bb{r} = \binom{2 + 3t}{3t + t^2}
$$

Derive each component of the vector function to find its derivative:

$$
\frac{d\bb{r}}{dt} = \binom{3}{3 + 2t}
$$


\subsection{Indefinite Integral}

Given the vector function:

$$
\bb{r} = \binom{2 + 4t}{8t + 3t^2}
$$

Integrate each component to find its indefinite integral:

$$
\int \bb{r} dt = \binom{2t + 2t^2 + c_1}{4t^2 + t^3 + c_2}
$$


\subsection{Definite Integral}

Same as normal calculus:

$$
\int_a^b \frac{d}{dt}\bb{r}(t) dt = \bb{r}(b) - \bb{r}(a)
$$




\section{Rectilinear Motion}

\subsection{Distance Travelled}

The distance travelled by a body between two points in time is the definite
integral of the magnitude of the velocity function:

$$
\int_{t_1}^{t_2} \lvert \bb{v}(t) \rvert dt
$$


\subsection{Circular or Elliptical Motion}

For a body travelling in a circle or an ellipse around a fixed point.

The velocity is perpendicular to the acceleration and to the direction of
motion:

$$
\begin{aligned}
\bb{v} \cdot \bb{r} & = 0
\bb{v} \cdot \bb{a} & = 0
\end{aligned}
$$


\subsection{Cartesian Equation of Path}

Convert $\bb{r}(t)$ to a cartesian equation.


\subsection{Graphing Circular Motion}

For increasing values of $t$, plot the corresponding value of $\bb{r}(t)$ on a
cartesian plane.


To indicate the direction of motion, find $\bb{r}(0)$ to find the initial
position of the body, then find $\bb{r}(1)$ to find where it moves to next,
indicating its direction of motion.


\subsection{Projectile Motion}

To find the maximum height reached, solve for when the $\bb{j}$ component of
the velocity is 0.




\section{Integration Techniques}

\subsection{Basic}

If given an integral in the form:

$$
\int f'(x) f(x) dx
$$

Then use standard integration techniques to solve it.


\subsection{Substitution}

Involves changing the variable used in the integration.

Given:

$$
\int 56x(2x - 3)^5 dx
$$

Find an appropriate substitution. Chose a part of the function which is raised
to a power.

Let $u = 2x - 3$:

$$
\int 56xu^5 dx
$$

Convert $dx$ to $du$:

$$
\begin{aligned}
\frac{du}{dx} & = 2 \\
dx & = \frac{du}{2} \\
\end{aligned}
$$

Thus:

$$
\int 56xu^5 \frac{1}{2} du
$$

Convert $x$ to $u$ by rearranging $u = 2x - 3$ for $x$:

$$
\int 14u^5(u + 3) du
$$

Expand the brackets:

$$
\int 14u^6 + 42u^5 du
$$

Perform the integration:

$$
2u^7 + 7u^6 + c
$$

Substitute in $u = 2x - 3$:

$$
2(2x - 3)^7 + 7(2x - 3)^6 + c
$$


\subsection{Substitution with Definite Integrals}

When solving definite integrals using substitution, convert the limits of the
integral to be in terms of $u$.

Given:

$$
\int_1^2 56x(2x - 3)^5 dx
$$

Let $u = 2x - 3$.

$u$ given $x = 1$ is -1, and given $x = 2$ is 1.

From above:

$$
\begin{aligned}
& = \int_{-1}^1 14u^6 + 42u^5 du \\
& = \big[2u^7 + 7u^6\big]_{-1}^1 \\
& = 2 + 7 - (-2 + 7) \\
& = 4 \\
\end{aligned}
$$


\subsection{Trigonometric Identities}

\subsubsection{Odd Power}

For $\sin^n{x}$ or $\cos^n{x}$, where $n$ is odd, pull out a $\sin{x}$ or
$\cos{x}$ to make the power even, then substitute $1 - \cos^2{x}$ or
$1 - \sin^2{x}$ from the Pythagorean identity:

$$
\begin{aligned}
\int \sin^3{x} dx & = \int \sin{x}\sin^2{x} dx \\
& = \int \sin{x}(1 - \cos^2{x}) dx \\
& = \int \sin{x} - \sin{x}\cos^2{x} dx \\
& = -\cos{x} + \frac{1}{3} \cos^3{x} + c \\
\end{aligned}
$$


\subsubsection{Even Power}

For $\sin^n{x}$ or $\cos^n{x}$, where $n$ is even, use the double angle formula
for $\cos{2x}$:

$$
\begin{aligned}
\cos{2x} & = 1 - 2\sin^2{x} \\
& = 2\cos^2{x} - 1 \\
\end{aligned}
$$

Rearrange the equation in terms of $\cos{2x}$:

$$
\begin{aligned}
\int \sin^2{x} dx & = \int \frac{1}{2} - \frac{1}{2}\cos{2x} dx \\
& = \frac{x}{2} - \frac{1}{4}\sin{2x} + c \\
\end{aligned}
$$


\subsubsection{Secant}

For $\tan^2{x}$, substitute $\sec^2{x} - 1$ from the identity:

$$
\sec^2{x} = \tan^2{x} + 1
$$


\subsubsection{Product to Sum}

For:

$$
\int \sin{ax}\cos{bx} dx
$$

Use the product to sum identities (these should be given in the question).




\subsection{Logarithmic Integration}

$$
\begin{aligned}
\int \frac{1}{x} dx & = \ln{|x|} + c \qquad x \neq 0 \\
\int \frac{f'(x)}{f(x)} dx & = \ln(|f(x)|) + c \qquad f(x) \neq 0 \\
\end{aligned}
$$

\subsubsection{Improper Fractions}

Given an improper fraction to integrate:

$$
\int \frac{2x}{x + 1} dx
$$

Convert the fraction to a proper one (use algebraic juggling):

$$
\int 2 - \frac{2}{x + 1} dx
$$

And integrate:

$$
2x - 2\ln{|x + 1|} + c
$$


\subsection{Logarithmic Definite Integrals}

Given:

$$
\int_a^b \frac{f'(x)}{f(x)} dx
$$

If $f(x) = 0$ for any $x$ in the interval $a \leq x \leq b$, then the integral
is undefined.


\subsection{Partial Fractions}

For proper fractions with a series of factors in the denominator.

The number of unknowns in the numerator will be the same as the degree of the
denominator.


\subsubsection{Linear Factors}

Given a proper fraction with a series of linear factors in the denominator:

$$
\int \frac{4x - 3}{(x + 3)(2x + 1)} dx
$$

Consider the inner fraction:

$$
\frac{4x - 3}{(x + 3)(2x + 1)}
$$

Split the fraction into parts, where the denominator of each is one of the
linear factors, and the numerator is some polynomial of one degree less than
the denominator:

$$
\frac{4x - 3}{(x + 3)(2x + 1)} = \frac{A}{x + 3} + \frac{B}{2x + 1}
$$

Multiply by $(x + 3)(2x + 1)$:

$$
4x - 3 = A(2x + 1) + B(x + 3)
$$

Compare coefficients to solve for $A$ and $B$:

$$
\begin{aligned}
4 & = 2A + B \\
-3 & = A + 3B \\
\end{aligned}
$$

Solve the system of equations by elimination:

$$
\begin{aligned}
10 & = -5B \\
B & = -2 \\
A & = 3 \\
\end{aligned}
$$

Thus the partial fractions are:

$$
\frac{3}{x + 3} - \frac{2}{2x + 1}
$$

Use logarithms to solve the integral:

$$
\begin{aligned}
\int \frac{4x - 3}{(x + 3)(2x + 1)} dx & = \int \frac{3}{x + 3} - \frac{2}{2x + 1} dx \\
& = 3\ln{|x + 3|} - \ln{|2x + 1|} \\
\end{aligned}
$$


\subsubsection{Cover Up Method}

A method to easily find the required partial fractions.
Only applies to fractions with linear factors.

Again, consider the fraction:

$$
\frac{4x - 3}{(x + 3)(2x + 1)}
$$

We require an expression in the form:

$$
\frac{A}{x + 3} + \frac{B}{2x + 1}
$$

For each linear factor, solve for $x$ when the factor equals 0.

Then cover up this linear factor in the original fraction, and substitute in
the value for $x$ to find the corresponding constant in the partial fractions.

To find $A$, solve $x + 3 = 0$, hence $x = -3$. Cover up the factor $x + 3$ in
the original fraction to give:

$$
\frac{4x - 3}{2x + 1}
$$

And substitute $x = -3$ to give $A = -3$.

To find $B$, solve $2x + 1 = 0$, hence $x = -0.5$. Substitute this into:

$$
\frac{4x - 3}{x + 3}
$$

To give $B = -2$.



\subsubsection{Quadratic Factors}

To find the integral:

$$
\int \frac{7x^2 - 2x + 5}{(x - 1)(x^2 + 1)} dx
$$

Use the partial fractions:

$$
\frac{A}{x - 1} + \frac{Bx + C}{x^2 + 1}
$$

Solve for the constants by equating this to the original fraction:

$$
\frac{7x^2 - 2x + 5}{(x - 1)(x^2 + 1)}  = \frac{A}{x - 1} + \frac{Bx + C}{x^2 + 1}
$$

Multiply through by $(x - 1)(x^2 + 1)$:

$$
7x^2 - 2x + 5  = A(x^2 + 1) + (Bx + C)(x - 1)
$$

Solve for $A$, $B$, and $C$ by comparing coefficients:

$$
\begin{aligned}
7 & = A + B \\
5 & = A - C \\
-2 & = -B + C \\
\end{aligned}
$$

Solve the system of equations by elimination:

$$
\begin{aligned}
5 & = A + C \\
10 & = 2A \\
A & = 5 \\
B & = 2 \\
C & = 0 \\
\end{aligned}
$$

Substitute these values into the partial fractions and solve the integral using
logarithms.


\subsubsection{Repeated Factors}

Given the integral:

$$
\int \frac{6x^2 + 6x - 20}{(x + 3)(x - 1)^2} dx
$$

Use the partial fractions:

$$
\frac{A}{x + 3} + \frac{B}{x - 1} + \frac{C}{(x - 1)^2}
$$

Solve for the constants by equating the two:

$$
\frac{6x^2 + 6x - 20}{(x + 3)(x - 1)^2} = \frac{A}{x + 3} + \frac{B}{x - 1} + \frac{C}{(x - 1)^2}
$$

Multiply by the denominator:

$$
6x^2 + 6x - 20 = A(x - 1)^2 + B(x + 3)(x - 1) + C(x + 3)
$$

Solve by comparing coefficients:

$$
\begin{aligned}
6 & = A + B \\
-20 & = A + 2B + 3C \\
6 & = -2A + B + C \\
\end{aligned}
$$

Solve the system of equations:

$$
\begin{aligned}
A & = -4 \\
B & = 10 \\
C & = -12 \\
\end{aligned}
$$

Perform the integration using logarithms.


\subsection{Integration by Parts}

Used to integrate expressions with two functions multiplied together.

The formula is derived from the product rule:

$$
\begin{aligned}
(f(x)g(x))' & = f'(x)g(x) + g'(x)f(x) \\
f(x)g(x) & = \int f'(x)g(x) dx + \int g'(x)f(x) dx \\
\int f'(x)g(x) dx & = f(x)g(x) - \int g'(x)f(x) dx \\
\end{aligned}
$$

Given the integral:

$$
\int e^x x dx
$$

Let $f'(x) = e^x$ and $g(x) = x$.

Hence $f(x) = e^x$ (ignore the integrating constant) and $g'(x) = 1$.

Using the formula from above:

$$
\begin{aligned}
\int e^x x dx & = e^x x - \int 1 \times e^x dx \\
& = x e^x - e^x + c \\
\end{aligned}
$$


\subsection{Summary}

Given a certain function in different forms, use several different methods for
integration:

Basic:

$$
\int f'(x) f(x) dx
$$

Substitution (let $u = f(x)$):

$$
\int g(x) (f(x))^n dx
$$

Use trigonometric identities for an integration involving some combination of
trigonometric functions.

Logarithms:

$$
\int \frac{f'(x)}{f(x)} dx
$$

Partial fractions, using the cover up method:

$$
\int \frac{f(x)}{(x + a)(x + b)}
$$

Partial fractions with a quadratic factor:

$$
\int \frac{f(x)}{(x^2 + bx + c)(x + d)}
$$

Partial fractions with a repeated factor:

$$
\int \frac{f(x)}{(x + a)^2 (x + b)}
$$

Integration by parts:

$$
\int f(x)g(x) dx
$$




\section{Numerical Integration}

Several functions cannot be algebraically integrated.
Numerical integration approximates a definite integral of such functions.

Given the definite integral:

$$
\int^2_0 e^{x^2} dx
$$

Different rules express the integral as an infinite sum of the areas of adjacent
shapes.

The integral can be approximated by taking a finite number of terms from such
sequences.

Summing more terms improves the accuracy of the approximation.


\subsection{Trapezium Rule}

The area of a trapezium with parallel side lengths $y_0$ and $y_1$, and
perpendicular height $h$ is:

$$
h\frac{y_0 + y_1}{2}
$$

Summing the areas of these trapeziums:

$$
\begin{aligned}
& = \sum_{i = 1}^n h\frac{y_{i - 1} + y_i}{2} \\
& = \frac{h}{2} \sum_{i = 1}^n y_{i - 1} + y_i \\
& = \frac{h}{2} (y_0 + 2\sum_{i = 1}^{n - 1} y_i + y_n) \\
\end{aligned}
$$

Given some definite integral:

$$
\int^a_b f(x) dx
$$

We have:

$$
\begin{aligned}
h & = \frac{a - b}{n} \\
y_0 & = f(b) \\
y_n & = f(a) \\
y_i & = f(b + ih) \\
\end{aligned}
$$


\subsection{Simpson's Rule}

Models the tops of the strips as parabolas.

Only applies for an even number of strips.

$$
\frac{h}{3}(y_0 + \sum_{i = 1}^{n - 2} (4y_i + 2y_{i + 2}) + 4y_{n - 1} + y_n)
$$

$h$ and $y_i$ are calculated the same as when applying the Trapezium rule.




\section{Volumes of Revolution}

The volume of the solid formed when an area under a curve is rotated in 3D
around a line (usually an axis).


\subsection{Around X Axis}

Split the area into vertical disks of thickness $\delta x$.

The volume of one of these disks is:

$$
\pi y^2 \delta x
$$

Summing the volume of an infinite number of these disks between $x = a$ and
$x = b$ gives:

$$
\lim_{\delta x \to 0} \sum_{x = a}^{b} \pi y^2 \delta x
$$

Using the definition of integration:

$$
\pi \int^b_a y^2 dx
$$


\subsubsection{Area with the Y Axis Rotated About the X Axis}

The volume of a solid formed when an area bounded by a function $f(x)$, the x
axis, and the lines $x = a$ and $x = b$ is rotated about the y axis.

Consider the inner surface area of a cylinder of radius $x$ and height $y$:

$$
2 \pi x y
$$

The volume of the solid is the sum of an infinite number of these cylinders
between $x = a$ and $x = b$:

$$
2 \pi \int^b_a x y dx
$$


\subsection{Around Y Axis}

Split the area into horizontal disks of thickness $\delta y$.

The volume of one disk is:

$$
\pi x^2 \delta y
$$

Summing the volumes of these disks between $y = a$ and $y = b$ gives:

$$
\lim_{\delta y \to 0} \sum_{y = a}^{b} \pi x^2 \delta y
$$

Using the definition of integration:

$$
\pi \int^b_a x^2 dy
$$


\subsubsection{Area with the X Axis Rotated About the Y Axis}

The volume of a solid formed when an area bounded by a function $f(x)$, the y
axis, and the lines $y = a$ and $y = b$ is rotated about the x axis.

Consider the inner surface area of a cylinder of radius $y$ and height $x$:

$$
2 \pi x y
$$

The volume of the solid is the sum of an infinite number of these cylinders
between $y = a$ and $y = b$:

$$
2 \pi \int^b_a x y dy
$$


\subsection{Around a Line}

Change the origin of the function to be centred around an axis.


\subsection{Area Between Two Curves}

The volume of a solid formed when the area between the curves $f(x)$ and $g(x)$
is rotated about one of the axes.

Given $f(x)$ is above $g(x)$ for the interval over which we're rotating,
subtract the volume of the solid formed using $g(x)$ from that of $f(x)$:

For example, rotating about the $x$ axis and bound by the lines $x = a$ and
$x = b$:

$$
\pi \int^b_a (f(x))^2 dx - \pi \int^b_a (g(x))^2 dx
$$

Use a graph of the two functions to determine the required areas in more
complicated scenarios.

\end{document}
