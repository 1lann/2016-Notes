
\documentclass[a4paper,11pt]{article}

% Math symbols
\usepackage{amsmath}
\usepackage{amsfonts}
\usepackage{esvect}

% Hyperlink contents page
\usepackage{hyperref}
\hypersetup{
	colorlinks,
	citecolor=black,
	filecolor=black,
	linkcolor=black,
	urlcolor=black
}

% \cis and \Arg
\DeclareMathOperator\cis{cis}
\DeclareMathOperator\Arg{Arg}

% No indent on new paragraphs
\setlength{\parindent}{0mm}
\setlength{\parskip}{0.2cm}

% Alias \boldsymbol to \bb for vectors
\newcommand{\bb}{\boldsymbol}


\begin{document}

\title{Functions}
\author{Ben Anderson}
\date{\today}
\maketitle
\pagebreak

\tableofcontents
\pagebreak


\section{Function and Relations}

A function is a series of operations that maps one or more input values to one
output value.

A relation is a series of operations that maps one or more input values to one
or more output values.


\subsection{Types}

A one-to-one function means each input value has one possible output value.
For example, $f(x) = x$.

A many-to-one function means multiple input values will map to the same output
value. For example, $f(x) = x^2$.

A one-to-many relation means one input value will map to multiple output values.
For example, $f(x) = \pm \sqrt{x}$.

A many-to-many relation means multiple input values will map to the same set of
multiple output values. For example, $x^2 + y^2 = 9$.


\subsection{Vertical Line Test}

Where a vertical line is moved from left to right over a graph.

If it intersects the curve more than once at any point, the graph is not of a
function (it is a relation).


\subsection{Horizontal Line Test}

Where a horizontal line is moved from top to bottom over a graph.

If it intersects the curve more than once at any point, the graph is of a
one-to-many function (not a one-to-one function).




\section{Domain}

The set of all possible input values for a function.


\subsection{Calculating Domain}

\subsection{Polynomials}

All polynomials have a domain of $x \in \mathbb{R}$.


\subsection{Fractional Polynomials}

For any function with $x$ in the denominator of a fraction, the entire
denominator cannot equal 0.

For example:

$$
f(x) = \frac{g(x)}{h(x)}
$$

Solve $h(x) = 0$ to determine the values of $x$ which are not in the domain.



\section{Range}

The set of all output values of a function.

The range is a subset of the codomain.


\subsection{Codomain}

The set of all possible output values of a function.

The codomain is part of the definition of the function. It is usually assumed
to be all real numbers, but can be defined as something else.


\subsubsection{Example}

Define the function $f(x) = x^2$ with a domain $x \in \mathbb{Z}$ and codomain
$y \in \mathbb{R}$.

The range of the function is ${1, 4, 9, 16, \ldots}$, yet the codomain was
defined to be all real numbers.

Thus the range is only a subset of the codomain.


\subsection{Calculating Range}

\subsubsection{Polynomials}

Polynomials with an odd degree have a domain and range across all real numbers.

Polynomials with an even degree have a domain across all real numbers and a
range above or below the lowest or highest turning point (depending if the
function has a maximum or minimum turning point).


\subsubsection{Hyperbolas}

Draw a rough sketch of the graph to determine what methods to use to determine
the range of $y$ values which the function never outputs.


\subsection{Square Root}

The contents of the square root must be greater than or equal to 0 (positive).




\section{Asymptotes}

To determine asymptotes for the function:

$$
f(x) = \frac{g(x)}{h(x)}
$$

Where $g(x)$ and $h(x)$ are polynomials.


\subsection{Vertical Asymptotes}

Occur when the denominator is 0.

Solve $h(x) = 0$.


\subsubsection{Behaviour Around Vertical Asymptotes}

Given there is a vertical asymptote at $x = a$.

The function will approach either positive or negative infinity either side of
$x = a$.

For the behaviour of the function as it approaches the asymptote from the left:

\begin{itemize}
\item If $f(a^-)$ is negative, the function approaches negative infinity.
\item If $f(a^-)$ is postiive, the function approaches positive infinity.
\end{itemize}

For the behaviour of the function as it approaches the asymptote from the right:

\begin{itemize}
\item If $f(a^+)$ is negative, the function approaches negative infinity.
\item If $f(a^+)$ is postiive, the function approaches positive infinity.
\end{itemize}


\subsubsection{Example}

Given the function:

$$
f(x) = \frac{2x - 1}{x^2 + 4x}
$$

Solving the denominator equal to 0 for vertical asymptotes:

$$
\begin{aligned}
x^2 + 4x & = 0 \\
x(x + 4) & = 0 \\
x & = 0, -4 \\
\end{aligned}
$$

Thus there are two vertical asymptotes with the equations $x = 0$ and $x = -4$.

Approaching the asymptote $x = -4$ from the left. Consider $f(-4^-)$:

\begin{itemize}
\item $2x$ is negative.
\item $2x - 1$ is still negative.
\item Consider the denominator as $x(x + 4)$.
\item $x + 4$ is still slightly negative.
\item $x(x + 4)$ is therefore two negatives multiplied together, forming a
	positive.
\item Thus overall the fraction is negative, so it approaches negative infinity.
\end{itemize}

Approaching the asymptote $x = -4$ from the right. Consider $f(-4^+)$:

\begin{itemize}
\item $2x - 1$ is negative.
\item $x + 4$ is slightly positive.
\item $x(x + 4)$ is therefore negative.
\item Thus overall the fraction is positive, so it approaches positive infinity.
\end{itemize}


\subsection{Oblique Asymptote}

A function will have at most 1 oblique asymptote.

It occurs when the degree of the numerator is 1 larger than the denominator.

The linear equation for the asymptote can be found by simplifying the fraction
using algebraic juggling.


\subsection{Horizontal Asymptote}

A function will have at most 1 horizontal asymptote.

If the degree of the numerator is larger, there is no horizontal asymptote.

If the degree of numerator and denominator are the same, then there is a
horizontal asymptote at:

$$
\frac{\text{leading coefficient of $g(x)$}}{\text{leading coefficient of $h(x)$}}
$$

If the degree of the denominator is larger, then there is a horizontal
asymptote at $y = 0$.


\subsubsection{Behaviour Approaching Infinity}

The horizontal asymptote is apparent due to the behaviour of the function as
$x$ approaches $\pm \infty$.

It is important whether the function approaches the horizontal asymptote from
above or below.

Consider:

$$
\frac{\text{leading term of $g(x)$}}{\text{leading term of $h(x)$}}
$$

If the fraction is positive for $x = \infty$, then the function approaches the
horizontal asymptote from above for large positive values of $x$.

If the fraction is negative for $x = \infty$, then the function approaches the
horizontal asymptote from below for large positive values of $x$.

If the fraction is positive for $x = -\infty$, then the function approaches the
horizontal asymptote from above for large negative values of $x$.

If the fraction is negative for $x = -\infty$, then the function approaches the
horizontal asymptote from below for large negative values of $x$.




\section{Graphing Hyperbolas}

To graph the function, where $g(x)$ and $h(x)$ are polynomials:

$$
f(x) = \frac{g(x)}{h(x)}
$$

First attempt to simplify the function to a set of transformations on a normal
hyperbola:

$$
f(x) = \frac{a}{b(x + c)} + d
$$

Otherwise:

\begin{enumerate}
\item Find all vertical, horizontal, and oblique asymptotes and mark them with
	dotted lines.
\item Find the behaviour of the function as it approaches either side of each
	of the asymptotes.
\item Plot all axis intercepts.
\item Plot all stationary points (using $f'(x) = 0$).
\end{enumerate}


\subsection{Simplification}

Given:

$$
f(x) = \frac{x + 3}{x + 3}
$$

This is equivalent to a graph of $f(x) = 1$, but with a point of discontinuity
at $x = -3$ (an open circle).


\subsection{Range}

For some $f(x)$ there is a gap in the middle of the graph for which no $y$ value
is defined.

Given:

$$
f(x) = \frac{3x - 2}{x(x + 2)}
$$

To solve for this gap algebraically:

$$
\begin{aligned}
y & = \frac{3x - 2}{x(x + 2)} \\
yx^2 + 2yx & = 3x - 2 \qquad x \neq 0, -2 \\
yx^2 + (2y - 3)x + 2 & = 0 \\
\end{aligned}
$$

This is a quadratic equation in terms of $x$.

For real valued $y$, the discriminant must be non-negative:

$$
\begin{aligned}
b^2 - 4ac & \geq 0 \\
(2y - 3)^2 - 8y & \geq 0 \\
4y^2 - 20y + 9 & \geq 0 \\
(2y - 9)(2y - 1) & \geq 0 \\
\end{aligned}
$$

Consider the sign of each factor between the roots of this quadratic:

\begin{center}
\begin{tabular}{c|c|c|c}
& $y < 0.5$ & $0.5 < y < 4.5$ & $y > 4.5$ \\
\hline
$2y - 9$ & - & - & + \\
$2y - 1$ & - & + & + \\
$(2y - 9)(2y - 1)$ & + & - & + \\
\end{tabular}
\end{center}

Thus the range for $f(x)$ is $y < 0.5$ and $y > 4.5$.


\section{Composite Functions}

Using the output of one function as the input for another:

$$
f(g(x))
$$


\subsection{Domain and Range}

\subsubsection{Method 1}

Find the resulting expression for the composite function, and determine the
domain and range of it.

Beware of simplifying, as it can lead to a wrong answer.


\subsubsection{Method 2}

Given the composite function $f(g(x))$.

$f(x)$ must be able to cope with the output of $g(x)$ for the composite
function to be defined.

Thus the range of $g(x)$ must be a subset of the domain of $f(x)$. It must not
output values $f(x)$ is not defined for.

Thus the domain of $g(x)$ must be restricted to exclude values from its range
for which $f(x)$ is not defined.

For example, given $f(x) = \frac{1}{x - 1}$ and $g(x) = x - 5$.

$f(x)$ has the domain $x \in \mathbb{R}, x \neq 1$ and range
$y \in mathbb{R}, y \neq 0$.

$g(x)$ has a domain and range $x, y \in \mathbb{R}$

For the domain of $f(g(x))$:

$$
\begin{aligned}
g(x) & \neq 1 \\
x - 5 & \neq 1 \\
x \neq 6 \\
\end{aligned}
$$

Thus the domain for $f(g(x))$ is $x \in \mathbb{R}, x \neq 6$.

The range for $f(g(x))$ will be the same as $f(x)$:
$y \in \mathbb{R}, y \neq 0$.




\section{Inverse Functions}

Reverses the operations performed by a function such that:

$$
f^{-1}(f(x)) = x
$$


\subsection{Conditions for an Inverse to Exist}

In order for the inverse of a function to exist as a function (and not a
relation), $f(x)$ must be a one-to-one function.

$f(x)$ must pass the horizontal line test.


\subsubsection{Quadratics}

The domain and range of the inverse may need to be restricted to accurately
reflect the original function.

The inverse of $f(x) = \sqrt{x - 5}$ is $f^{-1}(x) = x^2 + 5$.

The domain of the inverse must be restricted to accurately reflect $f(x)$.

The domain of $f^{-1}(x)$ will be the range of $f(x)$, thus
$x \in \mathbb{R}, x \geq 0$.

We must decide whether to chose the positive or negative side of the quadratic
depending on the original function.


\subsection{Calculating}

Given the function $y$:

$$
y = 3x + 4
$$

The inverse function can be found by rearranging for $x$:

$$
\begin{aligned}
3x & = y - 4 \\
x & = \frac{y - 4}{3} \\
f^{-1}(x) & = \frac{x - 4}{3} \\
\end{aligned}
$$


\subsection{Domain and Range}

The domain of $f^{-1}(x)$ is the range of $f(x)$.

The range of $f^{-1}(x)$ is the domain of $f(x)$.


\subsection{Graphical Relation}

The inverse of a function represents a reflection about the line $y = x$.

The points at which $f(x)$ will intersect with its inverse are the same as the
intersection between $f(x)$ and the line $y = x$.


\subsection{Many-To-One Functions}

The domain or range of many-to-one (or many-to-many) functions can be
restricted to make them one-to-one functions.

An inverse function can then be defined for the restricted domain or range.


\subsubsection{Example}

The function $f(x) = (x - 1)^2$ does not have an inverse for its natural domain.

$f(x)$ has a range $y \in \mathbb{R}, y \geq 0$.

The domain can be restricted to $x \geq 1$ or $x \leq 1$. The function is
one-to-one for this domain, and has an inverse.

The inverse is $f^{-1}(x) = 1 \pm \sqrt{x}$. We must chose either the positive
or negative square root based on our original domain restriction.

For the restriction $x \geq 1$ on $f(x)$, the inverse has a domain $x \geq 0$
and range $x \geq 1$.

To satisfy this range, we must chose the positive. The inverse is then
$f^{-1}(x) = 1 + \sqrt{x}$.




\section{Absolute Value Function}

The absolute value function $f(x) = \lvert x \rvert$ is defined as:

$$
\lvert x \rvert = \begin{cases}
\begin{aligned}
	x \qquad & x \geq 0, \\
	-x \qquad & x < 0, \\
\end{aligned}
\end{cases}
$$

It can also be defined as:

$$
\lvert x \rvert = \sqrt{x^2}
$$

The order of the square root and square is important.

All parts of the graph of any $f(x)$ that are below the x axis are reflected
about the x axis.


\subsection{Solving Equations}

$\lvert x \rvert$ represents the distance of $x$ from the origin.

If $\lvert x \rvert = k$, then either $x = k$ or $x = -k$.

Each solution must be checked by substituting it back into the original
equation, as false solutions can arise.


\subsection{Solving Inequalities}

$\lvert x \rvert < a$ implies the distance from $x$ to the origin is less than
$a$. This means:

$$
-a < x < a
$$

$\lvert x \rvert > a$ implies the distance from $x$ to the origin is greater
than $a$. This means:

$$
x < -a \qquad x > a
$$




\section{Reciprocal Function}

The reciprocal graph of $f(x)$ is:

$$
\frac{1}{f(x)}
$$


\subsection{Graphing}

Any roots of $f(x)$ will be vertical asymptotes on the reciprocal graph.

The $x$ coordinate of any stationary points for $f(x)$ will also be the $x$
coordinate for a stationary point on the reciprocal graph.

Positive/negative values of $f(x)$ will also be positive/negative on the
reciprocal graph.


\subsubsection{Behaviour Around Asymptotes}

Given $f(x)$ has a root at $x_1$.

Consider approaching the root from the left.

If $f(x)$ approaches $x_1$ from above the x axis (positive), the asymptote at
$x_1$ on the reciprocal graph will approach positive infinity from the left.

If $f(x)$ approaches $x_1$ from below the x axis (negative), the asymptote at
$x_1$ on the reciprocal graph will approach negative infinity from the left.

\end{document}
