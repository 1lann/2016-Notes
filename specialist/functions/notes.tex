
\documentclass[a4paper,11pt]{article}

% Math symbols
\usepackage{amsmath}
\usepackage{amsfonts}
\usepackage{esvect}

% Hyperlink contents page
\usepackage{hyperref}
\hypersetup{
	colorlinks,
	citecolor=black,
	filecolor=black,
	linkcolor=black,
	urlcolor=black
}

% \cis and \Arg
\DeclareMathOperator\cis{cis}
\DeclareMathOperator\Arg{Arg}

% No indent on new paragraphs
\setlength{\parindent}{0mm}
\setlength{\parskip}{0.2cm}

% Alias \boldsymbol to \bb for vectors
\newcommand{\bb}{\boldsymbol}


\begin{document}

\title{Functions}
\author{Ben Anderson}
\date{\today}
\maketitle
\pagebreak

\tableofcontents
\pagebreak


\section{Function and Relations}

A function is a series of operations that maps one or more input values to one
output value.

A relation is a series of operations that maps one or more input values to one
or more output values.


\subsection{Types}

A one-to-one function means each input value has one possible output value.
For example, $f(x) = x$.

A many-to-one function means multiple input values will map to the same output
value. For example, $f(x) = x^2$.

A one-to-many relation means one input value will map to multiple output values.
For example, $f(x) = \pm \sqrt{x}$.

A many-to-many relation means multiple input values will map to multiple output
values. For example, $x^2 + y^2 = 9$.


\subsection{Vertical Line Test}

Where a vertical line is moved from left to right over a graph.

If it intersects the curve more than once at any point, the graph is a relation,
not a function.


\subsection{Horizontal Line Test}

Where a horizontal line is moved from top to bottom over a graph.

If it intersects the curve more than once at any point, the function is
one-to-many (not one-to-one).




\section{Domain}

The set of all possible input values for a function.


\subsection{Polynomials}

All polynomials have a domain of $x \in \mathbb{R}$.


\subsection{Fractional Polynomials}

For any function with $x$ in the denominator of a fraction, the entire
denominator cannot equal 0.

For example:

$$
f(x) = \frac{g(x)}{h(x)}
$$

Solve $h(x) = 0$ to determine the values of $x$ which are not in the domain.


\subsection{Square Root}

The contents of the square root must be greater than or equal to 0 (positive).

For example:

$$
f(x) = \sqrt{g(x)}
$$

Solve $g(x) > 0$ to determine the values of $x$ in the domain.


\subsection{Sine and Cosine}

The domain for $\sin{x}$ and $\cos{x}$ is $x \in \mathbb{R}$.


\subsection{Tangent}

The domain for $\tan{x}$ excludes all vertical asymptotes,
$x \neq \frac{\pi}{2} + n\pi, n \in \mathbb{Z}$.


\subsection{Logarithm}

The domain for $\ln{x}$ is $x > 0$, for $\ln{|x|}$ its $x \neq 0$.




\section{Range}

The set of all output values of a function.

The range is a subset of the codomain.


\subsection{Codomain}

The set of all possible output values of a function.

The codomain is part of the definition of the function. It is usually assumed
to be all real numbers, but can be defined as something else.


\subsubsection{Example}

Define the function $f(x) = x^2$ with a domain $x \in \mathbb{Z}$ and codomain
$y \in \mathbb{R}$.

The range of the function is ${1, 4, 9, 16, \ldots}$, yet the codomain was
defined to be all real numbers.

Thus the range is only a subset of the codomain.


\subsection{Calculating Range}

\subsubsection{Polynomials}

Polynomials with an odd degree have a domain and range across all real numbers.

Polynomials with an even degree have a domain across all real numbers and a
range above or below the lowest or highest turning point (depending if the
function has a maximum or minimum turning point).


\subsubsection{Hyperbolas}

Draw a rough sketch of the graph to determine the range of $y$ values which the
function never outputs.


\subsubsection{Square Root}

The result of a square root is greater than or equal to 0.




\section{Asymptotes}

To determine asymptotes for the function:

$$
f(x) = \frac{g(x)}{h(x)}
$$

Where $g(x)$ and $h(x)$ are polynomials.


\subsection{Vertical Asymptotes}

Occur when the denominator is 0.

Solve $h(x) = 0$.


\subsubsection{Behaviour Around Vertical Asymptotes}

Given there is a vertical asymptote at $x = a$.

The function will approach either positive or negative infinity either side of
$x = a$.

For the behaviour of the function as it approaches the asymptote from the left:

\begin{itemize}
\item If $f(a^-)$ is negative, the function approaches negative infinity.
\item If $f(a^-)$ is postiive, the function approaches positive infinity.
\end{itemize}

For the behaviour of the function as it approaches the asymptote from the right:

\begin{itemize}
\item If $f(a^+)$ is negative, the function approaches negative infinity.
\item If $f(a^+)$ is postiive, the function approaches positive infinity.
\end{itemize}


\subsubsection{Example}

Given the function:

$$
f(x) = \frac{2x - 1}{x^2 + 4x}
$$

Solving the denominator equal to 0 for vertical asymptotes:

$$
\begin{aligned}
x^2 + 4x & = 0 \\
x(x + 4) & = 0 \\
x & = 0, -4 \\
\end{aligned}
$$

Thus there are two vertical asymptotes with the equations $x = 0$ and $x = -4$.

Approaching the asymptote $x = -4$ from the left. Consider $f(-4^-)$:

\begin{itemize}
\item $2x$ is negative.
\item $2x - 1$ is still negative.
\item Consider the denominator as $x(x + 4)$.
\item $x + 4$ is still slightly negative.
\item $x(x + 4)$ is therefore two negatives multiplied together, forming a
	positive.
\item Thus overall the fraction is negative, so it approaches negative infinity.
\end{itemize}

Approaching the asymptote $x = -4$ from the right. Consider $f(-4^+)$:

\begin{itemize}
\item $2x - 1$ is negative.
\item $x + 4$ is slightly positive.
\item $x(x + 4)$ is therefore negative.
\item Thus overall the fraction is positive, so it approaches positive infinity.
\end{itemize}


\subsection{Oblique Asymptote}

A function will have at most 1 oblique asymptote.

It only occurs when the degree of the numerator is 1 larger than the denominator.

The linear equation for the asymptote can be found by simplifying the fraction
using algebraic juggling.


\subsection{Horizontal Asymptote}

A function will have at most 1 horizontal asymptote.

If the degree of the numerator is larger, there is no horizontal asymptote.

If the degree of numerator and denominator are the same, then there is a
horizontal asymptote at:

$$
\frac{\text{leading coefficient of $g(x)$}}{\text{leading coefficient of $h(x)$}}
$$

If the degree of the denominator is larger, then there is a horizontal
asymptote at $y = 0$.


\subsubsection{Behaviour Approaching Infinity}

The horizontal asymptote is apparent due to the behaviour of the function as
$x$ approaches $\pm \infty$.

It is important whether the function approaches the horizontal asymptote from
above or below.

Consider:

$$
\frac{\text{leading term of $g(x)$}}{\text{leading term of $h(x)$}}
$$

If the fraction is positive for $x = \infty$, then the function approaches the
horizontal asymptote from above for large positive values of $x$.

If the fraction is negative for $x = \infty$, then the function approaches the
horizontal asymptote from below for large positive values of $x$.

If the fraction is positive for $x = -\infty$, then the function approaches the
horizontal asymptote from above for large negative values of $x$.

If the fraction is negative for $x = -\infty$, then the function approaches the
horizontal asymptote from below for large negative values of $x$.




\section{Graphing Hyperbolas}

To graph the function, where $g(x)$ and $h(x)$ are polynomials:

$$
f(x) = \frac{g(x)}{h(x)}
$$

First attempt to simplify the function to a set of transformations on a normal
hyperbola:

$$
f(x) = \frac{a}{b(x + c)} + d
$$

Otherwise:

\begin{description}
\item [Asymptotes] Find all vertical, horizontal, and oblique asymptotes and
	the behaviour of the function either side of them.
\item [Axis Intercepts] Place a point at all axis intercepts.
\item [Stationary Points] Place a point at all stationary points (using
	$f'(x) = 0$).
\end{description}


\subsection{Simplification}

Given:

$$
f(x) = \frac{x + 3}{x + 3}
$$

This is equivalent to a graph of $f(x) = 1$, but with a point of discontinuity
at $x = -3$ (an open circle).


\subsection{Range}

For some $f(x)$ there is a gap in the middle of the graph for which no $y$ value
is defined.

Given:

$$
f(x) = \frac{3x - 2}{x(x + 2)}
$$

To solve for this gap algebraically:

$$
\begin{aligned}
y & = \frac{3x - 2}{x(x + 2)} \\
yx^2 + 2yx & = 3x - 2 \qquad x \neq 0, -2 \\
yx^2 + (2y - 3)x + 2 & = 0 \\
\end{aligned}
$$

This is a quadratic equation in terms of $x$.

For real valued $y$, the discriminant must be non-negative:

$$
\begin{aligned}
b^2 - 4ac & \geq 0 \\
(2y - 3)^2 - 8y & \geq 0 \\
4y^2 - 20y + 9 & \geq 0 \\
(2y - 9)(2y - 1) & \geq 0 \\
\end{aligned}
$$

Consider the sign of each factor between the roots of this quadratic:

\begin{center}
\begin{tabular}{c|c|c|c}
& $y < 0.5$ & $0.5 < y < 4.5$ & $y > 4.5$ \\
\hline
$2y - 9$ & - & - & + \\
$2y - 1$ & - & + & + \\
$(2y - 9)(2y - 1)$ & + & - & + \\
\end{tabular}
\end{center}

Thus the range for $f(x)$ is $y < 0.5$ and $y > 4.5$.




\section{Composite Functions}

Using the output of one function as the input for another:

$$
f(g(x))
$$


\subsection{Domain and Range}

\subsubsection{Method 1}

Find the resulting expression for the composite function, and determine the
domain and range of it.

Beware of simplifying (especially with square roots), as it can lead to a wrong
answer.


\subsubsection{Method 2}

Given the composite function $f(g(x))$.

$f(x)$ must be able to cope with the output of $g(x)$ for the composite
function to be defined.

Thus the range of $g(x)$ must be a subset of the domain of $f(x)$. It must not
output values $f(x)$ is not defined for.

Thus the domain of $g(x)$ must be restricted to exclude values from its range
for which $f(x)$ is not defined.

For example, given $f(x) = \frac{1}{x - 1}$ and $g(x) = x - 5$.

$f(x)$ has the domain $x \in \mathbb{R}, x \neq 1$ and range
$y \in mathbb{R}, y \neq 0$.

$g(x)$ has a domain and range $x, y \in \mathbb{R}$

For the domain of $f(g(x))$:

$$
\begin{aligned}
g(x) & \neq 1 \\
x - 5 & \neq 1 \\
x \neq 6 \\
\end{aligned}
$$

Thus the domain for $f(g(x))$ is $x \in \mathbb{R}, x \neq 6$.

The range for $f(g(x))$ will be the same as $f(x)$:
$y \in \mathbb{R}, y \neq 0$.




\section{Inverse Functions}

Reverses the operations performed by a function such that:

$$
f^{-1}(f(x)) = x
$$


\subsection{Conditions for an Inverse to Exist}

In order for the inverse of a function to exist as a function (and not a
relation), $f(x)$ must be a one-to-one function.

$f(x)$ must pass the horizontal line test.


\subsection{Domain and Range}

The domain of $f^{-1}(x)$ is the range of $f(x)$.

The range of $f^{-1}(x)$ is the domain of $f(x)$.


\subsection{Graphical Relation}

The inverse of a function represents a reflection about the line $y = x$.

The points at which $f(x)$ will intersect with its inverse are the same as the
intersection between $f(x)$ and the line $y = x$.


\subsection{Calculating}

Given the function $y$:

$$
y = 3x + 4
$$

The inverse function can be found by rearranging for $x$:

$$
\begin{aligned}
3x & = y - 4 \\
x & = \frac{y - 4}{3} \\
f^{-1}(x) & = \frac{x - 4}{3} \\
\end{aligned}
$$


\subsection{Many-To-One Functions}

The domain or range of many-to-one (or many-to-many) functions can be
restricted to make them one-to-one functions.

An inverse function can then be defined for the restricted domain or range.


\subsubsection{Quadratics}

We must decide whether to chose the positive or negative side of the quadratic
depending on the original function.

The inverse of $f(x) = x^2$ for $x < 0$ is $f^{-1}(x) = -\sqrt{x}$. We must
chose the negative square root (rather than using $\pm\sqrt{x}$) due to the
original restriction on the domain of $f(x)$.

The inverse of $f(x) = \sqrt{x - 5}$ is $f^{-1}(x) = x^2 + 5$. We must restrict
the domain of this inverse to accurately reflect $f(x)$.

The range of $f(x)$ is the domain of $f^{-1}(x)$, thus
$x \in \mathbb{R}, x \geq 0$.


\subsubsection{Example}

The function $f(x) = (x - 1)^2$ does not have an inverse function for its
natural domain.

$f(x)$ has a range $y \in \mathbb{R}, y \geq 0$.

The domain can be restricted to $x \geq 1$ or $x \leq 1$. The function is
one-to-one for this domain, and has an inverse.

The inverse is $f^{-1}(x) = 1 \pm \sqrt{x}$. We must chose either the positive
or negative square root based on our original domain restriction.

For the restriction $x \geq 1$ on $f(x)$, the inverse has a domain $x \geq 0$
and range $x \geq 1$.

To satisfy this range, we must chose the positive. The inverse is then
$f^{-1}(x) = 1 + \sqrt{x}$.




\section{Absolute Value Function}

The absolute value function $f(x) = \lvert x \rvert$ is defined as:

$$
\lvert x \rvert = \begin{cases}
\begin{aligned}
	x \qquad & x \geq 0, \\
	-x \qquad & x < 0, \\
\end{aligned}
\end{cases}
$$

It can also be defined as:

$$
\lvert x \rvert = \sqrt{x^2}
$$

The order of the square root and square is important.

All parts of the graph of any $f(x)$ that are below the x axis are reflected
about the x axis.


\subsection{Solving Equations}

$\lvert x \rvert$ represents the distance of $x$ from the origin.

If $\lvert x \rvert = k$, then either $x = k$ or $x = -k$.

Each solution must be checked by substituting it back into the original
equation, as false solutions can arise.


\subsection{Solving Complicated Equations}

Given an equation such as $2 - \lvert x + 1 \rvert = \lvert 3x + 2 \rvert$.

There are four possible equations that can be formed:

\begin{itemize}
\item $2 - (x + 1) = 3x + 2$
\item $2 - (x + 1) = -(3x + 2)$
\item $2 + (x + 1) = 3x + 2$
\item $2 + (x + 1) = -(3x + 2)$
\end{itemize}

These represent the intersection of each branch of the left hand function with
each branch of the right hand one.

Two of these equations will give invalid solutions, so substitute each result
back into the original equation to check its validity.


\subsubsection{Given a Graph}

Given a graph of the left and right hand sides of the equation, we can eliminate
the two equations which give invalid solutions.

For each intersection between the two functions in the graph, determine on which
branch of each function the intersection point lies.

Equate the equation of each of these branches to solve for $x$.


\subsubsection{Squaring}

Given the equation $\lvert x + 1 \rvert = \lvert 3x + 2 \rvert$.

We can also use the definition of the absolute value as $\sqrt{x^2}$ to solve
the equation.

Square both sides:

$$
(x + 1)^2 = (3x + 2)^2
$$

Expand the brackets and solve the resulting quadratic.

Substitute each solution back into the original inequality to ensure their
validity.


\subsection{Solving Inequalities}

$\lvert x \rvert < a$ implies the distance from $x$ to the origin is less than
$a$. This means:

$$
-a < x < a
$$

$\lvert x \rvert > a$ implies the distance from $x$ to the origin is greater
than $a$. This means:

$$
x < -a \qquad x > a
$$

$\lvert x - b \rvert < a$ implies the distance from $x$ to $b$ on a number line
is less than $a$.

$\lvert x - b \rvert > a$ implies the distance from $x$ to $b$ on a number line
is greater than $a$.

Since $\lvert x \rvert$ is always positive, $\lvert x \rvert > -a$ implies
$x \in \mathbb{R}$.

Also, $\lvert x \rvert < -a$ has no solutions for $x$.


\subsection{Solving Complicated Inequalities}

Given an inequality such as $2 - \lvert x + 1 \rvert > \lvert 3x + 2 \rvert$.

Solve for each intersection point by equating the left and right hand sides.

Given two solutions, $x_1$ and $x_2$, there are 2 possible cases:

\begin{itemize}
\item $x_1 < x < x_2$
\item $x < x_1$ and $x > x_2$
\end{itemize}

Find a test value of $x$ for each case, substitute into the original inequality,
and eliminate the incorrect case.

Given there is only one solution, $x_1$, to the equation, there are also 2
possible cases:

\begin{itemize}
\item $x < x_1$
\item $x > x_1$
\end{itemize}

Again, find a test value of $x$ for each case, substitute into the original
inequality, and eliminate the incorrect case.


\subsubsection{Given a Graph}

A graph of the two functions ($2 - \lvert x + 1 \rvert$ and
$\lvert 3x + 2 \rvert$) allows us to determine which of the two possible cases
is correct by inspection.

Solve the inequality for 1 or 2 solutions, $x_1$ and $x_2$, using the graph to
chose the correct branches to equate.

For each case, observe which function is on top of the other, and select the
correct range of $x$ values.




\section{Reciprocal Function}

The reciprocal graph of $f(x)$ is:

$$
\frac{1}{f(x)}
$$


\subsection{Graphing}

Given a graph of $f(x)$ and asked to find the reciprocal:

\begin{itemize}
\item Roots of $f(x)$ will be vertical asymptotes on the reciprocal.
\item The $x$ coordinate of stationary points on $f(x)$ will also be the $x$
	coordinate of a stationary point on the reciprocal.
\item Positive and negative points on $f(x)$ will also be positive or negative
	on the reciprocal.
\item $f(x)$ and its reciprocal will intersect wherever $f(x) = 1$.
\end{itemize}


\subsubsection{Behaviour Around Asymptotes}

Given $f(x)$ has a root at $x_1$, the reciprocal will have a vertical asymptote
with the equation $x = x_1$.
We must determine the behaviour of the reciprocal around this asymptote.

When approaching the asymptote from the left:

\begin{itemize}
\item If $f(x)$ is positive, the asymptote will approach positive infinity.
\item If $f(x)$ is negative, the asymptote will approach negative infinity.
\end{itemize}

Use the same method for approaching the asymptote from the right.




\section{Continuity}

\subsection{Limit}

The limit of a function at $x = a$ is defined when:

$$
\lim_{x \to a^+} f(x) = \lim_{x \to a^-} f(x)
$$

The value obtained when approaching $a$ from the left is equal to that obtained
when approaching $a$ from the right.


\subsection{Continuous}

A function $f(x)$ is continuous at a point $x = a$ if:

\begin{enumerate}
\item $f(a)$ is defined
\item $\lim_{x \to a} f(x)$ is defined
\item $\lim_{x \to a} f(x) = f(a)$
\end{enumerate}

A function is continuous if it satisifes the above conditions for all $x$ in
the function's domain.

Informally, a function is continuous if it can be drawn without lifting the pen
from the page.


\subsection{Differentiable}

A function $f(x)$ is differentiable at point $x = a$ if it is continuous at $a$,
and:

$$
\lim_{h \to 0^+} \frac{f(a + h) - f(a)}{h} = \lim_{h \to 0^-} \frac{f(a + h) - f(a)}{h}
$$

A function is differentiable if it is continuous and satisfies the above
equation for all $x$ in its domain.

\end{document}
