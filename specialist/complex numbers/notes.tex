
\documentclass[a4paper,11pt]{article}

% Math symbols
\usepackage{amsmath}
\usepackage{amsfonts}
\usepackage{esvect}

% Hyperlink contents page
\usepackage{hyperref}
\hypersetup{
	colorlinks,
	citecolor=black,
	filecolor=black,
	linkcolor=black,
	urlcolor=black
}

% \cis and \Arg
\DeclareMathOperator\cis{cis}
\DeclareMathOperator\Arg{Arg}

% No indent on new paragraphs
\setlength{\parindent}{0mm}
\setlength{\parskip}{0.2cm}

% Alias \boldsymbol to \bb for vectors
\newcommand{\bb}{\boldsymbol}


\begin{document}

\title{Complex Numbers}
\author{Ben Anderson}
\date{\today}
\maketitle
\pagebreak

\tableofcontents
\pagebreak


\section{Number Sets}

\subsection{Natural Numbers}

Symbol: $\mathbb{N}$

All positive integers, and 0.


\subsection{Integers}

Symbol: $\mathbb{Z}$

All positive integers, negative integers, and 0.

Contains all natural numbers.


\subsection{Rational Numbers}

Symbol: $\mathbb{Q}$ (meaning quotient)

All numbers, positive and negative, that can be represented in the form
$\frac{a}{b}$ where $a, b \in \mathbb{Z}$.

Contains all integers.


\subsection{Irrational Numbers}

All numbers which cannot be represented in the form $\frac{a}{b}$.

Contains all real numbers not in the set of all rational numbers.


\subsection{Real Numbers}

Symbol: $\mathbb{R}$

The set of all rational and irrational numbers.


\subsection{Complex Numbers}

Symbol: $\mathbb{C}$

The set of all numbers with a real and imaginary component.

Contains all real numbers.



\section{Complex Numbers}

\subsection{i}

$i$ is defined as the solution to the equation:

$$
i^2 = -1
$$


\subsection{Representation}

A complex number is a number with both a real and imaginary component.

Written as:

$$
z = a + bi
$$

This representation is called cartesian or rectangular form.


\subsection{Components}

Given the complex number $z = a + bi$.

The real component is $Re(z) = a$.

The imaginary component is $Im(z) = b$.


\subsection{Conjugate}

The conjugate of $z = a + bi$ is $\overline{z} = a - bi$.

Found by negating the imaginary component.


\subsubsection{Properties}

The conjugate of a complex number has the following properties:

$$
\begin{aligned}
\overline{z_1 \pm z_2} & = \overline{z_1} \pm \overline{z_2} \\
\overline{z_1 \times z_2} & = \overline{z_1} \times \overline{z_2} \\
\overline{(\frac{z_1}{z_2})} & = \frac{\overline{z_1}}{\overline{z_2}} \\
\end{aligned}
$$




\section{Arithmetic}

\subsection{Surd Simplification}

Examples of simplifying complex surds:

$$
\begin{aligned}
\sqrt{-64} & = \sqrt{-1} \times \sqrt{64} \\
& = 8i \\
\end{aligned}
$$

$$
\begin{aligned}
\sqrt{-8} & = \sqrt{-1} \times \sqrt{8} \\
& = 2 \sqrt{2} i \\
\end{aligned}
$$


\subsection{Addition and Subtraction}

Add both real and complex components separately.

Given $z_1 = a + bi$ and $z_2 = c + di$:

$$
z_1 + z_2 = a + c + (b + d)i
$$


\subsection{Multiplication or Division by Scalar}

Multiply or divide each component by the scalar.

Given $z = a + bi$:

$$
\begin{aligned}
cz & = c(a + bi) \\
& = ca + cbi \\
\end{aligned}
$$


\subsection{Multiplication by Complex Number}

Distribute the multiplication across each component.

Given $z_1 = a + bi$ and $z_2 = c + di$:

$$
\begin{aligned}
z_1 z_2 & = (a + bi)(c + di) \\
& = ac + adi + bci + bdi^2 \\
& = ac - bd + (ad + bc)i \\
\end{aligned}
$$


\subsection{Multiplication by Conjugate}

Given $z = a + bi$ and its conjugate $\overline{z} = a - bi$:

$$
\begin{aligned}
z \overline{z} & = (a + bi)(a - bi) \\
& = a^2 + b^2 \\
\end{aligned}
$$


\subsection{Division by Complex Number}

Multiply the numerator and denominator by the conjugate of the denominator.

$$
\begin{aligned}
\frac{4 + 5i}{2 + i} & = \frac{4 + 5i}{2 + i} \times \frac{2 - i}{2 - i} \\
& = \frac{13 + 6i}{3} \\
& = \frac{13}{3} + 2i \\
\end{aligned}
$$


\subsection{Equality}

Two complex numbers are equal if and only if both their real and imaginary
components are equal.

Given $z_1 = z_2$, then $Re(z_1) = Re(z_2)$ and $Im(z_1) = Im(z_2)$.




\section{Quadratics}

\subsection{Roots}

Given the quadratic $ax^2 + bx + c$.

If the discriminant $b^2 - 4ac < 0$, then the quadratic will have 2 complex
roots.

Complex roots for any polynomial with real coefficients appear in conjugate
pairs.


\subsection{Solving}

\subsubsection{Quadratic Formula}

Apply the formula:

$$
x = \frac{-b \pm \sqrt{b^2 - 4ac}}{2a}
$$

To solve for the 2 complex conjugate roots.


\subsubsection{Completing the Square}

Complete the square:

$$
\begin{aligned}
x^2 + 4x + 10 & = 0 \\
(x + 2)^2 + 6 & = 0 \\
(x + 2)^2 & = -6 \\
x & = -2 \pm \sqrt{6}i
\end{aligned}
$$


\subsection{Expression From Roots}

Given a complex root for a quadratic $z = a + bi$.

The other root is its conjugate $\overline{z} = a - bi$.

The quadratic expression can be found by expanding:

$$
(x - z)(x - \overline{z}) = (x - (a + bi))(x - (a - bi))
$$


\subsection{Coefficients From Roots}

Given the quadratic $ax^2 + bx + c$ with roots $x_1$ and $x_2$ (real or
complex):

$$
\begin{aligned}
-\frac{b}{a} & = x_1 + x_2 \\
\frac{c}{a} & = x_1 x_2 \\
\end{aligned}
$$




\section{Argand Plane}

A cartesian plane, where the y axis is the imaginary axis, and the x axis the
real axis.

Complex numbers can be plot as vectors on this plane.


\subsection{Magnitude}

The length of the vector formed by the complex number when placed on the plane.

Given $z = a + bi$, the magnitude $\lvert z \rvert$ is:

$$
\lvert z \rvert = \sqrt{a^2 + b^2}
$$


\subsubsection{Properties}

The magnitude of a complex number has the following properties:

$$
\begin{aligned}
\lvert z_1 z_2 \rvert & = \lvert z_1 \rvert \lvert z_2 \rvert \\
\lvert \frac{z_1}{z_2} \rvert & = \frac{\lvert z_1 \rvert}{\lvert z_2 \rvert} \\
\end{aligned}
$$


\subsection{Argument}

The angle the complex number makes with the positive real axis.

Given $z = a + bi$, the argument $\Arg(z)$ is:

$$
\Arg(z) = \tan^{-1}(\frac{b}{a})
$$

Usually given in radians in the range $-\pi < \Arg(z) \leq \pi$.


\subsection{Transformations}

\subsubsection{Conjugate}

Reflection about the x axis.


\subsubsection{Multiplication by i}

Rotation anti-clockwise by $90^\circ$.

Expontents of $i$ represent multiples of this rotation:

\begin{center}
\begin{tabular}{c|c}
$n$ & Rotation for $i^n$ \\
\hline
-1 & $90^\circ$ clockwise \\
0 & No rotation \\
1 & $90^\circ$ anti-clockwise \\
2 & $180^\circ$ \\
3 & $90^\circ$ clockwise \\
4 & No rotation \\
\end{tabular}
\end{center}




\section{Polar Form}

Representation of a complex number by a magnitude and angle, measured
anti-clockwise from the positive real axis.

For a complex number of magnitude $r$, making an angle of $\theta$ with the
positive real axis:

$$
z = r (\cos{\theta} + i\sin{\theta})
$$

Also written:

$$
z = r \cis{\theta}
$$

Where $r > 0$ and $-\pi < \theta \leq \pi$.


\subsection{Simplifcation}

Add or subtract multiples of $2\pi$ to restrict $\theta$ to the range
$-\pi < \theta \leq \pi$.

If $r < 0$, take the absolute value of $r$, and add $\pi$ to $\theta$ to rotate
by $180^\circ$.


\subsection{Conversion}

\subsubsection{From Cartesian Form}

$r$ is the magnitude of the complex number.

$\theta$ is its argument.

Given the complex number $z = a + bi$, the polar form of this number will be:

$$
\begin{aligned}
z & = r \cis{\theta} \\
r & = \sqrt{a^2 + b^2} \\
\theta & = \tan^{-1}(\frac{b}{a}) \\
\end{aligned}
$$


\subsubsection{To Cartesian Form}

Expand the definition of $r \cis{\theta}$ to:

$$
z = r \cos{\theta} + ri\sin{\theta}
$$

And calculate the cosine and sine values.


\subsection{Arithmetic}

\subsubsection{Conjugate}

Negate the argument.

Given $z = r \cis{\theta}$:

$$
\overline{z} = r \cis(-\theta)
$$

Thus $\cos{-\theta} + i\sin{-\theta}$ can be written as
$\cos{\theta} - i\sin{\theta}$ (important for proofs).


\subsubsection{Negation}

Subtract or add $\pi$ to the argument.

Given $z = r \cis{\theta}$:

$$
-z = r \cis(\theta \pm \pi)
$$

Chose positive or negative depending on which will result in an argument within
the correct principle range.


\subsubsection{Addition and Subtraction}

Convert to cartesian form and perform the addition or subtraction.


\subsubsection{Multiplication}

Multiply radii and add arguments.

Given $z_1 = r_1 \cis{\theta_1}$ and $z_2 = r_2 \cis{\theta_2}$:

$$
z_1 z_2 = r_1 r_2 \cis(\theta_1 + \theta_2)
$$


\subsubsection{Division}

Divide radii and subtract arguments.

Given $z_1 = r_1 \cis{\theta_1}$ and $z_2 = r_2 \cis{\theta_2}$:

$$
\frac{z_1}{z_2} = \frac{r_1}{r_2} \cis(\theta_1 - \theta_2)
$$




\section{Regions}

\subsection{Lines Perpendicular to Axes}

\subsubsection{Real Axis}

For a line perpendicular to the real axis at $k$, where $k \in \mathbb{R}$:

$$
Re(z) = k
$$


\subsubsection{Imaginary Axis}

For a line perpendicular to the imaginary axis at $k$, where $k \in \mathbb{R}$:

$$
Im(z) = k
$$


\subsection{Circles}

For a circle centred at the origin, with radius $r$:

$$
\lvert z \rvert = r
$$

For a circle centred at $a + bi$, with radius $r$:

$$
\lvert z - (a + bi) \rvert = r
$$

For a region containing all points within a circle at center $a + bi$, with
radius $r$, including all points on the circumference:

$$
\lvert z - (a + bi) \rvert \leq r
$$

Excluding the circumference:

$$
\lvert z - (a + bi) \rvert < r
$$


\subsubsection{Maximum Modulus}

Given the equation of a circle:

$$
\lvert z - (a + bi) \rvert = r
$$

The maximum value of $\lvert z \rvert$ will be the distance to the centre of
the circle, plus the radius:

$$
= \sqrt{a^2 + b^2} + r
$$


\subsubsection{Minimum or Maximum Argument}

Given the equation of a circle:

$$
\lvert z - (a + bi) \rvert = r
$$

The minimum or maximum value of $\Arg(z)$ is a tangent to the circle,
perpendicular to the radius.

Find the angle between the tangent and the vector from the origin to the centre
of the circle using tan.

The minimum or maximum argument is then the argument to the centre of the
circle, plus or minus this angle.


\subsection{Perpendicular Bisector}

For a perpendicular bisector between $a + bi$ and $c + di$:

$$
\lvert z - (a + bi) \rvert = \lvert z - (c + di) \rvert
$$


\subsection{Rays}

For a ray extending from the origin in direction $\theta$:

$$
\Arg(z) = \theta
$$

Excludes the origin as the argument of 0 is undefined.

For a ray starting at $a + bi$ in direction $\theta$:

$$
\Arg(z - (a + bi)) = \theta
$$

Excludes the starting point $a + bi$, as the argument of 0 is undefined.


\subsection{Cartesian Equation}

Substitude $x + yi$ for $z$ in the complex equation and simplify.


\subsubsection{Circle}

$$
\begin{aligned}
\lvert (x + yi) - (a + bi) \rvert & = r \\
\lvert (x - a) + i(y - b) \rvert & = r \\
(x - a)^2 + (y - b)^2 & = r^2 \\
\end{aligned}
$$


\subsubsection{Perpendicular Bisector}

A perpendicular bisector equation will simplify to a straight line.


\subsubsection{Rays}

Rays can be represented by a line with a restriction on the $x$ value.

The gradient of the line will be $\tan{\theta}$ (given $\theta$ is in the first
quadrant):

$$
\begin{aligned}
\Arg(z) & = \theta \\
y & = \tan{\theta} x \quad x > 0 \\
\end{aligned}
$$

Must be $x > 0$ (not $x \geq 0$) since the ray is undefined at the origin.

If $\theta$ isn't in the first quadrant, then modify the restriction on $x$
accordingly.




\section{De Moivre's Theorem}

For the exponent of a complex number in polar form:

$$
(r \cis{\theta})^n = r^n \cis{n \theta}
$$

The theorem is valid for positive, negative, and fractional indices.


\subsection{Proof}

For $n = 1$:

$$
\begin{aligned}
(r \cis{\theta})^1 = r^1 \cis(1 \theta) \\
r \cis{\theta} = r \cis{\theta} \\
\end{aligned}
$$

Thus the theorem holds for $n = 1$.

Assume $n = k$:

$$
(r \cis{\theta})^k = r^k \cis(k \theta) \\
$$

Let $n = k + 1$:

$$
\begin{aligned}
(r \cis{\theta})^{k + 1} & = (r \cis(\theta))(r \cis(\theta))^k \\
& = r^{k + 1} (\cis(\theta) \cis(k \theta)) \\
& = r^{k + 1} (\cos(\theta) + i \sin(\theta))(\cos(k \theta) + i \sin(k \theta)) \\
& = r^{k + 1} (\cos(\theta) \cos(k \theta) - \sin(\theta) \sin(k \theta) + i(\sin(\theta) \cos(k \theta) + \cos(\theta) \sin(k \theta))) \\
& = r^{k + 1} (\cos((k + 1) \theta) + i \sin((k + 1) \theta)) \\
\end{aligned}
$$


\subsubsection{For Negative Integers}

$$
\begin{aligned}
(r \cis{\theta})^{-1} & = \frac{1}{r \cis{\theta}} \\
& = \frac{\cis{0}}{r \cis{\theta}} \\
& = \frac{1}{r} \cis{0 - \theta} \\
& = \frac{1}{r} \cis{-\theta} \\
\end{aligned}
$$


\subsection{For Cartesian Form}

To raise a complex number in cartesian form to a certain power:

\begin{enumerate}
\item Convert it to polar form and use De Moivre's Theorem.
\item Expand using binomial expansion and simplify.
\end{enumerate}


\subsection{Expansion of Trigonometric Functions}

Express $\cos{n \theta}$ or $\sin{n \theta}$ in terms of only $\cos{\theta}$ or
$\sin{\theta}$:

\begin{enumerate}
\item Express as $\cis{n \theta}$.
\item Expand to $\cos{n \theta} + i \sin{n \theta}$.
\item Reverse De Moivre's Theorem $(\cos{\theta} + i \sin{\theta})^n$.
\item Expand using binomial expansion.
\item Take only the real or complex component.
\end{enumerate}




\section{Roots of Real Number}

The equation:

$$
z^n = k
$$

Will have $n$ complex and real solutions.

The first solution will be the $n$th root of $k$ (ie. $\sqrt[n]{k}$).

The roots will evenly divide the complex plane into $n$ regions (evenly spaced
around a unit circle).

The roots will appear in conjugate pairs.


\subsection{Example}

For the equation:

$$
z^5 = 1
$$

The first solution is $\sqrt[5]{1} = 1$, equivalent to $\cis{0}$.

The angle between adjacent roots will be $\frac{2\pi}{5}$.

Thus all roots are:

\begin{itemize}
\item $\cis{0}$
\item $\cis{\frac{2\pi}{5}}$
\item $\cis{\frac{4\pi}{5}}$
\item $\cis{-\frac{4\pi}{5}}$
\item $\cis{-\frac{2\pi}{5}}$
\end{itemize}

Ensure all roots are given with $r > 0$ and $-\pi < \theta \leq \pi$.




\section{Roots of Complex Number}

The equation:

$$
z^n = a + bi
$$

Will have $n$ complex and real solutions.

The roots will evenly divide the complex plane into $n$ regions (evenly spaced
around a unit circle).

The roots do not necessarily appear in conjugate pairs.


\subsection{Given a Solution}

Given the equation:

$$
z^6 = -8i
$$

Has a solution $1 + i$, find the other solutions.

Implies $(1 + i)^6 = -8i$.

Writing $1 + i$ in polar form, $\sqrt{2} \cis{\frac{\pi}{4}}$.

Angle between roots will be $\frac{2\pi}{6} = \frac{\pi}{3}$.

Thus solutions are:

\begin{itemize}
\item $\sqrt{2} \cis{\frac{\pi}{4}}$ (given)
\item $\sqrt{2} \cis{\frac{7\pi}{12}}$
\item $\sqrt{2} \cis{\frac{11\pi}{12}}$
\item $\sqrt{2} \cis{-\frac{3\pi}{4}}$
\item $\sqrt{2} \cis{-\frac{5\pi}{12}}$
\item $\sqrt{2} \cis{-\frac{\pi}{12}}$
\end{itemize}


\subsection{Given No Solutions}

Given the equation:

$$
z^n = a + bi
$$

Write $a + bi$ in polar form with an additional constant:

$$
a + bi = r \cis{\theta + 2k\pi} \quad k \in \mathbb{Z}
$$

This implies:

$$
\begin{aligned}
z^n & = r \cis{\theta + 2k\pi} \\
(z^n)^{\frac{1}{n}} & = (r \cis{\theta + 2k\pi})^{\frac{1}{n}} \\
z & = \sqrt[n]{r} \cis{\frac{\theta}{n} + \frac{2k\pi}{n}} \\
\end{aligned}
$$

Substitute $n$ values of $k$ centred around 0 to find $n$ solutions for $z$.

Centre the values of $k$ around 0 to ensure the arguments are within the range
$-\pi < \theta \leq \pi$.




\section{Polynomials}

\subsection{Remainder Theorem}

The remainder when $f(x)$ is divided by $x - a$ is $f(a)$.


\subsubsection{Proof}

Consider the division identity:

$$
f(x) = D(x) Q(x) + R(x)
$$

When $f(x)$ is divided by $x - a$, the remainder must be at least 1 order less
than the divisor, thus $R(x)$ is a constant:

$$
f(x) = (x - a) Q(x) + k \quad k \in \mathbb{R}
$$

Consider $f(a)$:

$$
\begin{aligned}
f(a) & = (a - a) Q(a) + k \\
& = k \\
\end{aligned}
$$

Thus $f(a)$ is the remainder when $f(x)$ is divided by $x - a$.


\subsection{Factor Theorem}

$x - a$ is a factor of $f(x)$ if and only if $f(a) = 0$.


\subsubsection{Proof}

If $f(a) = 0$, then $f(x)$ will have a remainder of 0 when divided by $x - a$,
by the remainder theorem proved above.

Thus $x - a$ is a factor of $f(x)$.


\subsection{Factorising}

Find factors of $f(x)$ using the factor theorem by guessing values of $a$.

Try values of $a$ that are factors of the constant term.

Find as many real factors as possible.


\subsubsection{Inspection}

If sufficient real factors can be found such that we are left with a series of
linear factors and 1 complex quadratic factor, like so:

$$
\begin{aligned}
f(x) & = 8x^4 + 8x^3 - 4x^2 - 3x - 9 \\
& = (x - 1)(2x + 3)(ax^2 + bx + c) \\
\end{aligned}
$$

For $a$:

$$
\begin{aligned}
1 \times 2 \times a & = 8 \\
a & = 4 \\
\end{aligned}
$$

For $c$:

$$
\begin{aligned}
-1 \times 3 \times c & = -9 \\
c & = 3 \\
\end{aligned}
$$

For $b$ (using the cubed term):

$$
\begin{aligned}
1 \times 2 \times b + 1 \times 3 \times a - 1 \times 2 \times a & = 8 \\
2b + 12 - 8 & = 8 \\
b & = 2 \\
\end{aligned}
$$

Thus $f(x) = (x - 1)(2x + 3)(4x^2 + 2x + 3)$.


\subsubsection{Algebraic Juggling}

For example, $x^2 + 3x + 4$ divided by $x + 4$:

$$
\begin{aligned}
\frac{x^2 + 3x + 4}{x + 4} & = \frac{x(x + 4) - x + 4}{x + 4} \\
& = x + \frac{-x + 4}{x + 4} \\
& = x + \frac{-(x + 4) + 8}{x + 4} \\
& = x - 1 + \frac{8}{x + 4} \\
\end{aligned}
$$

Thus the quotient is $x - 1$, with a remainder of $8$.


\subsection{Solving}

Factorise the polynomial, leaving only linear or quadratic factors.

Apply the null factor law to find solutions.


\subsubsection{Real Coefficients}

All complex roots will appear in conjugate pairs.


\subsubsection{Complex Coefficients}

Complex roots do not necessarily appear in conjugate pairs.

Also try simple complex factors of the constant term when guessing values for
$a$.

\end{document}
