
\documentclass[a4paper,11pt]{article}

% Math symbols
\usepackage{amsmath}

% Chemistry
\usepackage{mhchem}
\usepackage{chemfig}

% Tables
\usepackage{tabularx}
\usepackage{multirow}
\newcolumntype{Y}{>{\centering\arraybackslash}X}

% Set style for diagrams
\setdoublesep{0.35700 em}
\setatomsep{1.78500 em}
\setbondoffset{0.18265 em}
\newcommand{\bondwidth}{0.06642 em}
\setbondstyle{line width = \bondwidth}

% No indent on new paragraphs
\setlength{\parindent}{0mm}

\begin{document}

\title{Esters Investigation}
\author{Ben Anderson}
\date{\today}
\maketitle


% CONTENTS
% - Aim
% - Calculations of the mass of ethanol and mass of ethanoic acid required
% - Equipment
% - Procedure
% - Results
% - Explanation for why we might have no ester layer
% 	- Water bath not hot enough
% 	- Not enough H2SO4
% 	- H2SO4 is dense so it might have sunk to the bottom and not mixed throughout solution
% 	- Not left in water bath for long enough due to time restrictions in class
% 	- Didn't rinse with water thoroughly enough to dissolve all alcohol and ethanoic acid
% 	- Didn't shake hard enough
% 	- Didn't add enough water
% - Results taken from another group
% - Calculation for theoretical yield
% - Percentage yield
% - Reasons for lower percentage yield
% 	- Equilibrium reaction
% 	- transfer costs
% - Measures taken and ones that could have been taken to increase percentage yield
% 	- High temperature
% 		- Endothermic reaction
% 		- Rate of reaction
% 	- Concentration
% 	- Leave in water bath for longer
% 		- Allow reaction more time to reach equilibrium
% 	- Continually stir mixture to ensure sulfuric acid is mixed throughout
% - Justification for separation
% 	- Solubility of ethanol and ethanoic acid
% 	- Bonding (hydrogen bonding vs. dispersion)
% - Errors
% 	- Solubility of alcohol and acid
% 		- Could perform more washes to reduce this
% 		- Would also wash out more ester because of its solubility
% 	- Excess water in weighing product
% 		-
% - Conclusion


\section{Aim}

The aim of the investigation is to synthesise ethyl ethanoate, extract it from
any unreacted acid and alcohol, and measure our percentage yield of ester.


\section{Preliminary Calculations}

$$
\ce{CH3CH2OH + CH3COOH <=> CH3COOCH2CH3 + H2O}
$$

$$
\ce{\chemfig{C(-[:90]H)(-[:180]H)(-[:270]H)(-[:0]C(-[:90]H)(-[:270]H)(-[:0]O(-[:0]H)))} +
	\chemfig{C(-[:90]H)(-[:180]H)(-[:270]H)(-[:0]C(=[:45]O)(-[:315]O(-[:0]H)))} <=>
	\chemfig{C(-[:90]H)(-[:180]H)(-[:270]H)(-[:0]C(=[:45]O)(-[:315]O(-[:0]C(-[:90]H)(-[:270]H)(-[:0]C(-[:90]H)(-[:0]H)(-[:270]H)))))} + H2O}
$$

We want the total volume of the esterification mixture to be about 25 mL. We want to
react the ethanol and ethanoic acid in a stoichiometric ratio, which means we
need to add an equal number of moles of the reactants. \\

If we use 0.2 mol of ethanol:

$$
\begin{aligned}
	\mbox{M($\ce{CH3CH2OH}$)} & = 2 \times 12.01 + 6 \times 1.008 + 16 \\
			& = 46.086\mbox{ g mol}^{-1} \\
	\mbox{m($\ce{CH3CH2OH}$)} & = 46.086 \times 0.2 \\
			& = 9.217\mbox{ g} \\
	\rho\mbox{($\ce{CH3CH2OH}$)} & = 0.789\mbox{ g mL}^{-1} \\
	\mbox{V($\ce{CH3CH2OH}$)} & = \frac{9.217}{0.789} \\
			& = 11.68\mbox{ mL} \\
\end{aligned}
$$

And 0.2 mol of ethanoic acid:

$$
\begin{aligned}
	\mbox{M($\ce{CH3COOH}$)} & = 2 \times 12.01 + 4 \times 1.008 + 2 \times 16 \\
							& = 60.052\mbox{ g mol}^{-1} \\
	\mbox{m($\ce{CH3COOH}$)} & = 60.052 \times 0.2 \\
			& = 12.01\mbox{ g} \\
	\rho\mbox{($\ce{CH3COOH}$)} & = 1.05\mbox{ g mL}^{-1} \\
	\mbox{V($\ce{CH3COOH}$)} & = \frac{12.01}{1.05} \\
			& = 11.44\mbox{ mL} \\
\end{aligned}
$$

This results in a total volume of $11.68 + 11.44 = 23.12\mbox{ mL}$ of
reactants. Since we will be adding an additional 2 mL of concentrated
$\ce{H2SO4}$ as a catalyst, this will make up our target of 25 mL. \\

Therefore the required masses are 9.217 g of ethanol, and 12.01 g of ethanoic
acid.

% Let the number of moles of ethanol and ethanoic acid be $x$. For ethanol:

% $$
% \begin{aligned}
% 	\mbox{M} & = 2 \times 12.01 + 6 \times 1.008 + 16 \\
% 			& = 46.086\mbox{ g mol}^{-1} \\
% 	\mbox{m} & = 46.086 x \\
% 	\rho & = 0.789\mbox{ g mL}^{-1} \\
% 	\mbox{V} & = \frac{46.086x}{0.789} \\
% \end{aligned}
% $$

% For ethanoic acid:

% $$
% \begin{aligned}
% 	\mbox{M} & = 2 \times 12.01 + 4 \times 1.008 + 2 \times 16 \\
% 							& = 60.052\mbox{ g mol}^{-1} \\
% 	\mbox{m} & = 60.052 x \\
% 	\rho & = 1.05\mbox{ g mL}^{-1} \\
% 	\mbox{V} & = \frac{60.052x}{1.05} \\
% \end{aligned}
% $$

% Therefore:

% $$
% \begin{aligned}
% 	\frac{46.086 x}{0.789} + \frac{60.052 x}{1.05} & = 25 \\
% 	x & = 0.216\mbox{ mol} \\
% \end{aligned}
% $$

% Hence we need 0.216 mol of ethanol and ethanoic acid. \\

% The required mass of each substance is:

% $$
% \begin{aligned}
% \mbox{m}(\ce{CH3CH2OH}) & = 46.086 \times 0.216 \\
% 							& = 9.955\mbox{ g} \\
% \mbox{m}(\ce{CH3COOH}) & = 60.052 \times 0.216 \\
% 							& = 12.97\mbox{ g} \\
% \end{aligned}
% $$

% The required volume of solution for each substance is:

% $$
% \begin{aligned}
% 	\mbox{V}(\ce{CH3CH2OH}) & = \frac{9.955}{0.789} \\
% 							& = 12.6\mbox{ mL} \\
% 	\mbox{V}(\ce{CH3COOH}) & = \frac{12.97}{1.05} \\
% 							& = 12.4\mbox{ mL} \\
% \end{aligned}
% $$

% Hence we need 12.6 mL of ethanol and 12.4 mL of ethanoic acid.




\section{Equipment}

\begin{itemize}
\item Ethanol
\item Ethanoic acid
\item Concentrated sulfuric acid
\item Set of scales
\item 300 mL of tap water
\item Kettle
\item 300 mL beaker
\item 100 mL beaker
\item Distilled water
\item Funnel
\item Separation flask and glass stopper
\item Retort stand
\item Clamp
\item Pipette
\end{itemize}


\section{Procedure}

\begin{enumerate}
\item Weigh the empty separation flask and record its mass
\item Weigh out 9.217 g of ethanol in the 100 mL beaker using the set of scales
\item Add the ethanol to the boiling tube
\item Weigh out 12.01 g of ethanoic acid in the 100 mL beaker using the set of
	scales
\item Add the ethanoic acid to the boiling tube
\item Add roughly 2 mL of concentrated sulfuric to the boiling tube using the pipette
\item Boil 300 mL of tap water in a kettle
\item Add the boiling water to the 300 mL beakers
\item Place the boiling tube in the hot water bath
\item Leave for at least 30 minutes
\item Place a funnel into the top of the separation flask
\item Pour contents of boiling tube into separation flask
\item Wash any residual mixture left in the boiling tube and on the funnel into the separation flask using distilled water
\item Add more distilled water to the separation flask to make up at least 50 mL
	of solution
\item Remove the funnel and plug the top of the separation flask with the glass stopper
\item Thoroughly shake the separation flask to completely mix its contents
\item Attach the separation flask to the retort stand using the clamp
\item Leave the separation flask still to allow its contents to separate into two
	distinct layers
\item Discard the contents of the 300 mL beaker and place it below the separation flask
\item Carefully open the tap on the separation flask to let the lower aqueous layer drain into the beaker
\item Close the tap just before all of the aqueous layer has been removed
\item Plug the top of the separation flask with the glass stopper
\item Remove the separation flask from the retort stand and weigh it and its
	contents using the set of scales, recording the measured mass


% \item Measure out 12.6 mL of ethanol using the measuring cylinder and add it to the boiling tube
% \item Measure out 12.4 mL of ethanoic acid using the measuring cylinder and add it to the boiling tube
% \item Place the boiling tube in the hot water bath
% \item Leave for 30 minutes
% \item Place a funnel into the top of the separation flask
% \item Pour contents of boiling tube into separation flask
% \item Wash any residual mixture left in the boiling tube and on the funnel into the separation flask using distilled water
% \item Add more distilled water to the separation flask to make up at least 50 mL of the mixture
% \item Plug the top of the separation flask using the rubber stopper
% \item Shake the separation flask so its contents are thoroughly mixed
% \item Attach the separation flask to the retort stand *EXPAND ON THIS POINT!!!!!!!!!!!!!!!!!!!!!!!!!!!!!!!!!!!!!!!!!!!!!!!!!!!!!!!!!!!!!!!!!!!!!!!!!!*
% \item Let the contents of the flask settle into two distinct layers
% \item Place the second 500 mL beaker below the separation flask
% \item Remove the rubber stopper from the top of the separation flask
% \item Open the tap on the separation flask to let the lower aqueous layer drain into the beaker
% \item Close the tap just before all of the aqueous layer has been removed
% \item Discard the contents of the beaker
% \item Repeat steps something to something another two times
% \item Plug the top of the separation flask with the rubber stopper
% \item Weigh the separation flask and its contents and record the mass
\end{enumerate}


\section{Results}

Unfortunately, the contents of our separation flask failed to separate into two
distinct layers, preventing us from obtaining a valid result. Some potential
reasons for this are discussed below. I have acquired another successful group's
set of results, which are used in a percentage yield calculation below: \\

The mass of the empty separation flask was 191.6 g. \\

The mass of the separation flask with the synthesised ester was 202.1 g. \\

The mass of ester produced is therefore $202.1 - 191.6 = 10.5\mbox{ g}$. \\

The theoretical yield of ester using 0.2 mol of ethanol and ethanoic acid is:

$$
\begin{aligned}
	\mbox{n(ester)} & = \mbox{n(ethanol)} \\
					& = 0.2\mbox{ mol} \\
	\mbox{M(ester)} & = 4 \times 12.01 + 8 \times 1.008 + 2 \times 16 \\
			& = 88.104\mbox{ g mol}^{-1} \\
	\mbox{m(ester)} & = 88.104 \times 0.2 \\
			& = 17.62\mbox{ g} \\
\end{aligned}
$$

Therefore the percentage yield is:

$$
\begin{aligned}
	& = \frac{10.5}{17.62} \times 100 \\
	& = 59.6 \% \\
\end{aligned}
$$


\section{Discussion}

\subsection{Justification for Separation}

Hydrogen bonding is where a hydrogen atom bonded to one of nitrogen, oxygen or
fluorine in a molecule is strongly attracted to a lone pair of electrons on an
oxygen atom in another molecule. Both ethanol and ethanoic acid are capable of
hydrogen bonding, with ethanol
having 1 potential site, and ethanoic acid with 2. Water also experiences
hydrogen bonding between its molecules. Therefore when dissolving ethanol and
ethanoic acid in water, the solute/solute and solvent/solvent bonds broken
(hydrogen bonds) are
of comparable strength to the solute/solvent bonds formed (also hydrogen
bonds), allowing both ethanol and ethanoic acid to dissolve in the added
water. \\

Ethyl ethanoate is not capable of forming hydrogen bonds with itself - only
dispersion and dipole/dipole forces. It is, however, capable of forming hydrogen
bonds with
water when partially dissolved in it, due to the lone pairs of electrons on the
oxygen atoms in the molecule. This means the solvent/solute bonds
formed when dissolved in water (hydrogen bonds) are not of comparable
strength to the
solute/solute bonds (dispersion and dipole/dipole) and solvent/solvent
bonds (hydrogen bonds) broken, meaning ethyl ethanoate is not very soluble in
water. \\

Hence by adding distilled water to the solution in the separation flask, we
dissolve any unreacted alcohol and acid in the water, but do not dissolve much
of the ester (although some is dissolved, discussed below). Due to the different
densities of the aqueous component of the solution (water containing the
dissolved alcohol and acid) and the ester, two distinct
layers will form, with the aqueous layer on the bottom. This allows the aqueous
layer to be removed using the tap on the separation flask, leaving only the
ester. Therefore we have effectively separated the ester from any unreacted
alcohol and acid.


\subsection{Reasons for Less Than 100\% Yield}

One assumption we made in the experiment was that the reaction producing the
ester proceeded to 100\% completion. The reality is that the reaction between
the alcohol and carboxylic acid to produce ethyl ethanoate is an equilibrium,
meaning that it is not possible for all the alcohol and acid to react to form
ester, but rather the system will approach an equilibrium concentration of ester,
alcohol, and acid. Therefore it is not possible to achieve a 100\% yield, as it
is impossible for all the alcohol and acid to react, resulting in a lower number
of moles and thus mass of ester than the corresponding theoretical value. \\

Another assumption we made in conducting the experiment was that none of the ester
dissolved in the added distilled water when separating the ester
from any unreacted alcohol and acid. In reality, the ester does dissolve
slightly in the water. This dissolved ester will then be removed when the
aqueous layer in the separation flask is drained. Therefore we remove some ester
during the separation process, reducing our final mass of ester, which reduces
our overall percentage yield. \\

Another possible reason for not achieving a 100\% yield includes failing to
transfer all of the
alcohol and acid from the original 100 mL beaker they were measured out in
before being
transferred to the boiling tube, as well as potentially leaving residual ester
in the boiling
tube after transferring it to the separation flask. Under the former scenario,
we would have reduced the moles of alcohol and acid reacting in the boiling
tube, reducing the amount of ester produced. The latter situation would have
also reduced the final amount of ester present in our separation flask. Both of
these aspects of the investigation would have reduced the percentage yield of
ester.


\subsection{Increasing Percentage Yield}

One feature of the investigation that was included in an attempt to increase our
percentage yield was the use of a hot water bath. This increased the rate of the
reaction producing the ester, as
a higher temperature increases the frequency of collisions between particles in
the system, and increases the proportion of particles with energy greater than
the activation energy of the reaction. This means the system would reach
equilibrium faster, meaning it was closer to equilibrium than it would've been
otherwise when we removed the boiling tube from the water bath (since we likely removed
the boiling tube before equilibrium was reached). This would have increased our
mass of produced ester, increasing the percentage yield. In addition, the
forward reaction is an
endothermic reaction, meaning that, according to Le Chatelier's Principle, if we
increase the temperature of the system, the system will attempt to counteract
this change by decreasing the temperature through favouring the endothermic
reaction. This shifts the position of
the equilibrium to the right, producing more ester. Both of these factors
would have worked to increase our percentage yield. \\

One thing we could have done in the experiment to increase the percentage yield of
ester further is to have left the boiling tube in the water bath for longer. This would
have allowed the system more time to reach equilibrium, since we likely removed
the tube from the water bath before this occurred. This would have increased the
mass of ester present in the final solution, increasing our percentage yield. \\

Another possible method of increasing the percentage yield is to continually
stir the mixture throughout the period it is in the water bath. The sulfuric
acid catalyst is quite dense relative to the ester, acid, and alcohol, with a density of
$1.84\mbox{ g mL}^{-1}$, compared to that of $0.902\mbox{ g mL}^{-1}$,
$0.789\mbox{ g mL}^{-1}$, and $1.05\mbox{ g mL}^{-1}$ for the ester, ethanol,
and ethanoic acid respectively. This means it is likely that the sulfuric acid
sunk to the bottom of the mixture, causing it to fail to catalyse the
reaction, reducing the rate of reaction. This means the
system would've taken more time to reach equilibrium, making it more likely that
we removed the boiling tube from the water bath before equilibrium was
established, which
would have reduced the amount of ester produced, and thus our percentage yield
would be lower.


\subsection{Reasons for Separation Failure}

There are a number of reasons that explain why our solution in the separation
flask failed to separate into two distinct layers. It is likely a combination of
the following aspects that caused this to happen. \\

One possible reason for the failure is that not enough sulfuric acid was added
to the solution before placing it in the hot water bath. The sulfuric acid acts
as a catalyst in this reaction, which reduces the time needed for the reaction
between the ethanol and ethanoic acid to reach equilibrium by increasing the
proportion of particles in the system with energy greater than the
activation energy for the reaction, which is achieved through providing an
alternate pathway for the reaction that has a lower activation energy. By not
adding enough of this catalyst, the rate of reaction would be so slow that the
system would not have come close to reaching equilibrium by the time we removed
the boiling tube from the water bath. This means very little ester would have
been produced, meaning that two distinct layers were unable to form in the
separation flask, or were simply not visible. \\

Another possible reason includes the fact that sulfuric acid is quite dense
relative to the ester, ethanol, and ethanoic acid, as described in the previous
section. This means that if the sulfuric acid was not
mixed into the solution sufficiently when it was first added, it would have sunk
to the bottom of the boiling tube and failed to catalyse the reaction
effectively. It follows from similar reasoning as in the previous paragraph that
little ester would have been produced, failing to create two distinct layers in
the separation flask. \\

A third possible reason includes that the water bath was not hot enough. The hot
water bath is used to increase the rate of reaction and shift the equilibrium
position to the right, as explained above. If the water was not sufficiently hot
enough, then the rate of reaction may not have been fast enough and the position
of the equilibrium would not have been far enough to the right to produce enough ester
so that we could achieve a successful separation of layers when the mixture is
added to the separation flask. \\

Another reason includes if the mixture was not left in the hot water bath for long enough,
the reaction wouldn't have had time to reach equilibrium (or somewhere close to
equilibrium), meaning not much ester would've been produced. This would also
cause the separation of layers in the separation flask to fail.


\subsection{Errors}

\subsubsection{Solubility of Alcohol and Acid}

In the investigation, we assumed that all of the alcohol and acid will dissolve
in the distilled water, allowing us to successfully remove all of it, leaving
only the ester. In fact, dissolving the alcohol and acid is an equilibrium
process. Not all of
the alcohol and acid will dissolve - some will remain in the final solution.
This will increase the mass of the final solution, increasing the apparent mass
of ester measured, thus increasing our percentage yield value. This decreases
the accuracy of our results. \\

One way to possibly reduce the impact of this error on the results is to perform
multiple washes with water, where steps 14 to 21 in the procedure
are repeated an additional 3 or 4 times. This would have the effect of
dissolving more ethanol and acid in the water with each wash, removing it from
the solution and thus increasing the accuracy of our results (by the same logic
as above). The downside to doing this is that, since the ester is partially
soluble in the water (as explained in a previous section), some ester will
dissolve in the
water with each wash, meaning we end up removing some product with each
wash. This reduces the final mass of ester in the separation flask, reducing our
percentage yield value, although increasing its accuracy by increasing the
purity of the final product.


\subsubsection{Excess Water}

Another source of error in the investigation includes failing to remove all of
the aqueous layer during the separation process. In order to not remove any
ester from our separation flask, we closed the tap on the flask just before all
the aqueous layer had been removed. This would have left a small amount of
the acid and alcohol dissolved in water in the flask, increasing the apparent
mass of ester we measured, artificially increasing our percentage yield. This
would also have reduced the accuracy of our results.


\subsubsection{Incorrect Theoretical Value}

A third source of error in the investigation is the method used to calculate the
theoretical mass of ester produced, used in the percentage yield calculation. In
calculating this theoretical mass, we assumed the reaction would proceed to
100\% completion, and thus the moles of ester produced would be equal to the
moles of ethanol and ethanoic acid reacted. Since this reaction is an
equilibrium in reality, such an assumption is false. In fact, the system would
reach an equilibrium concentration of ester. This means that not all ethanol and
ethanoic acid are reacted to form ester, making the number of moles of ester
produced less than the number of moles of acid and alcohol, meaning our
theoretical mass of ester for a ``100\% yield" is larger than what it should be. This would have
decreased our percentage yield, systematically reducing the accuracy of our
results.


\section{Conclusion}

Unfortunately, we failed at achieving our aim, as our solution in the separation
flask failed to separate into two distinct layers, rendering us unable to
retrieve any ester. \\

Using another group's results, we were able to calculate a percentage yield of
59.6\%, which is relatively close to the expected percentage yield of 65\% when
this process when performed at an industrial scale.

\end{document}
