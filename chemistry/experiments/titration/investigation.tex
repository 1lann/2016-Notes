
\documentclass[a4paper,11pt]{article}

% Math symbols
\usepackage{amsmath}

% Chemistry
\usepackage{mhchem}
\usepackage{chemfig}

% Tables
\usepackage{tabularx}
\usepackage{multirow}
\newcolumntype{Y}{>{\centering\arraybackslash}X}

% Set style for diagrams
\setdoublesep{0.35700 em}
\setatomsep{1.78500 em}
\setbondoffset{0.18265 em}
\newcommand{\bondwidth}{0.06642 em}
\setbondstyle{line width = \bondwidth}

% No indent on new paragraphs
\setlength{\parindent}{0mm}
\setlength{\parskip}{0.2cm}


\begin{document}

\title{Wine Acidity Investigation}
\author{Ben Anderson}
\date{\today}
\maketitle


\section{Aim}

The aim of this investigation is to determine the total acidity of a sample of
white wine, in both $\mbox{gL}^{-1}$ and percentage by mass.


\section{Introduction}

Wine consists of a number of acids that exist in varying concentrations. The
most significant is tartaric acid, but others such as malic, lactic, and citric
acid exist in lesser concentrations. For the purposes of this investigation, we
will assume a wine sample consists of only tartaric acid, a weak diprotic
acid. This will simplify our calculation of total acidity.

Tartaric acid is shown below:

\begin{center}
\chemfig{C(=[:135]O)(-[:-135]HO)(-[:0]C(-[:90]OH)(-[:-90]H)(-[:0]C(-[:90]H)(-[:-90]OH)(-[:0]C(=[:45]O)(-[:-45]OH))))}
\end{center}


\section{Equipment}

Required equipment for this investigation includes:

\begin{itemize}
\item White wine
\item 50 mL burette
\item 25 mL pipette
\item 50 mL beaker
\item 250 mL conical flask
\item Distilled water
\item Standardised $\ce{NaOH}$ solution
\item Phenolphthalein indicator solution
\item Pipette filler
\item Retort stand
\item Burette clamp
\item Funnel
\item Electronic scales
\end{itemize}

The concentration of the standard $\ce{NaOH}$ solution used in this experiment
is $0.1016\mbox{ molL}^{-1}$, with an associated percentage uncertainty of
$0.1\%$.


\subsection{Indicator Justification}

Tartaric acid is known to be a weak acid, which we are titrating against a
strong base ($\ce{NaOH}$). Tartaric acid will react with the $\ce{NaOH}$
solution according to the following equilibrium:

$$
\ce{
\chemfig{C(=[:135]O)(-[:-135]HO)(-[:0]C(-[:90]OH)(-[:-90]H)(-[:0]C(-[:90]H)(-[:-90]OH)(-[:0]C(=[:45]O)(-[:-45]OH))))} +
2OH- <=>
\chemfig{C(=[:135]O)(-[:-135]O^{-})(-[:0]C(-[:90]OH)(-[:-90]H)(-[:0]C(-[:90]H)(-[:-90]OH)(-[:0]C(=[:45]O)(-[:-45]O^{-}))))} +
2H2O
}
$$

This produces tartrate dianions, the conjugate base of a weak acid. These will
react with water according to the reverse reaction of the above equilibrium to
produce excess $\ce{OH-}$ ions.

This reduces the concentration of $\ce{H+}$ ions at the equivalence point,
shifting it into the basic region. Thus we must chose an indicator such that
the end point of the titration is also in the basic region.

Phenolphthalein indicator changes colour in the pH range 8.2 to 10.0 (the basic
region), making it suitable for this titration.

It will change from colourless to pink, as the solution in the conical flask
is initially acidic, but becomes basic as more $\ce{NaOH}$ is added.


\subsection{Pipette Justification}

A 25 mL pipette was selected for use in this experiment over a 20 mL one.

Both pipettes have the same absolute uncertainty in their measurements
($\pm 0.03\mbox{ mL}$). But, if we compare the percentage uncertainties for
each pipette:

$$
\begin{aligned}
\mbox{20 mL} & = \frac{0.03}{20} \times 100 \\
& = 0.15\% \\
\mbox{25 mL} & = \frac{0.03}{25} \times 100 \\
& = 0.12\% \\
\end{aligned}
$$

The 25 mL has a lower percentage uncertainty. Thus if the 25 mL pipette is
used, our final result (the total acidity of the wine) will have a lower
uncertainty associated with it, increasing its accuracy.

This justifies the use of a 25 mL pipette over a 20 mL one.


\section{Procedure}

To conduct this investigation:

\begin{enumerate}
\item Rinse the funnel with the $\ce{NaOH}$ solution.
\item Using the funnel, rinse the burette with the $\ce{NaOH}$ solution.
\item Rinse the 50 mL beaker with white wine.
\item Fill the 50 mL beaker with white wine.
\item Rinse the pipette using the pipette filler with white wine from the
	beaker
\item Set up the equipment as shown in the diagram below.
\item Rinse the conical flask with distilled water twice.
\item Place the conical flask on the set of scales and record its mass.
\item Measure a 25 mL aliquot of white wine from the beaker using the pipette
	and pipette filler.
\item Empty the contents of the pipette into the conical flask.
\item Record the mass of the conical flask and white wine together.
\item Add 2 drops of phenolphthalein indicator into the conical flask.
\item Place the conical flask under the burette as in the diagram.
\item Record the initial volume of $\ce{NaOH}$ in the burette.
\item Open the burette tap to add $\ce{NaOH}$ solution to the conical flask
	while swirling the flask with the other hand, to ensure $\ce{NaOH}$ is
	evenly distributed throughout the solution.
\item Close the tap just as the solution changes colour permanently.
\item Record the final volume of $\ce{NaOH}$ in the burette.
\item Subtract the initial volume from the final volume to calculate the titre
	titre volume.
\item Repeat steps 7 to 18 until three concordant results are obtained, using
	this initial rough titre volume as an indication of when to slow the flow
	of $\ce{NaOH}$ solution from the burette to obtain a more accurate titre
	volume. Ignore the instructions to calculate the mass of white wine any
	additional times.
\item Average the three concordant results to obtain the average titre volume.
\end{enumerate}

% For diagram
\pagebreak


\subsection{Dilution Justification}

Commonly in titrations, the solution of unknown concentration is diluted before
use in order to achieve an average titre volume between 20 and 25 mL.

The wine is known to have a total acidity between $6.0\mbox{ gL}^{-1}$ and
$8.5\mbox{ gL}^{-1}$.

If the total acidity of the wine is $6.0\mbox{ gL}^{-1}$, the expected average
titre volume, assuming no dilution is performed, is:

$$
\begin{aligned}
\big[\mbox{tartaric}\big] & = \frac{6.0}{150.088} \\
& = 0.03998\mbox{ molL}^{-1} \\
n(\mbox{tartaric}) & = 0.025 \times 0.03998 \\
& = 0.0009995\mbox{ mol} \\
n(\ce{NaOH}) & = 0.0009995 \times 2 \\
& = 0.001999 \\
V(\ce{NaOH}) & = \frac{0.001999}{0.1016} \\
& = 19.68\mbox{ mL} \\
\end{aligned}
$$

If the total acidity of the wine is $8.5\mbox{ gL}^{-1}$, the expected average
titre volume, assuming no dilution is performed, is:

$$
\begin{aligned}
\big[\mbox{tartaric}\big] & = \frac{8.5}{150.088} \\
& = 0.05663\mbox{ molL}^{-1} \\
n(\mbox{tartaric}) & = 0.025 \times 0.05663 \\
& = 0.001416\mbox{ mol} \\
n(\ce{NaOH}) & = 0.001416 \times 2 \\
& = 0.002832 \\
V(\ce{NaOH}) & = \frac{0.002832}{0.1016} \\
& = 27.87\mbox{ mL} \\
\end{aligned}
$$

Thus the expected average titre volume, assuming no dilution is performed, will
lie between 19.68 mL and 27.87 mL, which is sufficiently close to our desired
range of 20 mL to 25 mL to justify not performing any dilution.


\section{Uncertainty}

Measurements from the burette have an uncertainty of $\pm 0.05\mbox{ mL}$.

Measurements from the pipette have an uncertainty of $\pm 0.03\mbox{ mL}$.

The uncertainty in the concentration of the standardised $\ce{NaOH}$ solution
is $\pm 0.1\%$. This is a fractional uncertainty of $\frac{0.1}{100} = 0.001$.

Measurements from the set of scales have an uncertainty of
$\pm 0.005\mbox{ g}$.

The mass of 25 mL of wine was measured by subtracting the mass of the conical
flask from the mass of the conical flask and sample of wine. Thus the
uncertainty in this measurement is twice the uncertainty in a single
measurement from the set of scales.

Similarly, the final titre volume was calculated by subtracting the initial
volume in the burette from the final volume. Thus the uncertainty in this
measurement is also twice the uncertainty in a single measurement from the
burette.


\section{Results}

The mass of 25 mL of wine is $24.63 \pm 0.01\mbox{ g}$.

\makebox[\textwidth][c]{
\begin{tabular}{c|c c c c c}
& Rough Titre & Titre 1 & Titre 2 & Titre 3 & Titre 4 \\
\hline
Initial Volume ($\pm 0.05\mbox{ mL}$) &  0.40 &  0.90 &  0.70 &  0.40 &  0.90 \\
Final Volume ($\pm 0.05\mbox{ mL}$)   & 24.65 & 26.15 & 25.90 & 25.10 & 26.10 \\
Titre Volume ($\pm 0.10\mbox{ mL}$)   & 24.25 & 25.25 & 25.20 & 24.70 & 25.20 \\
\end{tabular}
}

Titre 1, 2, and 4 are concordant, and give an average titre volume of:

$$
\begin{aligned}
& = \frac{25.25 + 25.20 + 25.20}{3} \\
& = 25.22 \pm 0.10 \mbox{ mL} \\
\end{aligned}
$$


\section{Concentration Calculation}

We have the following measurements:

$$
\begin{aligned}
V(\ce{NaOH}) & = 25.22 \pm 0.10\mbox{ mL} \\
\big[\ce{NaOH}\big] & = 0.1016 \pm 0.0001\mbox{ molL}^{-1} \\
V(\mbox{tartaric}) & = 25.00 \pm 0.03\mbox{ mL} \\
m(\mbox{wine}) & = 24.63 \pm 0.01\mbox{ g} \\
\end{aligned}
$$

The equilibrium reaction occurring is:

$$
\ce{
\chemfig{C(=[:135]O)(-[:-135]HO)(-[:0]C(-[:90]OH)(-[:-90]H)(-[:0]C(-[:90]H)(-[:-90]OH)(-[:0]C(=[:45]O)(-[:-45]OH))))} +
2OH- <=>
\chemfig{C(=[:135]O)(-[:-135]O^{-})(-[:0]C(-[:90]OH)(-[:-90]H)(-[:0]C(-[:90]H)(-[:-90]OH)(-[:0]C(=[:45]O)(-[:-45]O^{-}))))} +
2H2O
}
$$


To determine the total acidity of the wine in $\mbox{gL}^{-1}$:

$$
\begin{aligned}
n(\ce{NaOH}) & = 0.02522 \times 0.1016 \\
& = 0.002562\mbox{ mol} \\
n(\mbox{tartaric}) & = 0.5 \times n(\ce{NaOH}) \\
& = 0.001281\mbox{ mol} \\
M(\mbox{tartaric}) & = 4 \times 12.01 + 6 \times 16 + 6 \times 1.008 \\
& = 150.088\mbox{ gmol}^{-1} \\
m(\mbox{tartaric}) & = 0.001281 \times 150.088 \\
& = 0.1923\mbox{ g} \\
\big[\mbox{tartaric}\big] & = \frac{0.1923}{0.02500} \\
& = 7.692\mbox{ gL}^{-1} \\
\end{aligned}
$$

The percentage mass of tartaric acid in the wine is, using the mass of tartaric
acid calculated above:

$$
\begin{aligned}
\%(\mbox{tartaric}) & = \frac{0.1923}{24.63} \times 100 \\
& = 0.7808\% \\
\end{aligned}
$$


\subsection{Uncertainty}

The uncertainty in the total acidity of the wine is the combination of the
uncertainty in the average titre volume, $\ce{NaOH}$ solution concentration,
and pipette:

$$
\begin{aligned}
\mbox{fractional} & = \frac{0.10}{25.22} + 0.001 + \frac{0.03}{25.00} \\
& = 0.006165 \\
\mbox{absolute} & = 0.006165 \times 7.692 \\
& = 0.047 \\
\end{aligned}
$$

The total acidity of the wine, including the associated uncertainty, is
$7.692 \pm 0.047\mbox{ gL}^{-1}$.

The uncertainty in the percentage mass of tartaric acid is the combination of
the uncertainty in the average titre volume, $\ce{NaOH}$ solution concentration,
and the mass of wine:

$$
\begin{aligned}
\mbox{fractional} & = \frac{0.10}{25.22} + 0.001 + \frac{0.01}{24.63} \\
& = 0.005371 \\
\mbox{absolute} & = 0.005371 \times 0.7808 \\
& = 0.0042 \\
\end{aligned}
$$

The percentage mass of tartaric acid in the wine, including the associated
uncertainty, is $0.7808 \pm 0.0042\%$.


\section{Errors}

\subsection{Rinsing}

\subsubsection{Burette}

The burette was correctly rinsed using the $\ce{NaOH}$ solution in this
experiment.

Had it been rinsed with distilled water instead, this would have diluted the
$\ce{NaOH}$ solution later added to it. This would increase the volume of $\ce{NaOH}$ required to
neutralise the acid in the wine sample, increasing the average titre volume.
This would have increased the apparent number of moles of tartaric acid in the
sample of wine, increasing the calculated total acidity and percentage mass.

Had it been rinsed with wine, this would have
also reduced the concentration of the $\ce{NaOH}$ solution, as acid present
in the wine would react with the $\ce{OH-}$ ions from the $\ce{NaOH}$ solution,
decreasing the concentration of $\ce{OH-}$ ions in the solution. This would increase the
volume of solution required to neutralise the acid in the wine sample, increasing the
average titre volume and thus the calculated total acidity and percentage mass.


\subsubsection{Pipette}

The pipette was correctly rinsed with the wine in this experiment.

Had it been rinsed with distilled water, this would have diluted the wine sample
added to the conical flask, reducing the number of moles of tartaric acid
present in the sample. This would have decreased the moles of $\ce{NaOH}$
required to neutralise the sample and reach the equivalence point, decreasing
the average titre volume. This would decrease the calculated total acidity and
percentage mass.

Had it been rinsed with $\ce{NaOH}$ solution, this would react with the wine
solution later added to the pipette, reducing the concentration of tartaric
acid in the wine. This would reduce the number of moles of $\ce{NaOH}$ required
to neutralise the remaining tartaric acid in the sample during
titration, decreasing the average titre volume. This would decrease the
calculated total acidity and percentage mass.


\subsubsection{Conical Flask}

The conical flask was correctly rinsed with distilled water in this experiment.

Had it been rinsed with wine instead, this would have increased the number of
moles of tartaric acid in the conical flask, requiring more moles of
$\ce{NaOH}$ to neutralise it during the titration. This would increase the
average titre volume, increasing the calculated percentage mass and total
acidity.

Had it been rinsed with $\ce{NaOH}$ solution, this would have reacted with the
wine sample when added, reducing the concentration of tartaric acid in the
sample. This would result in fewer moles of $\ce{NaOH}$ required to neutralise
the solution, reducing the average titre volume, decreasing the calculated
total acidity and percentage mass.


\subsection{Indicator Choice}

An indicator was chosen in this investigation such that the end point was in the
basic region, in order to be close to the equivalence point.

Consider if we had selected an indicator (like methyl orange or bromophenol blue) such that the end point of the
titration was in the acidic region. Since the solution in the conical flask was initially acidic, the end
point would occur before the equivalence point, decreasing the volume of $\ce{NaOH}$
solution added before the wine sample changes colour, decreasing the average
titre volume. This would decrease the apparent moles of tartaric acid in the sample of wine,
decreasing the calculated total acidity and percentage mass.


\subsection{Systematic Errors}

Systematic errors involve uncertainty in measurements due to inherent flaws
in the measuring equipment.


\subsubsection{Calibration Error}

There is a degree of error in the manufacture of glassware such as the burette,
conical flask, and pipette used in the experiment, as the manufacturing
processes used to produce these pieces of equipment cannot be infinitely
accurate. This decreases the accuracy of readings taken from these devices,
causing a certain uncertainty to be associated with every measurement. This
increases the uncertainty in our final calculated total acidity and percentage
mass. This decreases the overall accuracy of our experiment.


\subsection{Random Errors}

Random errors involve uncertainty in measurements due to inaccuracy in taking a
reading from a piece of equipment.


\subsubsection{Poor Technique}

Poor pipetting or titrating technique will decrease the accuracy of our results.

In regards to pipetting, if we had aligned the top of the wine's meniscus
to the measuring mark on the pipette, instead of the bottom, this would have resulted in less than 25 mL
of wine in the conical flask. This would have decreased our average titre
volume as less moles of $\ce{NaOH}$ would be required to neutralise the sample,
decreasing our calculated percentage mass and total acidity.

In regards to titrating using the burette, if we had not controlled the flow of
$\ce{NaOH}$ solution from the burette properly by opening the tap too far such
that the solution poured out too fast, we
may have not accurately closed the tap just as the indicator changed colour.
This would result in too much solution being added to the conical flask,
increasing our average titre volume and thus the calculated percentage mass and
total acidity.

Another potential issue is failing to effectively distribute the added $\ce{NaOH}$
solution throughout the wine sample in the conical flask by failing to swirl the flask properly.
This increases the difficulty in determining when the indicator has permanently
changed colour, decreasing our accuracy in closing the tap on the burette once
the end point has been reached. This decreases the accuracy of our average
titre volume, and calculated total acidity and percentage mass.


\subsubsection{Human Error}

Another source of random error in the experiment is in relation to reading
values off measuring devices such as the pipette and burette. There is a degree
of uncertainty associated with our ability to take effective measurements from
equipment, as our eyes are not infinitely accurate. This leads to uncertainty
in the volume of wine added to the conical flask (due to inaccuracy when using
the pipette), and the average titre volume (due to inaccuracy when using the
burette). This increases the uncertainty in our final calculated percentage
mass and total acidity.

Another factor that would contribute to this is parallax. This is where the position of
an object or line relative to a measuring device changes when viewed from
different positions. This makes reading measurements from the burette and
pipette more difficult for the human eye, contributing to the uncertainty
associated with these measurements.


\section{Reliability}

For an experiment to be reliable, it must be readily reproducible by an
independent third party using the same method to produce very similar
results. This is usually achieved through repetition of the experiment to
record multiple trials, which are then averaged.

Our experiment included averaging three concordant results from separate
trials, making it reasonably reliable.


\section{Accuracy}

Accuracy is a measure of how close our results are to an accepted value.

The percentage uncertainty in our final total acidity was:

$$
\begin{aligned}
& = \frac{0.047}{7.692} \times 100 \\
& = 0.6110\%
\end{aligned}
$$

This is a relatively small value, indicating our results are likely to be
reasonably accurate.


\section{Validity}

Validity is a measure of how well an experiment tests a hypothesis. This includes
assessing its reliability and accuracy.

Since this experiment (like many in Chemistry) did not test a hypothesis, but
rather performed an analysis, validity is not an applicable consideration.


\section{Conclusion}

We successfully achieved our aim of determining the total acidity of a sample
of white wine, expressing it both as a concentration in $\mbox{gL}^{-1}$ and
a percentage mass.

We determined the total acidity to be $7.692 \pm 0.047\mbox{ gL}^{-1}$, and
the percentage mass to be $0.7808 \pm 0.0042\%$.

We also concluded that our experiment was both reasonably reliable and
accurate.

\end{document}
