
\documentclass[a4paper,11pt]{article}

% Math symbols
\usepackage{amsmath}
\usepackage{amsfonts}
\usepackage{esvect}
\usepackage{mhchem}

% Hyperlink contents page
\usepackage{hyperref}
\hypersetup{
	colorlinks,
	citecolor=black,
	filecolor=black,
	linkcolor=black,
	urlcolor=black
}

% No indent on new paragraphs
\setlength{\parindent}{0mm}
\setlength{\parskip}{0.2cm}


\begin{document}

\title{Acids and Bases}
\author{Ben Anderson}
\date{\today}
\maketitle
\pagebreak

\tableofcontents
\pagebreak


\section{Electrolytes}

Substances that dissolve in water to conduct electricity.


\subsection{Terminology}

Dissociate means for a molecule to split into two ions that previously didn't
exist when dissolved (eg. $\ce{HCl -> H+ + Cl-}$).

Ionise means for a molecule formed from two ions to split into those ions when
dissolved (eg. $\ce{NaOH -> Na+ + OH-}$).


\subsection{Electrical Conductivity}

Substances that dissolve to produce more ions conduct electricity more
effectively.

For a series of different substances at different concentrations, multiply their
concentration by the number of ions produced when they dissolve to get an
effective measure for the conductivity of the solution.


\section{Strong and Weak}

Strong electrolytes fully ionise/dissociate when dissolved. There is none of the
original substance remaining in the solution.

Weak electrolytes only partially ionise/dissociate when dissolved. There is
still some of the original substance present in the solution.


\subsection{Concentration}

Concentrated acids/bases are solutions with a concentration greater than
$1\mbox{ mol L}^{-1}$.

Dilute acids/bases are solutions with a concentration less than or equal to
$1\mbox{ mol L}^{-1}$.


\section{Acid Reactions}

\subsection{With Metal}

The general reaction is:

$$
\ce{\text{Acid} (aq) + \text{Reactive Metal} (s) -> \text{Salt} + H2 (g)}
$$

For example:

$$
\ce{2H+ (aq) + Mg (s) -> Mg^2+ + H2 (g)}
$$

Reactive metals are all metals excluding group 11 metals ($\ce{Cu}$, $\ce{Ag}$,
$\ce{Au}$).


\subsection{With Carbonate}

$$
\ce{\text{Acid} (aq) + \text{Metal Carbonate} (s) -> \text{Salt} + CO2 (g) + H2O (l)}
$$

For example:

$$
\ce{2H+ (aq) + CaCO3 (s) -> Ca^2+ + CO2 (g) + H2O (l)}
$$


\subsection{With Hydrogen Carbonate}

$$
\ce{\text{Acid} (aq) + \text{Metal Hydrogen Carbonate} (s) -> \text{Salt} + CO2 (g) + H2O (l)}
$$

For example:

$$
\ce{H2SO4 (aq) + 2NaHCO3 (s) -> Na2SO4 (aq) + 2CO2 (g) + 2H2O (l)}
$$

$$
\ce{H+ (aq) + NaHCO3 (s) -> Na+ + CO2 (g) + H2O (l)}
$$


\subsection{With Base}

$$
\ce{\text{Acid} (aq) + \text{Base} (aq) -> \text{Salt} + H2O (l)}
$$

For example:

$$
\ce{H2SO4 (aq) + Ba(OH)2 (aq) -> 2BaSO4 (s) + H2O (l)}
$$

$$
\ce{H+ (aq) + OH- (aq) -> H2O (l)}
$$

$$
\ce{H2SO4 (aq) + CuO (s) -> CuSO4 (aq) + H2O (l)}
$$

$$
\ce{2H+ (aq) + O^2- (aq) -> H2O (l)}
$$


\subsection{Base With Ammonia}

$$
\ce{\text{Base} (aq) + \text{Ammonium Salt} -> \text{Salt} + H2O (l) + NH3 (g)}
$$

For example:

$$
\ce{2NaOH (aq) + (NH4)2SO4 (aq) -> Na2SO4 + 2H2O + 2NH3 (g)}
$$


\subsection{Weak Acids}

For all components of the equation, we write the most dominant species.

Weak acids exist mostly as the non-dissociated ion, and are written as such.

For example, ethanoic acid and calcium carbonate:

$$
\ce{CH3COOH (aq) + CaCO3 (s) -> CH3COO- (aq) + CO2 (g) + H2O (l)}
$$




\section{Definitions}

\subsection{Arrhenius}

\subsubsection{Acid}

Forms extra $\ce{H+}$ ions when dissolved in water.

For example:

$$
\ce{HCl (aq) -> H+ + Cl-}
$$


\subsubsection{Base}

Forms extra $\ce{OH-}$ ions when dissolved in water.

For example:

$$
\ce{NaOH (aq) -> Na+ + OH-}
$$


\subsection{Bronsted-Lowry}

\subsubsection{Acid}

Donates a proton to another species.

For example:

$$
\ce{HCl + H2O -> Cl- + H3O+}
$$

$\ce{H3O+}$ is the hydronium ion.


\subsubsection{Base}

Accepts a proton from another species.

For example:

$$
\ce{NH3 + H2O <=> NH4+ + OH-}
$$


\subsection{Strong and Weak}

Strong acids and bases are one way reactions. For example:

$$
\ce{HCl + H2O -> Cl- + H3O+}
$$

Weak acids and bases are reversible reactions. For example:

$$
\ce{CH3COOH + H2O <=> CH3COO- + H3O+}
$$

Strong acids include $\ce{HCl}$, $\ce{HNO3}$, $\ce{H2SO4}$.

Strong bases include $\ce{NaOH}$, $\ce{Ba(OH)2}$, all other oxide and hydroxide
salts.

Weak acids include $\ce{CH3COOH}$, $\ce{H3PO4}$, $\ce{H2CO3}$, $\ce{HNO2}$.

Weak bases include $\ce{NH3}$.


\section{Conjugate Pairs}

A conjugate acid/base pair is two species that differ by a proton (ie.
$\ce{H+}$ ion). Only applies for Bronsted-Lowry acids/bases.

For example, in the reaction:

$$
\ce{H2SO4 + H2O -> HSO4- + H3O+}
$$

$\ce{H2SO4}$ is an acid, and $\ce{HSO4-}$ is its conjugate base.

$\ce{H2O}$ is a base, and $\ce{H3O+}$ is its conjugate acid.


\subsection{Finding the Conjugate Acid/Base}

The conjugate acid of a species implies it is acting as a base, meaning it must
accept a proton. Thus the conjugate acid has 1 more proton.

The conjugate base of a species implies it is acting as an acid, meaning it must
donate a proton. Thus the conjugate base has 1 less proton.


\subsection{Amphiprotic}

A species is amphiprotic if it can act as both an acid and a base.

For example, $\ce{H2PO4-}$:

$$
\ce{H2PO4- + H2O <=> HPO4^2- + H3O+}
$$

$$
\ce{H2PO4- + H2O <=> H3PO4 + OH-}
$$


\section{Multiprotic Acids}

Acids with more than 1 $\ce{H+}$ ion they can donate.

\begin{itemize}
\item Monoprotic acids have 1 $\ce{H+}$ to donate.
\item Diprotic acids have 2.
\item Triprotic acids have 3.
\item Tetraprotic...
\end{itemize}

For example:

\begin{itemize}
\item $\ce{HNO3}$ and $\ce{HCl}$ are monoprotic.
\item $\ce{H2SO4}$ is diprotic.
\item $\ce{H3PO4}$ is triprotic.
\end{itemize}


\subsection{Sulfuric Acid}

$\ce{H2SO4}$ is a diprotic acid.

First ionisation is a one way reaction:

$$
\ce{H2SO4 + H2O -> HSO4- + H3O+}
$$

Second ionisation is reversible:

$$
\ce{HSO4- + H2O <=> SO4^2- + H3O+}
$$

Thus:

\begin{itemize}
\item $\ce{H2SO4}$ only acts as an acid, as it can't accept another proton.
\item $\ce{HSO4-}$ only acts as an acid, as the first ionisation is not
	reversible.
\item $\ce{SO4^2-}$ only acts as a base, as it hasn't got another proton to
	donate.
\item $\ce{HSO4-}$ is a weak base. The equilibrium lies to the left.
\end{itemize}


\section{pH}

A measure of the concentration of $\ce{H+}$ ions in a solution.

$$
\text{pH} = -\log{[\ce{H+}]}
$$

$$
[\ce{H+}] = 10^{-\text{pH}}
$$


\subsection{Scale}

pH of 7 is neutral (neither acidic nor basic).

pH less than 7 is acidic ($[\ce{H+}] > [\ce{OH-}]$).

pH greater than 7 is basic ($[\ce{OH-}] > [\ce{H+}]$).


\subsection{Water Self-Ionisation}

Water self ionises according to the following equilibrium:

$$
\ce{H2O + H2O <=> H3O+ + OH-}
$$

The equilibrium constant for this reaction is:

$$
[\ce{H+}][\ce{OH-}] = 10^{-14}
$$


\subsubsection{Self-Ionisation of Other Substances}

Given the equilibrium constant at a different temperature or of the
self-ionisation reaction of a different substance.

The concentration of $\ce{H+}$ and $\ce{OH-}$ ions are equal in a neutral
solution, thus:

$$
[\ce{H+}]^2 = K
$$


\subsection{Neutral Solution}

A solution is neutral if $[\ce{OH-}] = [\ce{H+}]$.

In a solution not at $25^\circ\mbox{ C}$, the pH of the solution may not be 7,
yet the solution is still neutral.

This is because the position of the equilibrium has changed, thus both the
concentration of $\ce{OH-}$ and $\ce{H+}$ have increased or decreased, but are
still equal.


\subsection{pH From a Base}

Given the concentration of $\ce{OH-}$ ions, use the equilibrium constant for
water to find the concentration of $\ce{H+}$ ions for the pH calculation:

$$
[\ce{H+}] = \frac{10^{-14}}{[\ce{OH-}]}
$$


\section{Acidity Constant}

Same as equilibrium constant for a weak acid.

The smaller the value, the more to the left the equilibrium lies for a weak
acid, the less acidic it is.

Thus larger values correlate with stronger acids.


\section{Acidity of Oxides}

The oxides of period 2 and 3 elements become more acidic across the period.

Metal oxides are basic when dissolved in water (eg. $\ce{Na2O}$ and $\ce{CuO}$).
For example, $\ce{Na2O + H2O -> 2Na+ + 2OH-}$.

Non-metal are acidic when dissolved in water (eg. $\ce{CO2}$ and $\ce{SO3}$).
For example, $\ce{CO2 + H2O -> H2CO3}$ and $\ce{SO3 + H2O -> H2SO4}$.


\section{Hydrolysis of Salts}

When a salt is dissolved in water, any subsequent reactions determine if the
resulting solution is neutral, acidic, or basic.


\subsection{Basic Solutions}

When the negative ion of the salt (anion) is from a weak acid.

Accepts a proton from the water, resulting in a basic solution.

For example, $\ce{Na2CO3}$ dissolves in water:

$$
\ce{Na2CO3 -> 2Na+ + CO3^2-}
$$

$\ce{CO3^2-}$ is the anion in carbonic acid ($\ce{H2CO3}$). It will react with
water like so:

$$
\ce{CO3^2- + H2O -> HCO3- + OH-}
$$

Increase in concentration of $\ce{OH-}$ ions decreases concentration of
$\ce{H+}$, increasing pH, resulting in a basic solution.


\subsection{Acidic Solutions}

When the positive ion of the salt (cation) is from a weak base.

Donates a proton to the water, resulting in an acidic solution.

For example, $\ce{NH4NO3}$ dissolves in water:

$$
\ce{NH4NO3 -> NH4+ + NO3-}
$$

$\ce{NH4+}$ is the conjugate acid of $\ce{NH3}$, a base. It will react with
water like so:

$$
\ce{NH4+ + H2O -> NH3 + H3O+}
$$

Increase in concentration of $\ce{H3O+}$ ions decreases pH, results in an
acidic solution.


\subsection{Neutral Solutions}

When the positive ion (cation) of the salt is not from a weak base, and the
negative ion (anion) of the salt is not from a weak acid.

Thus no subsequent hydrolysis reactions of these ions will take place.

For example, $\ce{NaCl}$ will dissolve in water:

$$
\ce{NaCl -> Na+ + Cl-}
$$

$\ce{Cl-}$ is the conjugate base of $\ce{HCl}$, but since $\ce{HCl}$ is a strong
acid no further reactions will take place.

$\ce{Na+}$ is not from a weak base. Again, no further reactions will take place.


\subsection{Amphiprotic Ions}

Amphiprotic ions as part of a salt could act as either an acid or a base when
dissolved.


\subsubsection{Sulfuric Acid}

Consider $\ce{NaHSO4}$ dissolved in water:

$$
\ce{NaHSO4 -> Na+ + HSO4-}
$$

$\ce{HSO4-}$ could act as either an acid or base, but remember the first
ionisation of $\ce{H2SO4}$:

$$
\ce{H2SO4 + H2O -> HSO4- + H3O+}
$$

Is a one way reaction. Thus $\ce{HSO4-}$ cannot act as a base and revert back
to $\ce{H2SO4}$.

It must act as an acid:

$$
\ce{HSO4- + H2O -> SO4^2- + H3O+}
$$

Increases the concentration of $\ce{H+}$ ions, decreasing pH, results in an
acidic solution.




\section{Buffers}

Solutions that resist changes to pH when small amounts of acid or base are
added.


\subsection{Mechanism}

\subsubsection{Acidic Buffer}

Work as an equilibrium:

$$
\ce{CH3COOH + H2O <=> H3O+ + CH3COO-}
$$

When $\ce{H+}$ is added:

\begin{itemize}
\item Concentration of $\ce{H3O+}$ increases.
\item Reacts according to $\ce{H3O+ + CH3COO- -> CH3COOH + H2O}$.
\item Le Chatelier's Principle states that the equilibrium will shift to the
	left to decrease the concentration of $\ce{H3O+}$.
\item Thus the concentration of $\ce{H3O+}$ returns to just above what it was
	before the acid was added.
\item Since pH is a logarithmic scale, this slight increase is insignificant.
\end{itemize}

When $\ce{OH-}$ is added:

\begin{itemize}
\item Reacts with $\ce{H+}$ according to the equation $\ce{H+ + OH- -> H2O}$.
\item Decreases concentration of $\ce{H3O+}$.
\item Le Chatelier's Principle states that the equilibrium will shift right to
	increase the concentration of $\ce{H3O+}$.
\item Thus the concentration of $\ce{H3O+}$ returns to just below what it was
	before the base was added.
\item Since pH is a logarithmic scale, this slight decrease is insignificant.
\end{itemize}


\subsubsection{Basic Buffer}

$$
\ce{NH3 + H2O <=> NH4+ + OH-}
$$

When $\ce{OH-}$ is added:

\begin{itemize}
\item Concentration of $\ce{OH-}$ increases.
\item Reacts according to $\ce{NH4+ + OH- -> NH3 + H2O}$.
\item Equilibrium shifts left according to Le Chatelier's Principle.
\item Concentration of $\ce{OH-}$ returns to just above original value.
\item Thus concentration of $\ce{H+}$ ions is slightly above original value.
\item Since pH is a logarithmic scale, there is no appreciable change in pH.
\end{itemize}

When $\ce{H+}$ is added:

\begin{itemize}
\item Reacts with $\ce{OH-}$ according to equation $\ce{H+ + OH- -> H2O}$.
\item Decreases concentration of $\ce{OH-}$ ions.
\item Equilibrium shifts right according to Le Chatelier's Principle.
\item Concentration of $\ce{OH-}$ returns to just below original value.
\item Thus concentration of $\ce{H+}$ ions is slightly below original value.
\item Since pH is a logarithmic scale, there is no appreciable change in pH.
\end{itemize}


\subsection{Formation of a Buffer Solution}

Must establish equilibrium between a weak acid or base and its conjugate
acid/base (ie. negative/positive ion).

For example, add $\ce{CH3COOH}$ and $\ce{CH3COONa}$. $\ce{CH3COONa}$ will ionise
to establish equilibrium between $\ce{CH3COOH}$ and $\ce{CH3COO-}$.

For example, add $\ce{NH3}$ and $\ce{NH4NO3}$. $\ce{NH4NO3}$ will ionise to
establish equilibrium between $\ce{NH3}$ and $\ce{NH4+}$.


\subsubsection{Using a Base}

Add $\ce{CH3COOH}$ and a strong base (eg. $\ce{NaOH}$). This causes the
reaction:

$$
\ce{CH3COOH + OH- -> CH3COO- + H2O}
$$

Thus leaving $\ce{CH3COO-}$ ions in the solution.

This establishes an equilibrium between $\ce{CH3COOH}$ and $\ce{CH3COO-}$,
creating the buffer solution.


\subsection{Maximum Buffer Capacity}

Buffer capacity is a measure of how much acid or base can be added to a
buffer solution before an appreciable change in pH is observed.

To maximise buffer capacity:

\begin{itemize}
\item Equal concentrations of the acid/base and its conjugate acid/base. For
	example, equal concentrations of $\ce{CH3COOH}$ and $\ce{CH3COO-}$. This
	gives the equilibrium reaction the largest range.
\item Highest possible concentration of the acid/base and its conjugate. Allows
	for a larger shift in the equilibrium position.
\end{itemize}




\section{Titrations}

Determine concentration of unknown solution using a solution of known
concentration.


\subsection{Equivalence Point}

Where excess $\ce{OH-}$ or $\ce{H+}$ ions are neutralised in a solution being
titrated.


\subsection{Primary Standards}

Solution of known concentration.


\subsubsection{Properties}

Properties of effective primary standards:

\begin{itemize}
\item Solid
\item High purity
\item Stable (doesn't react with any compounds in the air)
\item High molar mass (minimise fractional uncertainty when weighing)
\end{itemize}


\subsubsection{Examples}

Common primary standards include:

\begin{itemize}
\item \ce{Na2CO3}
\item \ce{H2C2O4.H2O}
\end{itemize}


\subsubsection{Preparation Procedure}

To produce a primary standard solution from a solid:

\begin{enumerate}
\item Wash out beaker with distilled water
\item Weigh mass of solid primary standard in beaker.
\item Dissolve in small amount of distilled water.
\item Wash volumetric flask with distilled water.
\item Add beaker contents to volumetric flask.
\item Rinse beaker multiple times into volumetric flask with distilled water.
\item Make up to line in volumetric flask.
\item Mix thoroughly.
\item Rinse container using some of the solution.
\item Add to storage container.
\end{enumerate}


\subsection{Indicators}

Coloured solution added to the titration solution.

Should change colour close to the equivalence point.

Indicates neutralisation of the solution and when to stop the titration.


\subsubsection{End Point}

The point at which the indicator changes colour, and the titration ends.


\subsubsection{Selection}

Indicator must be chosen such that the end point is close to the equivalence
point.


\subsubsection{Common Indicators}

\begin{description}
\item [Methyl Orange] Changes colour in acidic pH range (4 - 6).
\item [Phenolphthalein] Changes colour in basic pH range (8 - 10).
\end{description}

For neutral titration solutions, use either indicator.


\subsection{Strong Acid Added to Strong Base}

Forms a neutral solution at the equivalence point.

Can use either indicator.

For example, $\ce{HCl}$ added to $\ce{NaOH}$:

$$
\ce{H+ + OH- -> H2O}
$$


\subsection{Weak Acid Added to Strong Base}

Forms a basic solution at the equivalence point.

Must use an indicator that changes colour in the basic range.

For example, $\ce{CH3COOH}$ added to $\ce{NaOH}$.

The $\ce{CH3COO-}$ ion reacts with water to form a basic solution (hydrolysis
of salts):

$$
\ce{CH3COO- + H2O <=> CH3COOH + OH-}
$$


\subsection{Strong Acid Added to Weak Base}

Forms an acidic solution at the equivalence point.

Must use an indicator that changes colour in the acidic range.

For example, $\ce{HCl}$ added to $\ce{NH3}$ forms the salt $\ce{NH4Cl}$ (the
neutralisation reaction).

This dissociates into $\ce{NH4+}$ and $\ce{Cl-}$.

The $\ce{NH4+}$ ion reacts with water to form an acidic solution (hydrolysis
of salts);

$$
\ce{NH4+ + H2O <=> H3O+ + NH3}
$$

\end{document}
