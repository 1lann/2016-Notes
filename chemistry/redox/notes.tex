
\documentclass[a4paper,11pt]{article}

% Math symbols
\usepackage{amsmath}
\usepackage{amsfonts}
\usepackage{esvect}
\usepackage{mhchem}

% Hyperlink contents page
\usepackage{hyperref}
\hypersetup{
	colorlinks,
	citecolor=black,
	filecolor=black,
	linkcolor=black,
	urlcolor=black
}

% No indent on new paragraphs
\setlength{\parindent}{0mm}
\setlength{\parskip}{0.2cm}


\begin{document}

\title{Reduction and Oxidation}
\author{Ben Anderson}
\date{\today}
\maketitle
\pagebreak

\tableofcontents
\pagebreak


\section{Redox Reactions}

Redox reactions involve the transfer of electrons between species.

\subsection{Oxidation}

A species is oxidised when it loses electrons.

It is called the reductant or reducing agent, as it reduces the other species
in the reaction.

\subsection{Reduction}

A species is reduced when it gains electrons.

It is called the oxidant or the oxidising agent, as it oxidises the other
species in the reaction.

\subsection{Example}

$$
\ce{2Mg + O2 -> 2MgO}
$$

\begin{itemize}
\item $\ce{Mg}$ and $\ce{O2}$ both have a charge of 0 on the left side of the
	equation.
\item $\ce{Mg}$ donates 2 electrons to an $\ce{O}$ atom.
\item $\ce{Mg}$ has a charge of 2+ and $\ce{O}$ a charge of 2- on the right side
	of the equation.
\item Electrons have been transferred.
\item $\ce{Mg}$ has lost electrons, meaning it has been oxidised (it is the
	reductant).
\item $\ce{O}$ has gained electrons, meaning it has been reduced (it is the
	oxidant).
\end{itemize}



\section{Half Equations}

Half equations split a redox reaction into the interactions of each species
with electrons.

For the equation:

$$
\ce{2Mg (s) + O2 (g) -> 2MgO}
$$

The oxidation half equation is:

$$
\ce{Mg (s) -> Mg^{2+} + 2e^{-}}
$$

The reduction half equation is:

$$
\ce{O2 (g) + 4e^{-} -> 2O^{2-}}
$$

Each half equation must be simplified from the reaction so the ratios between
species are as low as possible.

\subsection{Example}

Another example, for the equation:

$$
\ce{Cl2 (g) + 2KI (aq) -> I2 (aq) + 2KCl (aq)}
$$

The ionic equation is:

$$
\ce{Cl2 (g) + 2I^{-} -> I2 (aq) + 2Cl^{-}}
$$

The oxidation half equation is:

$$
\ce{Cl2 (g) + 2e^{-} -> 2Cl^{-}}
$$

The reduction half equation is:

$$
\ce{2I^{-} -> I2 (aq) + 2e^{-}}
$$

\subsection{Balancing Half Equations}

In acidic conditions:

\begin{enumerate}
\item Balance the atoms in the equation as it is, before adding any other
	species.
\item Add $\ce{H2O}$ to balance oxygen.
\item Add $\ce{H+}$ to balance hydrogen.
\item Add electrons to balance charge.
\end{enumerate}

In basic conditions:

\begin{enumerate}
\item Balance the equation as if it were in acidic conditions.
\item Add $\ce{OH-}$ ions to each side of the equation to neutralise all
	$\ce{H+}$ ions.
\item Cancel out common species on each side of the equation (specifically,
	$\ce{H2O}$).
\end{enumerate}

\subsection{Combining Half Equations}

\begin{enumerate}
\item Multiply each species in each half equation by a factor such that the
	number of electrons in the oxidation and reduction half equations are
	equal.
\item Add all species on each side of each equation together to form one, larger
	equation.
\item Cancel out common species on each side of the equation.
\end{enumerate}



\section{Oxidation Numbers}

An oxidation number can be assigned to each atom in each molecule of an
equation.

A set of rules is used to determine the oxidation number of an atom.

\subsection{Rules}

Rules to determine an atom's oxidation number, listed in order of priority:

\begin{enumerate}
\item If a species is in its elemental state (eg. $\ce{Cl2} or \ce{P4}$), each
	atom has an oxidation number of 0.
\item Monoatomic ions have an oxidation number equal to the charge on the ion
	(eg. $\ce{Na^{+}}$ is +1, $\ce{Al^{3+}}$ is +3).
\item Oxygen is always -2, except in peroxides (eg. $\ce{H2O2}$) where it is -1,
	and in $\ce{F2O}$ where it is +2.
\item Hydrogen is always +1, except in metal hydrides (eg. $\ce{NaH}$) where the
	charge on the metal ion takes precedence.
\item \label{neutral} The oxidation numbers of all atoms in a neutral molecule
	add to 0.
\item \label{ions} The oxidation numbers of all atoms in a polyatomic ion add
	to the charge on the ion.
\end{enumerate}

\subsubsection{Example}

Rule \ref{neutral}:

\begin{itemize}
\item For $\ce{PH3}$, each $\ce{H}$ is +1, so $\ce{P}$ is -3.
\item For $\ce{Na2S2O3}$, each $\ce{Na}$ is +1, each $\ce{O}$ is -2, so each
	$\ce{S}$ is +2.
\item For $\ce{CH3OH}$, $\ce{C}$ is -2.
\end{itemize}

Rule \ref{ions}:

\begin{itemize}
\item For $\ce{NO2^{-}}$, each $\ce{O}$ is -2, so $\ce{N}$ is +3.
\item For $\ce{PO4^{3-}}$, each $\ce{O}$ is -2, so $\ce{P}$ is +5.
\end{itemize}

\subsection{Determining if an Equation is Redox}

An equation is redox if a species has a different oxidation number on each side
of the equation.

\subsection{Determining the Oxidant and Reductant}

A species is oxidised if the oxidation number of an atom in it increases
(consistent with a loss of electrons).

A species is reduced if the oxidation number of an atom in it decreases
(consistent with a gain in electrons).



\section{Standard Reduction Potentials Table}

Used to determine if a redox reaction will occur.

\begin{itemize}
\item Lists a number of common reduction half equations.
\item Gives the electrical potential ($E^\circ$) for each equation under
	standard laboratory conditions.
\end{itemize}

\subsection{Structure}

\begin{itemize}
\item Each half equation is given as a reduction.
\item Reactions that are more readily reduced are at the top of the table.
\itme The strongest oxidants are at the top of the table.
\item Reactions that are more readily oxidised are at the bottom of the table.
\item The strongest reductants are at the bottom of the table.
\end{itemize}

\subsection{Oxidation Half Equation}

Given the reduction half equation in the table, the equivalent oxidation half
equation is the reverse reaction.

To find the $E^\circ$ of an oxidation half equation, negate the value given for
the reduction half equation in the table.

\subsection{Electrical Potential}

Symbol: $E^\circ$

An indication of how spontaneous a redox half reaction is.

\begin{itemize}
\item Reactions with positive $E^\circ$ will occur spontaneously (with no
	activiation energy required from the surroundings).
\item Reactions with more positive $E^\circ$ will occur more readily than those
	with lower $E^\circ$.
\item Reactions with negative $E^\circ$ require energy input from the
	surroundings to occur.
\item $E^\circ$ values in the table are only valid for standard laboratory
	conditions.
\end{itemize}

\subsection{Standard Laboratory Conditions}

Standard laboratory conditions are:

\begin{description}
\item [Temperature] $25^\circ$ Celsius
\item [Pressure] 101.3 kPa
\item [Concentration] $1\text{ mol L}^{-1}$
\end{description}

\subsection{Determining if a Redox Reaction Occurs}

For a redox reaction to occur spontaneously (without energy input), the oxidant
(the species being reduced) must be a stronger oxidant than the reducant (the
species being oxidised).

A simple rule:

\begin{itemize}
\item On the standard reduction potentials table, find the reduction half
	equation with your left hand.
\item Find the oxidation half equation with your right hand.
\item If your left hand is above your right, the reaction occurs spontaneously.
\end{itemize}

The table is only a predictor. There are some exceptions (reactions that occur
spontaneously that are not predicted by the table to do so).



\section{Reactions}

\subsection{Disproportionation Reaction}

\subsection{Metal Displacement Reaction}

$$
\ce{Zn (s) + Cu^{2+} -> Cu (s) + Zn^{2+}}
$$

\begin{itemize}
\item Solid metal placed in aqueous solution containing ions of another metal.
\item Redox reaction may occur (depending on reduction potentials).
\item Coats the surface of the solid metal with another metal.
\end{itemize}

Observations for the above reaction:

\begin{itemize}
\item Silvery solid added to blue solution solution.
\item Salmon pink deposit on the surface of the silvery solid.
\item Blue colour fades from the solution.
\end{itemize}

\subsection{Halogen Displacement Reaction}

$$
\ce{Cl2 (aq) + 2I^{-} -> 2Cl^{-} + I2 (aq)}
$$

Aqueous solution of a halogen added to an ionic solution of another halogen.

Observations for the above reaction:

\begin{itemize}
\item Pale yellow solution added to colourless solution.
\item Solution turns brown.
\end{itemize}

\subsection{Oxidation of Alcohols}



\section{Electrolysis}




\section{Electroplating}



\section{Electrorefining}



\section{Electrochemical Cells}



\section{Comercial Cells}

\end{document}
