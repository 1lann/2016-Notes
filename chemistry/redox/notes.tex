
\documentclass[a4paper,11pt]{article}

% Math symbols
\usepackage{amsmath}
\usepackage{amsfonts}
\usepackage{esvect}
\usepackage{mhchem}
\usepackage{chemfig}

% Hyperlink contents page
\usepackage{hyperref}
\hypersetup{
	colorlinks,
	citecolor=black,
	filecolor=black,
	linkcolor=black,
	urlcolor=black
}

% No indent on new paragraphs
\setlength{\parindent}{0mm}
\setlength{\parskip}{0.2cm}

% Style for organic diagrams
\setdoublesep{0.35700 em}
\setatomsep{1.78500 em}
\setbondoffset{0.18265 em}
\newcommand{\bondwidth}{0.06642 em}
\setbondstyle{line width = \bondwidth}


\begin{document}

\title{Reduction and Oxidation}
\author{Ben Anderson}
\date{\today}
\maketitle
\pagebreak

\tableofcontents
\pagebreak


\section{Redox Reactions}

Redox reactions involve the transfer of electrons between species.

\subsection{Oxidation}

A species is oxidised when it loses electrons.

It is called the reductant or reducing agent, as it reduces the other species
in the reaction.

\subsection{Reduction}

A species is reduced when it gains electrons.

It is called the oxidant or the oxidising agent, as it oxidises the other
species in the reaction.

\subsection{Example}

$$
\ce{2Mg + O2 -> 2MgO}
$$

\begin{itemize}
\item $\ce{Mg}$ and $\ce{O2}$ both have a charge of 0 on the left side of the
	equation.
\item $\ce{Mg}$ donates 2 electrons to an $\ce{O}$ atom.
\item $\ce{Mg}$ has a charge of 2+ and $\ce{O}$ a charge of 2- on the right side
	of the equation.
\item Electrons have been transferred.
\item $\ce{Mg}$ has lost electrons, meaning it has been oxidised (it is the
	reductant).
\item $\ce{O}$ has gained electrons, meaning it has been reduced (it is the
	oxidant).
\end{itemize}



\section{Half Equations}

Half equations split a redox reaction into the interactions of each species
with electrons.

For the equation:

$$
\ce{2Mg (s) + O2 (g) -> 2MgO}
$$

The oxidation half equation is:

$$
\ce{Mg (s) -> Mg^{2+} + 2e^{-}}
$$

The reduction half equation is:

$$
\ce{O2 (g) + 4e^{-} -> 2O^{2-}}
$$

Each half equation must be simplified from the reaction so the ratios between
species are as low as possible.

\subsection{Example}

Another example, for the equation:

$$
\ce{Cl2 (g) + 2KI (aq) -> I2 (aq) + 2KCl (aq)}
$$

The ionic equation is:

$$
\ce{Cl2 (g) + 2I^{-} -> I2 (aq) + 2Cl^{-}}
$$

The oxidation half equation is:

$$
\ce{Cl2 (g) + 2e^{-} -> 2Cl^{-}}
$$

The reduction half equation is:

$$
\ce{2I^{-} -> I2 (aq) + 2e^{-}}
$$

\subsection{Balancing Half Equations}

In acidic conditions:

\begin{enumerate}
\item Balance the atoms in the equation as it is, before adding any other
	species.
\item Add $\ce{H2O}$ to balance oxygen.
\item Add $\ce{H+}$ to balance hydrogen.
\item Add electrons to balance charge.
\end{enumerate}

In basic conditions:

\begin{enumerate}
\item Balance the equation as if it were in acidic conditions.
\item Add $\ce{OH-}$ ions to each side of the equation to neutralise all
	$\ce{H+}$ ions.
\item Cancel out common species on each side of the equation (specifically,
	$\ce{H2O}$).
\end{enumerate}

\subsection{Combining Half Equations}

\begin{enumerate}
\item Multiply each species in each half equation by a factor such that the
	number of electrons in the oxidation and reduction half equations are
	equal.
\item Add all species on each side of each equation together to form one, larger
	equation.
\item Cancel out common species on each side of the equation.
\end{enumerate}



\section{Oxidation Numbers}

An oxidation number can be assigned to each atom in each molecule of an
equation.

A set of rules is used to determine the oxidation number of an atom.

\subsection{Rules}

Rules to determine an atom's oxidation number, listed in order of priority:

\begin{enumerate}
\item If a species is in its elemental state (eg. $\ce{Cl2} or \ce{P4}$), each
	atom has an oxidation number of 0.
\item Monoatomic ions have an oxidation number equal to the charge on the ion
	(eg. $\ce{Na^{+}}$ is +1, $\ce{Al^{3+}}$ is +3).
\item Oxygen is always -2, except in peroxides (eg. $\ce{H2O2}$) where it is -1,
	and in $\ce{F2O}$ where it is +2.
\item Hydrogen is always +1, except in metal hydrides (eg. $\ce{NaH}$) where the
	charge on the metal ion takes precedence.
\item \label{neutral} The oxidation numbers of all atoms in a neutral molecule
	add to 0.
\item \label{ions} The oxidation numbers of all atoms in a polyatomic ion add
	to the charge on the ion.
\end{enumerate}

\subsubsection{Example}

Rule \ref{neutral}:

\begin{itemize}
\item For $\ce{PH3}$, each $\ce{H}$ is +1, so $\ce{P}$ is -3.
\item For $\ce{Na2S2O3}$, each $\ce{Na}$ is +1, each $\ce{O}$ is -2, so each
	$\ce{S}$ is +2.
\item For $\ce{CH3OH}$, $\ce{C}$ is -2.
\end{itemize}

Rule \ref{ions}:

\begin{itemize}
\item For $\ce{NO2^{-}}$, each $\ce{O}$ is -2, so $\ce{N}$ is +3.
\item For $\ce{PO4^{3-}}$, each $\ce{O}$ is -2, so $\ce{P}$ is +5.
\end{itemize}

\subsection{Determine if Redox}

A reaction is a redox reaction if a species has a different oxidation number on
each side of the equation.

\subsection{Determine Oxidant and Reductant}

A species is oxidised if the oxidation number of an atom in it increases
(consistent with a loss of electrons).

A species is reduced if the oxidation number of an atom in it decreases
(consistent with a gain in electrons).



\section{Standard Reduction Potentials Table}

Used to determine if a redox reaction will occur.

\begin{itemize}
\item Lists a number of common reduction half equations.
\item Gives the electrical potential ($E^\circ$) for each equation under
	standard laboratory conditions.
\end{itemize}

\subsection{Structure}

\begin{itemize}
\item Each half equation is given as a reduction.
\item Reactions that are more readily reduced are at the top of the table.
\item The strongest oxidants are at the top of the table.
\item Reactions that are more readily oxidised are at the bottom of the table.
\item The strongest reductants are at the bottom of the table.
\end{itemize}

\subsection{Oxidation Half Equation}

Given the reduction half equation in the table, the equivalent oxidation half
equation is the reverse reaction.

To find the $E^\circ$ of an oxidation half equation, negate the value given for
the reduction half equation in the table.

\subsection{Electrical Potential}

Symbol: $E^\circ$

An indication of how spontaneous a redox half reaction is.

\begin{itemize}
\item Reactions with positive $E^\circ$ will occur spontaneously (with no
	activiation energy required from the surroundings).
\item Reactions with more positive $E^\circ$ will occur more readily than those
	with lower $E^\circ$.
\item Reactions with negative $E^\circ$ require energy input from the
	surroundings to occur.
\item $E^\circ$ values in the table are only valid for standard laboratory
	conditions.
\item All values are measured relative to the standard Hydrogen electrode (which
	has a value of 0).
\end{itemize}

\subsection{Standard Laboratory Conditions}

Standard laboratory conditions are:

\begin{description}
\item [Temperature] $25^\circ$ Celsius
\item [Pressure] 101.3 kPa
\item [Concentration] $1\text{ mol L}^{-1}$
\end{description}

\subsection{Spontaneous Reactions}

For a redox reaction to occur spontaneously (without energy input), the oxidant
(the species being reduced) must be a stronger oxidant than the reducant (the
species being oxidised).

A simple rule:

\begin{itemize}
\item On the standard reduction potentials table, find the reduction half
	equation with your left hand.
\item Find the oxidation half equation with your right hand.
\item If your left hand is above your right, the reaction occurs spontaneously.
\end{itemize}

The table is only a predictor. There are some exceptions (reactions that occur
spontaneously that are not predicted by the table to do so).

\subsection{Exceptions}

The standard reduction potential table is only a predictor of whether a redox
reaction will occur. There are exceptions to these predictions.

Acidified permanganate does not react with water:

$$
\begin{aligned}
\ce{MnO4- + 8H+ + 5e- & -> Mn^{2+} + 4H2O} \\
\ce{2H2O & -> O2 + 4H+ + 4e-} \\
\end{aligned}
$$

Acidified dichromate does not react with water:

$$
\begin{aligned}
\ce{Cr2O7^{2-} + 14H+ + 6e- & -> 2Cr^{3+} + 7H2O} \\
\ce{2H2O & -> O2 + 4H+ + 4e-} \\
\end{aligned}
$$



\section{Reactions}

\subsection{Disproportionation Reaction}

A disproportionation reaction is where the same species is both oxidised and
reduced.

For example:

$$
\ce{2NO2 + H2O -> HNO3 + HNO2}
$$

This reaction has the half equations:

$$
\begin{aligned}
\ce{NO2 + H2O & -> HNO3 + H+ + e-} \\
\ce{NO2 + H+ + e- & -> HNO2} \\
\end{aligned}
$$

\subsection{Metal Displacement Reaction}

$$
\ce{Zn (s) + Cu^{2+} -> Cu (s) + Zn^{2+}}
$$

\begin{itemize}
\item Solid metal placed in aqueous solution containing ions of another metal.
\item Redox reaction may occur (depending on reduction potentials).
\item Coats the surface of the solid metal with another metal.
\end{itemize}

Observations for the above reaction:

\begin{itemize}
\item Silvery solid added to blue solution solution.
\item Salmon pink deposit on the surface of the silvery solid.
\item Blue colour fades from the solution.
\end{itemize}

\subsection{Halogen Displacement Reaction}

$$
\ce{Cl2 (aq) + 2I^{-} -> 2Cl^{-} + I2 (aq)}
$$

Aqueous solution of a halogen added to an ionic solution of another halogen.

Observations for the above reaction:

\begin{itemize}
\item Pale yellow solution added to colourless solution.
\item Solution turns brown.
\end{itemize}

\subsection{Oxidation of Primary Alcohols}

A primary alcohol first oxidises to an aldehyde, then to a carboxylic acid.

Using ethanol, the oxidation half equation is:

\begin{center}
\ce{\chemfig{C(-[::0]C(-[::0]OH)(-[::90]H)(-[::270]H))(-[::90]H)(-[::180]H)(-[::270]H)} ->
\chemfig{C(-[::0]C(=[::45]O)(-[::-45]H))(-[::90]H)(-[::180]H)(-[::270]H)} + 2H+ + 2e-}
\end{center}

Using acidified potassium permanganate ($\ce{KMnO4}$) as the oxidising agent,
the reduction half equation is:

$$
\ce{MnO4- + 8H+ + 5e- -> Mn^{2+} + 4H2O}
$$

The full redox equation is:

\begin{center}
\ce{5\chemfig{C(-[::0]C(-[::0]OH)(-[::90]H)(-[::270]H))(-[::90]H)(-[::180]H)(-[::270]H)} + 2MnO4- + 6H+ ->
5\chemfig{C(-[::0]C(=[::45]O)(-[::-45]H))(-[::90]H)(-[::180]H)(-[::270]H)} + 2Mn^{2+} + 8H2O}
\end{center}

The second stage oxidation half equation is:

\begin{center}
\ce{\chemfig{C(-[::0]C(=[::45]O)(-[::-45]H))(-[::90]H)(-[::180]H)(-[::270]H)} + H2O ->
\chemfig{C(-[::0]C(=[::45]O)(-[::-45]OH))(-[::90]H)(-[::180]H)(-[::270]H)} + 2H+ + 2e-}
\end{center}

The reduction half equation is the same as in the first stage. The full redox
equation is:

\begin{center}
\ce{5\chemfig{C(-[::0]C(=[::45]O)(-[::-45]H))(-[::90]H)(-[::180]H)(-[::270]H)} + 2MnO4- + 6H+ ->
\chemfig{5C(-[::0]C(=[::45]O)(-[::-45]OH))(-[::90]H)(-[::180]H)(-[::270]H)} + 2Mn^{2+} + 3H2O}
\end{center}

\subsection{Oxidation of Secondary Alcohols}

A secondary alcohol oxidises to a keytone.

Using propan-2-ol, the oxidation half equation is:

\begin{center}
\ce{\chemfig{C(-[::0]C(-[::0]C(-[::0]H)(-[::90]H)(-[::270]H))(-[::90]O(-[::0]H))(-[::270]H))(-[::90]H)(-[::180]H)(-[::270]H)} ->
\chemfig{C(-[::0]C(-[::0]C(-[::0]H)(-[::90]H)(-[::270]H))(=[::90]O))(-[::90]H)(-[::180]H)(-[::270]H)} + 2H+ + 2e-}
\end{center}

The reduction half equation is the same as for a primary alcohol. The full redox
equation is:

\begin{center}
\ce{5\chemfig{C(-[::0]C(-[::0]C(-[::0]H)(-[::90]H)(-[::270]H))(-[::90]O(-[::0]H))(-[::270]H))(-[::90]H)(-[::180]H)(-[::270]H)} + 2MnO4- + 6H+ ->
5\chemfig{C(-[::0]C(-[::0]C(-[::0]H)(-[::90]H)(-[::270]H))(=[::90]O))(-[::90]H)(-[::180]H)(-[::270]H)} + 2Mn^{2+} + 8H2O}
\end{center}

\subsection{Oxidation of Tertiary Alcohols}

Tertiary alcohols do not undergo redox reactions.



\section{Electrochemical Cells}

Electrochemical cells produce a voltage across a wire:

\begin{itemize}
\item Redox reactions require the transfer of electrons.
\item We can separate the sites at which oxidation and reduction occur, and
	connect them via a wire.
\item This forces the electron transfer to occur through the wire, producing a
	voltage and current.
\end{itemize}

\subsection{Construction}

% TODO: Diagram

For the construction of an electrochemical cell with the two half equations:

$$
\begin{aligned}
\ce{Ag+ + e- & -> Ag} \\
\ce{Cu^{2+} + 2e- & -> Cu} \\
\end{aligned}
$$

\begin{itemize}
\item Two \textbf{half cells} (beakers) are constructed by filling one beaker
	with $\ce{Cu^{2+}}$ ions, and the other with $\ce{Ag+}$ ions.
\item Two metal \textbf{electrodes} are inserted into each half cell. One made
	from solid $\ce{Cu}$ is inserted into the $\ce{Cu^{2+}}$ beaker. One made
	from solid $\ce{Ag}$ is inserted into the $\ce{Ag+}$ beaker.
\item A wire connects the two electrodes.
\item A \textbf{salt bridge} is placed across the two beakers.
\end{itemize}

\subsection{Half Cells}

Since $\ce{Ag+}$ is a stronger oxidant than $\ce{Cu^{2+}}$ (left above right
rule), it will be reduced spontaneously.

The half equations of the two reactions occurring in the cell are therefore:

$$
\begin{aligned}
\ce{Ag+ + e- & -> Ag} \\
\ce{Cu & -> Cu^{2+} + 2e-} \\
\end{aligned}
$$

The full redox equation is:

$$
\ce{2Ag+ + Cu -> 2Ag + Cu^{2+}}
$$

When specifying which substances to use to construct an electrochemical cell,
full formula must be given. For example, we must write:

\begin{itemize}
\item $\ce{AgNO3}$ solution, rather than a solution of $\ce{Ag+}$ ions.
\item $\ce{CuSO4}$ solution, rather than a solution of $\ce{Cu^{2+}}$ ions.
\end{itemize}

\subsection{Anode and Cathode}

The reduction half equation occurs in the reduction half cell or
\textbf{cathode}:

$$
\ce{Ag+ + e- -> Ag}
$$

The oxidation half equation occurs in the oxidation half cell or
\textbf{anode}:

$$
\ce{Cu -> Cu^{2+} + 2e-}
$$

\subsection{Salt Bridge}

The salt bridge completes the circuit and allows the flow of charge.

It must not form a solid precipitate with any ions in solution in either half
cell.

It can be constructed from:

\begin{itemize}
\item Usually a piece of filter paper soaked in a Group 1 nitrate (eg.
	$\ce{NaNO3}$ or $\ce{KNO3}$).
\item A tube filled with a solution of ions.
\item A porous membrane.
\end{itemize}

\subsection{Cell Potential}

Each half cell has an associated potential given on the Standard Reduction
Potentials Table, in standard conditions.

The reduction half cell produces +0.80 V.

The oxidation half cell -0.34 V. The table lists +0.34 V, but we negate this
value since we require an oxidation potential, and the table lists reduction
potentials.

The \textbf{emf} (voltage) produced by the cell is:

$$
0.80 - 0.34 = +0.43\text{ V}
$$

\subsection{Workings}

\begin{itemize}
\item The solid $\ce{Cu}$ in the anode's electrode wants to be reduced (gain
	electrons).
\item This causes it to oxidise the $\ce{Ag+}$ in the cathode, producing
	electrons.
\item The electrons travel through the wire and allow the reduction of the
	$\ce{Cu}$.
\end{itemize}

Electron production:

\begin{itemize}
\item Electrons are produced by oxidation, making the anode the negative
	terminal.
\item Electrons are consumed in reduction, making the cathode the positive
	terminal.
\end{itemize}

Electron flow:

\begin{itemize}
\item Electrons flow from where they are produced by oxidation to where they
	are consumed by reduction.
\item Thus electrons flow from the anode to the cathode.
\item Conventional current is the flow of positive charge, opposite to the flow
	of electrons, and travels from the cathode to the anode.
\end{itemize}

Flow of ions in the salt bridge:

\begin{itemize}
\item Consider a salt bridge made from filter paper soaked in $\ce{NaNO3}$
	solution.
\item A surplus of positive ions forms at the anode (where $\ce{Cu^{2+}}$ ions
	are produced). The $\ce{NO3-}$ ions move toward the anode to counteract
	this.
\item A deficit of positive ions forms at the cathode (where $\ce{Ag+}$ ions
	are consumed). The $\ce{Na+}$ ions move toward the cathode to counteract
	this.
\end{itemize}

Ion concentration:

\begin{itemize}
\item The concentration of $\ce{Cu^{2+}}$ ions in the anode increases as the
	copper is reduced according to its half equation.
\item The concentration of $\ce{Ag+}$ ions in the cathode decreases as the
	silver is oxidised.
\end{itemize}

Electrode mass:

\begin{itemize}
\item The mass of the $\ce{Cu}$ electrode decreases as solid copper is oxidised,
	according to the oxidation half equation.
\item The mass of the $\ce{Ag}$ electrode increases as solid silver forms
	through reduction, according to the reduction half equation.
\end{itemize}

Observations:

\begin{itemize}
\item The solution of $\ce{Cu^{2+}}$ ions becomes more blue (as the ion
	concetration increases).
\item The copper electrode becomes smaller, decreasing in mass.
\item The silver electrode becomes larger, increasing in mass.
\end{itemize}

\subsection{Label Convention}

For an electrochemical cell constructed using $\ce{Ag}$ and $\ce{Cu}$, we write:

$$
\ce{Cu^{2+}} / \ce{Cu} // \ce{Ag+} / \ce{Ag}
$$

Where the first pair undergoes oxidation and the second reduction.

\subsection{Gaseous Half Cells}

% TODO: Diagram

Instead of using a solid and a solution of ions, we can use a gas and solution
of ions to construct a half cell:

\begin{itemize}
\item A beaker is filled with the ion solution.
\item A flared, open-ended glass container with a gas valve is partially
	submerged in the ion solution in the beaker.
\item An inert, platinum electrode connected to the wire is suspended in the
	glass container.
\item Gas is bubbled through the solution using the valve on the container.
\end{itemize}



\section{Electrolysis}

Electrolysis is splitting a compound into its constituent elements through the
use of electricity.

\subsection{Purpose}

Consider molten $\ce{NaCl}$ salt in a beaker. This will ionise into $\ce{Na+}$
and $\ce{Cl-}$.

The goal is to produce solid $\ce{Na}$ and gaseous $\ce{Cl2}$.

The two half equations involving these ions are:

$$
\begin{aligned}
\ce{Cl2 + 2e- & -> 2Cl-} \qquad E^\circ = +1.36\text{ V} \\
\ce{Na+ + e- & -> Na} \qquad E^\circ = -2.71\text{ V} \\
\end{aligned}
$$

To achieve the goal, we must reduce $\ce{Na+}$ and oxidise $\ce{Cl-}$.

This reaction does not happen simultaneously, as $\ce{Cl}$ is a stronger
oxidant than $\ce{Na}$.

Thus we must force the redox reaction to occur by supplying electrical energy
from an outside source (a battery).

\subsection{Electrical Potential}

The $E^\circ$ value for the above reaction where $\ce{Na+}$ is reduced and
$\ce{Cl-}$ is oxidised is:

$$
-2.71 - 1.36 = -4.07\text{ V}
$$

The negative voltage demonstrates that this reaction does not occur
simultaneously.

For the reaction to occur, we must supply at least 4.07 V.

\subsection{Construction}

% TODO: Diagram

Consider an electrolytic cell using molten $\ce{NaCl}$.

\begin{itemize}
\item A single beaker contains a molten salt (in this case, $\ce{NaCl}$).
\item Two inert electrodes (usually graphite) are partially submerged in the
	molten salt.
\item The two electrodes are connected to a battery that supplies a voltage of
	at least 4.07 V.
\end{itemize}

\subsection{Workings}

For reduction:

\begin{itemize}
\item The attached battery causes a surplus of electrons in the electrode
	attached to the negative terminal.
\item This acts as the supply of electrons for the reduction process.
\item Since $\ce{Cl-}$ in the molten salt can't be reduced, $\ce{Na+}$ must be
	reduced by these electrons.
\item This forms solid $\ce{Na}$ on the electrode.
\end{itemize}

For oxidation:

\begin{itemize}
\item The attached battery causes a deficit of electrons (surplus of positive
	charge) in the electrode attached to the positive terminal.
\item This allows $\ce{Cl-}$ to be oxidised at this electrode, supplying an
	electron to the wire.
\item This forms $\ce{Cl2}$ gas.
\end{itemize}

Terminals:

\begin{itemize}
\item As in an electrochemical cell, oxidation occurs at the anode, reduction
	at the cathode.
\item Unlike an electrochemical cell, the anode is the positive terminal, the
	cathode the negative terminal.
\end{itemize}



\section{Electrorefining}

The process of refining a metal to increase its purity.

\subsection{Construction}

% TODO: Diagram

Consider the electrorefining of copper:

\begin{itemize}
\item A solution of \textbf{acidified} $\ce{Cu^{2+}}$ ions (usually
	$\ce{CuSO4}$) is placed in a beaker.
\item An impure and a pure copper electrode are partially submerged in the
	solution.
\item Connect the impure electrode to the positive terminal to force oxidation
	to occur at it, making it the anode.
\item Connect the pure electrode to the negative terminal to force reduction
	to occur at it, making it the cathode.
\end{itemize}

\subsection{Equations}

Electrorefining involves:

\begin{itemize}
\item Take an impure copper anode, containing impurities like $\ce{Ag}$,
	$\ce{Zn}$, $\ce{Fe}$, etc.
\item Oxidise the substances in the anode to form ions in solution, or
	insoluble anodic sludge below the anode.
\item Reduce the $\ce{Cu^{2+}}$ ions in the solution to form pure copper on the
	cathode.
\end{itemize}

For example, consider a copper anode containing mostly $\ce{Cu}$, with some
$\ce{Ag}$, $\ce{Pb}$, and $\ce{Zn}$ as impurities.

The reduction half equations are:

$$
\begin{aligned}
\ce{Ag+ + e- & -> Ag} \qquad E^\circ = +0.80\text{ V} \\
\ce{Cu^{2+} + 2e- & -> Cu} \qquad E^\circ = +0.34\text{ V} \\
\ce{Pb^{2+} + 2e- & -> Pb} \qquad E^\circ = -0.13\text{ V} \\
\ce{Zn^{2+} + 2e- & -> Zn} \qquad E^\circ = -0.76\text{ V} \\
\end{aligned}
$$

\begin{itemize}
\item Solid $\ce{Cu}$ will be oxidised, increasing the concentration of
	$\ce{Cu^{2+}}$ ions in solution.
\item When a piece of solid $\ce{Pb}$ or $\ce{Zn}$ is encountered, these will
	be oxidised instead of the copper, as they are stronger reductants (meaning
	they are more readily reduced). These will form $\ce{Pb^{2+}}$ and
	$\ce{Zn^{2+}}$ ions in solution.
\item $\ce{Zn^{2+}}$ ions will remain as ions in solution. They will not form
	solid $\ce{Zn}$ on the cathode as they are less readily reduced compared to
	the $\ce{Cu^{2+}}$ ions in solution.
\item $\ce{Pb^{2+}}$ ions will form an insoluble precipitate with the
	$\ce{SO4^{2-}}$ ions from the original $\ce{CuSO4}$ solution, falling to the
	bottom as anodic sludge.
\item $\ce{Ag+}$ will remain as a solid and fall to the bottom of the beaker
	as anodic sludge, as it will not form ions due to being less readily
	oxidised than the copper, which makes up the majority of the anode.
\end{itemize}

\subsection{Observations}

\begin{itemize}
\item The impure anode will reduce in size.
\item The pure copper cathode will increase in size.
\end{itemize}



\section{Electroplating}

Coating a material with a thin layer of another metal.

\subsection{Construction}

% TODO: Diagram

Consider electroplating a copper cathode with silver:

\begin{itemize}
\item Fill a beaker with a solution of $\ce{Ag+}$ ions (eg. $\ce{AgNO3}$).
\item Submerge the copper and silver electrodes in the solution.
\item Connect the copper electrode to the negative terminal to force reduction
	to occur at it, making it the cathode.
\item Connect the silver electrode to the positive terminal to force oxidation
	to occur at it, making it the anode.
\end{itemize}

\subsection{Equations}

Electroplating involves:

\begin{itemize}
\item Take a silver anode and oxidise the silver to form $\ce{Ag+}$ ions in
	solution.
\item Reduce the $\ce{Ag+}$ ions at the copper cathode to coat it in a thin
	layer of silver metal.
\end{itemize}

The reduction half equations are:

$$
\begin{aligned}
\ce{Ag+ + e- & -> Ag} \qquad E^\circ = +0.80\text{ V} \\
\ce{Cu^{2+} + 2e- & -> Cu} \qquad E^\circ = +0.34\text{ V} \\
\end{aligned}
$$



\section{Lead Acid Battery}

An electrochemical cell that is rechargeable.

\subsection{Construction}

% TODO: Diagram

\begin{itemize}
\item A beaker is filled with $\ce{H2SO4}$.
\item Solid $\ce{Pb}$ and $\ce{PbO2}$ electrodes are submerged in the solution.
\item A wire connects the two electrodes.
\end{itemize}

\subsection{Discharging}

\textbf{Discharging} is the state of operation where a voltage is produced via
a redox reaction.

$\ce{PbO2}$ is reduced at the cathode, and $\ce{Pb}$ is oxidised at the anode.

The reduction half reaction at the cathode is:

$$
\ce{PbO2 + SO4^{2-} + 4H+ + 2e- -> PbSO4 + 2H2O} \qquad E^\circ = +1.69\text{ V}
$$

The oxidation half reaction at the anode is:

$$
\ce{Pb -> Pb^{2+} + 2e-} \qquad E^\circ = +0.13\text{ V}
$$

The $\ce{Pb^{2+}}$ ions form a solid precipitate with $\ce{SO4^{2-}}$ ions in
solution:

$$
\ce{Pb^{2+} + SO4^{2-} -> PbSO4}
$$

Thus the overall reaction at the anode is:

$$
\ce{Pb + SO4^{2-} -> PbSO4 + 2e-} \qquad E^\circ = +0.13\text{ V}
$$

The overall redox reaction is:

$$
\ce{Pb + PbO2 + 2SO4^{2-} + 4H+ -> 2PbSO4 + 2H2O} \qquad E^\circ = +1.82\text{ V}
$$

The solid $\ce{PbSO4}$ is formed at both the anode and cathode, and is deposited
as a thin layer on each.

Electron flow:

\begin{itemize}
\item Electrons are produced at the anode where $\ce{Pb}$ is oxidised, making
	it the negative terminal.
\item These flow to the cathode where they are consumed in the reduction of
	$\ce{PbO2}$, making it the positive terminal.
\end{itemize}

Changes in mass:

\begin{itemize}
\item At the anode, the mass of $\ce{Pb}$ decreases, the mass of $\ce{PbSO4}$
	increases.
\item At the cathode, the mass of $\ce{PbO2}$ decreases, the mass of
	$\ce{PbSO4}$ increases by the same amount as it did at the anode.
\end{itemize}

\subsection{Charging}

\textbf{Charging} is the state of operation where the battery is returned to a
condition where it is capable of continuing to discharge electricity.

$\ce{PbO2}$ is oxidised at the anode, and $\ce{Pb}$ is reduced at the cathode.

The reduction half reaction at the cathode is:

$$
\ce{PbSO4 + 2e- -> Pb + SO4^{2-}} \qquad E^\circ = -0.36\text{ V}
$$

The oxidation half reaction at the anode is:

$$
\ce{PbSO4 + 2H2O -> PbO2 + SO4^{2-} + 4H+ + 2e-} \qquad E^\circ = -1.69\text{ V}
$$

The full redox reaction is:

$$
\ce{2PbSO4 + 2H2O -> Pb + PbO2 + 2SO4^{2-} + 4H+} \qquad E^\circ = -1.82\text{ V}
$$

This is an example of a disproportionation reaction.

Electron flow:

\begin{itemize}
\item Electrons are produced at the anode where $\ce{PbSO4}$ is oxidised, making
	it the negative terminal.
\item These are consumed in the reduction of $\ce{PbSO4}$ at the cathode, making
	it the positive terminal.
\end{itemize}

Changes in mass:

\begin{itemize}
\item The mass of $\ce{PbSO4}$ on both the anode and cathode decreases by the
	same amount.
\item The mass of $\ce{Pb}$ increases on the cathode, where $\ce{PbSO4}$ is
	reduced.
\item The mass of $\ce{PbO2}$ increases on the anode, where $\ce{PbSO4}$ is
	oxidised.
\end{itemize}



\section{Hydrogen Fuel Cell}

An electrochemical cell that never stops producing a voltage as long as fuel is
supplied.

\subsection{Construction}

% TODO: Diagram

\begin{itemize}
\item Oxygen and Hydrogen gas are supplied on either side of a porous membrane.
\item Hydrogen is oxidised, producing electrons that travel across a wire to the
	other side of the membrane (supplying a voltage).
\item Oxygen gas is reduced using the Hydrogen ions that travel across the
	membrane, producing water.
\end{itemize}

\subsection{In Acidic Conditions}

$\ce{O2}$ is reduced at the cathode according to the half equation:

$$
\ce{O2 + 4H+ + 4e- -> 2H2O}
$$

$\ce{H2}$ is oxidised at the anode according to the half equation:

$$
\ce{H2 -> 2H+ + 2e-}
$$

The full redox equation is:

$$
\ce{O2 + 2H2 -> 2H2O}
$$

\subsection{In Alkaline Conditions}

The reduction half equation is:

$$
\ce{O2 + 2H2O + 4e- -> 4OH-}
$$

The oxidation half equation is:

$$
\ce{H2 + 2OH- -> 2H2O + 2e-}
$$

The full redox equation is (as in acidic conditions):

$$
\ce{O2 + 2H2 -> 2H2O}
$$



\section{Corrosion}

\end{document}
