
\documentclass[a4paper,11pt]{article}

% Math symbols
\usepackage{amsmath}
\usepackage{amsfonts}
\usepackage{esvect}
\usepackage{mhchem}
\usepackage{chemfig}

% No indent on new paragraphs
\setlength{\parindent}{0mm}
\setlength{\parskip}{0.2cm}

% AI files
\usepackage{graphicx}
\DeclareGraphicsRule{.ai}{pdf}{.ai}{}

% Style for organic diagrams
\setdoublesep{0.35700 em}
\setatomsep{1.78500 em}
\setbondoffset{0.18265 em}
\newcommand{\bondwidth}{0.06642 em}
\setbondstyle{line width = \bondwidth}


\begin{document}

\title{Organic}
\author{Ben Anderson}
\date{\today}
\maketitle
\pagebreak

\tableofcontents
\pagebreak


\section{Organic Molecules}

A molecule that contains only non-metal elements, and at least one carbon.

\subsection{Chain Length}

Carbon chain length prefixes:

\begin{enumerate}
\item meth-
\item eth-
\item prop-
\item but-
\item pent-
\item hex-
\item hept-
\item oct-
\item non-
\item dec-
\end{enumerate}




\section{Alkanes}

Suffix: -an

Only contain Carbon/Carbon single bonds.

They have the general formula $\ce{C_{n}H_{2n + 2}}$


\subsection{Examples}

Methane:

\begin{center}
\chemfig{C(-[::0]H)(-[::90]H)(-[::180]H)(-[::270]H)}
\end{center}

Butane:

\begin{center}
\chemfig{C(-[::0]C(-[::0]C(-[::0]C(-[::0]H)(-[::90]H)(-[::270]H))(-[::90]H)(-[::270]H))(-[::90]H)(-[::270]H))(-[::90]H)(-[::180]H)(-[::270]H)}
\end{center}


\subsection{Substitution Reaction}

Reaction between an alkane and a halogen.

Replaces a Hydrogen off a Carbon on the alkane.

Produces the hydride of the halogen.

A slow reaction, sped up in the presence of UV light.


\subsubsection{Multiple Substitutions}

Multiple Hydrogen atoms can be substituted for the halogen on the same molecule.

\begin{description}
\item [Monosubstitution] Replacement of 1 Hydrogen atom
\item [Disubstitution] Replacement of 2 Hydrogen atoms
\item [Trisubstitution] ...
\end{description}


\subsubsection{Examples}

Methane and chlorine mono-substitution:

\begin{center}
\ce{\chemfig{C(-[::0]H)(-[::90]H)(-[::180]H)(-[::270]H)} + Cl2 ->
	\chemfig{C(-[::0]H)(-[::90]Cl)(-[::180]H)(-[::270]H)} + HCl}
\end{center}

Methane and chlorine di-substitution:

\begin{center}
\ce{\chemfig{C(-[::0]H)(-[::90]Cl)(-[::180]H)(-[::270]H)} + Cl2 ->
	\chemfig{C(-[::0]Cl)(-[::90]Cl)(-[::180]H)(-[::270]H)} + HCl}
\end{center}




\section{Alkenes}

Suffix: -en

Contain at least 1 Carbon/Carbon double bond.

Formula (for 1 double bond): $\ce{C_{n}H_{2n}}$

Methene doesn't exist.


\subsection{Naming}

Position of double bond included in name.

Carbon atoms are numbered in sequence to minimise the number assigned to the
double bond.


\subsection{Examples}

Ethene:

\begin{center}
\chemfig{C(=[::0]C(-[::45]H)(-[::-45]H))(-[::135]H)(-[::-135]H)}
\end{center}

Propene:

\begin{center}
\chemfig{C(=[::0]C(-[::45]C(-[::0]H)(-[::90]H)(-[::270]H))(-[::-45]H))(-[::135]H)(-[::-135]H)}
\end{center}

But-1-ene:

\begin{center}
\chemfig{C(=[::0]C(-[::45]C(-[::0]C(-[::0]H)(-[::90]H)(-[::270]H))(-[::90]H)(-[::270]H))(-[::-45]H))(-[::135]H)(-[::-135]H)}
\end{center}


\subsection{Addition Reaction}

Reaction between an alkene and a halogen.

Replaces the double bond with a single bond, adding a halogen to each Carbon.

Fast reaction, no UV light needed.


\subsubsection{Examples}

Ethene and chlorine:

\begin{center}
\ce{\chemfig{C(=[::0]C(-[::45]H)(-[::-45]H))(-[::135]H)(-[::-135]H)} + Cl2 ->
	\chemfig{C(-[::0]C(-[::0]H)(-[::90]Cl)(-[::270]H))(-[::90]Cl)(-[::180]H)(-[::270]H)}}
\end{center}

But-1-ene and bromine:

\begin{center}
\ce{\chemfig{C(=[::0]C(-[::45]C(-[::0]C(-[::0]H)(-[::90]H)(-[::270]H))(-[::90]H)(-[::270]H))(-[::-45]H))(-[::135]H)(-[::-135]H)}
	+ Br2 ->
	\chemfig{C(-[::0]C(-[::0]C(-[::0]C(-[::0]H)(-[::90]H)(-[::270]H))(-[::90]H)(-[::270]H))(-[::90]Br)(-[::270]H))(-[::90]Br)(-[::180]H)(-[::270]H)}}
\end{center}


\subsection{Differentiate Between Alkene and Alkane}

Add liquid bromine (orange coloured).

Will turn colourless in an alkene due to the fast addition reaction.

Will remain orange in an alkane.




\section{Alkynes}

Suffix: -yn

Contains at least 1 Carbon/Carbon triple bond.

Position of the triple bond included in the name, like an alkene.

Methyne doesn't exist.


\subsection{Examples}

Ethyne:

\begin{center}
\chemfig{C(~[::0]C(-[::0]H))(-[::180]H)}
\end{center}

But-1-yne:

\begin{center}
\chemfig{C(~[::0]C(-[::0]C(-[::0]C(-[::0]H)(-[::90]H)(-[::270]H))(-[::90]H)(-[::270]H)))(-[::180]H)}
\end{center}




\section{Hydrocarbons}

Molecules containing only Carbon and Hydrogen.

Will combust in presence of Oxygen and a flame (activation energy).


\subsection{Complete Combustion}

$$
\ce{CH4 + 2O2 -> CO2 + H2O}
$$

Any hydrocarbon will combust in an excess of oxygen.

Will form only carbon dioxide and water.


\subsection{Partial Combustion}

$$
\ce{CH4 + O2 -> CO + H2O}
$$

Occurs when Oxygen supply is limited.

Forms leathal carbon monoxide.


\subsection{Incomplete Combustion}

$$
\ce{CH4 + O2 -> C + 2H2O}
$$

Occurs when Oxygen supply is very limited.

Forms just Carbon.




\section{Halogens}

A halogen can replace any carbon on an organic molecule.


\subsection{Naming}

The position and type of halogen are included in the name of the molecule.

The number of the carbon the halogen is attached to is used as its position.

Each halogen has an associated prefix:

\begin{center}
\begin{tabular}{c|c}
Chlorine & chloro- \\
Fluorine & fluoro- \\
Iodine   & iodo-   \\
Bromine  & bromo-  \\
\end{tabular}
\end{center}

More than 1 of the same halogen attached to the same carbon applies another
prefix specifying the number.

Minimising the number of the double bond takes precedence over any halogens.


\subsection{Examples}

Chloro methane:

\begin{center}
\chemfig{C(-[::0]H)(-[::90]Cl)(-[::180]H)(-[::270]H)}
\end{center}

Diiodo methane:

\begin{center}
\chemfig{C(-[::0]I)(-[::90]I)(-[::180]H)(-[::270]H)}
\end{center}

1,2-difluoro ethane:

\begin{center}
\chemfig{C(-[::0]C(-[::0]H)(-[::90]F)(-[::270]H))(-[::90]F)(-[::180]H)(-[::270]H)}
\end{center}

4,4-dibromo 2-iodo but-1-ene:

\begin{center}
\chemfig{C(=[::0]C(-[::45]C(-[::0]C(-[::0]H)(-[::90]Br)(-[::270]Br))(-[::90]H)(-[::270]H))(-[::-45]I))(-[::135]H)(-[::-135]H)}
\end{center}



\section{Alkyl Groups}

Suffix: -yl

Side Carbon chains branching off the main Carbon chain.

Position and length of chain is included in the name.

Numbering of the double bond takes precedence over alkyl groups.


\subsection{Examples}

Methyl propane:

\begin{center}
\chemfig{C(-[::0]C(-[::0]C(-[::0]H)(-[::90]H)(-[::270]H))(-[::90]C(-[::0]H)(-[::90]H)(-[::270]H))(-[::270]H))(-[::90]H)(-[::180]H)(-[::270]H)}
\end{center}

2-methyl butane:

\begin{center}
\chemfig{C(-[::0]C(-[::0]C(-[::0]C(-[::0]H)(-[::90]H)(-[::270]H))(-[::90]H)(-[::270]H))(-[::90]C(-[::0]H)(-[::90]H)(-[::270]H))(-[::270]H))(-[::90]H)(-[::180]H)(-[::270]H)}
\end{center}

2-ethyl butane:

\begin{center}
\chemfig{C(-[::0]C(-[::0]C(-[::0]C(-[::0]H)(-[::90]H)(-[::270]H))(-[::90]H)(-[::270]H))(-[::90]C(-[::0]C(-[::0]H)(-[::90]H)(-[::270]H))(-[::90]H)(-[::270]H))(-[::270]H))(-[::90]H)(-[::180]H)(-[::270]H)}
\end{center}

2-ethyl 3-methyl pentane:

\begin{center}
\chemfig{C(-[::0]C(-[::0]C(-[::0]C(-[::0]H)(-[::90]H)(-[::270]H))(-[::90]H)(-[::270]H))(-[::90]H)(-[::270]H))(-[::90]H)(-[::180]H)(-[::270]H)}
\end{center}




\section{Cyclic Compounds}


\section{Isomers}

\subsection{Structural Isomers}

\subsection{Geometric Isomers}


\section{Saturation}



\section{Functional Groups}

\subsection{Alcohols}

\subsection{Aldehydes}

\subsection{Ketones}

\subsection{Carboxylic Acids}

\subsection{Amines}


\section{Esters}


\section{Polymerisation}


\section{Proteins}

\end{document}
